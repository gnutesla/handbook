\documentclass[10pt,a5paper,twoside,,]{book}
\usepackage{lmodern}
\usepackage{amssymb,amsmath}
\usepackage{ifxetex,ifluatex}
\usepackage{fixltx2e} % provides \textsubscript
\ifnum 0\ifxetex 1\fi\ifluatex 1\fi=0 % if pdftex
  \usepackage[T1]{fontenc}
  \usepackage[utf8]{inputenc}
\else % if luatex or xelatex
  \ifxetex
    \usepackage{mathspec}
  \else
    \usepackage{fontspec}
  \fi
  \defaultfontfeatures{Ligatures=TeX,Scale=MatchLowercase}
\fi
% use upquote if available, for straight quotes in verbatim environments
\IfFileExists{upquote.sty}{\usepackage{upquote}}{}
% use microtype if available
\IfFileExists{microtype.sty}{%
\usepackage{microtype}
\UseMicrotypeSet[protrusion]{basicmath} % disable protrusion for tt fonts
}{}
\usepackage[bindingoffset=1cm,hcentering]{geometry}
\usepackage[unicode=true]{hyperref}
\hypersetup{
            pdfborder={0 0 0},
            breaklinks=true}
\urlstyle{same}  % don't use monospace font for urls
\ifnum 0\ifxetex 1\fi\ifluatex 1\fi=0 % if pdftex
  \usepackage[shorthands=off,main=]{babel}
\else
  \usepackage{polyglossia}
  \setmainlanguage[]{}
\fi
\usepackage{longtable,booktabs}
\usepackage{graphicx,grffile}
\makeatletter
\def\maxwidth{\ifdim\Gin@nat@width>\linewidth\linewidth\else\Gin@nat@width\fi}
\def\maxheight{\ifdim\Gin@nat@height>\textheight\textheight\else\Gin@nat@height\fi}
\makeatother
% Scale images if necessary, so that they will not overflow the page
% margins by default, and it is still possible to overwrite the defaults
% using explicit options in \includegraphics[width, height, ...]{}
\setkeys{Gin}{width=\maxwidth,height=\maxheight,keepaspectratio}
\IfFileExists{parskip.sty}{%
\usepackage{parskip}
}{% else
\setlength{\parindent}{0pt}
\setlength{\parskip}{6pt plus 2pt minus 1pt}
}
\setlength{\emergencystretch}{3em}  % prevent overfull lines
\providecommand{\tightlist}{%
  \setlength{\itemsep}{0pt}\setlength{\parskip}{0pt}}
\setcounter{secnumdepth}{0}
% Redefines (sub)paragraphs to behave more like sections
\ifx\paragraph\undefined\else
\let\oldparagraph\paragraph
\renewcommand{\paragraph}[1]{\oldparagraph{#1}\mbox{}}
\fi
\ifx\subparagraph\undefined\else
\let\oldsubparagraph\subparagraph
\renewcommand{\subparagraph}[1]{\oldsubparagraph{#1}\mbox{}}
\fi
\usepackage{quotchap} % changes style of chapter headings
\usepackage{setspace}
\usepackage{fancyhdr}

\newcommand{\copyleft}{\reflectbox{©}}

\definecolor{grey}{cmyk}{0,0,0,0.6}

\renewcommand{\chapnumfont}{
    \fontsize{44}{46}
    \selectfont
    \color{grey}
}

\pagestyle{fancy}
\setlength{\headheight}{15.2pt}
\fancyhead[]{}
\fancyhead[LE]{\footnotesize{\leftmark}}
\fancyhead[RE]{}

\renewcommand{\chaptermark}[1]{\markboth{#1}{}}
\renewcommand{\sectionmark}[1]{\markright{#1}{}}

\date{}

\begin{document}

{
\setcounter{tocdepth}{2}
\tableofcontents
}
\chapter{Introducción a la versión en español del Manual de
CryptoParty}\label{introducciuxf3n-a-la-versiuxf3n-en-espauxf1ol-del-manual-de-cryptoparty}

La siguiente es la versión en español del CryptoParty Handbook realizada
por el \href{http://www.partidopirata.com.ar}{Partido Pirata de
Argentina}. Antes de que siga leyendo, creemos necesario hacer algunas
aclaraciones.

Hemos respetado fielmente el original traduciendo lo más literalmente
posible al texto, algunas veces lo hemos logrado, otras no tanto. Por
ejemplo, click puede traducirse por hacer click, hacer clic, cliquear,
presionar o pulsar. La redundancia típica de estas palabras muchas veces
en el mismo párrafo, incluso en la misma oración, hace que la traducción
de la misma no sea uniforme. Por cuestión de estilo, la repetición de
palabras en la misma oración no es muy agradable en castellano.

A determinadas palabras las hemos traducido por respeto al idioma y a
sus expresiones locales. Aunque \emph{email} es de amplio uso en
Argentina, preferimos usar correo electrónico, ya que desconocemos la
aceptación del original en inglés en la totalidad de las comunidades
hispanohablantes.

\textbf{IMPORTANTE: el manual está inmerso en una profunda cultura open
source. En el apéndice podrá ver un artículo llamado `La necesidad del
open source'. Prácticamente no hay mención a la importancia del software
libre. Disentimos con esta postura. Sin embargo, por respeto al
original, dejamos el artículo. Pero añadimos otro, que expresa `Por qué
se debería usar software libre y no open source'.}

\textbf{Hemos cambiado `Linux' por una expresión más adecuada,
`GNU/Linux'. Para una explicación, consulte el artículo
\href{https://www.gnu.org/gnu/why-gnu-linux.es.html}{¿Qué hay en un
nombre?}.}

\textbf{Ubuntu no es software completamente libre. No lo recomendamos,
al igual que tampoco recomendamos Windows ni Mac OS. Todos los ejemplos
de este manual se pueden aplicar perfectamente en
\href{https://trisquel.info/es}{Trisquel}, que sí es totalmente libre.
Para obtener una lista completa, consulte la
\href{https://www.gnu.org/distros/free-distros.es.html}{guía de
distribuciones GNU/Linux 100\% libres}.}

\textbf{¿Por qué no usar Ubuntu? Muy sencillo. Ubuntu provee
repositorios específicos de software que no es libre, y Canonical
promueve y recomienda explícitamente, bajo el nombre de Ubuntu, software
que no es libre en algunos de sus canales de distribución. También
ofrece la opción de instalar aplicaciones que no son libres. Además, la
versión del kernel Linux que incluye contiene objetos binarios de
firmware (blobs).Las políticas de marca registrada de Ubuntu prohíben la
redistribución comercial de copias exactas, negando una importante
libertad. Además,desde el mes de octubre de 2012, Ubuntu transmite datos
personales acerca de las búsquedas realizadas por el usuario a un
servidor de Canonical que restituye avisos publicitarios para comprar en
Amazon. En sentido estricto, esto no influye en el hecho de si Ubuntu es
o no es software libre, sino que se trata de una violación de la
privacidad de los usuarios. Además, anima a comprar en Amazon, una
empresa involucrada en la DRM (Digital Restrictions Management, Gestión
digital de restricciones) como así también en el maltrato de los
trabajadores, autores y editores. La inclusión de esta publicidad
involuntaria (adware) es uno de los raros casos en que un programador de
software libre persiste en conservar una funcionalidad maligna en su
versión de un programa.}

Bueno, usted decide. Nuestro consejo es que use software libre, no open
source (y menos software privativo). A partir de aquí, la traducción
completa del original en inglés.

\chapter{Manual de CryptoParty}\label{manual-de-cryptoparty}

https://cryptoparty.org/wiki/CryptoPartyHandbook

\textbf{Por favor, siéntase libre de hacer un fork de este repositorio.
Añada y edite contenido. Responda a las solicitudes recibidas.}

Los comentarios y preguntas acerca del contenido del manual son más que
bienvenidos, por favor envíelas usando un asunto nuevo y creando una
solicitud.

\section{Prerrequisitos}\label{prerrequisitos}

Para dar formato al manual (PDF, LaTeX, etc\ldots{}) se requiere lo
siguiente: - GNU make - pandoc - pdflatex

En Ubuntu se pueden instalar con la siguiente línea de comandos:

\begin{verbatim}
sudo apt-get install build-essential pandoc texlive-full
\end{verbatim}

\section{Proceso de revisión de
pares}\label{proceso-de-revisiuxf3n-de-pares}

Todavía no se ha implementado un proceso de revisión por pares para el
contenido ya existente en el manual, así como para futuras
incorporaciones. Esperamos que el contenido esté completo para
finalmente ser revisado de acuerdo a la investigación en seguridad hasta
al día y las mejores prácticas.

\section{Publicación}\label{publicaciuxf3n}

El Manual de CryptoParty pretende ser - y lucir - profesional, por lo
que debe ser empaquetado y publicado de manera adecuada con un buen
motor de composición tipográfica. Si usted tiene algún conocimiento o
experiencia con la publicación de libros y archivos de texto, por favor
involúcrese

\section{Licencia}\label{licencia}

El contenido CryptoParty Manual está disponible bajo la licencia
\href{https://creativecommons.org/licenses/by-sa/3.0/}{Creative Commons
Attribution-ShareAlike 3.0 Unported (CC BY-SA 3.0)}.

© Todos los capítulos de los contribuyentes a menos que se indique lo
contrario.

\chapter{Acerca de este libro}\label{acerca-de-este-libro}

El Manual de CryptoParty nació como una sugerencia de Marta Peirano
(\href{http://petitmedia.es}{http://petitemedia.es}) y Adam Hyde
(\url{http://booksprints.net}) después de realizar la primera
CryptoParty de Berlín, el 29 de agosto del 2012. Julian Olivier
\href{http://julianoliver.com}{(http://julianoliver.com}) y Danja
Vasiliev (\href{http://koala.net}{http://k0a1a.net}), coorganizadores de
la CryptoParty de Berlín junto con Marta estaban muy entusiasmados con
la idea, viendo la necesidad de contar con un libro práctico de fácil
comprensión para usar en las próximas CryptoParties. Asher Wolf, creador
del movimiento CryptoParty, fue invitado a participar con el incipiente
proyecto.

Este libro se escribió en los 3 primeros días de octubre del 2012 en
Studio Weise7, Berlín, rodeado por buena comida y un pequeño océano de
café. Estuvieron involucradas en su creación unas 20 personas, algunas
más que otras, unas cerca, y otras más lejos.

La metodología usada para escribir, Booksprint
(\url{http://booksprints.net}), trata de minimizar los problemas que
acarrea el proceso de publicación de las páginas creadas. La discusión
cara a cara y la asignación dinámica de tareas fueron una parte muy
importante de la realización del trabajo, ¡como en toda CryptoParty!

Para la tarea de edición usamos la plataforma de escritura open source
Booktype (\url{http://booktype.pro}), basada en la web (HTML5 y CSS),
que nos ayudó enormemente a desarrollarlo en forma paralela con relativa
facilidad. Asher también abrió un par de páginas TitanPad para obtener
financiamiento público para los capítulos del Manifiesto y Cómo hacer
una CryptoParty.

Combinado, se convirtió en el manual oficial de CryptoParty en la
medianoche del 3 de octubre, GMT+1.

La carrera por el libro duró 3 días y la lista completa de colaboradores
incluye a:

\begin{itemize}
\tightlist
\item
  Adam Hyde (facilitador)
\item
  Marta Peirano
\item
  Julian Oliver
\item
  Danja Vasiliev
\item
  Asher Wolf (\url{http://cryptoparty.org})
\item
  Jan Gerber
\item
  Malte Dik
\item
  Brian Newbold
\item
  Brendan Howell{]} (\url{http://wintermute.org})
\item
  AT
\item
  Carola Hesse
\item
  Chris Pinchen
  (\href{http://chokenpointproject.net}{http://chokepointproject.net/})
\item
  Arte de tapa a cargo de Emile Denichaud
  (\url{http://about.me/denichaud})
\end{itemize}

Esta versión del manual ha sido movido a Github para editarlo en forma
colaborativa. Encuéntrelo en
\url{https://github.com/cryptoparty/handbook}. Si encuentra errores o
partes que necesiten mejoras, cree una cuenta de GitHub y empiece a
editarlo, comentarlo o cree nuevas secciones. Si necesita más
información referida al uso de git y github, consulte
\url{https://help.github.com/}.

Créditos del Manual de CryptoParty

Facilitador:

\begin{itemize}
\tightlist
\item
  Adam Hyde
\end{itemize}

Equipo principal:

\begin{itemize}
\tightlist
\item
  Marta Peirano
\item
  Asher Wolf
\item
  Julian Oliver
\item
  Danja Vasiliev
\item
  Malte Dik
\item
  Jan Gerber
\item
  Brian Newbold
\item
  Brendan Howell
\end{itemize}

Asistentes:

\begin{itemize}
\tightlist
\item
  Teresa Dillon
\item
  AT
\item
  Carola Hesse
\item
  Chris Pinchen
\item
  `LiamO'
\item
  `l3lackEyedAngels'
\item
  `Story89'
\item
  Travis Tueffel
\end{itemize}

Migración a GitHub, empaquetado y mantenimiento:

\begin{itemize}
\tightlist
\item
  Yuval Adam
\item
  Samuel Carlisle
\item
  Daniel Kinsman
\item
  petter
\item
  Jens Kubieziel
\item
  Uwe Lippmann
\item
  Kai Engert
\end{itemize}

Imagen de portada:

\begin{itemize}
\tightlist
\item
  Emile Denichaud.
\end{itemize}

Traducción al español:

\begin{itemize}
\tightlist
\item
  gnu\_tesla@riseup.net
\end{itemize}

Otros manuales incluidos:

\begin{itemize}
\tightlist
\item
  \href{https://www.flossmanuals.net/bypassing-censorship}{http://www.flossmanuals.net/bypassing-censorship}
\end{itemize}

Los manuales usados en la segunda mitad de este libro se basan en 2
libros impresos por FLOSS Manuals:

\begin{itemize}
\item
  ``How to Bypass Internet Censorship'' 2008 \& 2010 Adam Hyde
  (Facilitador), Alice Miller, Edward Cherlin, Freerk Ohling, Janet
  Swisher, Niels Elgaard Larsen, Sam Tennyson, Seth Schoen, Tomas Krag,
  Tom Boyle, Nart Villeneuve, Ronald Deibert, Zorrino Zorrinno, Austin
  Martin, Ben Weissmann, Ariel Viera, Niels Elgaard Larsen, Steven
  Murdoch, Ross Anderson, Helen Varley Jamieson, Roberto Rastapopoulos,
  Karen Reilly, Erinn Clark, Samuel L. Tennyson, A Ravi
\item
  ``Basic Internet Security'' 2011 Adam Hyde (Facilitador), Jan Gerber,
  Dan Hassan, Erik Stein, Sacha van Geffen, Mart van Santen, Lonneke van
  der Velden, Emile den Tex y Douwe Schmidt
\end{itemize}

El contenido del Manual de CryptoParty está cubierto por la siguiente
licencia \href{https://creativecommons.org/licenses/by-sa/3.0/}{Creative
Commons Attribution-ShareAlike 3.0 Unported (CC BY-SA 3.0)}.

\begin{itemize}
\tightlist
\item
  © Todos los capítulos de los contribuyentes a menos que se indique lo
  contrario a continuación.
\end{itemize}

\chapter{Prefacio}\label{prefacio}

Este libro es un esfuerzo continuado y colaborativo basado en dos
\href{https://www.flossmanuals.net}{manuales FLOSS}
\href{https://flossmanuals.net/bypassing-censorship}{How to Bypass
Internet Censorship} y
\href{https://flossmanuals.net/basic-internet-security/}{Basic Internet
Security} y editados colaborativamente en
\href{https://github.com/cryptoparty/handbook}{Github} aunque se están
investigando otras formas posibles de colaboración.

Su objetivo es brindar un recurso completo para la gente que quiera
asistir u organizar una CryptoParty pero simplemente carece de
experiencia o de la confianza para llevarla acabo. Todos los capítulos
están escritos para ser consultados independientemente unos de otros.

Todos los contenidos del \emph{Manual de la CryptoParty} están cubiertos
por la licencia
\href{https://creativecommons.org/licenses/by-sa/3.0/}{Creative Commons
Attribution-ShareAlike 3.0 Unported (CC BY-SA 3.0)}. La lista de autores
se encuentra en \emph{Apéndice A: Contribuciones}. Por qué es tan
importante la privacidad =======================================

La privacidad es un derecho humano fundamental. Está reconocida en
muchos países como fundamental para la dignidad individual y para los
valores sociales de libertad de asociación y de expresión. En pocas
palabras, la privacidad es la frontera que separa nuestra intimidad de
la intrusión de la sociedad.

Los países difieren en la definición de la privacidad. En el Reino
Unido, por ejemplo, las leyes acerca de la privacidad se remontan al
siglo XIV cuando la monarquía inglesa creó leyes para proteger a las
personas de curiosos y mirones. Estas regulaciones se refieren a la
intrusión en la intimidad de las personas, donde ni siquiera el rey de
Inglaterra podía ingresar al hogar de la persona más humilde sin su
permiso. Con esta perspectiva, la privacidad se define en términos de
espacio personal y propiedad privada. En 1880, los abogados
estadounidenses, Samuel Warren y Louis Brandeis describieron a la
privacidad como el ``derecho a estar solo''. En este caso, la privacidad
es sinónimo del derecho a la vida privada. En 1948, la Declaración
Universal de los Derechos Humanos protegió en forma específica la
privacidad territorial y de las comunicaciones, que posteriormente se
convirtió en parte de las constituciones de todo el mundo. La Comisión
Europea de Derechos Humanos y el Tribunal Europeo de Derechos Humanos
también señalaron en 1978 que la privacidad incluye el derecho a
establecer relaciones con los demás y desarrollar el bienestar
emocional.

Hoy en día, una faceta cada vez más importante de la vida privada son
los datos personales que proporcionamos a las organizaciones, tanto
online como offline. Cómo utilizan nuestros datos personales y cómo
acceden a ellos es una temática que domina el debate sobre las leyes que
rigen nuestro comportamiento y la sociedad. Esto, a su vez, tiene
efectos en cadena sobre los servicios públicos a los cuales accedemos y
cómo las empresas interactúan con nosotros. Incluso tiene efectos sobre
cómo nos definimos. Si la privacidad está referida acerca de los límites
que determinan a quién le damos permiso de vernos y seguir los aspectos
de nuestras vidas, entonces la cantidad y tipo de información personal
recopilada, procesada y diseminada es de suma importancia para nuestras
libertades civiles fundamentales.

Un argumento oído a menudo, cuando se tratan las cuestiones de la
privacidad y el anonimato, sigue la línea de, ``Yo sólo hago cosas
aburridas. Nadie va a estar interesado en eso de todos modos'', o ``no
tengo nada que ocultar''. Ambas argumentos son fácilmente rebatidos.

En primer lugar, una gran cantidad de empresas están muy interesados en
estas cosas aburridas que usted hace porque ellos tienen la oportunidad
de ofrecer ``excelentes'' productos adecuados a sus intereses. De esta
manera, su publicidad se vuelve mucho más eficiente - son capaces de
adaptarla específicamente a las necesidades asumidas y a los deseos. En
segundo lugar usted tiene mucho que ocultar. Tal vez no lo exprese
explícitamente en los mensajes que envíe a sus amigos y colegas, pero su
navegación por la web - si no está protegida por las técnicas expuestas
en este libro - les dirá mucho acerca de las cosas que usted quisiera
mantener en secreto: una antigua pareja que usted busca a través de
Google, las enfermedades que investiga o las películas que ve son sólo
algunos ejemplos.

Otra consideración es que sólo porque usted no tenga algo que ocultar en
este momento, no significa que no lo tenga que hacer en el futuro.
Reunir todas las herramientas y habilidades necesarias para protegerse
de la vigilancia requiere práctica, confianza y un poco de esfuerzo.
Estas son cosas que podría no ser capaz de lograr y configurar justo
cuando más lo necesite aunque no sea un espía. Un obsesionado, un
acosador persistente, por ejemplo, es suficiente para alterar mucho su
vida. Cuanto más fielmente siga las sugerencias de este libro, estos
ataques tendrán menor impacto sobre usted. Las empresas también pueden
acechar demasiado, encontrando más y más maneras de llegar a su vida
diaria a medida que el uso de las redes de computadoras en sí mismo se
profundiza.

Por último, la falta de anonimato y la privacidad puede que no le
afecte, pero sí a toda la gente de su entorno. Si un tercero, como su
proveedor de servicios de Internet, lee su correo electrónico, también
se viola la privacidad de todas las personas de su libreta de
direcciones. Este problema se empieza a ver aún más dramáticamente
cuando nos fijamos en los problemas de los sitios web de redes sociales
como Facebook. Cada vez es más común ver fotos cargadas y etiquetadas
sin el conocimiento o permiso de las personas afectadas.

Mientras lo animamos a ser políticamente activo para mantener su derecho
a la privacidad, hemos escrito este libro con el fin de capacitar a las
personas que sienten que el mantenimiento de la privacidad en Internet
es también una responsabilidad personal. Esperamos que estos capítulos
le ayuden a llegar a un punto donde puede sentir que tiene algo de
control acerca de cuánto saben de usted otras personas. Cada uno de
nosotros tiene el derecho a una vida privada, el derecho a explorar,
buscar y comunicarse con los demás como uno desee, sin tener que vivir
con el temor de miradas indiscretas.

\chapter{El manifiesto de la
CryptoParty}\label{el-manifiesto-de-la-cryptoparty}

\begin{quote}
\textbf{``El hombre no es él mismo cuando habla en nombre propio. Dale
una máscara y te dirá la verdad.'' - Oscar Wilde}
\end{quote}

En 1996, John Perry Barlow, cofundador de la
\href{https://www.eff.org/}{Electronic Frontier Foundation (EFF)},
escribió ``Una declaración de independencia del ciberespacio''.
Extrajimos los párrafos siguientes:

\begin{quote}
El ciberespacio consiste en transacciones, relaciones y opiniones en sí
mismo, formando una onda estacionaria en la telaraña de nuestras
comunicaciones. El nuestro es un mundo que está en todas partes y en
ninguna a la vez, pero no está donde viven los cuerpos.
\end{quote}

\begin{quote}
Estamos creando un mundo al cual todos pueden entrar sin privilegios o
prejuicios debidos a la raza, el poder económico, la fuerza militar, o
el lugar de nacimiento.
\end{quote}

\begin{quote}
Estamos creando un mundo donde cualquiera, en cualquier sitio, puede
expresar sus creencias, sin importar lo singulares que sean, sin miedo a
ser coaccionado al silencio o el conformismo.
\end{quote}

Dieciséis años pasaron, e Internet ha cambiado nuestra forma de vivir.
Nos ha proporcionado el conocimiento combinado de la humanidad en la
punta de nuestros dedos. Podemos establecer nuevas relaciones y
compartir nuestros pensamientos y nuestras vidas con amigos de todo el
mundo. Podemos organizarnos, comunicarnos y colaborar de formas que
nunca nos hubiésemos imaginado posibles. Este es el mundo que queremos
legar a nuestros hijos, un mundo con una Internet libre.

Desafortunadamente, no toda la visión de John Perry Barlow se ha
cumplido. Sin acceso al anonimato online, no podemos librarnos de los
privilegios ni de los prejuicios. La libre expresión no existe sin
privacidad.

Los problemas que enfrentamos en el siglo 21 requieren que la humanidad
trabaje junta. Los problemas que enfrentamos son serios: cambio
climático, crisis energética, censura de los estados, vigilancia masiva
y guerras sin fin. Debemos ser libres para comunicarnos y asociarnos sin
miedo. Debemos apoyar los proyectos de software libre y de código
abierto que nos ayuden a incrementar el conocimiento común de
tecnologías de las cuales todos dependemos, por ejemplo,
\url{http://opensourceecology.org/wiki} ¡Contribuya!

Para ejercer nuestro derecho a la privacidad y al anonimato online,
necesitamos soluciones creadas mediante colaboración, abierta y
distribuida, revisadas por pares. Las CryptoParties proporcionan la
oportunidad de conocer y aprender a utilizar estas soluciones para
darnos todos los medios necesarios para hacer valer nuestro derecho a la
privacidad y al anonimato online.

\begin{enumerate}
\def\labelenumi{\arabic{enumi}.}
\item
  Todos somos usuarios, luchamos por el usuario y nuestra misión es
  fortalecerlo. Afirmamos que los pedidos de los usuarios son la razón
  de existir de las computadoras. Confiamos en la sabiduría colectiva de
  los seres humanos, no en los proveedores de software, corporaciones o
  gobiernos. Rechazamos los grilletes de los gulags digitales, montados
  sobre los intereses vasallos de los gobiernos y las corporaciones.
  Somos los ciberpunks revolucionarios.
\item
  El derecho al anonimato personal, a los seudónimos y a la privacidad
  son derechos humanos básicos. Estos derechos incluyen la vida, la
  libertad, la dignidad, la seguridad, el derecho a una familia, y el
  derecho a vivir sin temor o intimidación. Ningún gobierno,
  organización o individuo debe evitar que las personas tengan acceso a
  la tecnología que pone de relieve estos derechos humanos básicos.
\item
  La privacidad es el derecho absoluto del individuo. La transparencia
  es un requisito de los gobiernos y las empresas que actúan en nombre
  de las personas.
\item
  El individuo es el único dueño del derecho a su identidad. Sólo el
  individuo puede elegir que compartir. Los intentos coercitivos para
  obtener acceso a la información personal sin el consentimiento
  explícito es una violación de los derechos humanos.
\item
  Todas las personas tienen derecho a la criptografía y a los derechos
  humanos que las herramientas criptográficas involucran,
  independientemente de su raza, color, condición sexual, idioma,
  religión, opinión política o de otra índole, origen nacional o social,
  nacimiento, posición económica, política, jurídica o internacional del
  país o territorio en el que reside.
\item
  Así como los gobiernos deben existir sólo para servir a sus ciudadanos
  - así también, la criptografía debe pertenecer a gente. La tecnología
  no debe ser inaccesible para la gente.
\item
  La vigilancia no se puede separar de la censura y la esclavitud que
  implica. Ninguna máquina debe estar al servicio de la vigilancia y la
  censura. La criptografía es una clave para nuestra libertad colectiva.
\item
  El código es discurso: es un lenguaje humano creado. Prohibir,
  censurar o bloquear la criptografía para que la gente no tenga acceso
  a ella es privar a los seres humanos de un derecho humano, la libertad
  de expresión.
\item
  Aquellos que buscan detener la propagación de la criptografía se
  asemejan a los clérigos del siglo 15 que trataban de prohibir la
  imprenta, temerosos de que su monopolio del conocimiento fuera
  socavado. Fiesta como en el 31 de diciembre de 1983
  =========================================
\end{enumerate}

\section{¿Qué es una CryptoParty?}\label{quuxe9-es-una-cryptoparty}

\emph{CryptoParty} es una iniciativa global, descentralizada, con el
objetivo de introducir herramientas básicas de criptografía - tales como
la red de anonimato Tor, la clave de cifrado pública (PGP/GPG), y OTR
(Off The Record messaging) - al público en general.

La idea de una CryptoParty fue concebida como respuesta a la
\href{http://theconversation.edu.au/cybercrime-bill-makes-it-through-but-what-does-that-mean-for-you-8953}{Australian
Cybercrime Legislation Amendment Bill 2011} y su razón de ser es que
leyes como estas se vuelven inútiles cuando todos cifran sus
comunicaciones.

Las CryptoParties no tienen fines comerciales ni políticos, y son libres
y abiertas para todos aquellos que sigan sus \emph{principios guía}:

\subsection{Sean amables los unos con los
otros}\label{sean-amables-los-unos-con-los-otros}

Las CryptoParties son eventos en donde las personas se sienten
bienvenidas y seguras para aprender y enseñar sin importar sus
conocimientos ni su nivel de experiencia. Todas las preguntas son
relevantes, todas las explicaciones deberían estar dirigidas a las
personas con menos conocimientos.

Esto también significa que toda forma de acoso u otro comportamiento que
incomode a las personas no tiene cabida en las CryptoParties. De acuerdo
a nuestra experiencia, estas situaciones (aunque no suceden muy a
menudo) se deben más a ineptitud en el trato social que a malicia y
puden ser resueltas instando a las personas a ser más cuidadosas con sus
comportamientos, pero la responsabilidad final recae sobre los
organizadores de la CryptoParty quienes deben invitar a retirarse a
aquellas personas que no adhieran a esta sencilla regla, Sean amables
los unos con los otros. La concientización es la clave en este respecto.

\subsection{Haga cosas}\label{haga-cosas}

En las CryptoParties suceden cosas porque las personas hacen cosas. Las
experiencias de aprendizaje más sorprendentes e inesperadas suceden
porque la gente hace que sucedan. Si no está seguro de lo que tiene en
mente o de si otras personas están interesadas en ello haga lo
siguiente: propóngalo de todas formas y fíjese si alguien tiene algo que
decir. Si es demasiado tímido para proponerlo a toda la audiencia,
diríjase a la persona más cercana a su lado.

A una escala más global, existe una lista de correo
\href{https://cryptoparty.is/mailman/listinfo/global}{\textless{}global@cryptoparty.is\textgreater{}}
que está abierta a todas las preguntas y discusiones de todo tipo,
también pueden encontrarse listas de correo específicas para cada ciudad
y país y otros recursos en https://cryptoparty.in.

Para una guía acerca de cómo organizar una CryptoParty por favor
consulte el capítulo con el mismo nombre.

DIY, movimiento autoorganizado, inmediatamente se volvió viral, con una
docena de CryptoParties autónomas organizadas en horas en ciudades a lo
largo de Australia, EEUU, el Reino Unido, y
Alemania.``{]}(http://en.wikipedia.org/wiki/CryptoParty)

Actualmente, dieciséis CryptoParties se han realizado en una docena de
países diferentes a nivel mundial, y muchos más están previstas. El uso
de Tor en Australia se ha incrementado después de cuatro CryptoParties,
y la CryptoParty de Londres tuvo que ser trasladada del Hackspace al
campus de Google para acomodar el gran número de participantes ansiosos,
con 125 asistentes y 40 personas en lista de espera. Del mismo modo, la
CryptoParty de Melbourne despertó gran interés superando la capacidad
del lugar - originalmente prevista para aproximadamente 30 participantes
- cuando se presentaron más de 70 personas.

La CryptoParty ha recibido mensajes de apoyo de la Electronic Frontier
Foundation, de AnonyOps, del informante de la NSA Thomas Drake, del ex
editor central de WikiLeaks Heather Marsh, y del reportero de Wired,
Quinn Norton. Eric Hughes, el autor hace veinte años de \emph{Un
manifiesto ciberpunk}, pronunció un discurso de apertura en la primera
CryptoParty en Amsterdam.

\chapter{Cómo organizar una
CryptoParty}\label{cuxf3mo-organizar-una-cryptoparty}

\section{Introducción}\label{introducciuxf3n}

CryptoParty es un movimiento comunitario global y descentralizado. Como
tal, varía mucho según el lugar en donde se desarrolle. Este tutorial
está escrito para poder brindarle algunas ideas acerca de qué es lo que
funciona bien y qué no, pero todo como una acción directa: los planes no
seon nada, la planificación es todo, y todo está bien siempre y cuando
esté de acuerdo con los siguientes principios
(\url{https://cryptoparty.in/guiding_principles}): sean amables unos con
otros y hagan cosas.

Si prefiere el video sobre el texto:
(\url{https://va.ludost.net/files/initlab/20140502cparty.mp4})(grabado
en \href{https://initlab.org}{initlab} en abril del 2014).

O únase a nosotros en
\href{https://cryptoparty.in/connect/contact/irc}{IRC}
(\#cryptoparty.oftc.net) o la
\href{https://cryptoparty.in/connect/contact/mailinglists}{lista de
correo}
\href{mailto:global@cryptoparty.is}{\nolinkurl{global@cryptoparty.is}}

Por favor, contáctese con nosotros a través de dichos canales de
comunicación si tiene alguna duda o necesita ayuda.

Una CryptoParty no puede enseñarle todo lo que deba saber acerca de las
computadoras y la seguridad en Internet en una tarde. El objetivo
principal es derribar las barreras mentales que impiden que las personas
piensen acerca de estos temas o los enfrenten a medida que aparecen en
sus vidas, en artículos periodísticos, en blogs, en el ámbito educativo
y memes. Existe una gran cantidad de información acerca de las
computadoras y la seguridad en internet ahí afuera. Lamentablemente,
muchas personas no se consideran capaces de procesar dicha información,
y mucho menos de intentarlo. Esto es lo que nosotros queremos cambiar.
Despejando el miedo a las cosas crítpticas y técnicas (dos propiedades
inherentes a todas las herramientas criptográficas) podrá continuar
aprendiendo y enseñando a otros.

Con una CryptoParty usted creará un ambiente en donde personas de
diferente formación se juntarán y aprenderán unas de otras. Por lo
tanto, sería deseable incluir personas de diferente edad, género, nivel
cultural y experiencia.

Con las puertas abiertas, la gente llega, busca un asiento y socializa.
Una breve introducción inaugura oficialmente el evento y luego todos se
dirigen a las mesas. Cada mesa debe cubrir un tema y la gente debe
decidir que le gustaría aprender y/o enseñar.

Las personas se sentirán más cómodas si cuentan con el tiempo suficiente
para socializar. Se sentirán más a gusto para formular preguntas. Esto
sucederá en un ambiente adecuado. Preparar la escena es su tarea.

El discurso de apertura debería ser tan breve como sea posible (no más
de veinte minutos) y debería dar un vistazo general acerca de qué se
debería esperar (consulte el capítulo dedicado a tal fin para más
detalles). En algunas ciudades, también hay charlas. Funciona muy bien
cuando la gente busca una introducción en profundidad. La mayoría de las
veces querrán pasar a la acción rápidamente. Dependiendo del grupo
podría ofrecer ambas opciones en habitaciones separadas.

Apenas concluya la introducción, la gente se debe dirigir a las mesas
con el tema de su preferencia. No se procupe si todo luce algo caótico
durante algunos minutos. Cada mesa abre con una introducción más
específica antes de instalar, configurar o usar cualquier herramienta.
Insista, aliente a que todos realicen preguntas todo el tiempo.

La capacidad de improvisar es muy útil en la CryptoParty como en todo
aprendizaje en donde habitualmente surgen situaciones inesperadas.;-

Si todos trabajan se sorprenderá por la energía positiva, por el
compromiso, concentración y diversión de la gente. Las mejores
CryptoParties generalmente duran hasta bien entrada la noche aún después
de un largo día (o semana) de trabajo.

La duración recomendada para una CryptoParty es de 3 a 5 horas.

Aquí hay una lista completa de cosas que hemos aprendido de las pasadas
CryptoParties:

\section{Antes de la fiesta}\label{antes-de-la-fiesta}

\subsection{Lugares, infraestructura y
comida}\label{lugares-infraestructura-y-comida}

El público de una CryptoParty y su conducta general estarán
condicionados en gran medida por el lugar en donde suceda todo. No
importa si es un bar, un club, un centro social, una escuela, una
universidad, una biblioteca, una sala de prens, una ONG o una empresa:
mientras sea de libre acceso y gratuita, y sin banderías políticas ni
fines de lucro estará bien. Pero sea cuidadoso con las oficinas
empresariales. Si tiene alguna duda, por mínima que sea, pregunte a otro
organizador de CryptoParties por medio de la lista de correo local o
global si ellos se sentirían cómodos realizando un evento en el lugar en
cuestión.

Una cosa muy importante - posiblemente aún más que la electricidad e
Internet - es la comida. Es casi imposible que alguien no se comporte
amablemente después de una buena comida. Bueno, sí, generalmente usted
querrá tener también electricidad y una buena conexión a Internet para
ser capaz de brindarles a todos el software y la experiencia que
vinieron a buscar.

Los criterios generales para tener un buen lugar son:

\begin{itemize}
\tightlist
\item
  debe ser acogedor
\item
  tener bebidas
\item
  idealmente, tener comidas
\item
  tener sillas y mesas suficientes
\item
  tener alargues y zapatillas eléctricas
\item
  tener una conexión a Internet lo suficientemente rápida
\end{itemize}

Recuerde que su público mayormente no está familiarizado con la escena
hacker.

Lugares adecuado puede ser:

\begin{itemize}
\tightlist
\item
  cafés
\item
  espacios comunitarios
\item
  bibliotecas
\item
  escuelas
\item
  universidades
\item
  hacklabs
\item
  clubes nocturnos
\item
  cualquier lugar que le parezca
\end{itemize}

\subsection{Página web}\label{puxe1gina-web}

Ya sea que hospede la página web de su CryptoParty en su propio servidor
o use \href{https://cryptoparty.in}{cryptoparty.in}, asegúrese de poner
al menos un enlace en nuestra página, ya que la CryptoParty es un
esfuerzo global, colaborativo. y así otras personas podrán enterarse de
su existencia y unírseles.

Algunos elementos que deberían tener son:

\begin{itemize}
\tightlist
\item
  un texto de bienvenida
\item
  fechas de reuniones futuras
\item
  lista de lugares
\item
  información de contacto (preferiblemente correo electrónico)
\end{itemize}

Supongamos que su ciudad no posee página web. Cree una para la ciudad
entera (por ejemplo, ``https://cryptoparty.in/ciudad-gotica''). Aquí
debería ir la información. Los lugares o las reuniones deberían tener
subpáginas. Eto le permitirá a otros organizar reuniones sin tener que
crear una segunda versión de la página de dicha ciudad.

Usted puede ayudar creando una página básica para ser copiada por otros,
para que la ajusten a sus necesidades, y puedan brindarle su opinión
para mejorarla.

Por favor añada las fechas de sus reuniones enla
\href{https://cryptoparty.in/parties/upcoming}{lista global de fechas}
aunque no use la wiki como su sitio web principal. Ayúdenos a mostrar
cuan global es el movimiento. Podrá encontrar un tutorial dedicado aquí
\url{https://cryptoparty.in/parties/add-a-date}.

\subsection{Difusión}\label{difusiuxf3n}

La difusión ayudará a dar a conocer su CryptoParty. Primero debe pensar
en su público promedio. Deberían ser personas que aún no han cifrado
nada. Ellos conocen como encender una computadora pero no tienen
conocimientos avanzados en el tema.

Comience de a poco. Dependiendo del número de ángeles disponibles
probablemente no querrá cientos de asistentes a su CryptoParty. Los
lugares tienden a venir con una comunidad. Si están interesados en
hospedar una CryptoParty entonces su comunidad estará interesada en
asistir a ella. Si logra que estén contentos ellos le dirán a sus amigos
lo maravilloso que fue todo.

Anuncie la CryptoParty en la wiki, en las listas de correo, y en todos
los canales relevantes que pueda, a saber:

\begin{itemize}
\tightlist
\item
  online
\item
  listas de correo
\item
  blogs
\item
  redes sociales
\item
  offline (consulte
  \href{https://github.com/cryptoparty/flyers}{github.com/cryptoparty/flyers}
\item
  folletos
\item
  stickers
\item
  posters
\item
  boca a boca
\item
  medios de comunicación locales
\end{itemize}

¿Podrá poner posters o folletos en algún sector del lugar? ¿Tendrá el
lugar alguna lista de correo? Considere crear una lista de correo y
cuentas en las redes sociales para su ciudad. Las listas de correo,
Diaspora y Twitter suelen ser muy populares en la comunidad de la
CryptoParty.

Conectarse con la comunidad global de la CryptoParty puede ser muy útil
para aprender de las experiencias pasadas y retomarlas con nuevo
impulso.

\begin{itemize}
\tightlist
\item
  \href{https://cryptoparty.in/communication/mailinglists}{listas de
  correo}
\item
  \href{https://cryptoparty.in/communication/irc}{IRC}
\item
  \href{https://twitter.com/search?q=\%23cryptoparty\&mode=users}{Twitter}
\item
  \href{https://wk3.org/u/cryptoparty}{Diaspora}
\end{itemize}

\subsection{Ángeles}\label{uxe1ngeles}

Decida qué tema le gustaría enseñar. Para ver cómo enseñan otros revise
la \href{https://www.cryptoparty.in/learn/links\#handbooks}{lista de
manuales}. Las explicaciones deben estar dirigidas a los principiantes.
Tenga esto siempre en mente. \url{https://www.level-up.cc/} tiene una
sección específic acerca de cómo ser un mejor instructor.

No sea crítico, Respete las decisiones de la gente acerca de cuáles
herramientas usar y cuánto comprometerse en su decisión de proteger su
privacidad. No responda sus preguntas como si fueran estúpidas. Todas
las preguntas son buenas.

Contacte al organizador y hágale saber lo que desea enseñar. Ayúdelo a
planificar la CryptoParty. Cuantos más ángeles haya y cuanto más
pequeños sean los grupos la experiencia será mucho mejor para todos los
participantes. También debería hacerse de algo de tiempo para aprender
de otros ángeles.

Lleve bolígrafos y papel a la CryptoParty para dibujar diagramas
mientras explica cómo funciona algo.

\subsection{Materiales}\label{materiales}

Lista de cosas que debería tener a mano durante una CryptoParty:

\begin{itemize}
\tightlist
\item
  \href{https://github.com/cryptoparty/handouts/tree/master/en}{Formularios
  varios para las mesas}
\item
  folletos (CryptoParty o grupos similares)
\item
  stickers
\item
  memorias usb
\item
  todo el software relevante descargado (y verificado en medida de lo
  posible)
\item
  huellas y firmas digitales de Tor, Tails, PGP y otros proyectos.
\end{itemize}

\section{La fiesta}\label{la-fiesta}

\subsection{Configurando la escena}\label{configurando-la-escena}

Quizás sea la parte más importante. La fortaleza de las CryptoParties se
basa en la unión de diferentes personas de las más diversas experiencias
que se comprometen a aprender unos de otros. Pero también implica un
desafío para que personas de diferentes opiniones ``sean amables unas
con otras''.

Un primer paso, aún antes de iniciar oficialmente la fiesta, es darle la
bienvenida a cada persona y grupo a medida que lleguen y asegurarse que
no se sientan solos o perdidos. Esto es especialmente importante cuando
se demore el inicio, lo que sucede más a menudo de lo que debería, pero
esto aplica a toda la reunión. Solo asegúrese de que todos sean amables
unos con otros y hagan cosas (por ejemplo, tomar un te y charlar
amigablemente). Derribe murallas.

\subsection{Discurso introductorio}\label{discurso-introductorio}

Según nuestra experiencia es muy útil disponer de un plan general acerca
de los temas potenciales a cubrir

El discurso introductorio inaugura oficialmente la CryptoParty.
Dependiendo de su estilo propio, puede ser preferible ser realista o
algo más osado. Pero sea breve (no más de 20 minutos) y no entre en
detalles técnicos ya que eso es lo que hará cada grupo de aprendizaje
individualmente.

Si prefiere mostrar un video, fíjese el
\href{https://github.com/cryptoparty/video}{video de introducción a la
CryptoParty}.

Los puntos potenciales a tratar pueden ser:

\begin{itemize}
\tightlist
\item
  Saludo y bienvenida
\item
  Agradecimiento a la gente que cedió el lugar
\item
  Sean amables unos con otros
\item
  que es una CryptoParty
\item
  movimiento global y descentralizado
\item
  todos pueden ser parte de él
\item
  un tema por mesa
\item
  cada persona elige su tema
\item
  advertencia de seguridad
\item
  no existe el nivel de seguridad al 100\% (ni online ni offline)
\item
  usar cifrado es legal, pero no en todos los países
\item
  la CryptoParty es para principiantes
\item
  en segundo lugar, es para los periodistas, los activistas y hasta para
  los expertos (por ejemplo, {[}EFF{]} (https://www.eff.org/ ),
  \href{https://tacticaltech.org}{Tactical Tech},
  \href{https://www.accessnow.org/}{AccessNow})
\item
  existe el prejuicio de que la criptografía es difícil
\item
  la seguridad es un proceso
\item
  no es un producto
\item
  no es algo que usted instala
\item
  es algo que usted hace
\item
  software libre
\item
  servicios descentralizados
\item
  no controlados por una sola organización
\item
  lista de temas presentados en una específica CryptoParty
\end{itemize}

No ofrezca una false sensación de seguridad, pero tampoco atemorice a la
gente con todas las formas en que las cosas podrían empeorar. Algunas
personas \emph{quieren} escuchar todas las cosas que podrían empeorar y
no son temerosas, pero usted necesitará discutir el tema individualmente
con cada persona, no en un discurso introductorio.

\subsection{Temas}\label{temas}

Esta lista solamente es una sugerencia. Los grupos de aprendizaje se
formarán alrededor de estos temas y dependiendo del espacio disponible,
de la cantidad de personas dispuestas a aprender y de la cantidad de
personas dispuestas a conducir un grupo de aprendizaje podrán ser
agrupados de forma más amplia o más específica.

Su oferta dependerá de los ángeles disponibles. Posteriores sugerencias
se enlistarán en un {[}resumen de
herramientas{]}/https://cryptoparty.in/learn/tools ) separado. Todo es
software libre y de código abierto. Ypor supuesto, nos agradan también
los servicios descentralizados.

\begin{itemize}
\tightlist
\item
  discusión y mesa de orientación
\item
  cifrado de correo electrónico con
  \href{https://es.wikipedia.org/wiki/PGP}{PGP}
\item
  cifrado de mensajería instantánea con
  \href{https://es.wikipedia.org/wiki/Extensible_Messaging_and_Presence_Protocol}{XMPP}
  y {[}OTR{]} (https://es.wikipedia.org/wiki/Off\_the\_record\_messaging
  )
\item
  navegación web anónima \href{https://torproject.org}{Tor}
\item
  plugins para mejorar la navegación privada
\item
  seguridad en telefonía movil
  (\href{https://cryptoparty.in/learn/tools\#android}{Android},
  \href{https://cryptoparty.in/learn/tools\#ios}{iOS}
\item
  cifrado de discos y archivos con
  \href{https://veracrypt.codeplex.com/}{VeraCrypt}
\item
  cifrado de discos con \href{https://es.wikipedia.org/wiki/LUKS}{LUKS}
\item
  seguridad en contraseñas y su administración
\item
  instalación de distribuciones GNU/Linux
\item
  \href{https://tails.boum.org/}{Tails} (sistema operativo anónimo y
  seguro\ldots{} no olvide decirle a la gente que lleve una memoria usb)
\end{itemize}

Una mesa de orientación y discusión debería tratar el tema de cuanta
vigilancia es necesariae y por qué todos tenemos algo que ocultar. La
mayoría de las personas no son exhibicionistas y valoran su privacidad.
Por lo tanto piense en alguien curioso pero aún no convencido del
beneficio de la criptografía.

En la mesa de discusión y orientación, deberá tratar con un montón de
preguntas inesperadas. Algunas pueden parecerles irrelevantes. No
intente dirigir la conversación, responda todas las preguntas sin
juzgarlas.

\subsection{Pidiendo ayuda}\label{pidiendo-ayuda}

Siempre pida ayuda. Las CryptoParties no equivalen a trabajo duro
individual. Si el estrés sobrepasa la diversión, deténgase un momento y
vea que todo continúa mágicamente aún sin usted. Lo principal es invitar
a la gente para que ayude. decirle que pueden ayudar y que su ayuda será
muy apreciada. Para la CrptoParty en cuestión o para todas. Si ve gente
que se presenta por tercera vez consecutiva, pregúntele si desea
hospedar una mesa, si la gente habla acerca del bar en donde trabajan o
la casa en donde viven, pregúntele si esos lugares son adecuados para
una CryptoParty y si ellos pueden organizar una. Las oportunidades son
innumerables.

\section{Variaciones en el formato}\label{variaciones-en-el-formato}

Las CryptoParties para un grupo específico pueden ayudar a reducir la
barrera de ``Este no es mi campo de experiencia, no comprenderé nada''.
Considere organizar CryptoParties para periodistas, estudiantes, grupos
específicos de activistas, etc. Aún, incluso considerando la preparación
dedicada, todos los que quieran ayudar deben ser bienvenidos.

Si desea llegar a las personas que son demasiado tímidas para participar
o si encontrar un lugar se hace muy difícil quizás querrá visitar gente
en su lugar para lo que denominamos un ``Cryptoparty en una hogar/en un
living room''.

\section{Roles}\label{roles}

Así como existen diferentes eventos, también hay diferentes roles en una
CryptoParty. Esta sección resume todos los roles.

\begin{itemize}
\tightlist
\item
  organizadores
\item
  oradores
\item
  criptoángeles
\item
  meta-ángeles
\end{itemize}

\section{Organizador}\label{organizador}

Como organizador de una CryptoParty, necesita hallar un lugar donde la
gente se sienta cómoda. Debe mantenerse en contacto con la gente que
administra el espacio y hallar una fecha adecuada para todos. Los
lugares puede ser espacios comunitarios, cafés, bibliotecas, escuelas,
universidades, hacklabs\ldots{} y cualquier otro lugar acogedor y con
suficiente cantidad de sillas, mesas, conexiones eléctricas y una buena
conexión a Internet.

Debería ser cuidadoso de la difusión además del alcance de la palabras
acerca de cuando y donde sucede la CryptoParty. La difusión tiene una
sección separada (consulte más arriba).

Por último pero no por eso menos importante necesitará contar con una
suficiente cantidad de personas para explicar las herramientas
específicas y llevar adelante la CryptoParty con usted. Si usted tiene
suficiente experiencia podrá sacarla adelante casi sólo y organizar a
sus ayudantes (algunos llamados criptoángeles) sobre la marcha. Pero es
mucho mejor simplemente preguntarle a aquellos que saben si pueden
ayudarle.

Mantenga grupos tan pequeños como le sea posible. La experiencia
demuestra que la mejor relación entre instructor-aprendiz es de 1.5 o
menos.

Una CryptoParty es exitosa si la atmósfera es correcta. No importa
cuanta gente asista. Sea paciente cuando intente establecer una
CryptoParty en su ciudad o comunidad. Difundirla toma tiempo.

\chapter{Orador}\label{orador}

Un orador tiene la tarea de abrir la CryptoParty antes de que la gente
se dirija a las mesas de su elección. Debe dar un discurso calmo,
referido a ``que es la CryptoParty y cuales son sus temas'' o uno más
intenso acerca de ``su privacidad, su libertad, cifrar ahora y hacer que
los bastardos que impulsan la vigilancia masiva sufran''. Cada orador
tiene su propio estilo.

Para más detalles, consulte la sección //Discurso de apertura// más
arriba.

\subsection{Criptoángeles}\label{criptouxe1ngeles}

Como criptoángel, su tarea es explicar la criptografía a nivel
conceptual, de que lo protege (y de que no) y ayudar con la instalación
y usar el software relacionado.

Siempre explíquele a las personas de su grupo con menor nivel de
conocimiento, esté atento a las caras de sorpresa y pregunte si todos
comprendieron lo que dijo. Aliente la participación y responda las
preguntas cada vez que surjan. Cuando alguien dice saber la respuesta
deje que la responda.

El aprendizaje debe ser práctico. Nunca toque la computadora de un
participante a menos que note que la persona está atascada en algo, y
siempre pida permiso para hacerlo. La mayoría de las personas aprenden
visualmente, por eso haga pequeños bocetos o diagramas para ayudarlos un
poco y puedan entender los conceptos abstractos detrás del software. Si
no conoce la respuesta a una pregunta, transmítala a otros participantes
o a los criptoángeles, o intente encontrarla con su grupo.

La idea es que la gente sepa como usar las herramientas que aprendan a
un nivel básico cuando se vayan de la CryptoParty. Aún mejor, que puedan
contarles a sus amigos que ellos ahora usan la ``herramienta xyz'' y de
este modo lograr que comprendan que no fue tan difícil aprenderlas (y
que se lo puedan decir a sus amigos, también).

Si necesita más criptoángeles, encuéntrelos en:

\begin{itemize}
\tightlist
\item
  una CryptoParty
\item
  un hacklab
\item
  una universidad
\item
  o entre sus amigos
\end{itemize}

\subsection{Meta-ángeles}\label{meta-uxe1ngeles}

Cuando mayor o más caótica sea una CryptoParty en general, lo mejor es
tener un meta-ángel, una persona cuya única tarea es asegurarse que
todos tengan la mejor experiencia de aprendizaje y que nadie quede
afuera.

Como meta-ángel usted no tendrá un tema ni una mesa. En su lugar, tendrá
un resumen de los ángeles disponibles, cuáles son sus fortalezas y qué
mesa está cubriendo cada tema.

También debe ayudar a quienes arriben tardíamente a encontrar una mesa.

Debe facilitar la comunicación entre las mesas. suma que existen
preguntas en cada mesa que su ángel no podrá responder. Ayude hallando
alguien que sí pueda hacerlo.

Si alguien parece perdido ayúdlo a encontrar la mesa correcta.

Si alguien está indeciso acerca de qué aprender, charle con él para
ayudarlo a entender qué le gustaría saber y elijan una mesa.

\section{Construcción de una
comunidad}\label{construcciuxf3n-de-una-comunidad}

Cosas que puede hacer para que crezca la comunidad local y global de la
CryptoParty.

\subsection{Encuentros nocturnos
regulares}\label{encuentros-nocturnos-regulares}

Para construir un movimiento sustentable, debe establecer lazos
sociales. Un encuentro nocturno habitual de criptoángeles,
organizadores, etc. puede servir para este propósito. Diviértase y pase
el rato con amigos.

\subsection{Sesiones de entrenamiento}\label{sesiones-de-entrenamiento}

Los ángeles necesitan una oportunidad para aprende cosas ellos mismos.
En una CryptoParty no hay tiempo para ello. Por eso, puede ser una buena
idea organizar una sesión de ``entrenamiento para entrenadores''.

\subsection{Conferencias}\label{conferencias}

Se ha producido una asamblea de CryptoParty en el CCC desde que el
movimiento inició. Esto ayudó a conectarnos mundialmente y para
intercambiar experiencias.

Aplicar el concepto a más conferencias puede ayudar a la difusión. Y por
supuesto, será muy divertido encontrarse con personas de otras ideas. O
quizás alguien ya tenga la misma idea y querrá unírseles.

\section{Recursos}\label{recursos}

\begin{longtable}[]{@{}lll@{}}
\toprule
\begin{minipage}[b]{0.14\columnwidth}\raggedright\strut
Lenguaje\strut
\end{minipage} & \begin{minipage}[b]{0.11\columnwidth}\raggedright\strut
Enlace\strut
\end{minipage} & \begin{minipage}[b]{0.18\columnwidth}\raggedright\strut
Descripción\strut
\end{minipage}\tabularnewline
\midrule
\endhead
\begin{minipage}[t]{0.14\columnwidth}\raggedright\strut
inglés\strut
\end{minipage} & \begin{minipage}[t]{0.11\columnwidth}\raggedright\strut
\url{https://www.level-up.cc/}\strut
\end{minipage} & \begin{minipage}[t]{0.18\columnwidth}\raggedright\strut
recurso para la comunidad de entrenamiento acerca de cuidados digitales
globales\strut
\end{minipage}\tabularnewline
\begin{minipage}[t]{0.14\columnwidth}\raggedright\strut
inglés\strut
\end{minipage} & \begin{minipage}[t]{0.11\columnwidth}\raggedright\strut
\url{https://www.cryptoparty.in/learn/links\#handbook}\strut
\end{minipage} & \begin{minipage}[t]{0.18\columnwidth}\raggedright\strut
enlaces a varios manuales\strut
\end{minipage}\tabularnewline
\begin{minipage}[t]{0.14\columnwidth}\raggedright\strut
alemán\strut
\end{minipage} & \begin{minipage}[t]{0.11\columnwidth}\raggedright\strut
\url{https://wiki.piratenpartei.de/HowTo_Cryptoparty}\strut
\end{minipage} & \begin{minipage}[t]{0.18\columnwidth}\raggedright\strut
como hacer una cryptoparty según el Partido Pirata alemán\strut
\end{minipage}\tabularnewline
\begin{minipage}[t]{0.14\columnwidth}\raggedright\strut
alemán\strut
\end{minipage} & \begin{minipage}[t]{0.11\columnwidth}\raggedright\strut
\url{https://www.ak-vorrat.org/wiki/cryptoparty}\strut
\end{minipage} & \begin{minipage}[t]{0.18\columnwidth}\raggedright\strut
como hacer una cryptoparty por el grupo activista alemán AK Vorrat\strut
\end{minipage}\tabularnewline
\begin{minipage}[t]{0.14\columnwidth}\raggedright\strut
español\strut
\end{minipage} & \begin{minipage}[t]{0.11\columnwidth}\raggedright\strut
\url{https://wiki.partidopirata.com.ar/Aprender} Tutoriales, videos y
manuales del Partido Pirata de Argentina\strut
\end{minipage}\tabularnewline
\bottomrule
\end{longtable}

\chapter{Introducción a la versión en
español}\label{introducciuxf3n-a-la-versiuxf3n-en-espauxf1ol}

La siguiente es la versión en español del CryptoParty Handbook realizada
por el \href{http://partidopirata.com.ar/}{Partido Pirata de Argentina}.
Antes de que siga leyendo, creemos necesario hacer algunas aclaraciones.

Hemos respetado fielmente el original traduciendo lo más literalmente
posible al texto, algunas veces lo hemos logrado, otras no tanto. Por
ejemplo, click puede traducirse por hacer click, hacer clic, cliquear,
presionar o pulsar. La redundancia típica de estas palabras muchas veces
en el mismo párrafo, incluso en la misma oración, hace que la traducción
de la misma no sea uniforme. Por cuestión de estilo, la repetición de
palabras en la misma oración no es muy agradable en castellano.

A determinadas palabras las hemos traducido por respeto al idioma y a
sus expresiones locales. Aunque \emph{email} es de amplio uso en
Argentina, preferimos usar correo electrónico, ya que desconocemos la
aceptación del original en inglés en la totalidad de las comunidades
hispanohablantes.

IMPORTANTE: el manual está inmerso en una profunda cultura open source.
En el apéndice podrá ver un artículo llamado ``La necesidad del open
source''. Prácticamente no hay mención a la importancia del software
libre. Disentimos con esta postura. Sin embargo, por respeto al
original, dejamos el artículo. Pero añadimos otro, que expresa ``Por qué
se debería usar software libre y no open source''.

Hemos cambiado ``Linux'' por una expresión más adecuada, ``GNU/Linux''.
Para una explicación, consulte el artículo
\href{https://www.gnu.org/gnu/why-gnu-linux.es.html}{¿Qué hay en un
nombre?}.

Ubuntu no es software completamente libre. No lo recomendamos, al igual
que tampoco recomendamos Windows ni Mac OS. Un distribución GNU/Linux
derivada de Ubuntu pero completamente libre es
\href{https://trisquel.info/es}{Trisquel}. Para obtener una lista
completa, consulte la
\href{https://www.gnu.org/distros/free-distros.es.html}{guía de
distribuciones GNU/Linux 100\% libres}.

¿Por qué no usar Ubuntu? Muy sencillo. Ubuntu provee repositorios
específicos de software que no es libre, y Canonical promueve y
recomienda explícitamente, bajo el nombre de Ubuntu, software que no es
libre en algunos de sus canales de distribución. También ofrece la
opción de instalar aplicaciones que no son libres. Además, la versión
del kernel Linux que incluye contiene objetos binarios de firmware
(blobs).Las políticas de marca registrada de Ubuntu prohíben la
redistribución comercial de copias exactas, negando una importante
libertad. Además,desde el mes de octubre de 2012, Ubuntu transmite datos
personales acerca de las búsquedas realizadas por el usuario a un
servidor de Canonical que restituye avisos publicitarios para comprar en
Amazon. En sentido estricto, esto no influye en el hecho de si Ubuntu es
o no es software libre, sino que se trata de una violación de la
privacidad de los usuarios. Además, anima a comprar en Amazon, una
empresa involucrada en la DRM (Digital Restrictions Management, Gestión
digital de restricciones) como así también en el maltrato de los
trabajadores, autores y editores. La inclusión de esta publicidad
involuntaria (adware) es uno de los raros casos en que un programador de
software libre persiste en conservar una funcionalidad maligna en su
versión de un programa.

Bueno, usted decide. Nuestro consejo es que use software libre, no open
source (y menos software privativo). A partir de aquí, la traducción
completa del texto original en inglés.

\chapter{Consejos básicos}\label{consejos-buxe1sicos}

Al igual que con otras formas de comunicación en la web, siempre se
deben tomar algunas precauciones básicas para poder proteger nuestra
privacidad de manera efectiva.

\section{Brevemente:}\label{brevemente}

\begin{itemize}
\tightlist
\item
  Las contraseñas no deben estar relacionadas con detalles personales y
  deben contener una combinación de 8 o más letras y otros caracteres.
\item
  Verifique siempre que su conexión es segura cuando lee correos
  electrónicos o cuando navega en redes inalámbricas, especialmente en
  sitios con acceso público a internet.
\item
  Los archivos temporarios (el ``caché'') de la computadora que usted
  usa para revisar sus correos electrónicos pueden presentar riesgos.
  Bórrelos periódicamente.
\item
  Cree y mantenga cuentas de correo electrónico separadas para distintas
  tareas e intereses.
\item
  Cifre todos los mensajes que no se atrevería a escribir en una tarjeta
  postal.
\item
  Sea precavido con los riesgos que implica que su correo electrónico
  esté hospedado en una empresa u organización.
\end{itemize}

\section{Contraseñas}\label{contraseuxf1as}

Las contraseñas son el punto más vulnerable en la comunicación de
correos electrónicos. Incluso una contraseña segura puede ser
interceptada a menos que la conexión sea segura (consulte TLS/SSL en el
glosario). Además, que una contraseña sea larga no significa que no
pueda ser adivinada usando conocimientos de su persona y de su vida
privada.

La regla general para crear contraseñas es que deben ser largas (8
caracteres o más) y tener una mezcla de letras y otros caracteres
(números y símbolos, lo que significa que usted no debe elegir una
oración breve). Combinar la fecha de su cumpleaños con un nombre
familiar es un gran ejemplo de lo que no debe hacerse. Este tipo de
información es fácil de encontrar usando recursos públicos. Un truco
popular es basarse en una frase favorita y entonces, sólo para
confundir, se mezcla con algunos números. Lo mejor de todo es el uso de
un generador de contraseñas, ya sea en el sistema local o en forma
online.

A menudo, las contraseñas son difíciles de recordar y por eso aparece un
segundo punto de vulnerabilidad -el descubrimiento de su registro
escrito. Puesto que no hay mejor medio de almacenar una contraseña que
en su propio cerebro, servicios como
OnlinePasswordGenerator(\url{http://www.onlinepasswordgenerator.com/})
ofrecen un gran compromiso por generar contraseñas al azar que recuerdan
vagamente a las palabras y les presentará una lista para elegir.

Si usted no elige memorizar sus contraseñas, deberá escribirlas o usar
un software de cadena de claves. Esto puede ser una decisión riesgosa,
especialmente si la cuenta de correo electrónico y la contraseña son las
mismas para dispositivos diferentes tales como su teléfono o su
computadora.

El software de cadena de claves, tal como Keepass, reúne varias
contraseñas y frases de paso en un lugar y posibilita su acceso a través
de una contraseña o frase de paso maestra. Esto le pone presión a la
elección de esta clave maestra. Si decide usar un software de cadena de
claves, recuerde elegir contraseñas seguras.

Por último, debe usar una contraseña diferente para cada cuenta. De esta
manera, si una de ellas es robada, las otras cuentas permanecerán
seguras. Nunca use la misma contraseña para las cuentas de correo
electrónico laborales y para las personales. Vea la sección
\textbf{Contraseñas} para aprender más acerca de cómo protegerse.

\section{Leyendo correos electrónicos en lugares
públicos}\label{leyendo-correos-electruxf3nicos-en-lugares-puxfablicos}

Uno de las principales ventajas de las redes inalámbricas y la
``computación en la nube'' es la posibilidad de trabajar en cualquier
lugar. A menudo, puede revisar su correo en un café con conexión a
internet o en algún otro lugar público. Los espías, criminales y todo
tipo de malvivientes a menudo frecuentan estos lugares para aprovechar
las grandes oportunidades que ofrecen para el robo de identidad, el
espionaje electrónico y el saqueo de cuentas bancarias.

Aquí nos encontramos a menudo con un riesgo a menudo subestimado de que
alguien escuche nuestras comunicaciones usando un \emph{paquete de
sniffing de redes}. No es tan importante que la red sea abierta o está
asegurada por una contraseña. Si alguien se une a la misma red cifrada,
puede capturar y leer fácilmente todo el tráfico inseguro (vea el
capítulo \textbf{Conexión segura}) de todos los otros usuarios que están
dentro de la misma red. Una clave de acceso inalámbrica se puede
adquirir por el precio de una taza de café y les da -a las personas con
conocimientos de lectura y captura de paquetes en la red- la oportunidad
de leer su contraseña mientras usted revisa sus correos electrónicos.

Aquí tiene una sencilla regla que siempre debe cumplir: si el café
ofrece una conexión cableada, ¡úsela! Además, asegúrese de que nadie
está viendo sobre su hombro cuando tipee su contraseña.

\section{Almacenamiento malicioso}\label{almacenamiento-malicioso}

Una vez más la conveniencia rápidamente nos lleva por mal camino. Debido
a la molestia general de tener que escribir las contraseñas una y otra
vez, las almacenamos en el navegador o cliente local de correo
electrónico. Esto no es malo en sí mismo, pero cuando nos roban la
computadora o el teléfono móvil, el ladrón puede acceder a nuestra
cuenta de correo electrónico. Lo más recomendable es limpiar la memoria
caché siempre que cierre su navegador. Todos los navegadores populares
tienen una opción para borrar la memoria caché al salir.

Si aún así decides almacenar en memoria tus contraseñas, deberías cifrar
tu disco. Si tu computadora es robada y el ladrón reinicia la máquina,
va a encontrarse con un disco cifrado. También es aconsejable tener un
bloqueo de pantalla instalado. Si le roban la máquina mientras navega,
no podrán acceder a ella.

\section{Asegurando su
comunicación}\label{asegurando-su-comunicaciuxf3n}

Mientras escriba y envíe correos electrónicos mediante un navegador o un
programa (Outlook Express, Mozilla Thunderbird, Mail.app o Mutt),
asegúrese de que la sesión completa esté cifrada. Esto es fácil de hacer
debido al uso de conexiones \emph{TLS/SSL (Secure Socket Layer)} en los
servidores de correo electrónico (Consulte en el glosario
\textbf{TLS/SSL}).

Si usa un navegador para revisar su correo, verifique que su servidor
soporta sesiones SSL comprobando que la URL comienza con https://. Si
este no es el caso, asegúrese de activarlo en la configuración de
cuentas de correo electrónico, tales como Gmail o Hotmail.Esto asegura
que no sólo la parte de la sesión de inicio de sesión de correo
electrónico está cifrado, sino también la escritura y el envío de
correos electrónicos. Además verifique los detalles del certificado,
tenga en cuenta el \emph{TLS pinning} y respalde las extensiones del
navegador web que advierten acerca de los cambios o los certificados
disfuncionales (por ejemplo, \emph{Certificate Patrol}) y haga uso de la
versión segura TLS del sitio web como default (por ejemplo \emph{HTTPS
everywhere}).

El proveedor de servicios de correo electrónico que usted elija debería
brindarle a usted detalles de su servidor de correos. Estos detalles
pueden hallarse a menudo en la sección de configuración. Si su servicio
de correo electrónico no proporciona TLS/SSL para cifrar sus datos,
entonces le aconsejamos que deje de usarlo. Incluso si sus mensajes no
son importantes, puede que un día se encuentre ``inhabilitado'' para
acceder a su cuenta ¡porque su contraseña ha sido cambiada!

Cuando use un programa de correo electrónico para ver sus mensajes,
asegúrese de usar la opción TLS/SSL. Por ejemplo, en Mozilla Thunderbird
la opción para asegurar su correo saliente se encuentra en
\texttt{Edit\ -\textgreater{}\ Account\ Settings\ -\textgreater{}\ Outgoing\ Server\ (SMTP)},
para correo entrante está en
\texttt{Edit\ -\textgreater{}\ Account\ Settings\ -\textgreater{}\ Server\ Settings}.
Esto nos asegura que la descarga y envío de mensajes esté cifrada,
dificultando su lectura o la de sus registros para cualquier persona de
su propia red o que se encuentre entre usted y su servidor de correo
electrónico. Además, cifre el mensaje en sí mismo. Nota del traductor:
cuando ejemplifiquemos con Thunderbird, usaremos la última versión al
momento de escribir esta traducción, febrero del 2013, que difiere
sensiblemente de las anteriores. Por ejemplo, las configuraciones
mencionadas más arriba, en las versiones de Thunderbird anteriores se
encuentran en Tools y no en Edit. Asimismo, la barra de menú no aparece
visible por defecto en la versión actual, para verla debe hacer un click
con el botón derecho del ratón en la barra donde se encuentra la solapa
\texttt{Inbox} y tildar la opción \texttt{Menu\ bar}.

Aunque la línea esté cifrada usando un sistema como SSL, el proveedor de
correo electrónico aún tiene acceso a los mensajes porque tiene un
acceso completo al dispositivo de almacenamiento de su correo
electrónico. Si desea usar su servicio web asegúrese que su proveedor no
pueda leer sus mensajes, para eso necesitará algo conocido como
\emph{GPG} (en el apéndice, \textbf{GnuPG}) con el cual podrá cifrar su
mensaje. El encabezado de su mensaje, sin embargo, aún contiene la IP
(Internet address, dirección IP) a partir de la cual se envió y otros
detalles comprometedores. Vale la pena mencionar que usar \emph{GPG} en
webmail no es tan sencillo como en los clientes localmente instalados,
tales como \emph{Thunderbird} o \emph{Outlook Express}.

\section{DNSSEC \& DANE}\label{dnssec-dane}

La información del certificado puede estar almacenada en registros DNS y
sin embargo ser más fiable y segura. Verifique la disponibilidad de
\emph{DNSSEC} y especialmente considere los servicios de correo
electrónico \emph{DANE} con su proveedor de servicios. En este punto,
nuevamente las extensiones del navegador web (por ejemplo
\emph{DNSSEC/TLSA Validator}) pueden ayudarlo para controlar la
disponibilidad de estas medidas de seguridad.

\section{Separación de cuentas}\label{separaciuxf3n-de-cuentas}

Debido a la conveniencia de servicios como Gmail, es cada vez más común
que las personas usen una única cuenta de correo electrónico. Esto
aumenta considerablemente el daño potencial que provocaría si tuviéramos
algún problema con ella. Más aún, nada impide que algún empleado
disgustado de Google borre o robe su cuenta, sin olvidar que el propio
Google puede ser hackeado. Estas cosas suceden.

Una estrategia práctica es mantener su cuenta personal de esa manera.
personal. Si dispone del servicio de correo electrónico en su trabajo,
cree una cuenta nueva si su empleador aún no lo ha hecho por usted. Lo
mismo para todo club u organización a la cual pertenezca, con
contraseñas diferentes. No sólo mejora su seguridad, sino que también
reduce el riesgo de un robo completo de identidad y disminuye
enormemente la cantidad de spam.

\section{Nota acerca del almacenamiento de correos
electrónicos}\label{nota-acerca-del-almacenamiento-de-correos-electruxf3nicos}

Los proveedores de servicios de almacenamiento, envío, descarga y
lectura de correos electrónicos no se destacan precisamente por el uso
de TLS/SSL. Al almacenarlos, pueden leer y registrar sus mensajes en
texto plano. Pueden cumplir con los pedidos de las agencias de seguridad
locales que deseen acceder a su cuenta. También pueden analizar sus
mensajes para obtener patrones, palabras claves o signos de sus
afinidades con determinados grupos políticos, ideologías o marcas
comerciales. Por eso es muy importante leer el contrato de licencia de
uso del usuario final de su proveedor de correo electrónico y realizar
una pequeña investigación acerca de sus afinidades e intereses antes de
elegirlo. Todo lo referido anteriormente también se aplica a los
destinatarios de sus mensajes.

\chapter{Tipos de correo
electrónico}\label{tipos-de-correo-electruxf3nico}

El correo electrónico se puede usar de dos maneras:

\begin{itemize}
\item
  Lectura, escritura y envío de mensajes desde un \emph{navegador web}
  (webmail), o
\item
  Lectura, escritura y envío usando un \emph{programa de correo
  electrónico}, como Mozilla Thunderbird, Mail.App o Outlook Express
  utilizando protocolos tales como \emph{SMTP}, \emph{POP} e
  \emph{IMAP}.
\end{itemize}

Estos dos modelos pueden ser mixtos en la práctica, especialmente si se
usa \emph{IMAP}. Aunque el webmail es la solución más adecuada para usar
y más facil de mantener para usuarios finales que usen diferentes
computadoras comparada con las soluciones más poderosas (más
almacenamiento, mejores opciones de búsqueda y control directo de los
datos) basadas en las aplicaciones nativas

\section{\texorpdfstring{Correo electrónico almacenado remotamente
(``webmail'') usando un navegador
web}{Correo electrónico almacenado remotamente (webmail) usando un navegador web}}\label{correo-electruxf3nico-almacenado-remotamente-webmail-usando-un-navegador-web}

Los mensajes enviados por medio del \emph{browser}, a veces llamado
\emph{webmail}, consisten en una cuenta con un almacenamiento remoto de
correo electrónico tal como Google (Gmail), Microsoft (Hotmail) o Yahoo
(Yahoo Mail). Las oportunidades de negocios abiertas al almacenar
mensajes de correo de otras personas son muchas: contacto con otros
servicios ofrecidos por la empresa, exposición de marcas comerciales y
lo más importante, búsqueda entre sus mensajes de patrones que puedan
ser usados para evaluar sus intereses -- algo de gran valor en la
industria de la publicidad (aunque también para determinados gobiernos).
Por razones de data mining, dichas compañías \emph{no están interesadas}
en alentar a sus usuarios para que usen \emph{cifrado para asegurar la
privacidad} y/o \emph{firmas para la integridad/autenticidad} de la
comunicación.

\section{Correo electrónico almacenado remotamente usando un programa o
un navegador
web}\label{correo-electruxf3nico-almacenado-remotamente-usando-un-programa-o-un-navegador-web}

Un programa de correo electrónico tal como Outlook, Thunderbird o
Mail.App también puede ser usado con un servicio de webmail como Gmail o
su compañía proveedora de servicio de correo electrónico. En cualquier
caso, los mensajes aún pueden ser descargados en su computadora pero
están retenidos en su servidor de correo (por ejemplo Gmail). De esta
manera, para acceder a los mensajes no se requiere del uso del navegador
todo el tiempo, pero aún estará usando Gmail, Hotmail, etc. como
servicio. La diferencia entre almacenar los mensajes en su computadora
con un programa de correo y hacerlo remotamente en un servidor (por
ejemplo Hotmail, Gmail o el servidor de su universidad) en Internet
puede parecer algo confuso al principio.

Finalmente, también se pueden enviar mensajes a un servidor de correo
electrónico sin que se almacenen allí en absoluto, simplemente lo
reenvía a su destino tan pronto como llega al servidor de reenvío de
correo electrónico. Google y Microsoft no permiten este tipo de
configuración. Más bien esto suele ser algo que su universidad o empresa
proveerá para usted. Tenga en cuenta que esto conlleva el riesgo de que
el administrador del sistema haga copias secretamente de sus mensajes a
medida que entran y salen del servidor.

En general, el uso de webmail combinado con la descarga de los mensajes
usando un programa de correo electrónico es la mejor opción. Este
enfoque añade redundancia (copias de seguridad locales) junto a la
opción de borrar todo el correo electrónico desde el servidor remoto una
vez descargado. Esta última opción es ideal para la información de
contenido sensible donde la posibilidad de robo de cuentas es alto, pero
corre el riesgo de pérdida total de los mensajes si la máquina local
falla y no se dispone de copias de seguridad. En segundo lugar, cuando
se utiliza un programa de correo electrónico, tenemos la opción de
cifrar los mensajes, como el popular GPG, algo que no es fácil de
configurar y utilizar en servicios de correo web con uso exclusivo del
navegador. En cualquier caso, el cifrado del disco rígido en el equipo
local es altamente recomendable (consulte el Apéndice \textbf{Cifrado de
disco}).

\section{Consideraciones de contexto}\label{consideraciones-de-contexto}

Usted puede administrar un servidor y correr su propio servicio de
correo electrónico. O almacenar sus mensajes en su empresa o en el
servidor de sus jefes. Finalmente, usted puede usar un servicio mediante
una corporación, por ejemplo Google (Gmail) or Microsoft (Hotmail). Cada
uno presenta una interesante combo de consideraciones que se refieren
precisamente al hecho básico de que a menos que la propia dirección de
correo electrónico está cifrada, el administrador del servidor de correo
electrónico aún puede copiar secretamente el correo electrónico en el
momento que llegue al servidor. No importa que usted esté utilizando
\emph{TLS/SSL} (consulte el Apéndice \textbf{SSL}) para ingresar y
consultar su correo electrónico, ya que sólo protege la conexión entre
el equipo local y el servidor.

Como siempre, si conoce los riesgos y se siente preocupado es sabio
escuchar estos consejos - no envíe correos electrónicos sensibles
utilizando un servicio que no sean de confianza.

\section{Empleador/Organización}\label{empleadororganizaciuxf3n}

Su empleador o la organización que esté involucrada está en excelente
posición para aprovecharse de su confianza y leer los mensajes de su
cuenta de correo electrónico laboral que se almacenan en el servidor,
tal vez en un esfuerzo por aprender acerca de usted, de sus
motivaciones, agendas e intereses. Estos casos de espionaje del
empleador hacia el empleado son tan comunes que no merecen atención. La
única solución es el cifrado del correo electrónico usando, por ejemplo,
GPG (consulte el Apéndice \textbf{GPG}).

\section{Correos electrónicos \&
metadata}\label{correos-electruxf3nicos-metadata}

La información del contenido actual de los correos puede ser preservada
usando \emph{OpenPGP} o \emph{S/MIME} pero los metadatos - la asociación
de personas, direcciones, tiempo y software y/o servicios usados - son
almacenados por diversas plataformas. Los servicios gubernamentales
pueden almacenar datos así como también las compañías involucradas en
transmitirlos. Con respecto a la información del encabezado del mensaje
de correo, permanece en riesgo durante la comunicación así como también
las cuentas usadas pueden ser conectadas con individuos o grupos

\section{Servidor de correo auto
administrado}\label{servidor-de-correo-auto-administrado}

Esta es la configuración ideal de almacenamiento, pero requiere un alto
grado de conocimientos técnicos. Aquí, en general, los riesgos a la
privacidad no son sólo proteger su propia cuenta contra intentos de
exploits (contraseñas débiles, sin SSL) sino que conlleva una gran
responsabilidad, y tal vez sucumba a la tentación de leer los correos
electrónicos de aquellas personas a las cuales les presta servicio.

\section{\texorpdfstring{Servicios de correo electrónico
``gratuitos''}{Servicios de correo electrónico gratuitos}}\label{servicios-de-correo-electruxf3nico-gratuitos}

Como se mencionó anteriormente los riesgos de almacenar y enviar
mensajes con un servicio prestado por una empresa son bastante altos si
valora su derecho ciudadano a la privacidad. Las empresas que almacenan
sus cartas de amor, sus expresiones y sus diarios corren el riesgo de
ceder a las presiones de los intereses de orden político, económico y de
las fuerzas de seguridad del país al que están legalmente sujetas. Un
usuario de Gmail Malasia, por ejemplo, corre el riesgo de exponer sus
intereses y sus propósitos a un gobierno que no eligieron, por no hablar
de los socios comerciales de Google interesados en ampliar su alcance en
el mercado.

\section{Sin fines de lucro}\label{sin-fines-de-lucro}

Distintos servidores web ofrecen cuentas de correo electrónico gratuitas
a las organizaciones sin ánimo de lucro o filantrópicos como ellos.
Algunos incluso ofrecen wikis, listas de correo, chats y redes sociales.
Una consideración para las organizaciones que trabajan en el campo
político: puede haber diferencias de intereses entre el estado en el que
se aloja el correo electrónico y los intereses políticos de la
organización por medio de ese servicio. Tales riesgos idealmente se
deben reflejar en el Acuerdo de Licencia de Usuario Final.

\section{Notas sobre reenvío de correo
electrónico}\label{notas-sobre-reenvuxedo-de-correo-electruxf3nico}

Los servicios de reenvío de mensajes proporcionan la ventaja de
``enlazar'' una cuenta con otra de la forma que el usuario crea
conveniente. Esto por supuesto es más comúnmente utilizado cuando el
titular de la cuenta está de vacaciones y quiere que sus mensajes sean
derivados desde su cuenta de trabajo a otra que utilizará durante el
viaje o que está inaccesible fuera del lugar de trabajo. El riesgo con
cualquier servicio de reenvío de correo electrónico externo es el mismo
que el riesgo de alojarlo de forma remota en servicios como Gmail, por
ejemplo: puede ser copiado y almacenado. Aquí, el cifrado usando un
sistema como \emph{GPG} (consulte el Apéndice \textbf{GPG}) le asegurará
de que si se copia por lo menos no se podrá leer.

\chapter{Temores}\label{temores}

\emph{¿Quién puede leer los mensajes de correo electrónico que he
enviado o recibido?}

\emph{¿Quién puede leer los correos electrónicos que envío cuando viajan
a través de Internet?}

\emph{¿Las personas que reciben mis mensajes pueden compartirlos con
alguien?}

Los correos electrónicos que se envían ``en texto plano'', sin ningún
tipo de cifrado (la gran mayoría de los correos electrónicos enviados y
recibidos en la actualidad) se pueden leer, registrar, e indexar por
medio de cualquier servidor o router a lo largo del camino, mientras el
mensaje viaja del emisor al receptor. Suponiendo que utiliza una
conexión cifrada (ver el glosario para TLS/SSL) entre sus dispositivos y
su proveedor de servicios de correo electrónico (lo que todo el mundo
debería hacer), esto significa en la práctica que las siguientes
personas todavía pueden leer cualquier mensaje enviado:

\begin{enumerate}
\def\labelenumi{\arabic{enumi}.}
\tightlist
\item
  Usted
\item
  Su proveedor de correo electrónico
\item
  Los operadores y los dueños de cualquier conexión intermedia de red (a
  menudo conglomerados multinacionales o inclusos estados soberanos)
\item
  El proveedor de servicio de correo electrónico del destinatario
\item
  El destinatario previsto
\end{enumerate}

Muchos proveedores de correo web (como Gmail) automáticamente
inspeccionan todos los mensajes enviados y recibidos por sus usuarios
con el fin de mostrar anuncios dirigidos. Si bien esto puede ser un
compromiso razonable para algunos usuarios (¡libertad al correo
electrónico!), la mayoría de las veces es preocupante para muchas
personas ya que incluso sus comunicaciones más íntimas son
inspeccionados y catalogadas como parte de un perfil mantenido oculto y
potencialmente muy interesante para cualquier poderoso gigante
corporativo con fines de lucro.

Además, alguien que legalmente puede presionar a los grupos
anteriormente mencionados podría solicitar o exigir:

\begin{enumerate}
\def\labelenumi{\arabic{enumi}.}
\tightlist
\item
  metadatos registrados sobre los mensajes (listas de mensajes enviados
  o recibidos por cualquier usuario, asunto de los mensajes,
  destinatarios), injustificada en algunas jurisdicciones.
\item
  mensajes enviados y recibidos por un grupo específico de usuarios o
  grupos, con justificación u orden judicial en algunas jurisdicciones.
\item
  una conexión dedicada a desviar \emph{todos} los mensajes y
  \emph{todo} el tránsito, para ser analizados e indexados fuera del
  sitio.
\end{enumerate}

En los casos donde un usuario tiene una relación comercial o de servicio
con su proveedor de correo electrónico, la mayoría de los gobiernos van
a defender los derechos de privacidad del usuario contra la lectura no
autorizada e injustificada o el intercambio de mensajes, aunque a menudo
es el propio gobierno quien busca información, y con frecuencia los
usuarios renuncian a algunos de estos derechos como parte de su acuerdo
de servicio. Sin embargo, cuando el proveedor de correo electrónico es
el empleador del usuario o institución académica, los derechos de
privacidad con frecuencia no se aplican. Dependiendo de la jurisdicción,
las empresas en general tienen el derecho legal a leer todos los
mensajes enviados y recibidos por sus empleados, incluso los mensajes
personales enviados después de hora o en las vacaciones.

Históricamente, era posible ``eludirlos'' con el uso de correo
electrónico en texto plano, porque el costo y el esfuerzo de almacenar e
indexar el creciente volumen de los mensajes era demasiado alto:
alcanzaba para que los mensajes fueran entregados confiablemente. Por
ello, muchos sistemas de correo electrónico no contienen mecanismos para
preservar la privacidad de sus contenidos. Ahora bien, el costo de la
vigilancia ha bajado mucho más rápidamente que el crecimiento del
tráfico de Internet y es razonable esperar la vigilancia a gran escala y
la indexación de todos los mensajes (ya sea en el remitente o del lado
del receptor) aún para usuarios y los mensajes más inocuos. {[}CITA:
espionaje/archivado de correo electrónico corporativo, bluecoats, el
seguimiento de Siria, centro de datos en Utah, EEUU, los escándalos de
intercepción en EEUU{]}

Para más información sobre la protección legal de los mensajes de correo
electrónico ``en reposo'' (término técnico para los mensajes almacenados
en el servidor después de haber sido enviados), en especial con respecto
a los accesos del gobierno a estos mensajes de correo, vea:

\begin{itemize}
\tightlist
\item
  https://ssd.eff.org/3rdparties/govt/stronger-protection (USA)
\item
  http://en.wikipedia.org/wiki/Data\_Protection\_Directive (EU)
\end{itemize}

Así como hay ciertas fotos, cartas, y credenciales que usted no sube
``en texto plano'' en Internet porque no quiere que esa información sea
indexada accidentalmente y se muestre en los resultados de búsqueda,
nunca se deben enviar mensajes de correo electrónico ``en texto plano''
si no quiere que un empleador o un oficial de seguridad del aeropuerto
disgustado tenga fácil acceso al mensaje.

\section{Abusos al azar y robo por parte de hackers
maliciosos}\label{abusos-al-azar-y-robo-por-parte-de-hackers-maliciosos}

\emph{¿Qué pasa si alguien toma el control completo de mi cuenta de
correo electrónico?}

\emph{Me he conectado desde un lugar inseguro \ldots{} ¿cómo puedo saber
ahora si mi cuenta ha sido hackeada?}

\emph{No he hecho nada malo\ldots{} ¿qué tengo que esconder?}

\emph{¿Por qué alguien se preocupa por mí?}

Por desgracia, hay muchos incentivos prácticos, sociales y económicos
para que los hackers maliciosos irrumpan en las cuentas de individuos al
azar de Internet. El incentivo más evidente es el robo de identidad y
financiero, cuando el atacante puede estar tratando de obtener acceso a
los números de tarjetas de crédito, credenciales de compras del sitio, o
información bancaria para robar dinero. Un hacker malicioso no tiene
manera de saber de antemano que usuarios son mejores blancos que otros,
por lo que sólo tratan de irrumpir en todas las cuentas, incluso si el
usuario no tiene nada para robar o toma recaudos para no exponer su
información.

Menos evidentes son los ataques para obtener acceso a las cuentas de
usuario válidas y confiables para recolectar direcciones de correo
electrónico de contactos y luego distribuir spam en forma masiva, o para
acceder a determinados servicios vinculados a una cuenta de correo
electrónico, o para usarla como un ``trampolín'' en sofisticados ataques
de ingeniería social. Por ejemplo, una vez controlada la cuenta un
hacker malicioso podría rápidamente enviar correos electrónicos a sus
socios o compañeros de trabajo solicitando un acceso de emergencia a los
sistemas informáticos más seguros.

Un último problema inesperado que afecta incluso a los usuarios de bajo
perfil de correo electrónico, es el secuestro masivo de cuentas en
grandes proveedores de servicios, cuando los hackers maliciosos acceden
a la propia infraestructura de hosting y extraen contraseñas e
información privada en grandes cantidades, y luego vender o publicar
listas de información de inicio de sesión en mercados online.

\section{Abuso dirigido, acoso y
espionaje}\label{abuso-dirigido-acoso-y-espionaje}

\emph{Algo que escribí enfureció a una persona en el poder\ldots{} ¿Cómo
puedo protegerme?}

Si usted es un objetivo individual de atención por parte de poderosas
organizaciones, gobiernos o individuos determinados, entonces deberá
aplicar las mismas técnicas y principios para mantener la seguridad y la
privacidad de su correo electrónico, pero deberá tomar medidas
adicionales para protegerse de hackers maliciosos que utilizan técnicas
sofisticadas para socavar sus dispositivos y cuentas. Si un hacker
malicioso toma el control de alguno de sus dispositivos de computación o
tiene acceso a cualquiera de sus cuentas de correo electrónico, es
probable que pueda acceder en forma inmediata tanto a la totalidad de su
correspondencia, como a cualquiera de los servicios externos vinculados
a su cuenta de correo electrónico.

Los esfuerzos para protegerse contra este tipo de ataques pueden sufrir
una gran escalada hasta extenderse rápidamente en una batalla de
voluntades y recursos, pero algunas pautas básicas pueden ayudarlo. Use
dispositivos exclusivos para las comunicaciones. Desconecte y apague los
dispositivos inmediatamente después de que haya terminado de usarlos. Lo
mejor es utilizar herramientas de cifrado, navegadores web y sistemas
operativos de software libre ya que pueden sus problemas de seguridad
pueden ser revisados públicamente y solucionados con los parches de
seguridad.

\emph{Ten cuidado al abrir archivos PDF con Adobe Reader u otros
lectores de PDF propietarios.} Los lectores de PDF de código cerrado han
sido utilizadas para ejecutar código maligno incorporado en el cuerpo
del PDF. Si recibe un pdf como archivo adjunto primero debe considerar
si se conoce el presunto emisor y si usted está esperando un documento
de él. En segundo lugar, puede utilizar lectores de PDF que han sido
probados en busca de vulnerabilidades conocidas y no ejecutar código a
través de javascript.

GNU/Linux: Evince, Sumatra PDF

OS X: Preview

Windows: Evince

Use contraseñas generadas al azar siempre que sea posible.

\section{Cuando el cifrado funciona
mal}\label{cuando-el-cifrado-funciona-mal}

\emph{¿Qué pasa si pierdo mis ``claves''? ¿Pierdo mi correo
electrónico?}

Estrictamente hablando, el cifrado GPG de correo electrónico no deja de
tener sus propios problemas.

Si almacena su correo electrónico cifrado y pierde todas las copias de
su clave privada, será absolutamente incapaz de leer los mensajes
antiguos almacenados, y si usted no tiene una copia de su certificado de
revocación para la clave privada sería muy difícil probar que cualquier
nueva clave que genere es válida, al menos hasta que la clave privada
original expire.

Si usted firma un mensaje con su clave privada, tendrá grandes
dificultades para convencer a alguien de que no lo hizo si el
destinatario revela el mensaje y la firma públicamente. El término para
esto es \emph{sin repudio}: cualquier mensaje que envíe firmado es una
excelente evidencia en una corte. Además, si la clave privada es robada,
podría ser utilizada para leer todos los mensajes cifrados enviados a
usted alguna vez con su clave pública: los mensajes pueden estar seguros
cuando se encuentran en tránsito y en el momento en que se reciben, pero
las copias son una responsabilidad y dependen de que la clave privada
nunca sea revelada. En particular, incluso si se destruye cada mensaje
justo después de leerlo, cualquiera pueda interceptar el mensaje en el
hilo, guardar una copia y tratar de descifrarlo más tarde si obtiene la
clave privada.

La solución es utilizar un protocolo de mensajería que proporciona un
\emph{secreto-perfecto-hacia-adelante} mediante la generación de forma
aleatoria de una nueva clave de sesión única para cada conversación de
intercambio de mensajes de tal manera que no puedan ser generadas a
posteriori, aunque sean conocidas las claves privadas. El protocolo de
chat (\url{https://es.wikipedia.org/wiki/Perfect_forward_secrecy}) lo
garantiza en el caso de la mensajería instantánea en tiempo real, y el
protocolo SSH lo proporciona para las conexiones de shell remoto, pero
no existe un sistema equivalente para el correo electrónico en este
momento.

Puede ser difícil de sopesar la comodidad del acceso móvil a sus claves
privadas con el hecho de que los dispositivos móviles son mucho más
fáciles de perder, ser robados o hackeados que las máquinas fijas.
Cualquier urgencia podría ser el momento exacto en que usted más desea
enviar un mensaje confidencial o un mensaje firmado para verificar su
identidad, pero también es el momento en el puede que no tenga acceso a
las claves privadas si su dispositivo móvil está intervenido o no está
cargado con todas sus claves.

\chapter{Conexiones seguras}\label{conexiones-seguras}

\section{¿Otras personas pueden leer mis mensajes mientras verifico mi
correo
electrónico?}\label{otras-personas-pueden-leer-mis-mensajes-mientras-verifico-mi-correo-electruxf3nico}

Como vimos en el Capítulo \textbf{Basic Tips}, aunque utilice correo web
o un programa de correo electrónico usted siempre debe asegurarse de
utilizar el cifrado para la sesión entera, desde el inicio hasta cerrar
la sesión. Esto evitará que cualquier persona espíe su comunicación con
su proveedor de correo electrónico. Afortunadamente, esto es fácil de
hacer debido al uso popular de las conexiones \emph{TLS/SSL} en los
servidores de correo electrónico (ver apéndice \textbf{TLS/SSL}).

Una conexión TLS/SSL en el navegador, cuando se utiliza correo web,
aparecerá con \texttt{https} en la URL, en lugar de la norma
\texttt{http}, por ejemplo:

\texttt{https://gigglemail.com}

Si su servidor de correo web no ofrece un servicio de TLS/SSL, entonces
debería considerar la suspensión del uso de esa cuenta, aunque sus
propios correos electrónicos no son especialmente privados o
importantes, su cuenta puede ser fácilmente hackeada descubriendo su
contraseña por ``sniffing''. Si aún no está activado, asegúrese de
hacerlo en las opciones de su cuenta. Al momento de escribir el libro,
Gmail de Google y Hotmail/Microsoft Live cambian automáticamente su
explorador para utilizar una conexión segura.

Si está utilizando un programa de correo electrónico como Thunderbird,
Mail.app o Outlook, asegúrese de comprobar que está utilizando TLS/SSL
en las opciones del programa. Consulte el capítulo \textbf{Configuración
de conexiones seguras} en la sección de \textbf{Seguridad del correo
electrónico}.

\section{Notas}\label{notas}

Es importante tener en cuenta que los administradores de los proveedores
como Hotmail o Google, que albergan, reciben o envían su correo
electrónico, pueden leerlo aunque utilice conexiones seguras. También es
destacable que las claves criptográficas que protegen una conexión
TLS/SSL pueden ser deliberadamente reveladas por los operadores de los
sitios web, o copiadas sin su permiso, violando la confidencialidad de
su conexión. También es posible que un Certificado de Autorización esté
corrupto o comprometido entonces se creará un certificado falso de las
claves mantenido por espías, lo que facilita el ataque man in the middle
sobre las conexiones que utilizan TLS/SSL (Véase el Glosario para
``Ataque Man in the Middle''). A modo de ejemplo, vea el siguiente caso
\url{http://cryptome.info/0001/nsa-ssl-email.htm} que involucra a la NSA
de Estados Unidos y a varios proveedores de correo electrónico.

También notemos que el uso de una \emph{Red Privada Virtual} es otra
buena manera de asegurar sus conexiones al enviar y leer correo
electrónico, pero requiere el uso de un cliente VPN en el equipo local
que se conecta a un servidor. Consulte el capítulo \textbf{Red Privada
Virtual} en la sección de \textbf{navegación}.

\chapter{Correo electrónico seguro}\label{correo-electruxf3nico-seguro}

Es posible enviar y recibir mensajes seguros utilizando los programas
estándares de correo electrónico actuales mediante la adición de unos
pocos complementos. La función esencial de estos complementos es hacer
que el cuerpo del mensaje (pero no los campos Para:, De:, CC: y Asunto:)
sea ilegible para cualquier tercera parte que intercepte o acceda de
otro modo a su correo electrónico o a la de su compañero de
conversación. Este proceso se conoce como cifrado.

Para asegurar los mensajes se utiliza generalmente una técnica llamada
\emph{Criptografía de clave pública}. La criptografía de clave pública
es una técnica inteligente que utiliza dos claves de codificación para
enviar un mensaje. Cada usuario tiene un \emph{clave pública}, la cual
sólo puede ser utilizada para cifrar un mensaje, pero no para
descifrarlo. Las claves públicas son bastante seguras para no tener que
preocuparse de que alguien pudiera descubrirlos. La \emph{clave privada}
es mantenida en secreto por la persona que recibe el mensaje y se puede
utilizar para descifrar los mensajes codificados con la clave pública
correspondiente.

En la práctica, eso significa que si Rosa quiere enviarle un mensaje
seguro a Heinz, sólo necesita su clave pública para codificar el texto.
Al recibir el correo electrónico, Heinz a continuación utiliza su clave
privada para descifrar el mensaje. Si quiere responder, tendrá que
utilizar la clave pública de Rosa para cifrar la respuesta, y así
sucesivamente.

\section{¿Qué software puedo usar para cifrar mi correo
electrónico?}\label{quuxe9-software-puedo-usar-para-cifrar-mi-correo-electruxf3nico}

La configuración más popular para la criptografía de clave pública es el
uso de \emph{Gnu Privacy Guard (GPG)} para crear y administrar claves y
un complemento para integrarla con el software de correo electrónico
estándar. El uso de GPG le dará la opción de cifrar correo electrónico
sensible y decodificar el coreo entrante que ha sido cifrada pero no
estará obligado a usarlo todo el tiempo. Años atrás era muy difícil de
instalar y configurar, pero avances recientes han hecho que este proceso
sea relativamente simple.

Consulte la sección \textbf{Cifrado de correo electrónico} para trabajar
con GPG en el ámbito de su sistema operativo y su programa de correo
electrónico.

Si utiliza un servicio de \emph{webmail}, es más difícil cifrar el
correo electrónico. Puede utilizar un programa GPG en su computadora
para cifrar el texto utilizando su clave pública o puede utilizar un
complemento, como Lock The Text
(\url{http://lockthetext.sourceforge.net/}). Si desea mantener los
mensajes privados, le sugerimos que utilice un programa de correo
electrónico dedicado como Thunderbird en lugar de webmail.

\chapter{Consejos básicos}\label{consejos-buxe1sicos-1}

\section{Brevemente:}\label{brevemente-1}

\begin{itemize}
\tightlist
\item
  Cuando visite un sitio web no dé ninguna información acerca de usted
  mismo al dueño del sitio sin tomar algunas precauciones.
\item
  Su navegación en Internet puede ser rastrada por los sitios que visita
  y por los socios de estos sitios. Use software antiseguimiento.
\item
  La visita a un sitio web nunca es una conexión directa. Muchas
  computadoras, de distintos dueños, están involucradas. Use una
  conexión segura para evitar que su sesión web sea registrada.
\item
  Lo que usted busca es lo que más le importa a los proveedores de los
  buscadores. Use software de búsqueda anónima para proteger su
  privacidad.
\item
  Es más prudente confiar en los navegadores de código abierto como
  Mozilla Firefox, ya que su seguridad puede ser auditada más
  fácilmente.
\end{itemize}

\section{Su navegador habla de usted por detrás
suyo}\label{su-navegador-habla-de-usted-por-detruxe1s-suyo}

Todos los navegadores comunican información al servidor web que almacena
la página que usted visita. Esta información incluye el nombre y la
versión del navegador, la información de referencia (un enlace desde
otro sitio, por ejemplo) y el sistema operativo utilizado.

Los sitios web suelen utilizar esta información para personalizar su
experiencia de navegación, lo que sugiere descargas para su sistema
operativo y formatear la página web para adaptarla mejor a su navegador.
Naturalmente, esto presenta un problema en lo que al anonimato del
usuario ya que esta información forma parte de un conjunto más amplio de
datos que pueden ser utilizados para identificarlo en forma individual.

Detener la charla de su navegador no es fácil de hacer. Usted puede, sin
embargo, falsificar alguna parte de la información enviada a los
servidores web mientras navega por la alteración de los datos contenidos
en el archivo \emph{User Agent}, la identidad del navegador. Hay un
plugin muy útil para Firefox, por ejemplo, el llamado \emph{User Agent
Switcher} que le permite establecer la identidad del navegador a otro
perfil seleccionado de una lista desplegable de opciones.

\section{Los sitios web pueden rastrear por dónde usted
navega}\label{los-sitios-web-pueden-rastrear-por-duxf3nde-usted-navega}

A menudo, los sitios web escriben en su computadora pequeños archivos
llamados cookies. Estos cookies presentan ciertas ventajas, como
almacenar datos de inicio de sesión, información de sesión y otros datos
que hacen a su experiencia de navegación más llevadera. Estas pequeñas
piezas de información son muy peligrosas para su derecho al anonimato en
la web: pueden ser usadas para identificarlo si retorna al sitio y
también puede registrar como navega entre diferentes sitios. Junto con
el User-Agent, representan un medio poderoso y secreto para identificar
remotamente a su persona.

La solución ideal para este problema es denegar todos los intentos del
sitio web para escribir cookies en su sistema sin embargo esto puede
reducir significativamente la calidad de su experiencia en la web.

Consulte la sección \textbf{Seguimiento} para ver guías de cómo impedir
el rastreo de sitios web sobre usted.

\section{Búsqueda online de información acerca de usted
mismo}\label{buxfasqueda-online-de-informaciuxf3n-acerca-de-usted-mismo}

Cuando usamos buscadores tales como Bing o Google ponemos en riesgo
nuestro derecho a la privacidad, mucho más que cuando respondemos, por
ejemplo, a una persona del sector de Informaciones en un aeropuerto.

La información combinada del uso de datos de User Agent y las cookies
pueden usarse para construir un retrato suyo en tiempo real. Los
publicistas consideran a esta información muy valiosa, y la usan para
hacer hipótesis acerca de sus intereses y del mercado de los productos
de una manera más específica.

Mientras que algunos clientes pueden cantar alabanzas de la publicidad
dirigida y a otros los tiene sin cuidado, los riesgos son a menudo mal
entendidos. En primer lugar, la información recopilada acerca de usted
puede ser solicitada por un gobierno, incluso un gobierno que no
eligieron (Google, por ejemplo, es una empresa estadounidense y por lo
tanto debe cumplir con los procesos judiciales estadounidenses y sus
intereses políticos). En segundo lugar, existe el riesgo que la mera
búsqueda de información pueda ser mal interpretada como la intención o
el apoyo político. Por ejemplo, el estudio de un artista de la estética
de las diferentes formas de extremismo religioso lo pone en peligro de
ser asociado con el apoyo de las organizaciones estudiadas. Por último,
existe el riesgo de que este perfil oculto pueda ser vendido a los
agentes de seguros, a sus posibles empleadores o a los clientes de la
empresa cuyo servicio de búsqueda está utilizando.

Incluso aunque se haya asegurado que las cookies se borraron, su
\emph{User Agent} ha sido cambiado (vea más abajo y en el capítulo de
\textbf{Seguimiento}) y todavía está informando un dato crucial: la
dirección de Internet de dónde se conecta (vea el capítulo \textbf{¿Qué
sucede mientras navega}). Para evitar esto, puedes usar un servicio de
anonimato como Tor (ver capítulo \textbf{Anonimato}). Si usted es un
usuario de Firefox (recomendado), asegúrese de instalar el excelente
complemento \emph{Google Sharing}, que mantiene su anonimato mientras
realiza una búsqueda en Google. Incluso si no usa Google, debe cuidarse
de un gran número de sitios web que utilizan una barra personalizada de
búsqueda de Google como un medio para explorar su contenido

Por lo dicho anteriormente, no se puede confiar en Google, en Yahoo ni
en Bing. Nosotros recomendamos cambiar por un servicio de búsqueda que
toma en cuenta su derecho a la privacidad muy seriamente: DuckDuckGo
(\url{http://duckduckgo.com/}).

\section{Más ojos de los que usted puede
ver}\label{muxe1s-ojos-de-los-que-usted-puede-ver}

Internet es un enorme lugar y no es una única red, sino que es una gran
red formada por muchas redes pequeñas interconectadas entre sí. Cuando
usted solicita una página a un servidor de Internet su solicitud puede
atravesar muchas máquinas antes de alcanzar al servidor que hospeda la
página. Este trayecto se conoce como encaminamiento y típicamente
incluye al menos 10 máquinas a través de la ruta. Como los paquetes se
mueven de una máquina a otra, deben copiarse en la memoria, reescribirse
y traspasarse.

Cada una de las máquinas a través del encaminamiento en la red pertenece
a alguien, normalmente una empresa u organización y puede estar en
diferentes países. Si bien se están realizando esfuerzos para
estandarizar las leyes de comunicación entre los países, existe en la
actualidad uno amplia variedad según la jurisdicción. Así, mientras que
puede que no haya una ley que exige el registro de su navegación por la
web en su país, tales leyes pueden existir en otro lugar a lo largo de
la ruta de su paquete.

La única forma de proteger el tráfico a lo largo de la ruta de que sea
grabado o manipulado es utilizar cifrado de extremo a extremo como el
proporcionado por TLS/Secure Socket Layer (Vea el capítulo
\textbf{Cifrado}) o una red privada virtual (Vea el capítulo
\textbf{VPN}).

\section{Su derecho a permanecer en el
anonimato}\label{su-derecho-a-permanecer-en-el-anonimato}

Más allá del deseo de minimizar las fugas de privacidad para los
proveedores de servicios específicos, usted debe considerar ocultar la
dirección de Internet desde la cual se conecta habitualmente (vea el
capítulo \textbf{Qué sucede cuando navega}). El deseo de lograr este
anonimato impulsó la creación del \emph{Proyecto Tor}.

\emph{Tor} usa una red de nodos en constante evolución para enrutar la
conexión a un sitio de una manera que no se puede rastrear de nuevo
hasta usted. Es un medio muy robusto para asegurar que su dirección de
Internet no se puede registrar en un servidor remoto. Vea el capítulo
\textbf{Anonimato} para obtener más información acerca de cómo funciona
y cómo empezar con Tor.

\chapter{Temores}\label{temores-1}

\section{Redes sociales - ¿cuáles son los
peligros?}\label{redes-sociales---cuuxe1les-son-los-peligros}

El fenómeno de las redes sociales han cambiado no sólo a la forma en
cómo la gente usa Internet. Grandes servidores alrededor del mundo,
particularmente en EEUU, se han construido para atender al deseo
repentino y enorme de la gente de subir contenido sobre sí mismos, sus
intereses y sus vidas con el fin de participar en las redes sociales.

Las redes sociales, tal como conocemos a Facebook, Twitter (y
anteriormente MySpace) están lejos de ser ``libres''. Más bien, se trata
de empresas que buscan desarrollarse sobre una angustia muy básica para
luego poder explotarla: el miedo a la irrelevancia social. Como animales
sociales que somos no podemos soportar la idea de aislarnos y por lo
tanto muchos se creen integrados colocando sus expresiones más íntimas
en el disco duro de un hombre de negocios, enclavado en un centro de
datos en otro país - uno que nunca se le permitirá visitar.

A pesar de esto, muchos podrían argumentar que el calor social y el
reconocimiento personal adquirido a través del compromiso con las redes
sociales, compensa la posible pérdida de la privacidad. Tal afirmación,
sin embargo es válida sólo cuando se conoce completamente la magnitud de
los riesgos.

Las amenazas de las redes sociales al derecho básico de las personas a
la privacidad son las siguientes:

\begin{itemize}
\item
  El alcance y la intimidad de las contribuciones individuales de los
  usuarios.
\item
  Un usuario que publica frecuentemente e incluye muchos detalles
  personales construye un cuerpo de información de enorme utilidad para
  marketing directo.
\item
  La preparación del usuario para asumir riesgos sociales.
\item
  Un usuario que establece conexiones sociales sin cuidado corre un gran
  riesgo ante ataques de ingeniería social y los depredadores.
\item
  Los intereses económicos y los socios de las organizaciones que
  proveen el servicio.
\item
  Estudios encargados de clientes, minería de datos, análisis de
  sentimientos.
\item
  Demandas político/legales ejercidas por el Estado contra la
  organización en las jurisdicciones en las cuales reside.
\item
  Órdenes de los jueces para obtener datos de un usuario particular (sin
  importar si es nativo o extranjero).
\item
  Agendas de vigilancia por aplicación de leyes o socios de la
  organización.
\item
  Análisis de sentimientos: proyecciones de intentos políticos.
\end{itemize}

Con estas cosas en mente, es posible señalar un gran contraste entre
proyectos como Diáspora y Facebook: el primero promete un cierto nivel
de transparencia organizativa, el compromiso con la privacidad y una
apertura general, mientras que Facebook ha demostrado ser una empresa
turbia económicamente capaz de especular con la privacidad de sus
usuarios y gestionar demandas civiles antes que lo hagan sus clientes.
Por lo tanto es más probable de que su reputación sea analizada por una
compañía de seguros o un empleador potencial si usa la gran red social
en lugar de otra más pequeña y transparente.

\section{¿Quién puede robar mi
identidad?}\label{quiuxe9n-puede-robar-mi-identidad}

La respuesta depende del contexto en el cual usted trabaja con su
navegador. Una contraseña débil y única para múltiples servicios de
redes sociales, banca, webmail, etc. es muy peligrosa y es muy factible
que sea robada. Una contraseña fuerte y única en una red inalámbrica
compartida con otros (ya sea abierta o cifrada) es vulnerable. La regla
general es que usted se asegure de tener una contraseña personal fuerte
(ver la sección de \textbf{Contraseñas}).

\subsection{Redes inalámbricas}\label{redes-inaluxe1mbricas}

Aquí nos encontramos en medio de un riesgo a menudo subestimado de que
alguien escuche nuestras comunicaciones usando un \emph{paquete de
sniffing de red}. Poco importa si la red es abierta o posee una
contraseña segura. Si alguien utiliza la misma red cifrada, puede
fácilmente capturar y leer todo el tráfico inseguro de otros usuarios en
la misma red. Una clave inalámbrica puede ser adquirida por el costo de
una taza de café y le da a todos aquellos que saben cómo capturar y leer
los paquetes de red la oportunidad de leer su contraseña mientras usted
revisa su correo electrónico.

Existe una regla simple que debe aplicarse siempre: si el café ofrece
una conexión de cable de red, ¡úsela! Finalmente, así como en un cajero
automático, asegúrese de que no miran por encima del hombro cuando
escribe la contraseña.

\subsection{El caché del navegador
web}\label{el-cachuxe9-del-navegador-web}

Debido a la molestia general de tener que escribir su contraseña en
repetidas ocasiones, muchas personas permiten que el navegador o el
cliente de correo local lo almacenen por usted. Esto no es malo en sí
mismo, pero cuando un ordenador portátil o teléfono es robado, le
permite al ladrón acceder a la cuenta del propietario del correo
electrónico. Lo más recomendable es limpiar la caché siempre que cierre
su navegador. Todos los navegadores populares tienen una opción para
borrar la memoria caché al salir.

El uso de la memoria caché se justifica si toma la precaución de cifrar
su disco. Si su dispositivo portátil es robado y el ladrón reinicia la
máquina, van a encontrarse con un disco cifrado. También es aconsejable
tener un bloqueo de pantalla instalado en su ordenador o teléfono. Si la
máquina es robada mientras está ejecutando una sesión de usuario, no se
podrá acceder a ella.

\subsection{Asegurando su línea}\label{asegurando-su-luxednea}

Mientras esté registrado en cualquier servicio debería asegurarse de
usar cifrado para la sesión completa. Esto se puede hacer fácilmente con
el popular \emph{TLS/SSL (Secure Socket Layer)}.

Compruebe que el servicio que está usando (ya sea correo electrónico,
redes sociales o banca en línea) es compatible con sesiones TLS/SSL
viendo la existencia de \texttt{https://} al comienzo de la URL. Si no
es así, asegúrese de activarlo en todas las configuraciones
proporcionadas por el proveedor del servicio. Para entender mejor cómo
funciona la navegación web, consulte el capítulo \textbf{¿Qué sucede
cuando navego en Internet?}

\section{¿Puedo meterme en problemas por usar Google con cosas
raras?}\label{puedo-meterme-en-problemas-por-usar-google-con-cosas-raras}

Google y otras compañías de búsqueda pueden cumplir con las órdenes
judiciales dirigidas a determinadas personas. Un sitio web con un campo
personalizado de búsqueda de Google para encontrar contenido en su sitio
puede ser obligado a registrar y suministrar todas las consultas de
búsqueda a la justicia local. Académicos, artistas e investigadores
están particularmente en riesgo de ser mal entendidos, ya que suponen
motivaciones donde sólo existen intereses aparentes.

\section{¿Quién mantiene un registro de mi navegación? ¿puedo esconderme
de
ellos?}\label{quiuxe9n-mantiene-un-registro-de-mi-navegaciuxf3n-puedo-esconderme-de-ellos}

Está completamente cubierto por sus derechos humanos básicos, y
comúnmente protegido constitucionalmente, para poder visitar los sitios
web de forma anónima. Del mismo modo que se le permita visitar una
biblioteca pública, hojear libros y ponerlos de nuevo en la estantería
sin que nadie tome nota de las páginas y los títulos de su interés,
usted es libre de navegar de forma anónima en Internet.

\section{¿Cómo hacer para no revelar mi
identidad?}\label{cuxf3mo-hacer-para-no-revelar-mi-identidad}

Consulte el capítulo sobre \textbf{Anonimato}.

\section{¿Cómo evitar ser
rastreado?}\label{cuxf3mo-evitar-ser-rastreado}

Vea el capítulo de \textbf{Seguimiento}.

\chapter{Qué sucede cuando usted
navega}\label{quuxe9-sucede-cuando-usted-navega}

Navegar por la web es comunicarse. Puede que usted no envíe mucho texto
en términos de cantidad de palabras, pero siempre es el navegador el que
inicia y mantiene la comunicación al solicitar los bits y las piezas de
datos que están involucrados con lo que usted eventualmente ve en su
pantalla.

Los navegadores tales como Mozilla Firefox, Google Chrome, Opera, Safari
e Internet Explorer trabajan todos de manera similar. Cuando escribe una
URL (por ejemplo ``http://happybunnies.com'') en la barra de
direcciones, el navegador solicita el sitio web (el cual es sólo un tipo
especial de texto) de un servidor remoto y entonces lo transforma en
bloques coloridos, textos e imágenes para ser mostrados en la ventana
del navegador. Para ver el texto de la manera en que el navegador lo ve,
sólo debe hacer click en
\texttt{Ver\ -\/-\textgreater{}\ Código\ de\ la\ página}. Lo que verá
será la misma página web pero en HTML -- un lenguaje que se ocupa
principalmente del contenido, el contexto y los enlaces a otros recursos
(CSS y JavaScript) que gobiernan la forma en que los contenidos son
mostrados y cómo se comportan.

Cuando el navegador intenta abrir una página web - y suponiendo que no
hay proxies involucrados - lo primero que hace es comprobar su propia
caché. Si no tiene registros de visitas anteriores a dicho sitio,
intentará resolver el nombre en una dirección que realmente puede
utilizar. Se trata de un programa de internet, por lo que necesita una
dirección de Protocolo de Internet (dirección IP o simplemente IP). Para
obtener esta dirección se le pide a un servidor DNS (una especie de guía
telefónica para los programas de Internet) que se instala en el router
de su conexión a internet de forma predeterminada1. La dirección IP es
una etiqueta numérica asignada a cada dispositivo en la red (global),
como la dirección de una casa en el sistema postal - y como la dirección
de su casa, usted debe tener mucho cuidado a quién se la da (por defecto
es visible para todos). Una vez que la dirección IP ha sido recibida, el
navegador abre una conexión TCP (un protocolo de comunicación) al host
de destino y comienza a enviar paquetes a un puerto en esta dirección,
por lo general el puerto 80 (los puertos son como puertas a los
servidores, hay muchos, pero por lo general sólo unos pocos están
abiertos)2 a menos que se especifique otra ruta. Estos paquetes viajan a
través de una serie de servidores en Internet (hasta un par de docenas
en función de donde se encuentra la dirección de destino). Después, el
servidor busca la página solicitada y, si lo encuentra, la entrega
utilizando el protocolo HTTP. (Para evitar que otras personas lean o
alteren los datos, se debe usar TLS/SSL junto con HTTP para asegurar la
conexión).

Cuando llega la respuesta HTTP, el navegador puede cerrar la conexión
TCP o reutilizarlo para las solicitudes posteriores. La respuesta puede
ser una entre muchas, desde algún tipo de redirección hasta el clásico
error interno del servidor (500). Siempre que la respuesta es la
esperada, el navegador guarda la página en la memoria caché para su uso
posterior, la decodifica (la descomprime si está comprimida, la
renderiza si es un códec de vídeo, etc) y la muestra en pantalla o la
ejecuta de acuerdo con las instrucciones.3

Ilustremos el proceso con una pequeña conversación entre el navegador
(B) y el servidor (S):

B: ``Hola.''

S: ``Hola!''

B: ``¿Puede alcanzarme la página con los conejitos felices, por favor?''

S: ``Bien, aquí la tiene.''

B: ``Oh, tal vez usted podría alcanzarme una versión más grande de esa
imagen en la cual el conejito bebé abraza un oso de peluche.''

S: ``Seguro, por qué no.''

{[}\ldots{}{]}

B: ``Esto es todo por ahora. Muchas gracias. Adiós.''

Tenga en cuenta que hay un montón de actividades que suceden
paralelamente a este intercambio de TCP/IP. Dependiendo de cómo haya
configurado las opciones, el navegador podría añadir la página a la
historia del navegador, guardar cookies, comprobar plugins y
actualizaciones RSS y comunicarse con una gran variedad de servidores,
todo mientras está haciendo otra cosa.

\section{Una topografía suya:
huellas}\label{una-topografuxeda-suya-huellas}

Lo más importante: siempre dejará rastros. Algunos permanecerán en su
propia computadora -- una colección de datos en caché, la historia de
navegación y pequeños archivos malvados con memoria de elefante llamados
cookies. Son todos ellos muy ventajosos; aceleran su navegación, reducen
su descarga de datos o recuerdan sus contraseñas y preferencias en las
Redes Sociales. También estudian sus hábitos de navegación y recopilan
registros de todos los lugares que visita y de todo lo que hace en
ellos. Esto debería preocuparlo si está usando una computadora pública
en una biblioteca, en su trabajo o en un cibercafé, o si comparte el
departamento con un compañero entrometido.

Incluso si configura su navegador para no registrar el historial de
navegación, rechaza las cookies y borra los archivos almacenados en
caché (o asignar cero MB de espacio en caché), seguiría dejando rastros
por toda Internet. Su dirección IP queda registrada de forma
predeterminada en todas partes y para todo el mundo y los paquetes
enviados son supervisados por un número cada vez mayor de entidades -
comerciales, gubernamentales o criminales, junto con algunos cretinos y
acosadores potenciales.

Los gobiernos democráticos en todas partes están rediseñando las
regulaciones para exigir a los proveedores de Internet que conserven una
copia de todo para poder tener acceso a ella más tarde. En los EE.UU.,
el artículo 215 de la Ley Patriótica Estadounidense \emph{`prohíbe a un
individuo u organización revelar que les ha dado sus registros al
gobierno federal, siguiendo una investigación'}. Eso significa que la
empresa a la cual usted le paga cada mes para poder tener acceso a
Internet puede ser obligada a entregar su historial de navegación y sus
registros de correo electrónico sin su conocimiento.

La mayor parte del tiempo, sin embargo, la vigilancia no es un asunto de
1984. Google recopila sus búsquedas, junto con su identificación del
navegador (\emph{user agent}), su dirección IP y un montón de datos que
eventualmente puede conducir a su puerta, pero el objetivo final no
suele ser la represión política, sino la investigación de mercado. Los
anunciantes no se preocupan solamente por el espacio publicitario, ellos
quieren saberlo todo sobre usted. Ellos quieren saber sus hábitos de
medicación y dietarios, el número de hijos que tiene y dónde se toman
las vacaciones, qué hace con su dinero, cuánto gana y cómo le gusta
gastarlo. Aún más: quieren saber qué \emph{sienten} acerca de
determinadas cosas. Ellos quieren saber si sus amigos respetar esos
sentimientos lo suficiente para que pueda convencerlos de que cambien
sus hábitos de consumo. Esto no es una conspiración, sino más bien la
naturaleza del capitalismo en la era de la información. Parafraseando
una famosa observación de la situación actual, las mentes más brillantes
de nuestra generación está pensando en cómo hacer que la gente haga
click en los avisos comerciales.4

Algunas personas piensan que los avisos comerciales pueden ignorarse o
que los publicistas satisfacen nuestras necesidades específicas en una
situación de ganar-ganar, porque el spam que reciben se refiere a cosas
que eventualmente desean. Si este fuera el caso (y no lo es):
¿deberíamos confiarle a Google aspectos íntimos y detallados de nuestra
vida? Aunque creamos que Google `no es el diablo', puede ser comprado
por alguien en quien no confiamos; los benevolentes Larry Page y Sergey
Brin pueden ser destituidos por su propio Consejo, o su base de datos
puede ser secuestrada por un gobierno fascista. Uno de sus 30.000
empleados en todo el mundo puede irse con nuestros datos. Sus servidores
pueden ser hackeados. Después de todo, sólo están interesados en sus
clientes, \emph{las empresas que pagan por publicidad}. Sólo somos el
producto que se vende.

Más aún; en las Redes Sociales nuestros hábitos de navegación generan un
registro permanente, una colección de datos tan vasta que la información
que Facebook recopila acerca de un sólo usuario puede llenar 880
páginas. Nadie podrá sorprenderse al saber que el propósito de Facebook
no es hacernos más felices -- de nuevo: si usted no paga por algo, no es
un cliente, es el producto. Pero aunque no le preocupen sus objetivos
comerciales, piense en esto: la plataforma tiene publicidades que
permiten que hackers maliciosos irrumpan en cientos de miles de cuentas
de Facebook todos los días.

Para una muestra de lo que se esconde detrás de las cortinas de los
sitios web que visita, instale un plugin/add-on llamado \emph{Ghostery}
en tu navegador. Es como una radiografía de la máquina que revela toda
la tecnología de vigilancia que puede estar (y a menudo lo está)
incrustada en una página web, normalmente invisible para el usuario. En
esta misma línea, \emph{Do Not Track Plus} y \emph{Trackerblock} le
darán un mayor control sobre el seguimiento online, a través del bloqueo
de cookies, las cookies persistentes opt-out, etc. Nuestro capítulo
siguiente \textbf{Seguimiento} le enseñará mucho acerca de dichos temas.

Incluso entre el ordenador y el router, los paquetes pueden ser
fácilmente interceptadas por cualquier persona que utilice la misma red
inalámbrica en el ambiente informal de un café. Es una jungla allá
afuera, pero todavía elegimos contraseñas como ``password'' y
``123456'', realizamos transacciones económicas y compramos entradas en
las redes públicas inalámbricas y hacemos click en enlaces de correos
electrónicos no solicitados. No se trata solamente de nuestro derecho a
preservar nuestra intimidad, también tenemos la responsabilidad de
defender ese derecho contra las intrusiones de los gobiernos, empresas y
cualquier persona que intentan despojarnos de ellos. Si no ejercemos
esos derechos hoy en día, nos merecemos lo que suceda mañana.

\begin{enumerate}
\def\labelenumi{\arabic{enumi}.}
\tightlist
\item
  Si es un usuario de Unix, puede usar el comando tcpdump en el bash y
  ver el tráfico dns en tiempo real. ¡Está cargado de diversión! (y
  disturbios) \^{}
\item
  Vea la lista de números de puertos TCP y UDP
  (\url{https://es.wikipedia.org/wiki/Anexo:Lista_de_números_de_puerto}).
  \^{}
\item
  Si el intercambio se produce bajo una conexión HTTPS, el proceso es
  mucho más complicado y también mucho más seguro, hallará más
  información acerca de esto en el fascinante capítulo llamado
  Criptografía. \^{}
\item
  This Tech Bubble Is Different
  (\href{http://www.businessweek.com/magazine/content/11_17/b4225060960537.htm}{http://www.businessweek.com/magazine/content/11\_17//b4225060960537.htm}),
  Ashlee Vance (Businessweek magazine) \^{} Cuentas y seguridad
  ===================
\end{enumerate}

Cuando navega por Internet, puede estar conectado con varios servicios,
a veces en forma simultánea. Puede estar en el sitio web de una empresa,
viendo su correo electrónico o en una red social. Nuestras cuentas son
importantes porque almacenan información altamente sensible acerca de
nosotros y de otras personas en máquinas a lo largo de toda Internet.

Mantener sus cuentas seguras requiere algo más que una contraseña segura
(véase la sección \textbf{Contraseñas}) y un vínculo de comunicación
segura con el servidor a través de TLS/SSL (véase el capítulo
\textbf{Conexión segura}). A menos que se especifique lo contrario, la
mayoría de los navegadores almacenan sus datos de acceso en pequeños
archivos llamados cookies, reduciendo la necesidad de volver a escribir
la contraseña cuando vuelva a conectarse a estos sitios. Esto significa
que alguien con acceso a su computadora o teléfono celular puede acceder
a sus cuentas sin tener que robar la contraseña o hacer espionaje
sofisticado.

Desde que los teléfonos inteligentes se han vuelto muy populares ha
habido un aumento dramático en el secuestro de cuentas por robo de
teléfonos. El robo de computadoras portátiles presenta un riesgo
similar. Si usted elige que el navegador guarde sus contraseñas entonces
usted tiene varias opciones para protegerse:

\begin{itemize}
\tightlist
\item
  Utilice un bloqueo de pantalla. Si usted tiene un teléfono y prefiere
  un sistema de patrón de desbloqueo debe adquirir el hábito de limpiar
  la pantalla para que un atacante no pueda adivinar el patrón de
  manchas de los dedos. En una computadora portátil, debe configurar su
  salvapantallas para que le pida una contraseña, así como una
  contraseña en el arranque.
\item
  Cifrar el disco duro. TrueCrypt es un sistema de cifrado de disco
  abierto y seguro para Windows 7/Vista/XP, Mac OS X y GNU/Linux. OSX y
  muchas distribuciones de GNU/Linux ofrecen la opción de cifrado de
  disco en la instalación.
\item
  Desarrolladores Android: no habilitar la depuración USB en el teléfono
  de forma predeterminada. Esto permite a un atacante utilizar el
  \emph{adb shell} de Android en una computadora para acceder al disco
  duro de su teléfono sin desbloquearlo.
\end{itemize}

\section{¿Puede un sitio web malicioso apoderarse de mis
cuentas?}\label{puede-un-sitio-web-malicioso-apoderarse-de-mis-cuentas}

Aquellos cookies especiales que contienen sus datos de inicio de sesión
son el punto primario de vulnerabilidad. Una técnica muy popular para
robo de datos es el llamado clickjacking, donde el usuario es engañado
al hacer click en un enlace aparentemente inofensivo, ejecutando un
script que se aprovechará del hecho de que usted está logueado. Los
datos de inicio de sesión pueden ser robados, permitiéndole al atacante
remoto acceder a su cuenta. Aunque es una técnica complicada, ha probado
ser muy efectiva en varias ocasiones. Tanto en Twitter como en Facebook
se han registrado casos de inicio de sesión robadas usando esta técnica.

Es importante desarrollar hábitos para pensar antes de hacer click en
enlaces a sitios mientras está logueado en sus cuentas. Una técnica es
utilizar otro navegador que no registre las cuentas como una herramienta
para probar la seguridad de un enlace. Confirme siempre la dirección
(URL) en el enlace para asegurarse de que esté escrita correctamente.
Puede ser un sitio con un nombre muy similar al del sitio en el cual
confía. Tenga en cuenta que los enlaces con acortadores de URL (como
http://is.gd y http://bit.ly) son riesgosos ya que no puede ver el
enlace real al cual usted está solicitando los datos.

Si utiliza Firefox en su dispositivo, utilice el complemento
\href{http://noscript.net}{NoScript}, ya que mitiga muchas de las
técnicas de \emph{Cross Site Scripting} que permitan que su cookie de
login sea robado, pero tenga en cuenta que se deshabilitarán muchas
características de algunos sitios web.

\chapter{Seguimiento}\label{seguimiento}

Cuando navega por la web pequeños rastros de su presencia van quedando
por el camino. Muchos sitios web inofensivos usan estos datos para
compilar estadísticas y ver cuánta gente está visitando el sitio y qué
páginas son populares, pero algunos sitios usan estas técnicas para
rastrear usuarios individuales, tratando de ir más allá para
identificarlos personalmente. Sin embargo, no se detienen acá. Algunas
empresas almacenan datos en su navegador para registrarlo a usted en
otros sitios. esta información puede ser recopilada y pasada a otras
organizaciones sin su conocimiento o permiso.

Todo esto suena ominoso ¿pero a quién le importa realmente si alguna
gran empresa sabe de unos pocos sitios web que hemos visto? Los sitios
web grandes recopilan y utilizan estos datos para ``publicidad
comportamental'' donde los anuncios están diseñados para satisfacer
exactamente sus intereses. Es por eso que después de mirar la entrada de
Wikipedia para Mallorca, uno de repente puede comenzar a ver un montón
de anuncios para la paquetes de vacaciones envasados y sombreros de
fiesta. Esto puede parecer bastante inocente, pero después de hacer una
búsqueda de ``tratamientos para el herpes'' o ``comunidades fetiches'' y
ver listados de repente a los productos pertinentes, se puede empezar a
sentir que la web se está volviendo demasiado familiar.

Esta información es también de interés para otros, como su compañía de
seguros. Si ellos saben que usted ha estado buscando en los sitios de
paracaidismo o en los foros de enfermedades congénitas, sus primas
misteriosamente puede empezar a aumentar de precio. Los empleadores
potenciales o los propietarios de alquileres pueden perder interés
basado en sus intereses en la web. En casos extremos, las autoridades
policiales o fiscales pueden empezar a observarlo sin que siquiera haya
cometido un delito, simplemente sobre la base de sospechas.

\section{¿Cómo lo siguen?}\label{cuxf3mo-lo-siguen}

Cada vez que carga una página web, el software del servidor en el sitio
web genera un registro de la página vista en un archivo de log. Esto no
es siempre una mala idea. Cuando se loguea en un sitio web, es necesario
establecer su identidad y mantener registros ordenados para grabar sus
preferencias, o presentarle información personalizada. Esto se logra
pasándole un pequeño archivo a su navegador y almacenando una referencia
correspondiente en el servidor web. Este archivo se denomina
\emph{cookie}. Suena hermoso, pero el problema es que esta información
se mantiene en el equipo incluso después de salir del sitio web y podrá
llamar a casa para decirle al dueño de la cookie acerca de otros sitios
web que está visitando. Algunos sitios importantes, como Facebook y
Google han sido descubiertos usándolos para realizar un seguimiento de
su navegación, incluso después de haber cerrado la sesión.

\section{¿Cómo puedo evitar el
seguimiento?}\label{cuxf3mo-puedo-evitar-el-seguimiento}

La manera más simple de evitar el seguimiento es borrar las cookies en
su navegador.

En \textbf{Firefox}:

\begin{enumerate}
\def\labelenumi{\arabic{enumi}.}
\tightlist
\item
  Pulse \textbf{Firefox menu}.
\item
  Pulse \textbf{Options}.
\item
  Pulse \textbf{Privacy}.
\item
  Pulse \textbf{Clear your recent history}.
\end{enumerate}

\includegraphics{firefox_delete_cookies_01.png} 5. Asegúrese de
configurar \textbf{Time range to clear} como \textbf{Everything}. 6.
Tilde \textbf{Cookies}.

\includegraphics{firefox_delete_cookies_02.png} 7. Haga click en
\textbf{Clear now}.

En \textbf{Chrome}: 1. Pulse \textbf{Chrome menu}. 2. Pulse
\textbf{Tools}. 3. Pulse \textbf{Clear browsing data}. 4. Asegúrese de
configurar \textbf{Obliterate the following items from} como \textbf{The
beginning of time}. 5. Tilde \textbf{Delete cookies and other site and
plug-in data}. 6. Pulse \textbf{Clear browsing data}.

\begin{figure}[htbp]
\centering
\includegraphics{chrome_delete_cookies_02.png}
\caption{Borrado de cookies en Chrome}
\end{figure}

En \textbf{Internet Explorer}: 1. Pulse el botón \textbf{Tools} (en
forma de engranaje). 2. Pulse \textbf{Safety}. 3. Pulse \textbf{Delete
Browsing History}. 4. Tilde \textbf{Cookies}. 5. Pulse \textbf{Delete}.

\begin{figure}[htbp]
\centering
\includegraphics{ie_delete_cookies_02.png}
\caption{Borrado de cookies e Internet Explorer}
\end{figure}

La limitación de esta aproximación es que usted recibirá nuevas cookies
tan pronto como vuelva al sitio o cuando vaya a otras páginas con
componentes de seguimiento. Otras desventajas son que usted perderá
todas sus sesiones iniciadas para todas sus pestañas abiertas,
forzándolo a tipear sus nombres de usuario y contraseña nuevamente. Una
opción más conveniente, soportada por los navegadores actuales es
navegación privada o modo incógnito. Esto abre una ventana de un
navegador temporario que no grabará la historia de las páginas
visitadas, contraseñas, archivos descargados o cookies. Después de
cerrar la ventana de navegación privada, toda la información será
borrada.

En \textbf{Firefox}: 1. Pulse \textbf{Firefox menu}. 2. Pulse
\textbf{Start Private Browsing}.

\includegraphics{firefox_private_browsing_01.png} 3. Si se lo solicita,
pulse \textbf{Start Private Browsing} nuevamente.

\includegraphics{firefox_private_browsing_02.png} 4. El botón
\textbf{Firefox menu} aparece en color púrpura, mostrando que se está
navegando en forma privada.

\begin{figure}[htbp]
\centering
\includegraphics{firefox_private_browsing_03.png}
\caption{Navegación privada en Firefox}
\end{figure}

En \textbf{Chrome}: 1. Pulse \textbf{Chrome menu}. 2. Pulse \textbf{New
incognito window}.

\includegraphics{chrome_private_browsing_01.png} 3. El \textbf{ícono
espía} en la parte superior izquierda de la ventana del navegador
muestra que se está navegando en forma privada.

\begin{figure}[htbp]
\centering
\includegraphics{chrome_private_browsing_02.png}
\caption{Navegación privada en Chrome}
\end{figure}

En \textbf{Internet Explorer}: 1. Pulse en el menú \textbf{Tools}, en
forma de engranaje. 2. Pulse \textbf{Safety}. 3. Pulse \textbf{InPrivate
Browsing}.

\includegraphics{ie_private_browsing_01.png} 4. El logo
\textbf{InPrivate} aparecerá en la parte superior izquierda de la
ventana del navegador: se está navegando en forma privada.

\begin{figure}[htbp]
\centering
\includegraphics{ie_private_browsing_02.png}
\caption{Navegación privada en IE}
\end{figure}

Esta solución también tiene sus limitaciones. Nosotros no podemos grabar
marcadores, registrar contraseñas, o sacar ventajas de la conveniencia
ofrecida por navegadores modernos. Afortunadamente, existen distintos
plugins especialmente diseñados para direccionar los problemas del
seguimiento. El más extenso, en términos de características y
flexibilidad, es Ghostery. El plugin le permite bloquear servicios
individuales o por categorías que registran usuarios.

\begin{enumerate}
\def\labelenumi{\arabic{enumi}.}
\tightlist
\item
  En Firefox, pulse el menú \textbf{Firefox} y elija \textbf{Add-ons}.
\end{enumerate}

\includegraphics{ghostery01.png} 2. En la casilla \textbf{Search}, tipee
``ghostery'', luego pulse el ícono \textbf{Search} o presione
\textbf{Enter}.

\includegraphics{ghostery02.png} 3. Busque Ghostery en la lista de
Add-ons, y pulse \textbf{Install}.

\includegraphics{ghostery03.png} 4. Reinicie su navegador pulsando
\textbf{Restart Now}.

\includegraphics{ghostery04.png} 5. Pulse \textbf{Ghostery toolbar} y
seleccione \textbf{Options}. Recorre las opciones y prueba diversos
ajustes para Ghostery, si así lo desea.

\includegraphics{ghostery05.png} 6. Visite una página web y observe sus
rastreadores.

\begin{figure}[htbp]
\centering
\includegraphics{ghostery06.png}
\caption{Ghostery}
\end{figure}

Otra opción es instalar un plugin bloqueador de publicidad como
AdBlockPlus. Este plugin automáticamente bloqueará muchos de las cookies
de seguimiento enviadas por empresas de publicidad pero no los
utilizados por Google, Facebook y otras empresas de análisis web.

\section{¿Cómo puedo ver quién me está
siguiendo?}\label{cuxf3mo-puedo-ver-quiuxe9n-me-estuxe1-siguiendo}

La forma más fácil de ver quién lo está rastreando es usar el plugin
Ghostery. Hay un pequeño ícono en la esquina superior derecha o inferior
derecha de la ventana del navegador que le dirá qué servicios lo están
siguiendo a usted en un sitio web específico.

(Sugerencia: Añada el complemento Do Not Track de Abine.com Sugerimos
utilizar tanto Ghostery como DNT, porque a veces bloquean cookies
distintas. Abine también tiene Privacy Suite, recientemente desarrollado
que puede darle un proxy telefónico y de correo electrónico, similar a
10 Minute Mail o Guerrilla Mail para rellenar correos electrónicos para
formularios.)

\section{Una palabra de advertencia}\label{una-palabra-de-advertencia}

Si bloquea a sus rastreadores tendrá un elevado nivel de privacidad
cuando navegue por la web. Sin embargo, las agencias de gobierno, los
jefes, los hackers y los administradores de red inescrupulosos aún
podrán interceptar su tráfico y averiguar qué está viendo. Si quiere
asegurar su conexión, necesitará leer el capítulo de cifrado. Su
identidad puede ser visible para otras personas en internet. Si quiere
proteger completamente su identidad mientras navega, tendrá que dar
algunos pasos más hacia el anonimato en línea que se explican en otra
sección de este libro.

\chapter{Anonimato}\label{anonimato}

\section{Introducción}\label{introducciuxf3n-1}

Artículo 2 de la Declaración Universal de los Derechos Humanos:

\begin{quote}
``Toda persona tiene todos los derechos y libertades proclamados en esta
Declaración, sin distinción alguna, por motivos de raza, color, sexo,
idioma, religión, opinión política o de otra índole, origen nacional o
social, posición económica, nacimiento o cualquier otra condición.
\end{quote}

\begin{quote}
Además, no se hará distinción alguna en función de la condición
política, jurídica o internacional del país o territorio del cual
dependa una persona, tanto si es independiente, fiduciaria, no autónomo
o bajo cualquier otra limitación de soberanía``.
\end{quote}

Una forma de aplicación de este derecho básico en ambientes hostiles es
por medio del anonimato, donde los intentos para conectar un agente
activo a una persona específica están bloqueados.

Actuar anónimamente es también de gran ayuda para aquellos con una gran
necesidad de protección - cuanto más grande es el rebaño de ovejas, más
difícil es encontrar una en particular. Una manera fácil de hacerlo es
mediante el uso de TOR, una técnica que enruta el tráfico de Internet
entre los usuarios de un software especial, por lo que es imposible de
rastrear a cualquier dirección IP específica o persona sin que tenga
autoridad sobre toda la red (y que nadie tiene aún en el caso de la
Internet). Un medio muy funcional para proteger la identidad de los
propios es el uso de servidores proxy anónimos y redes privadas
virtuales (VPN).

\section{Proxy}\label{proxy}

\begin{quote}
``Un \textbf{anonymizer} o un \textbf{proxy anónimo} es una herramienta
que ayuda a hacer que la actividad en Internet no pueda ser rastreada.
Es una computadora que es un proxy {[}servidor{]} que actúa de
intermediaria y como escudo de la privacidad entre un cliente y el resto
de internet. Accede a Internet en representación del usuario,
protegiendo su información personal al ocultarla información que pudiera
identificar a la computadora del cliente.''
(\url{http://en.wikipedia.org/wiki/Anonymizer})
\end{quote}

El objetivo principal detrás del uso de un proxy es ocultar o cambiar la
dirección de Internet (dirección IP) asignada a la computadora del
usuario. Puede haber varias razones por las que necesitan hacerlo, por
ejemplo:

\begin{itemize}
\tightlist
\item
  Para acceder en forma anónima a determinados servidores y/o para
  ocultar los rastros que quedan en los archivos de registro de un
  servidor web. Por ejemplo, un usuario podría necesitar acceder a
  materiales sensibles en línea (materiales especiales, temas de
  investigación u otra cosa) sin llamar la atención de las autoridades.
\item
  Para atravesar los cortafuegos de las empresas o de los regímenes
  represivos. Un gobierno/corporación puede limitar o restringir
  completamente el acceso a Internet a una dirección IP específica o un
  rango de direcciones IP. Al esconderse detrás de un proxy ayudará a
  engañar a estos filtros y acceder a sitios prohibidos de otra manera.
\item
  Para ver los videos online prohibidos en su país debido a cuestiones
  legales.
\item
  Para acceder a los sitios web y/o materiales disponibles sólo para las
  direcciones IP que pertenecen a un país específico. Por ejemplo, un
  usuario quiere ver un vídeo en la BBC (Reino Unido solamente),
  mientras que no residen en el Reino Unido.
\item
  Para acceder a Internet desde una dirección IP parcialmente
  prohibida/bloqueada. Las direcciones IP públicas a menudo puede tener
  ``mala fama'' (abuso del ancho de banda, estafa o distribución de
  correo electrónico no solicitado) y ser bloqueadas por algunos sitios
  web y servidores.
\end{itemize}

Aunque el proxy debería utilizarse para acceder a la Web (HTTP), en la
práctica el protocolo de Internet puede ser ``proxificado'', es decir,
enviado vía servidor remoto. A diferencia de un router, un servidor
proxy no envía directamente las peticiones de usuarios remotos, sino que
interviene en las solicitudes y respuestas hechos a la computadora del
usuario remoto.

El proxy (a menos que esté configurado como ``transparente'') no permite
la comunicación directa a Internet por eso las aplicaciones tales como
navegadores web, clientes de chat o aplicaciones de descargas deben
tenerlo en cuenta al conectarse (vea el capítulo \textbf{Navegación web
segura/Configuración de proxy})

\section{Tor}\label{tor}

\begin{quote}
\begin{itemize}
\tightlist
\item
  Tor impide que alguien conozca su localización o aprenda acerca de sus
  hábitos de navegación.
\item
  Tor funciona con navegadores web, clientes de mensajería instantánea,
  sesiones remotas, etc.
\item
  Tor es software libre y está disponible para Windows, Mac, GNU/Linux,
  Unix y Android. (\url{https://www.torproject.org})
\end{itemize}
\end{quote}

Tor es un sistema destinado a permitir el anonimato en línea, compuesto
por un software cliente y una red de servidores que pueden ocultar
información sobre la ubicación de los usuarios y otros factores que
pudieran identificarlos. Imagine un mensaje que está envuelto en varias
capas de protección: cada servidor tiene que quitarle una capa, con lo
que inmediatamente elimina la información del remitente del servidor
anterior.

El uso de este sistema hace que sea más difícil de rastrear el tráfico
en Internet del usuario, que incluye visitas a sitios web, publicaciones
online, mensajes instantáneos y otras formas de comunicación. Su
objetivo es proteger la libertad personal de los usuarios, la privacidad
y la capacidad de hacer negocios confidencialmente, al evitar que sus
actividades en Internet sean monitoreadas. El software es libre y la red
de uso gratuito.

Tor no puede y no intenta protegerlo del monitoreo del tráfico que entra
y sale de la red. Mientras que Tor proporciona protección contra el
análisis de tráfico, no puede evitar el tráfico de confirmación (también
llamado correlación de extremo a extremo). La \emph{correlación de
extremo a extremo} es una manera de hacer coincidir una identidad online
con una persona real.

Un ejemplo reciente involucra al FBI que quería demostrar que un hombre,
Jeremy Hammond, estaba detrás de un alias que se sabía responsable de
varios ataques anónimos. Sentado frente a su casa, el FBI estaba
monitoreando su tráfico inalámbrico junto a un canal de chat que sabía
que visitaba el alias. Cuando Jeremy se conectó en su apartamento, la
inspección de los paquetes inalámbricos reveló que estaba usando Tor en
el mismo momento en que el alias sospechado asociado con él se conectó
al canal de chat vigilado. Esto fue suficiente para incriminar a Jeremy
y él fue arrestado.

Consulte la sección \textbf{Navegación segura/Uso de Tor} para
instrucciones de configuración.

\chapter{VPN}\label{vpn}

La forma en que los datos van y vuelven entre el servidor y su
computadora portátil o dispositivo móvil no es tan sencilla como podría
parecer.Supongamos que está conectado a una red inalámbrica en casa y
abre una página, por ejemplo, wikipedia.org. La ruta de su solicitud
(datos) va a consistir en múltiples puntos medios o \emph{``saltos''} -
en la terminología de arquitectura de red. En cada uno de estos saltos
(probablemente más de 5) sus datos pueden ser potencialmente recogidos,
copiados y modificados por:

\begin{itemize}
\tightlist
\item
  Su red inalámbrica (sus datos pueden ser husmeados desde el aire)
\item
  Su ISP (en la mayoría de los países están obligados a mantener
  registros detallados de la actividad del usuario)
\item
  Un Internet Exchange Point (IXP) en algún lugar en otro continente
  (generalmente más seguro que cualquier otro \emph{salto})
\item
  El ISP de la empresa de hosting que aloja el sitio (probablemente está
  manteniendo registros)
\item
  La red interna a la que está conectado el servidor
\item
  Y varios saltos entre \ldots{}
\end{itemize}

Cualquier persona con acceso físico a las computadoras o las redes que
están en el camino de usted con el servidor remoto, de forma deliberada
o no, puede recoger y mostrar los datos que se pasan desde que el
servidor remoto y viceversa. Esto es especialmente cierto para
situaciones llamadas de ``última milla'' - los últimos saltos que hace
una conexión a Internet para llegar a un usuario. Eso incluye redes
inalámbricas o cableadas, privadas o públicas, redes móviles y
telefónicas, redes en bibliotecas, hogares, escuelas, hoteles. El ISP no
puede ser considerado seguro, ni una instancia `neutral a los datos' -
en muchos países las agencias estatales ni siquiera necesitan una orden
judicial para acceder a sus datos, y siempre existe el riesgo de
intrusión por parte de atacantes pagos que trabajan para adversarios de
bolsillos profundos.

VPN - una red privada virtual - es una solución para esta filtración de
``última milla''. VPN es una tecnología que permite la creación de una
red virtual en la parte superior de una infraestructura existente. Tal
red VPN funciona usando los mismos protocolos y estándares como la red
física subyacente. Utiliza a los programas y al sistema operativo de
forma transparente, como si se tratara de una conexión de red por
separado, sin embargo, su topología o la forma en cómo los nodos de la
red (usted, el servidor VPN y, potencialmente, otros miembros o
servicios disponibles en VPN) están conectados entre sí en relación con
el espacio físico es totalmente nueva.

Imagínese que en vez de tener que confiar sus datos a todos y cada uno
de los intermediarios (su red local, ISP, el Estado) tiene la opción de
pasar a través de un servidor de un proveedor VPN en quien usted confía
(después de una recomendación o de una investigación) - desde el cual
los datos viajarán a la ubicación remota. VPN le permite recrear su
contexto local y geopolítico todo junto - desde el momento en que sus
datos dejan a su equipo y se meten en la red VPN están plenamente
asegurados con cifrado tipo TLS/SSL. Y como tal, aparecerá como puro
ruido aleatorio a cualquier nodo que podrían estar espiando detrás suyo.
Es como si los datos se desplazaran dentro de un tubo de aleación de
titanio, irrompible en todo el camino desde su computadora hasta el
servidor VPN. Por supuesto, uno podría argumentar que con el tiempo,
cuando los datos están fuera del puerto seguro de la VPN se vuelve tan
vulnerables como lo eran antes - pero esto es sólo parcialmente cierto.
Una vez que los datos salen del servidor VPN están lejos suyo - más allá
del alcance de algunos cretinos que husmean en la red local inalámbrica,
su venal ISP o un gobierno local obsesionado con las leyes
antiterroristas. Un proveedor VPN serio tendría sus servidores
instalados en un lugar de intercambio en Internet de alta seguridad,
dificultando seriamente el acceso físico humano, la grabación o el
registro.

\begin{quote}
``Todo lo que usted hace hoy en día en Internet está monitoreado y
nosotros queremos cambiar esto. Con nuestro servicio de VPN rápida usted
tendrá un anonimato total en Internet. Podrá navegar en sitios web
censurados, que su escuela, ISP, trabajo o país están bloqueando.
{[}DarkVPN{]} no solo ayudará a la gente a navegar anónimamente, también
ayudará a la gente de países como China para que puedan navegar por
páginas web censuradas. Lo cual es su derecho democrático. DarknetVPN le
da a todos los usuarios VPN una dirección IP anónima. Todos los
registros electrónicos terminarán en usted. Nosotros no grabamos ningún
archivo de registros para alcanzar el máximo anonimato posible. Con
nosotros usted siempre puede navegar en forma anónima, segura y
cifrada.'' (\url{http://www.darknetvpn.com/about.php})
\end{quote}

Otra característica interesante y a menudo subestimada de una VPN está
codificada en su nombre - además de ser \textbf{V}irtual y
\textbf{P}rivada es también una red (\textbf{N}etwork). La VPN permite
no sólo conectarse por medio de su servidor al resto del mundo sino
también comunicarse con otros miembros de la misma red VPN sin tener que
abandonar la seguridad de su espacio cifrado. Por medio de esta
funcionalidad las VPN's se convierten en algo asi como una
\emph{Darknet} ( en el sentido amplio de la palabra) - una red aislada
de Internet e inaccesible a ``otros''. Ya que una conexión con un
servidor VPN, y esto la red privada lo facilita, requiere una clave para
\emph{certificar}, solamente se permite a los usuarios ``invitados''. No
existe chance de que un extraño de Internet pudiera obtener acceso a una
VPN sin registrarse como usuario o sin robar alguna clave. Mientras no
sea referida como tal, cualquier tipo de red intranet corporativa es
también una Darknet.

\begin{quote}
``Una red privada virtual (VPN) permite una extensión segura sobre una
red pública o no controlada como Internet. Permite que la computadora en
la red envíe y reciba datos sobre redes compartidas o públicas como si
fuera una red privada con toda la funcionalidad, seguridad y políticas
de gestión de una red privada.''
(\url{https://es.wikipedia.org/wiki/vpn})
\end{quote}

Muchos proveedores comerciales de VPN hacen hincapié en el anonimato que
proporciona su servicio. Citando la página Ipredator.org (un servicio
VPN iniciado por la gente detrás del proyecto Pirate Bay):

\begin{quote}
``Usted cambiará la dirección IP que recibe de su proveedor de Internet
para obtener una dirección IP anónima. Usted obtiene una conexión
segura/cifrada entre su computadora e Internet''.
(\url{https://www.ipredator.se})
\end{quote}

En efecto, cuando se accede a Internet a través de una conexión VPN
parece como si la conexión se originara en la dirección IP de los
servidores de IPredator.

\chapter{Publicaciones anónimas}\label{publicaciones-anuxf3nimas}

Si usted es un activista que opera bajo un régimen totalitario, un
empleado determinado a exponer algunas malas acciones de su empresa o un
escritor vengativo que compone un retrato insidioso de su ex esposa,
necesita proteger su identidad. Si no va a colaborar con los demás, debe
enfocarse en el anonimato y no en el cifrado o la privacidad.

Si el mensaje es urgente y hay mucho en juego, una manera fácil es
simplemente salir, ir a un lugar con Internet que no frecuente, crear
una cuenta de correo electrónico específicamente para la ocasión,
entregar los datos y descartar posteriormente esas cuentas. Si usted
está en un apuro, considere usar MintEmail
(\href{http://www.mintemail.com/}{http://mintemail.com/}) o FilzMail
(\url{http://www.filzmail.com/}), donde su cuenta expira a partir de 3 y
24 horas, respectivamente. No haga nada más mientras está allí, no
marque su cuenta de gmail, no visite Facebook y borre toda la caché, los
cookies y el historial y cierre el navegador web antes de salir.

Si sigue estas reglas básicas, lo peor que podría suceder - aunque
altamente improbable - es que el equipo estuviera comprometido y
registrara las pulsaciones del teclado, que revelarían las contraseñas o
incluso la cara, en el caso de una cámara web conectada y operada
remotamente. No haga esto en el trabajo o en un lugar donde usted sea un
usuario registrado o visitante regular, como un club o una biblioteca.

Si desea mantener un flujo constante de comunicación e incluso
establecer una conferencia, este método rápidamente se vuelve bastante
engorrosos, y también podría quedarse sin cafés de Internet para usar.
En este caso se puede usar una máquina de su propiedad, pero, si no se
puede dedicarla especialmente a este fin, arranque la computadora con un
sistema operativo diferente. Esto puede hacerse fácilmente mediante el
uso de una memoria USB para arrancar un sistema operativo live por
ejemplo TAILS (\url{https://tails.boum.org/}), el cual viene con TOR
habilitado por defecto e incluye herramientas criptográficas
actualizadas. En cualquier caso, use Tor para ocultar su IP.

Deshabilite todas las cookies, el historial y las opciones de caché y
nunca utilice el mismo perfil o el mismo navegador para otras
actividades. No sólo eso sería agregar datos a su topografía como
usuario en la red, sino que también abre una ventana muy amplia para los
errores. Si desea ayuda adicional, instale \emph{Do Not Track Plus} y
\emph{Trackerblock} o \emph{Ghostery} en el menú de complementos de tu
navegador.

Utilice contraseñas apropiadas y distintas para diferentes cuentas o
incluso frases de paso (vea más sobre esto en la sección de consejos
básicos). Proteja su sistema con una contraseña general, cámbiela a
menudo y no la comparta con nadie, \emph{especialmente} con su amante.
Instale un capturador de teclado para ver si alguien se cuela en su
correo electrónico, especialmente su amante. Configure sus preferencias
generales para desconectarse de todos los servicios y las plataforma
después de 5 minutos de inactividad. Mantenga su identidad de superhéroe
en secreto.

Aunque usted pueda MANTENER tal nivel de disciplina, incluso debe ser
capaz de usar su propia conexión a Internet. Pero considere esto: no
utilizar un sistema dedicado hace que sea muy difícil mantener todos los
diferentes identidades separadas de forma segura, y la sensación de
seguridad a menudo conduce a la falta de cuidado. Mantenga un nivel
adecuado de neurosis.

Hoy en día existen muchas posibilidades de publicación, desde sitios de
blogs sin costo (Blogspot, Tumblr, WordPress, Identi.ca) a PasteBins
(ver glosario) y algunos contemplan específicamente a los usuarios
anónimos como BlogACause. Global Voices Advocacy recomienda el uso de
WordPress a través de la red Tor. Mantenga un nivel sano de
escepticismo, todos tienen un interés comercial para que usted utilice
estas plataformas ``libres'' y por lo tanto no puede confiar plenamente
en ellas, especialmente en cuanto a que pueden estar sujeto a las
demandas de una jurisdicción legal que no es la suya. Todos los
proveedores son, cuando se llega a este punto, unos traidores.

Si el registro de estos servicios requiere una dirección de correo
electrónico, cree una cuenta dedicada exclusivamente a este fin. Evite
Gmail, Yahoo, Hotmail y otras grandes plataformas comerciales con
antecedentes de entregar a sus usuarios y vaya a un servicio
especializado como Hushmail (\url{https://www.hushmail.com/}). Para más
información sobre correo electrónico anónimo, por favor, consulte el
capitulo sobre Anonimato en la sección anterior.

\section{Distintos no}\label{distintos-no}

\textbf{No registre un dominio.} Existen servicios que protegen su
identidad en una consulta simple acerca de quién es, como Anonymous
Speech o Silent Register, pero tendrá que hacerlo mediante pago. A menos
que tenga la posibilidad de comprar uno en BitCoins, limítese a uno de
los dominios ofrecidos por su plataforma de blogging tal como
yourblogname.blogspot.com y elija una configuración fuera de su propio
país. También, encuentre un mombre que no lo delate fácilmente. Si tiene
problemas con esto, use un generador online de nombres de blogs.

\textbf{No abra una cuenta de red social asociada a su blog.} Si debe
hacerlo, mantenga el nivel de higiene que mantiene para blogging y nunca
jamás se conecte mientras usa su navegador habitual. Si tiene una vida
en una red social pública, cuídese de tener todo junto. Tarde o temprano
cometerá un error.

\textbf{No suba videos, fotos u archivos de audio} sin usar un editor
que modifique o borre todos los metadatos (las fotos contienen
información acerca de las coordenadas del lugar donde la fotografía fue
tomada con cámaras digitales estándares, SmartPhones, registradores y
otros dispositivos añadidos por defecto. The \emph{Metadata
Anonymisation Toolkit} podría serle útil.

\textbf{No se olvide de las historias.} Añada X-Robots-Tag a sus
encabezados http para detener a los robots de búsqueda que indexan su
sitio web. Esto debería incluir repositorios como Wayback Machine de
archive.org. Si no sabe qué es esto, busque a través de las líneas de
``Robots Text File Generator''.

\textbf{No se olvide de los comentarios.} Si debe hacerlo, mantenga los
niveles de higiene que usa para blogging y siempre cierre la sesión
cuando haya terminado y por el amor de dios no se comporte como un
troll. El infierno es agradable comparado con un blogger despreciado.

\textbf{No espere a lo último.} Si usted golpea la olla y se convierte
en una sensación del blogging (como \emph{Belle de Jour}, la estudiante
británica de doctorado que se convirtió en una sensación y vendió un
libro y reflexionó en dos programas de televisión acerca de su doble
vida como prostituta de lujo) habrá una legión de periodistas,
inspectores fiscales y fanáticos obsesivos que escudriñen todos sus
movimientos. Usted es solamente una persona: ellos la atraparán.

\textbf{No se detenga.} Si se da cuenta que ha cometido errores, aunque
nadie lo haya atrapado, cierre todas sus cuentas, descubra las pistas y
comience una identidad totalmente nueva. Internet tiene memoria
infinita: un solo golpe y quedará fuera de combate.

\chapter{Correo electrónico
anónimo}\label{correo-electruxf3nico-anuxf3nimo}

Cada paquete de datos que viaja a través de Internet contiene
información acerca de su emisor y su destinatario. Esto se aplica al
correo electrónico, así como a cualquier otra red de comunicación.
Existen varias maneras de reducir la información de identificación pero
no hay manera de eliminarla completamente.

\section{Envío de mensajes por medio de cuentas de correo electrónico
desechables}\label{envuxedo-de-mensajes-por-medio-de-cuentas-de-correo-electruxf3nico-desechables}

Una opción es utilizar una cuenta de correo electrónico desechable. Se
trata de una cuenta configurada en un servicio como Gmail o Hotmail,
usada una vez o dos veces para el intercambio anónimo. Al registrarse en
la cuenta, usted tendrá que proporcionar información falsa acerca de su
nombre y ubicación. Después de usar la cuenta durante un corto periodo
de tiempo, digamos 24 horas, nunca se debe volver a iniciar sesión. Si
necesita comunicarse posteriormente, cree una nueva cuenta.

Es muy importante tener en cuenta que estos servicios llevan un registro
de las direcciones IP de dónde las utilizan. Si desea enviar información
altamente sensible, usted tendrá que combinar el alta de una cuenta de
correo electrónico con Tor para mantener su dirección IP oculta.

Si usted no está esperando una respuesta, un repetidor de correo anónimo
como AnonEmail o Silentsender puede ser una solución útil. Un remailer
es un servidor que recibe mensajes con instrucciones sobre dónde enviar
los datos y actúa como intermediario, reenviándolo a partir de una
dirección genérica sin revelar la identidad del remitente original. Esto
funciona mejor cuando se combina con un proveedor de correo electrónico
como Hushmail o Riseup que están especialmente configurados para
conexiones seguras de correo electrónico.

Ambos métodos son útiles, pero sólo si usted recuerda siempre que el
intermediario sabe de dónde viene el mensaje original y puede leerlos
mensajes a medida que le llegan. A pesar de sus reclamos para proteger
su identidad, estos servicios suelen tener acuerdos de usuario que
indican su derecho ``a divulgar a terceros ciertos datos de registro
sobre usted'' o si sospechan que pueden estar en peligro por los
servicios secretos. La única forma de utilizar esta técnica en forma
segura es no confiar en estos servicios plenamente, y aplicar medidas de
seguridad adicionales: el envío a través de Tor utiliza una dirección de
correo electrónico desechable.

Si sólo necesita recibir correo electrónico, servicios como Mailinator
MintEmail y darle una dirección de correo electrónico que se
autodestruye después de unas pocas horas. Al registrarse en una cuenta,
usted debe proporcionar información falsa acerca de su nombre y la
ubicación y protegerse mediante el uso de Tor.

\section{¡Sea cuidadoso con lo que
dice!}\label{sea-cuidadoso-con-lo-que-dice}

El contenido de su mensaje puede revelar su identidad. Si menciona
detalles acerca de su vida, su geografía, relaciones sociales o
apariencia social, las personas pueden ser capaces de determinar quién
envió el mensaje. Cada palabra elegida y el estilo de escritura se
pueden usar para descubrir quién está detrás de los mensajes anónimos.

No debe usar el mismo nombre de usuario para diferentes cuentas o un
nombre que esté relacionado con usted tal como un apodo de su niñez o un
personaje de su libro favorito. Nunca use una cuenta secreta para una
comunicación personal habitual. Si alguien conoce sus secretos, no se
comunique con esta persona usando esta dirección de correo electrónico.
Si su vida depende de esto, cambie su cuenta secreta tan a menudo como
le sea posible entre distintos proveedores.

Finalmente, una vez que tenga su cuenta de correo electrónico totalmente
configurado para proteger su identidad, la vanidad es su peor enemigo.
Debe evitar ser distinto. No trate de ser inteligente, extravagante o
único. Incluso la forma de comenzar sus párrafos son datos valiosos para
la identificación, especialmente en estos días en que cada ensayo de la
escuela y la entrada del blog que ha escrito está disponible en la
Internet. Poderosas organizaciones pueden efectivamente utilizar estos
textos para construir una base de datos que pueda ``rastrearlo
digitalmente'' por lo que ha escrito.

\chapter{Compartir archivos}\label{compartir-archivos}

El término \emph{compartir archivos} se refiere a la práctica de
compartir archivos en la red, a menudo con la distribución más amplia
posible en mente. Desafortunadamente en los últimos años se ha asociado
popularmente con la distribución de contenido registrado bajo ciertas
licencias de derechos de autor que no permiten la distribución de copias
(por ejemplo supuesta actividad criminal). A pesar de esta nueva
asociación, el intercambio de archivos sigue siendo una herramienta
vital para todo el mundo: desde grupos académicos a las redes
científicas y las comunidades de software libre.

En este libro intentamos ayudarlo a que aprenda a distribuir archivos en
forma privada, con el consentimiento de algunas personas, sin que otras
puedan acceder al contenido que intercambia ni que la transacción sea
interceptada. Usted está protegido por su derecho básico al anonimato y
a no ser espiado. La sospecha de que los contenidos podrían haber sido
robados y no ser suyos no es razón suficiente para socavar su derecho a
la privacidad.

La historia de Internet está plagada de ataques de diferentes tipos de
nodos de publicación y distribución, realizadas por diferentes medios
(orden judicial, ataques de denegación de servicio). Lo que este tipo de
eventos han demostrado es que si uno quiere que la información esté
disponible en forma persistente y resistente contra los ataques, es un
error confiar en la neutralización de un único nodo.

Esto ha sido demostrado recientemente por la clausura del servicio de
descarga directa Megaupload, cuya desaparición provocó la pérdida de
grandes cantidades de datos de sus usuarios, en gran parte ajeno incluso
a las supuestas infracciones de copyright que sirvieron de pretexto para
su cierre. En la misma línea los ISPs suelen acabar con los sitios web
que contengan material dudoso simplemente porque les resulta más barato
hacerlo que acudir a los tribunales y que un juez decida. Estas
políticas dejan la puerta abierta a la intimidación por parte de todo
tipo de empresas, organizaciones e individuos listos y dispuestos a
hacer un uso agresivo de cartas legales. Tanto los servicios de descarga
directa como los ISP son ejemplos de estructuras centralizadas que no
pueden ser invocados porque son puntos débiles para el ataque, y debido
a que sus intereses comerciales no están alineados con los de sus
usuarios.

Difundir a través de los archivos de distribución, la descentralización
de los datos, es la mejor manera de defenderse contra estos ataques. En
la siguiente sección dos ámbitos de intercambio de archivos se perfilan.
El primero son las tecnologías estándar de p2p cuya técnica de diseño
está determinado por la eficiencia de las redes para permitir la
velocidad de distribución y descubrimiento de contenido a través de
mecanismos de búsqueda asociados. El segundo se centra en I2P como un
ejemplo de una darknet llamada, su diseño da prioridad a la seguridad y
el anonimato durante otros criterios que ofrecen una robusta, aunque
menos eficiente de los recursos, ruta de acceso a la disponibilidad
persistente.

Los medios de compartir archivos mencionados a continuación son sólo
algunos ejemplos de las muchas tecnologías P2P que se han desarrollado
desde 1999. BitTorrent y Soulseek tienen enfoques muy diferentes, sin
embargo ambos fueron diseñados para la facilidad de uso por un público
amplio y tienen importantes comunidades de usuarios. I2P, de más
reciente desarrollo, tiene una base de usuarios pequeña.

\textbf{BitTorrent} se ha convertido en el sistema P2P para compartir
archivos más popular. La controversia que lo rodea hoy en día
irónicamente ha ayudado a la comunidad a crecer, mientras que la
policía, impulsada por los poderosos dueños de los derechos de autor,
aprovechan los registros de los servidores para perseguir a sus
operadores, a veces hasta el punto de encarcelarlos como en el caso de
The Pirate Bay.

\textbf{Soulseek} - si bien nunca ha sido la más popular entre las
plataformas de intercambio de archivos, tampoco ese es su objetivo.
Soulseek se centra en el intercambio de música entre los simpatizantes,
productores independientes, aficionados e investigadores. El sistema y
la comunidad que lo rodea está completamente aislada de la web: no hay
enlaces externos a los archivos de Soulseek. Estos archivos se mantienen
exclusivamente en los discos duros de sus usuarios. El contenido de la
red depende totalmente de cuántos miembros están conectados y cuántos lo
comparten. Los archivos se transfieren sólo entre dos usuarios a la vez
y nadie más que esos dos usuarios se involucran. Debido a esta
``introvertido'' carácter - y la especificidad de su contenido -
Soulseek se ha mantenido fuera de la vista de los defensores de los
derechos de autor y la legislación anticopia.

\textbf{I2P} es uno de varios sistemas desarrollados para resistir la
censura (otros incluyen FreeNet y Tor) y cuenta con una comunidad de
usuarios mucho menor, se destaca aquí por su inclusión de la
funcionalidad de Bit Torrent en su instalación básica. Estos sistemas
también pueden ser utilizados para proporcionar servicios ocultos, entre
otros, lo que le permite publicar páginas Web en sus entornos..

\section{BitTorrent}\label{bittorrent}

BitTorrent es protocolo P2P que facilita la distribución de los datos
almacenados en varios nodos / participantes de la red. No hay servidores
centrales o concentradores, cada nodo es capaz de intercambiar datos con
cualquier otro nodo, a veces cientos de ellos al mismo tiempo. El hecho
de que los datos se intercambian en partes entre numerosos nodos permite
grandes velocidades de descarga de los contenidos más populares en las
redes BitTorrent, por lo que se ha convertido rápidamente en el estándar
de facto entre las plataformas de intercambio de archivos P2P.

Si utiliza BitTorrent para distribuir material de dudosa legalidad,
usted debe saber que las fuerzas de seguridad habitualmente recopilan
información sobre presuntos infractores participando en enjambres
torrent, observando y documentando el comportamiento de los otros pares.
El gran número de usuarios en constante aumento crea un problema para
aplicar este sistema - simplemente no tienen recursos suficientes para
perseguir a todos los usuarios. Cualquier caso judicial requerirá
evidencia real de transferencia de datos entre el cliente y otro par (y
por lo general la evidencia de la carga del archivo); sin embargo, es
suficiente que usted proporciona una parte del archivo, no el archivo en
su totalidad, para ser acusado. Si usted prefiere ser mas precavido,
debe utilizar una VPN para enrutar el tráfico de BitTorrent, como se
detalla en el capítulo \textbf{Uso de VPN}.

La descarga de un archivo de la red BitTorrent comienza con un archivo
torrent o con un enlace magnet. Un archivo torrent es un archivo pequeño
que contiene información sobre los archivos de mayor tamaño que desea
descargar. El archivo torrent le dice a su cliente de torrent los
nombres de los archivos que se comparten, una dirección URL para el
seguidor y un código hash, que es un código único derivado del archivo
subyacente al cual representa - algo así como una identificación o
número de catálogo. El cliente puede utilizar el hash para encontrar la
semillas de otros (para subir) esos archivos, así puede descargar desde
su computadora y comprobar la autenticidad de los fragmentos a medida
que llegan.

Un \emph{enlace magnet} elimina la necesidad de un archivo torrent y es
esencialmente un hipervínculo que contiene una descripción de ese
torrent que su cliente puede usar inmediatamente para empezar a buscar
personas que compartan el fichero que está dispuesto a descargar. Los
enlaces magnet no requieren de un tracker, sino que dependen de la
\emph{tabla hash distribuidas (DHT)} - se puede leer más en el Glosario
-- y en \emph{Mecanismo de intercambio}. Los enlaces magnet no hacen
referencia a un archivo por su ubicación (por ejemplo, mediante las
direcciones IP de las personas que tienen el archivo o URL) sino que
definen los parámetros de búsqueda que permiten encontrar el archivo.
Cuando se carga un enlace magnet, el cliente torrent inicia una búsqueda
de disponibilidad que se transmite a otros nodos y es básicamente una
nota -``¿quién tiene algo que coincida con el hash?''. El cliente
torrent se conecta a los nodos que respondieron a la nota de salida y
comienza a descargar el archivo.

BitTorrent utiliza cifrado para evitar que los proveedores y otros
man-in-the-middle bloqueen o espíen el tráfico basándose en el contenido
que usted intercambia. Ya que los enjambres de BitTorrent (rebaños de
semillas y leechers) son libres para que todo el mundo pueda unirse a
ellos es posible que alguien lo haga y recopile información acerca de
todos los pares conectados. El uso de enlaces magnet no evitará que lo
detecten entre la multitud, ya que todos los nodos que comparten el
mismo archivo deben comunicarse entre sí -y, por tanto, aunque sólo uno
de los nodos del enjambre sea un tramposo, será capaz de ver su
dirección IP. También será capaz de determinar si usted está sembrando
los datos mediante el envío de su nodo de una solicitud de descarga.

Un aspecto importante del uso de BitTorrent es digno de una mención
especial. Cada fragmento de datos que recibe (leecher) está siendo
compartida instantáneamente (sin semillas) con otros usuarios de
BitTorrent. Por lo tanto, un proceso de descarga se transforma en un
proceso (involuntario) de publicación, utilizando un término legal pone
a disposición los datos, antes de que la descarga se complete. Mientras
BitTorrent se utiliza a menudo para volver a distribuir el software
libremente disponible y legítimo, películas, música y otros materiales,
su capacidad de ``puesta a disposición'' ha creado mucha controversia y
dio lugar a un sinfín de batallas legales entre los titulares de
derechos de autor y los facilitadores de plataformas BitTorrent. En el
momento de escribir este texto, el co-fundador de The Pirate Bay
Gottfrid Svartholm se encuentra detenido por la policía sueca tras una
orden internacional dictada contra él.

Por estas razones, y por las campañas de relaciones públicas de los
titulares de los derechos de autor, el uso de las plataformas de
BitTorrent se ha convertido prácticamente en sinónimo de piratería. Y
aunque todavía no está claro el significado de términos como piratería,
derechos de autor y propiedad en el contexto digital, muchos usuarios
comunes de BitTorrent han sido perseguidos acusados de violar las leyes
de derechos de autor.

La mayoría de los clientes torrent le permiten bloquear direcciones IP
de los trolls conocidos de derechos de autor usando listas negras. En
lugar de usar torrents públicos también se puede unir a trackers de
BitTorrent cerrados o utilizarlos a través de VPN o Tor.

En las situaciones en las que usted cree que debería estar preocupado
por el tráfico de BitTorrent y su anonimato debería verificar lo
siguiente:

\begin{itemize}
\tightlist
\item
  Compruebe si su cliente soporta listas negras de pares.
\item
  Compruebe si las definiciones de las listas negras de pares se
  actualizan diariamente.
\item
  Asegúrese que su cliente soporte la totalidad de los protocolos más
  recientes - DHT, PEX y enlaces Magnet.
\item
  Elija un cliente torrent que soporte cifrado de pares y habilítelo.
\item
  Actualice o cambie su cliente torrent si algo de lo mencionado más
  arriba no está disponible.
\item
  Use una conexión VPN para ocultarle su tráfico BitTorrent a su ISP.
  Asegúrese que su proveedor de VPN permite el tráfico P2P. Vea más
  consejos y recomendaciones en el capítulo Uso de VPN.
\item
  No siembre ni guarde semillas si no sabe mucho acerca de ello.
\item
  Sospeche de los enlaces muy populares o con comentarios muy positivos.
\item
  Verifique si su cliente torrent soporta listas negras de pares.
\end{itemize}

\section{SoulSeek}\label{soulseek}

Como en el caso de los programas para compartir archivos entre pares
(P2P), los usuarios de Soulseek determinan el contenido disponible, y
cuáles archivos se pueden compartir. La red tuvo históricamente una
mezcla de música diversa, incluyendo artistas independientes y
alternativos, música inédita, tales como demos y cintas de las mezclas,
grabaciones piratas, etc. Está totalmente financiado por donaciones, sin
publicidad ni cobro de tarifas a usuarios.

\begin{quote}
Soulseek no avala ni aprueba el intercambio de materiales con copyright.
Sólo debe compartir y descargar archivos para los cuales usted está
legalmente autorizado, o ha recibido permiso." (consulte su
\href{http://www.soulseekqt.net\%5D}{página web})
\end{quote}

La red de Soulseek depende de un par de servidores centrales. Uno
soporta al cliente original y a la red, y el otro soporta a la red más
nueva. Mientras que estos servidores centrales son claves para coordinar
búsquedas y hospedar salas de chat, no juegan ningún rol en la
transferencia de archivos entre usuarios, que se desarrolla directamente
entre ellos.

Los usuarios pueden buscar por ítem, los resultados devueltos serán una
lista de los archivos cuyos nombres coinciden con el término de búsqueda
utilizado. Las búsquedas pueden ser explícitas o pueden utilizar
comodines/patrones o condiciones para ser excluidos. Una característica
específica al motor de búsqueda Soulseek es la inclusión de los nombres
de las carpetas y las rutas de archivo en la lista de búsqueda. Esto
permite a los usuarios buscar por nombre de carpeta.

La lista de resultados de búsqueda muestra los detalles, como el nombre
completo y la ruta del archivo, su tamaño, el usuario que aloja el
archivo, junto con la tasa promedio de transferencia de los usuarios y,
en el caso de archivos mp3, detalles breves acerca de la pista
codificada en sí, tales como la velocidad de bits, longitud, etc. La
lista de búsqueda resultante puede entonces ser ordenada en una variedad
de formas y archivos individuales (o carpetas) seleccionados para su
descarga.

A diferencia de BitTorrent, Soulseek no es compatible con la descarga
desde fuentes múltiples o ``swarming'' como otros clientes post-Napster,
y deben buscar un archivo solicitado desde una sola fuente.

Si bien el software Soulseek es libre, existe un sistema de donación
para apoyar el esfuerzo de programación y el costo de mantenimiento de
los servidores. A cambio de donaciones, a los usuarios se les concede el
privilegio de ser estar por delante de los usuarios que no hagan
donaciones en una cola de descarga de archivos (pero sólo si los
archivos no se comparten en una red de área local). Los algoritmos del
protocolo de búsqueda Soulseek no se publican, ya que esos algoritmos se
ejecutan en el servidor. Sin embargo, existen muchas implementaciones
libres de software para clientes y servidores en GNU/Linux, OS X y
Windows.

En cuanto a los temas de privacidad y copyright Soulseek está bastante
lejos de BitTorrent también. Soulseek ha sido llevado ante los
tribunales sólo una vez, en 2008, pero incluso eso no iba a ninguna
parte. No hay indicios de usuarios Soulseek que hayan sido llevados ante
la corte o acusados de distribución ilegal de material con copyright u
otros crímenes del `milenio digital'.

Si desea usar la red Soulseek con algún grado de anonimato, deberá
usarla sobre una VPN.

\section{I2P}\label{i2p}

I2P comenzó como una ramificación del proyecto Freenet, originalmente
concebida como un método de publicación y distribución resistente a la
censuran. Desde su sitio web:

\begin{quote}
El proyecto I2P se formó en el 2003 para apoyar los esfuerzos de
aquellos que tratan de construir una sociedad más libre, ofreciéndoles
un sistema de comunicación incensurable, anónimo y seguro. I2P es un
esfuerzo de desarrollo que produce una red de baja latencia, totalmente
distribuida, autónoma, escalable, anónima y resistente. El objetivo es
operar con éxito en entornos hostiles - incluso cuando una organización
con recursos financieros o políticos lo ataca. Todos los aspectos de la
red son de código abierto y están disponibles sin costo, ya que debe
asegurar a las personas que lo usan que el software hace lo que dice, al
igual que permite que otros puedan contribuir y mejorarlo para derrotar
los intentos agresivos que quieren sofocar la libertad de expresión.
(\url{http://www.i2p2.de/})
\end{quote}

Para una guía de instalación del software y la configuración de su
navegador web consulte la sección sobre Intercambio seguro de archivos -
Instalación de I2P. Una vez terminado, el lanzamiento lo llevará a una
página de la consola que contiene enlaces a otros sitios y servicios
populares. Además de las páginas web habituales (conocidas como
eePsites) hay una amplia gama de servicios de aplicaciones disponibles
desde la herramienta de blogging para Syndie construido en un cliente de
BitTorrent que funciona a través de una interfaz web.

\chapter{Llamadas seguras}\label{llamadas-seguras}

Las llamadas telefónicas hechas a través del sistema normal de
telecomunicaciones tienen algunas formas de protección contra la
intercepción de terceros, por ejemplo, los teléfonos móviles GSM cifran
las llamadas. Sin embargo, no están cifradas de extremo a extremo, y los
proveedores de telefonía están cada vez más obligados a dar a los
gobiernos y a las instituciones de la ley acceso a sus llamadas. Además,
el cifrado utilizado en GSM se ha roto y cualquier persona con interés y
un capital suficiente puede comprar el equipo para interceptar llamadas.
Un interceptor GSM
(\href{http://en.intercept.ws/catalog/2087.html}{http://en.interceptor.ws/catalog/2087.html})
es un dispositivo disponible para la plataforma para grabar
conversaciones de teléfono móvil cuando se encuentra en las proximidades
de la llamada. Los sistemas centralizados o propietarios como Skype
también cifran las llamadas, pero están construidos con puertas traseras
para que accedan los servicios secretos y los gobiernos con conocimiento
de sus propietarios (en el caso de Skype, Microsoft). Adicionalmente,
existe una una amplia clasificación de dispositivos llamados receptores
IMSI los cuales pueden recolectar más información acerca de los
teléfonos móviles, incluso el contenido de su comunicación.

Sin embargo, existen varias herramientas que usted puede utilizar para
asegurar su teléfono usando cifrado punto a punto.

\section{iOS - Instalando Signal}\label{ios---instalando-signal}

Los creadores de TextSecure proporcionan una herramienta FLOSS llamada
Signal.
\href{https://itunes,apple.com/us/app/signal-private-messenger/id874139669?mt=8}{(https://itunes.apple.com/us/app/signal-private-messenger/id874139669?mt=8}
Signal usa métodos de cifrado similares a SilentCircle pero provee su
servicio con herramientas FLOSS. Además, su GUI (interfaz gráfica de
usuario) es extremadamente fácil de usar. Signal detectará en forma
transparente si usted está hablando con un usuario de Signal y le
preguntará si desea establecer una ``llamada segura'' (con Signal) o una
``llamada insegura'' (sin cifrado punto a punto).

\section{Android - Instalando
RedPhone}\label{android---instalando-redphone}

También de los creadores de Signal, existe una herramienta FLOSS llamada
RedPhone.
\href{https://play.google.com/store/apps/details?id=org.thoughtcrime.redphone\&hl=en}{https://play.google.com/store/apps/details?id=org.thoughtcrime.redphone\&hi=en}
Nuevamente, RedPhone usa métodos de cifrado similares a SilentCircle
pero provee sus servicios usando herramientas FLOSS. También en este
caso, su GUI detectará en forma transparente si usted está hablando con
otros usuarios de Signal o RedPhone y le preguntará si desea establecer
una ``llamada segura'' (con RedPhone) o una ``llamada insegura'' (sin
cifrado punto a punto). Desafortunadamente, RedPhone requiere el
framework de Google Play sino no trabajará en dichos teléfonos
(Cyanogenmod u otras ROMs similares). Mensajería segura
=================

Los SMS son mensajes cortos enviados entre teléfonos móviles. El texto
se envía sin cifrar y pueden ser leídos y almacenados por los
proveedores de telefonía móvil y otras partes que tienen acceso a la
infraestructura de la red a la que está conectado. Para evitar que sus
mensajes sean interceptados usted tiene que utilizar cifrado punto a
punto en sus mensajes de texto.

\section{Android}\label{android}

\begin{itemize}
\tightlist
\item
  \textbf{TextSecure} - WhisperSystems provee un sistema de cifrado de
  SMS para Android llamado TextSecure, basado en la criptografía de
  clave pública que asegura que los mensajes se cifren desde la conexión
  y que también se almacenen en una base de datos cifrada en el
  dispositivo, sin embargo, para asegurar el cifrado de la conexión,
  ambas partes deben usar la aplicación, Es
  \href{https://github.com/WhisperSystems/TextSecure/}{Open Source} y
  está disponible a través de
  \href{https://play.google.com/store/apps/details?id=org.thoughtcrime.securesms\&hl=en}{PlayStore}
\end{itemize}

La tecnología detrás del cifrado (llamada //axolotl//) extiende el
protocolo OTR para que el mensaje pueda ser cifrado y enviado aunque no
estén online todas las partes intervinientes en la comunicación.

\chapter{Usando Thunderbird}\label{usando-thunderbird}

\begin{figure}[htbp]
\centering
\includegraphics{thunderbird.jpg}
\caption{Thunderbird}
\end{figure}

En las secciones siguientes vamos a usar el programa de correo
electrónico Thunderbird de Mozilla para mostrarle cómo configurar su
cliente para mayor seguridad. Al igual que el navegador Mozilla Firefox,
Thunderbird tiene muchas ventajas sobre sus contrapartes de seguridad
como Apple Mail y Outlook.

Thunderbird es un ``agente de usuario de correo'' (MUA). Esto es
diferente de web basadas en servicios de correo electrónico como Gmail
de Google. Usted debe instalar la aplicación Thunderbird en el equipo.
Thunderbird tiene una interfaz agradable y las características que le
permiten gestionar varios buzones, organizar los mensajes en carpetas, y
la búsqueda a través de correos con facilidad.

Thunderbird puede ser configurado para trabajar con su actual cuenta de
correo electrónico, ya sea un proveedor de servicios de Internet (como
Comcast) o un proveedor de correo electrónico basado en la web (como
Gmail).

Thunderbird presenta muchas ventajas sobre las interfaces web de correo
electrónico. Estas serán discutidos en el capítulo siguiente. La más
importante es que Thunderbird permite mucho mayor privacidad y
seguridad.

Esta sección proporciona información sobre cómo instalar Thunderbird en
Windows, Mac OS X y Ubuntu.

\section{Instalación de Thunderbird en
Windows}\label{instalaciuxf3n-de-thunderbird-en-windows}

La instalación de Thunderbird involucra dos pasos: primero, descargar el
software y luego ejecutar el programa de instalación.

\begin{enumerate}
\def\labelenumi{\arabic{enumi}.}
\tightlist
\item
  Visite la página de descarga de Thunderbird
  \url{http://www.mozillamessaging.com/en-US/thunderbird/}. Esta página
  detectará el sistema operativo de su computadora y el idioma,
  recomendándole la mejor versión disponible para su uso.
\end{enumerate}

\begin{figure}[htbp]
\centering
\includegraphics{thunderbird_inst_1.jpg}
\caption{Instalación de Thunderbird}
\end{figure}

Si desea usar Thunderbird en un idioma y/o sistema operativo diferente,
pulse \emph{Other Systems and Languages} en el lado derecho de la página
y elija la versión de su agrado.

\begin{enumerate}
\def\labelenumi{\arabic{enumi}.}
\setcounter{enumi}{1}
\tightlist
\item
  Haga click en el botón de descarga para grabar el programa de
  instalación en su computadora.
\end{enumerate}

\begin{figure}[htbp]
\centering
\includegraphics{thunderbird_inst_2.jpg}
\caption{Descarga de Thunderbird}
\end{figure}

Pulse el botón \textbf{Save} para grabar el archivo de configuración de
Thunderbird en su computadora.

\begin{enumerate}
\def\labelenumi{\arabic{enumi}.}
\setcounter{enumi}{2}
\tightlist
\item
  Cierre todas las aplicaciones en uso.
\item
  Busque el archivo de configuración (generalmente está en su carpeta de
  descargas o en el escritorio) y haga doble click para iniciar la
  instalación. La primer cosa que el instalador hará será mostrarle en
  pantalla la bienvenida del asistente de configuración \textbf{Welcome
  to the Mozilla Thunderbird Setup Wizard}.
\end{enumerate}

\begin{figure}[htbp]
\centering
\includegraphics{thunderbird_inst_3.jpg}
\caption{Comenzando la instalación de Thunderbird}
\end{figure}

Pulse el botón \textbf{Next} para comenzar la instalación. Si desea
detenerla, haga click en el botón \textbf{Cancel}.

\begin{enumerate}
\def\labelenumi{\arabic{enumi}.}
\setcounter{enumi}{4}
\tightlist
\item
  Lo próximo que verá es la pantalla de tipo de configuración,
  \textbf{Setup Type}. Para la mayoría de los usuarios la opción
  estándar es buena aunque no suficiente para todas sus necesidades. La
  opción de configuración personalizada se recomienda exclusivamente
  para usuarios con experiencia. Note que Thunderbird se instalará él
  mismo como su aplicación de correo por defecto. Si no desea que esto
  ocurra, desmarque la casilla de verificación etiquetada como
  \textbf{Use Thunderbird as my default mail application}.
\end{enumerate}

\begin{figure}[htbp]
\centering
\includegraphics{thunderbird_inst_4.jpg}
\caption{Configuración}
\end{figure}

Pulse el botón \textbf{Next} para continuar con la instalación.

\begin{enumerate}
\def\labelenumi{\arabic{enumi}.}
\setcounter{enumi}{5}
\tightlist
\item
  Después que Thunderbird ha sido instalado, pulse el botón
  \textbf{Finish} para cerrar el asistente de configuración.
\end{enumerate}

\begin{figure}[htbp]
\centering
\includegraphics{thunderbird_inst_5.jpg}
\caption{Finalizando la instalación}
\end{figure}

Si marca la opción \textbf{Launch Mozilla Thunderbird} en la casilla de
verificación, Thunderbird iniciará después de haber sido instalado.

\section{Instalación de Thunderbird en
Ubuntu}\label{instalaciuxf3n-de-thunderbird-en-ubuntu}

Existen dos procedimientos de instalación diferentes según la versión de
Ubuntu: uno para la versión 10.04 o posterior, y otra para las
anteriores. Describiremos ambas más abajo.

Thunderbird no se ejecutará sin las siguientes librerías o paquetes
instalados en su computadora:

\begin{itemize}
\tightlist
\item
  GTK+ 2.10 o superior
\item
  GLib 2.12 o superior
\item
  Pango 1.14 o superior
\item
  X.Org 1.0 o superior
\end{itemize}

Mozilla recomienda que un sistema GNU/Linux tenga también las siguientes
librerías o paquetes:

\begin{itemize}
\tightlist
\item
  NetworkManager 0.7 o superior
\item
  DBus 1.0 o superior
\item
  HAL 0.5.8 o superior
\item
  GNOME 2.16 o superior
\end{itemize}

\section{Instalación de Thunderbird en Ubuntu 12.04 o
posteriores}\label{instalaciuxf3n-de-thunderbird-en-ubuntu-12.04-o-posteriores}

Si está usando 12.04 o una versión posterior, la manera más sencilla de
instalar Thunderbird es mediante el Ubuntu Software Center.

\begin{enumerate}
\def\labelenumi{\arabic{enumi}.}
\tightlist
\item
  Tipee Software en la ventana de búsqueda Unity.
\end{enumerate}

\begin{figure}[htbp]
\centering
\includegraphics{thunderbird_inst_ubuntu_1.jpg}
\caption{Buscando Thunderbird}
\end{figure}

\begin{enumerate}
\def\labelenumi{\arabic{enumi}.}
\setcounter{enumi}{1}
\tightlist
\item
  Haga click en `Ubuntu Software Center'
\item
  Tipee ``Thunderbird'' en la caja de búsqueda y pulse Enter en su
  teclado. El Ubuntu Software Center encontrará a Thunderbird en su
  lista de software disponible.
\item
  Haga click en el botón \textbf{Install}. Si Thunderbird necesita
  alguna librería adicional, el Ubuntu Software Center le avisará y lo
  instalará junto con Thunderbird.
\end{enumerate}

Usted puede encontrar el acceso directo a Thunderbird en las opciones de
Internet dentro del menú de aplicaciones:

\begin{figure}[htbp]
\centering
\includegraphics{thunderbird_inst_ubuntu_2.jpg}
\caption{Menú de aplicaciones}
\end{figure}

\section{Instalación de Thunderbird en Mac OS
X}\label{instalaciuxf3n-de-thunderbird-en-mac-os-x}

Para instalar Thunderbird en su Mac, siga los siguientes pasos:

\begin{enumerate}
\def\labelenumi{\arabic{enumi}.}
\tightlist
\item
  Vaya a la página de descargas de Thunderbird
  \url{http://www.mozillamessaging.com/en-US/thunderbird/}. Esta página
  detectará el sistema operativo de su computadora y el idioma,
  recomendándole la mejor versión disponible para su uso.
\end{enumerate}

\begin{figure}[htbp]
\centering
\includegraphics{thunderbird_inst_mac_1.jpg}
\caption{Instalación de Thunderbird}
\end{figure}

\begin{enumerate}
\def\labelenumi{\arabic{enumi}.}
\setcounter{enumi}{1}
\tightlist
\item
  Descargue la imágen de disco Thunderbird. Cuando complete la descarga,
  la imagen de disco se abrirá automáticamente y se montará un nuevo
  volumen denominado \emph{Thunderbird}.
\end{enumerate}

Si el volumen no se monta automáticamente, abra la carpeta de descargas
y haga doble click en la imagen del disco para montarla. Aparecerá una
ventana de localización:

\begin{figure}[htbp]
\centering
\includegraphics{thunderbird_inst_mac_2.jpg}
\caption{Abriendo la imagen}
\end{figure}

\begin{enumerate}
\def\labelenumi{\arabic{enumi}.}
\setcounter{enumi}{2}
\tightlist
\item
  Arrastre el ícono de Thunderbird dentro de su carpeta de
  aplicaciones.¡Thunderbird ya está instalado!
\item
  Opcionalmente, arrastre el ícono de Thunderbird desde la carpeta de
  aplicaciones dentro del Dock. Seleccionar el ícono de Thunderbird
  desde el Dock le permitirá abrirlo rápidamente.
\end{enumerate}

\begin{figure}[htbp]
\centering
\includegraphics{thunderbird_inst_mac_3.jpg}
\caption{Abriendo Thunderbird}
\end{figure}

\textbf{Nota:} Cuando ejecute Thunderbird por primera vez, las versiones
más recientes de Mac OS X (10.5 o posterior) le avisarán que la
aplicación Thunderbird.app fue descargada desde Internet.

Si descargó Thunderbird desde el sitio web de Mozilla, haga click en el
botón \textbf{Open}.

\begin{figure}[htbp]
\centering
\includegraphics{thunderbird_inst_mac_4.jpg}
\caption{Abriendo Thunderbird}
\end{figure}

\section{Usando Thunderbird por primera
vez}\label{usando-thunderbird-por-primera-vez}

Al usar Thunderbird por primera vez será guiado a través de la
configuración de su cuenta de correo electrónico. Estos parámetros son
definidos por su proveedor de correo electrónico (su ISP o su proveedor
de servicios de correo electrónico basado en la web). El próximo
capítulo describe cómo configurar su cuenta con la máxima seguridad.

\chapter{Configuración de cuentas
seguras}\label{configuraciuxf3n-de-cuentas-seguras}

Existe una manera correcta (segura) de configurar su conexión con los
servidores de correo electrónico de su proveedor y una manera incorrecta
(insegura). El aspecto más importante de la seguridad en los correos
electrónicos es el tipo de conexión que establecerá con el servidor de
su proveedor.

Siempre que sea posible, debería conectarse usando los protocolos
\textbf{SSL} (Secure Socket Layer) y \textbf{TLS} (Transport Layer
Security). (\textbf{STARTTLS}, otra versión disponible cuando configura
una cuenta, es una variación de SSL/TLS.) Estos protocolos impiden que
su propio sistema (más allá de Thunderbird) y todos los puntos entre sus
sistema y el servidor de correo puedan ser interceptados y robadas sus
contraseñas.Además, también impiden que los la lectura del contenido de
sus mensajes.

Estos protocolos, sin embargo, sólo aseguran la conexión entre el
ordenador y el servidor de correo. No protegen el canal de información
en todo el camino hasta el destinatario del mensaje. Una vez que los
servidores de correo reenvían el mensaje para la entrega, el mensaje
puede ser interceptado y leído por los puntos intermedios entre el
servidor de correo y el destinatario.

Aquí es donde \textbf{PGP} (Pretty Good Privacy) entra, lo cual se
describe en el capítulo siguiente.

El primer paso para asegurar al correo electrónico es tener una conexión
segura entre su sistema y los servidores de correo. En este capítulo se
describe cómo configurar su cuenta de correo electrónico de la manera
correcta.

\section{Requisitos de
configuración}\label{requisitos-de-configuraciuxf3n}

Al configurar una cuenta, Thunderbird intenta determinar los parámetros
de conexión (de la cuenta de correo electrónico y los datos de la cuenta
que usted proporciona) con su proveedor de correo electrónico. Aunque
Thunderbird conoce los parámetros de conexión para muchos proveedores de
correo electrónico, no los conoce a todos. Si los parámetros no son
conocidos por Thunderbird, usted tendrá que proporcionar la siguiente
información para configurar su cuenta:

\begin{itemize}
\tightlist
\item
  \textbf{Su nombre de usuario}
\item
  \textbf{Su contraseña}
\item
  \textbf{Servidor entrante:} nombre (como \texttt{imap.example.com}),
  protocolo (POP o IMAP), el puerto (por defecto, 110), y el protocolo
  de seguridad
\item
  \textbf{Servidor saliente:} nombre (como \texttt{smtp.example.com}),
  el puerto (por defecto, 25), y el protocolo de seguridad
\end{itemize}

Debería haber recibido esta información de su proveedor de hosting.
Alternativamente, usted puede encontrar esta información en las páginas
de soporte en el sitio Web de su proveedor de hosting. En nuestro
ejemplo vamos a utilizar la configuración del servidor de Gmail. Puede
utilizar Thunderbird con su cuenta de Gmail. Para ello, es necesario
cambiar una opción de configuración de su cuenta. Si no está usando una
cuenta de Gmail, saltee la siguiente sección.

\section{Preparación de una cuenta de Gmail para usar con
Thunderbird}\label{preparaciuxf3n-de-una-cuenta-de-gmail-para-usar-con-thunderbird}

Accede a tu cuenta de Gmail en tu navegador. Seleccione
\textbf{Configuración} de opciones en la parte superior derecha, luego
vaya a la pestaña \textbf{Forwarding and POP/IMAP}. Pulse \textbf{Enable
IMAP} y luego \textbf{Save Changes}.

\begin{figure}[htbp]
\centering
\includegraphics{gmail_imap.png}
\caption{Habilitación de IMAP en Gmail}
\end{figure}

\section{Configurar Thunderbird para usar SSL /
TLS}\label{configurar-thunderbird-para-usar-ssl-tls}

Al iniciar Thunderbird por primera vez, se entra en un proceso de
configuración paso a paso para configurar su primera cuenta. (Se puede
invocar la interfaz de configuración de la cuenta en cualquier momento
seleccionando \textbf{File \textbar{} New \textbar{} Mail Account}). En
la primera pantalla, se le pedirá su nombre, su dirección de correo
electrónico y su contraseña. El valor que introduzca por nombre no tiene
por qué ser su nombre real. Se muestra al destinatario de sus mensajes.
Introduzca la información y haga click en \textbf{Continue}.

\begin{figure}[htbp]
\centering
\includegraphics{thunderbird_conf_1.png}
\caption{Configurando Thunderbird}
\end{figure}

En la siguiente pantalla, Thunderbird intentará determinar los nombres
de los servidores basados en su dirección de correo electrónico. Esto
puede llevar algún tiempo, y sólo funcionará si Thunderbird sabe la
configuración de los servidores de tu proveedor. En cualquier caso, se
le presentará una ventana donde se puede modificar la configuración. En
el siguiente ejemplo, Thunderbird ha detectado la configuración de forma
automática. Usted puede ver el protocolo en la parte derecha de los
nombres de servidor. Esto debe ser tanto \textbf{SSL/TLS} o
\textbf{STARTTLS}. \emph{De lo contrario su conexión es insegura y usted
debe tratar de configurarla manualmente.}

\begin{figure}[htbp]
\centering
\includegraphics{thunderbird_conf_2.png}
\caption{Configurando las cuentas}
\end{figure}

Cuando haya terminado, haga click en \textbf{Create account}. Si
Thunderbird no pudo determinar la configuración del servidor, haga click
en \textbf{Manual setup} para configurar los nombres de servidor usted
mismo.

\section{Configuración manual}\label{configuraciuxf3n-manual}

Utilice la interfaz para configurar manualmente las cuentas en
Thunderbird. El cuadro de diálogo de configuración de la cuenta se
abrirá automáticamente si se selecciona \textbf{Manual setup} en el
asistente de configuración. En este caso, sólo estamos interesados en
los nombres de servidores de correo entrante y saliente, y el protocolo
que se utiliza para conectarse con ellos. Como se puede ver en los
ejemplos siguientes, ingresamos los nombres de los servidores de Gmail y
los forzamos a utilizar \textbf{TLS/SSL}, un método seguro para
conectarse a los servidores.

\begin{figure}[htbp]
\centering
\includegraphics{thunderbird_conf_3.png}
\caption{Configurando la seguridad}
\end{figure}

En `Configuración del servidor', encontraremos sólo el servidor entrante
(\textbf{IMAP}) y la configuración de esa cuenta específica.

\begin{figure}[htbp]
\centering
\includegraphics{thunderbird_conf_4.png}
\caption{Configuración de correo entrante}
\end{figure}

Bajo \textbf{Server Name} introduzca el nombre del servidor IMAP, en
este caso \texttt{mail.gmail.com}.

\emph{Como usted puede ver, hemos seleccionado \textbf{`SSL/TLS'} en la
configuración de seguridad de conexión. o fuerza el cifrado.} No te
asustes por el método de autenticación \textbf{contraseña normal}. La
contraseña se cifra automáticamente debido a nuestras conexiones seguras
al servidor.

Por último, configure el servidor de correo saliente para la cuenta.
Haga click en \textbf{Outgoing Server (SMTP)} en el panel izquierdo.

\begin{figure}[htbp]
\centering
\includegraphics{thunderbird_conf_5.png}
\caption{Configuración de correo saliente}
\end{figure}

Una vez más, hemos seleccionado \textbf{SSL/TLS} en \textbf{Connection
security}. El puerto por defecto será 465 y generalmente no debería ser
cambiado.

\section{Finalizando la configuración, diferentes métodos de
cifrado}\label{finalizando-la-configuraciuxf3n-diferentes-muxe9todos-de-cifrado}

Pruebe la configuración de Thunderbird intentando enviar y recibir
mensajes. Algunos proveedores de almacenamiento de correo electrónico no
soportan el protocolo SSL/TLS, la opción favorita. Debería aparecer un
mensaje de error diciendo que el protocolo de autenticación no está
soportado por el servidor. Entonces, pruebe a utilizar STARTTLS. En las
dos pantallas de más abajo, seleccione ``STARTTLS'' en ``Connection
security''. Si este método también falla, póngase en contacto con su
proveedor de almacenamiento de correo electrónico y pregúntele ellos
ofrecen otra manera de conectarse de forma segura a sus servidores. Si
no le permiten hacerlo entonces usted debe quejarse y considerar
seriamente la posibilidad de cambiar a un proveedor diferente.

\section{De regreso a las pantallas de
configuración}\label{de-regreso-a-las-pantallas-de-configuraciuxf3n}

En cualquier momento usted puede reconfigurar sus cuentas de correo
yendo a la barra de menú de Thunderbird y pulsando \textbf{Edit
\textbar{} Account Settings} (GNU/Linux), \textbf{Tools \textbar{}
Account Settings} (Windows y Mac OS X).

\chapter{Parámetros adicionales de
seguridad}\label{paruxe1metros-adicionales-de-seguridad}

Thunderbird provee medidas de seguridad adicional para protegerlo del
correo basura, robo de identidad, virus (con la ayuda de su software
antivirus, por supuesto), robo de propiedad intelectual, y sitios web
maliciosos.

Veremos las siguientes características de seguridad de Thunderbird.
Primero un poco de historia sobre por qué debe tener en cuenta algunas
de estas medidas:

\begin{itemize}
\tightlist
\item
  \textbf{Controles de correo basura adaptables}. Le permiten entrenar a
  Thunderbird para que pueda identificar correo electrónico no deseado
  (SPAM) y eliminarlo de su bandeja de entrada. También puede marcar los
  mensajes como correo basura manualmente si el sistema de su proveedor
  de correo electrónico falla el correo basura y lo deja pasar.
\item
  \textbf{Integración con el software anti-virus.} Si su software
  antivirus es compatible con Thunderbird, puede utilizarlo para poner
  en cuarentena a los mensajes que contienen virus u otros contenidos
  maliciosos. Si usted se pregunta qué software antivirus trabaja con
  Thunderbird, puede encontrar una lista aquí:
  \url{http://kb.mozillazine.org/Antivirus_software}.
\item
  \textbf{Contraseña maestra.} Para su conveniencia, usted puede hacer
  que Thunderbird recuerde cada una de las contraseñas individuales de
  sus cuentas de correo electrónico. Puede especificar una contraseña
  maestra que se introduzca cada vez que inicie Thunderbird. Esto le
  permitirá a Thunderbird abrir todas tus cuentas de correo electrónico
  con sus contraseñas guardadas.
\item
  \textbf{Restricciones a las cookies.} Algunos blogs y sitios web
  intentan enviar cookies (una pieza de código que almacena información
  de los sitios web en su computadora) con sus feeds RSS. Estas cookies
  son utilizan con frecuencia por los proveedores de contenido para
  ofrecer publicidad dirigida. Thunderbird rechaza las cookies de forma
  predeterminada, pero se puede configurar para aceptar algunas o todas
  las cookies.
\end{itemize}

En la caja de diálogo Options/Preferences de la sección Security
Preferences puede establecer las preferencias para estas funciones.

\begin{itemize}
\tightlist
\item
  En Windows y Mac OS X, vaya al menú `Herramientas' y haga clic en
  `Opciones'.
\item
  En Ubuntu u otras versiones de Linux, vaya al menú `Editar' y haga
  clic en `Preferencias'.
\end{itemize}

\section{Configuración de correo
basura}\label{configuraciuxf3n-de-correo-basura}

\begin{enumerate}
\def\labelenumi{\arabic{enumi}.}
\tightlist
\item
  En la caja de diálogo de Preferences/Options, pulse `Security' y luego
  seleccione la pestaña `Junk'.
\end{enumerate}

\begin{figure}[htbp]
\centering
\includegraphics{thunderbird_sec_1.jpg}
\caption{Seguridad en Thunderbird}
\end{figure}

\begin{enumerate}
\def\labelenumi{\arabic{enumi}.}
\setcounter{enumi}{1}
\tightlist
\item
  Haga lo siguiente:

  \begin{itemize}
  \tightlist
  \item
    Dígale a Thunderbird que maneje los mensajes marcados como basura,
    seleccionando la casilla de verificación etiquetada como `When I
    mark message as junk'.
  \item
    Para que Thunderbird mueva estos mensajes a la carpeta de correo
    basura, seleccione el botón de opción `Move them to account's 'Junk'
    folder'.
  \item
    Para que Thunderbird borre el correo basura recibido, seleccione el
    botón de inicio `Delete them'.
  \end{itemize}
\item
  Thunderbird marcará los correos basura como leídos si usted selecciona
  la casilla de verificación etiquetada `Mark messages determined to be
  Junk as read'.
\item
  Si desea mantener un registro de correo basura recibido, seleccione la
  casilla de verificación `Enable junk filter logging'.
\item
  Haga click en el botón `OK' para cerrar la casilla de verificación
  `Options/Preferences'.
\end{enumerate}

\section{Alerta y detección de
estafas}\label{alerta-y-detecciuxf3n-de-estafas}

\begin{enumerate}
\def\labelenumi{\arabic{enumi}.}
\tightlist
\item
  En el cuadro de diálogos Preferences/Options, haga click en `Security'
  y luego en la pestaña `E-mail Scams'.
\end{enumerate}

\begin{figure}[htbp]
\centering
\includegraphics{thunderbird_sec_2.jpg}
\caption{Configurando la seguridad}
\end{figure}

\begin{enumerate}
\def\labelenumi{\arabic{enumi}.}
\setcounter{enumi}{1}
\item
  Para que Thunderbird le advierta sobre posibles estafas por correo
  electrónico, seleccione la casilla de verificación `Tell me if the
  message I'm read is a suspected email scam'. Para desactivar esta
  característica, desmárquela.
\item
  Pulse `OK' para cerrar el cuadro de diálogo `Options/Preferences'.
\end{enumerate}

\section{Integración con el
antivirus}\label{integraciuxf3n-con-el-antivirus}

\begin{enumerate}
\def\labelenumi{\arabic{enumi}.}
\tightlist
\item
  En el cuadro de diálogos Preferences/Options, haga click en `Security'
  y luego en la pestaña `Antivirus'.
\end{enumerate}

\begin{figure}[htbp]
\centering
\includegraphics{thunderbird_sec_3.jpg}
\caption{Integrando al antivirus}
\end{figure}

\begin{enumerate}
\def\labelenumi{\arabic{enumi}.}
\setcounter{enumi}{1}
\tightlist
\item
  Para activar la integración de antivirus, seleccione la casilla de
  verificación `Allow anti-virus clients to quarantine individual
  incoming messages'. Para desactivar esta característica, desmárquela.
\item
  Haga clic en el botón ``Aceptar'' para cerrar el cuadro de diálogo
  `Options/Preferences'.
\end{enumerate}

\section{Establezca una contraseña
maestra}\label{establezca-una-contraseuxf1a-maestra}

\begin{enumerate}
\def\labelenumi{\arabic{enumi}.}
\tightlist
\item
  En el cuadro de diálogos Preferences/Options, haga click en `Security'
  y luego en la pestaña `Passwords'.
\end{enumerate}

\begin{figure}[htbp]
\centering
\includegraphics{thunderbird_sec_4.jpg}
\caption{Contraseña maestra}
\end{figure}

\begin{enumerate}
\def\labelenumi{\arabic{enumi}.}
\setcounter{enumi}{1}
\tightlist
\item
  Seleccione la casilla de verificación `Use a master password'.
\item
  Ingrese su contraseña en los campos `Enter new password' y `Re-enter
  password'.
\end{enumerate}

\begin{figure}[htbp]
\centering
\includegraphics{thunderbird_sec_5.jpg}
\caption{Contraseña}
\end{figure}

\begin{enumerate}
\def\labelenumi{\arabic{enumi}.}
\setcounter{enumi}{3}
\tightlist
\item
  Haga click en el botón ``OK'' para cerrar el cuadro de diálogos Change
  Master Password.
\item
  Si desea ver las contraseñas que ha grabado en Thunderbird, haga click
  en el botón `Saved Passwords'. Esto abrirá el cuadro de diálogos
  `Saved Passwords'.
\end{enumerate}

\begin{figure}[htbp]
\centering
\includegraphics{thunderbird_sec_6.jpg}
\caption{Guardando contraseñas}
\end{figure}

\begin{enumerate}
\def\labelenumi{\arabic{enumi}.}
\setcounter{enumi}{5}
\tightlist
\item
  Para ver las contraseñas, haga click en el botón `Show Passwords'.
\end{enumerate}

\begin{figure}[htbp]
\centering
\includegraphics{thunderbird_sec_7.jpg}
\caption{Mostrando las contraseñas}
\end{figure}

\begin{enumerate}
\def\labelenumi{\arabic{enumi}.}
\setcounter{enumi}{6}
\tightlist
\item
  Haga click en el botón `Close' para cerrar el cuadro de diálogos
  `Saved Passwords'.
\item
  Haga click en el botón `OK' para cerrar el cuadro de diálogos
  `Options/Preferences'.
\end{enumerate}

\section{Controles adaptables para correo
basura}\label{controles-adaptables-para-correo-basura}

Necesita primero abrir la ventana de configuración de cuentas (Account
Settings). Tenga en cuenta que los ajustes configurados sólo se aplican
a la cuenta que haya seleccionado en el panel de carpetas. Debe
configurar las carpetas locales por separado.

\begin{enumerate}
\def\labelenumi{\arabic{enumi}.}
\tightlist
\item
  En el panel Carpetas haga clic derecho en un nombre de cuenta y
  seleccione ``Configuración''.
\end{enumerate}

\begin{figure}[htbp]
\centering
\includegraphics{thunderbird_sec_8.jpg}
\caption{Configuración anti spam}
\end{figure}

\begin{enumerate}
\def\labelenumi{\arabic{enumi}.}
\setcounter{enumi}{1}
\item
  En Windows o Mac, vaya al menú `Tools' y seleccione `Accounts
  settings'. En Linux, vaya `Edit menu' y seleccione `Account Settings'.
\item
  Para configurar los controles adaptables de correo basura para una
  cuenta específica, elija una cuenta y seleccione `Junk Settings'.
\end{enumerate}

\begin{figure}[htbp]
\centering
\includegraphics{thunderbird_sec_9.jpg}
\caption{Configurando controles adaptables}
\end{figure}

\begin{enumerate}
\def\labelenumi{\arabic{enumi}.}
\setcounter{enumi}{3}
\item
  Para activar los controles, seleccione la casilla de verificación
  ``Activar controles adaptables de correo basura para esta cuenta. Para
  desactivarlos, desmárquela.
\item
  Si desea que los controles de ignorar el correo de los remitentes en
  la libreta de direcciones, seleccione las casillas de verificación
  junto a cualquiera de las libretas de direcciones de la lista.
\item
  Para usar un filtro de correo como SpamAssassin o SpamPal, active la
  casilla de verificación denominada `Trust junk mail headers sent by:'
  y elija un filtro en el menú.
\item
  Seleccione la casilla de verificación denominada `Move new junk
  messages to' si desea mover el correo no deseado a una carpeta
  especificada. A continuación, seleccione la carpeta de destino para su
  proveedor de correo electrónico o una carpeta local en el equipo.
\item
  Seleccione la opción `Automatically delete junk mail other 14 days' en
  la casilla de verificación para que Thunderbird regularmente elimine
  los mensajes de correo basura. Para cambiar el período de tiempo para
  este proceso, introduzca un número diferente (en días) en el cuadro de
  texto.
\item
  Haga clic en `OK' para guardar los cambios.
\end{enumerate}

\chapter{Introducción al cifrado de correo electrónico
(PGP)}\label{introducciuxf3n-al-cifrado-de-correo-electruxf3nico-pgp}

\begin{figure}[htbp]
\centering
\includegraphics{pgp.jpg}
\caption{PGP}
\end{figure}

Este capítulo lo introducirá en algunos conceptos básicos acerca de
cifrado de correo electrónico. Es importante que lo lea para tener una
idea general de cómo funciona, cuáles son sus alcances y cuáles sus
limitaciones. \textbf{PGP} (Pretty Good Privacy) es el protocolo que se
utiliza para cifrar correo electrónico. Este protocolo nos permite
firmar digitalmente y cifrar mensajes. Funciona de extremo a extremo:
los mensajes se cifran en su propia computadora y sólo pueden ser
descifrados por el destinatario del mensaje. No hay posibilidad de que
un ``man-in-the-middle'' lo haga. Esto \emph{excluye} las líneas de
`asunto' y las direcciones `desde' y `hasta', que por desgracia no se
cifran en este protocolo.

Los siguientes capítulos le proporcionarán una guía práctica para
instalar las herramientas necesarias en su sistema operativo y poner en
marcha el cifrado. Nos centraremos en el uso de Enigmail que es una
extensión para Firefox que lo ayudará a administrar el cifrado PGP para
su correo electrónico. El proceso de instalación de Enigmail/PGP es
diferente para Mac OSX, Windows y Ubuntu así que por favor consulte los
capítulos correspondientes de esta sección para obtener instrucciones.

\begin{figure}[htbp]
\centering
\includegraphics{gpg-schema.jpg}
\caption{GPG Schema}
\end{figure}

\section{Uso de un par de claves para cifrar su correo
electrónico}\label{uso-de-un-par-de-claves-para-cifrar-su-correo-electruxf3nico}

Un concepto crucial en el cifrado es el llamado \emph{pares de claves}.
Un par de claves son dos archivos separados almacenados en su disco
rígido o memoria USB. Siempre que pretenda cifrar mensajes de una cierta
cuenta de correo, necesitará disponer de estos archivos de alguna
manera. Si los almacena en su computadora, no podrá descifrar mensajes
en su trabajo. Una solución posible sería llevarlo en una memoria USB.

Un par de claves consiste de dos claves diferentes: una clave pública y
una secreta.

La clave pública: usted puede pasarla a otra persona para que ellos
pueden enviarle a usted correos cifrados. Este archivo no se debe
mantener en secreto.

La clave secreta: es básicamente su archivo secreto para descifrar
mensajes que las personas le envían. No debe darse \emph{nunca} a nadie.

\section{Envío de mensajes cifrados a otras personas: usted necesita sus
clave
públicas}\label{envuxedo-de-mensajes-cifrados-a-otras-personas-usted-necesita-sus-clave-puxfablicas}

Si usted tiene compañeros de trabajo y desea enviarles mensajes cifrados
necesita la clave pública de cada uno de ellos. Se las podrían enviar
por correo electrónico, o podrían dársela en persona, o grabarla en una
memoria USB. No importa, siempre que pueda confiar en que esas llaves
pertenecen realmente a ellos. Su software pondrá las llaves en su
`llavero', por lo que su aplicación de correo sabrá cómo enviar mensajes
cifrados.

\section{Recepción de mensajes de correo electrónico de otras personas:
ellos necesitan su clave
pública}\label{recepciuxf3n-de-mensajes-de-correo-electruxf3nico-de-otras-personas-ellos-necesitan-su-clave-puxfablica}

Para que sus compañeros puedan enviarle a \emph{usted} mensajes
cifrados, debe distribuir su clave pública a cada uno de ellos.

\section{Conclusión: el cifrado de mensajes requiere la distribución de
las claves
públicas}\label{conclusiuxf3n-el-cifrado-de-mensajes-requiere-la-distribuciuxf3n-de-las-claves-puxfablicas}

Todas las personas de una red de amigos o colegas que esperan enviarse
unos a otros mensajes cifrados, necesitan distribuir sus claves públicas
unos a otros, mientras mantienen sus claves secretas guardadas en un
lugar seguro. El software descripto en este capítulo lo ayudará a
administrar sus claves.

\chapter{Instalación de PGP en
Windows}\label{instalaciuxf3n-de-pgp-en-windows}

Para complicar un poco las cosas, varios programas de software utilizan
el protocolo PGP para cifrar correo electrónico. Para trabajar con PGP
en Thunderbird necesitamos instalar GPG - una implementación libre de
PGP \emph{y} Enigmail - una extensión de Thunderbird que le permite
utilizar GPG \ldots{} ¿Confundido? No se preocupe por eso, todo lo que
tiene que saber es cómo cifrar su correo electrónico con PGP y para eso,
necesita instalar a \emph{ambos}, GPG y Enigmail. Ahora explicaremos
como hacerlo\ldots{}

\section{Instalación de PGP (GPG) en Microsoft
Windows}\label{instalaciuxf3n-de-pgp-gpg-en-microsoft-windows}

Para enviar mensajes cifrados con PGP o para firmarlos, necesitamos el
software GNU Privacy Guard (GnuPG). Es necesario instalarlo antes de
hacer cualquier tipo de cifrado.

Vaya a la página web del proyecto \href{http://gpg4win.org/}{Gpg4win}

En la parte izquierda de la página, encontrará el enlaces a la sección
`Descargar'. Haga click en él.

\begin{figure}[htbp]
\centering
\includegraphics{gpg_win.png}
\caption{Página de descarga}
\end{figure}

Esto le llevará a una página donde se puede descargar Gpg4Win. Haga clic
en el botón que le ofrece la última versión estable (no beta).

\begin{figure}[htbp]
\centering
\includegraphics{gpg_win_2.png}
\caption{Inicio de descarga}
\end{figure}

Esto descargará un archivo .exe. Dependiendo de su navegador, es posible
que tenga que hacer doble clic en el archivo descargado (que se llamará
algo así como \texttt{gpg4qin-2.1.0.exe}) antes de que algo suceda.
Windows le preguntará si está seguro de que desea instalar este
programa. Conteste sí.

Luego complete la instalación, aceptando la licencia, seleccionando el
idioma apropiado y aceptando las opciones por defecto haciendo clic en
`Next', a menos que tenga una razón para no hacerlo.

El programa de instalación le preguntará dónde colocar la aplicación en
su computadora. La configuración por defecto debería estar bien, pero
tome nota de ella porque es posible que la necesitemos más. Haga clic en
``Siguiente'' cuando esté de acuerdo.

\section{Instalación con la extensión
Enigmail}\label{instalaciuxf3n-con-la-extensiuxf3n-enigmail}

Después de haber instalado correctamente el software \textbf{PGP} como
hemos descrito anteriormente, ahora está listo para instalar el
complemento \textbf{Enigmail}.

Enigmail es un complemento de Firefox que le permite proteger la
privacidad de sus mensajes de correo electrónico. Enigmail es
simplemente una interfaz que le permite utilizar el cifrado PGP desde
dentro de Thunderbird.

Enigmail se basa en la criptografía de clave pública. En este método,
cada individuo debe generar su propio par de claves personales. La
primera clave se conoce como la clave privada. Debe estar protegida por
una contraseña o frase de acceso, guardada en un lugar secreto y nunca
debe compartirse con nadie.

La segunda clave es conocida como la clave pública. Esta clave puede ser
compartida con alguno de sus contactos. Una vez que tenga la clave
pública del destinatario puede comenzar a enviar mensajes de correo
electrónico cifrados a esta persona. Sólo ella será capaz de descifrar y
leer sus correos electrónicos, porque ella es la única persona que tiene
acceso a la clave privada coincidente.

Del mismo modo, si usted envía una copia de su clave pública propia a
sus contactos de correo electrónico y mantiene la correspondiente clave
privada en secreto, sólo usted podrá leer los mensajes cifrados de esos
contactos.

Enigmail también permite adjuntar firmas digitales a sus mensajes. El
destinatario del mensaje que tiene una copia original de su clave
pública podrá verificar que el correo electrónico proviene de usted, y
que su contenido no ha sido alterado en el camino. Del mismo modo, si
usted tiene la clave pública del destinatario, puede verificar las
firmas digitales en sus mensajes.

\section{Pasos para la instalación}\label{pasos-para-la-instalaciuxf3n}

Para empezar a instalar Enigmail, lleve a cabo los siguientes pasos:

\begin{enumerate}
\def\labelenumi{\arabic{enumi}.}
\item
  Abra \textbf{Thunderbird}, luego
  \texttt{Select\ tools\ \textgreater{}\ Add-ons} para activar la
  ventana de complementos, por defecto aparecerá habilitado el panel
  \emph{Get add-ons}.
\item
  Ingrese enigmail en la barra de búsqueda, como abajo, y haga click en
  el ícono de búsqueda.
\end{enumerate}

\begin{figure}[htbp]
\centering
\includegraphics{enigmail_inst_1.png}
\caption{Buscando Enigmail}
\end{figure}

\begin{enumerate}
\def\labelenumi{\arabic{enumi}.}
\setcounter{enumi}{2}
\item
  Simplemente haga click en el botón `Add to Thunderbird' para iniciar
  la instalación.
\item
  Thunderbird le preguntará si está seguro de que desea instalar este
  complemento. Confiamos en esta aplicación por lo que debemos hacer
  click en el botón `Install now'.
\end{enumerate}

\begin{figure}[htbp]
\centering
\includegraphics{enigmail_inst_2.png}
\caption{Instalando Enigmail}
\end{figure}

\begin{enumerate}
\def\labelenumi{\arabic{enumi}.}
\setcounter{enumi}{4}
\tightlist
\item
  Después de algún tiempo, la instalación se completará y la siguiente
  ventana debe aparecer. Por favor, haga click en el botón `Restart
  Thunderbird'.
\end{enumerate}

\begin{figure}[htbp]
\centering
\includegraphics{enigmail_inst_3.png}
\caption{Reiniciando}
\end{figure}

\chapter{Instalación de PGP en OSX}\label{instalaciuxf3n-de-pgp-en-osx}

GNU Privacy Guard (GnuPG) es un programa de software que le permite
enviar mensajes de correo electrónico cifrados con PGP o firmados. Es
necesario instalarlo previamente para poder realizar cualquier tipo de
cifrado. Este capítulo cubre los pasos requeridos para instalar GnuPG en
Mac OSX.

\section{Comenzando}\label{comenzando}

En este capítulo supondremos que usted tiene instalado la última versión
de:

\begin{itemize}
\tightlist
\item
  OSX (10.6.7)
\item
  Thunderbird (3.1.10)
\end{itemize}

\textbf{Nota acerca de OSX Mail:} Es posible usar PGP con el programa de
correo electrónico incorporado de OSX. Sin embargo, no recomendamos esta
opción se basa en un hack del programa que no es abierto ni está
soportado por su creador, por lo que se rompe con cada actualización del
programa de correo. Así que que realmente no tenemos otra opción que
recomendarle cambiar a Mozilla Thunderbird como su programa de correo
predeterminado si desea utilizar PGP.

\section{Descarga e instalación del
software}\label{descarga-e-instalaciuxf3n-del-software}

\begin{enumerate}
\def\labelenumi{\arabic{enumi}.}
\tightlist
\item
  Para OSX existe disponible un paquete que instalará todo lo necesario
  en un solo paso. Para obtenerlo, vaya a
  \href{http://www.gpgtools.org/}{gpgtools} y haga click en el gran
  disco azul con la inscripción ``Download GPGTools Installer'' debajo.
  Será redirigido a
  \href{http://www.gpgtools.org/installer/index.html}{otra página} donde
  podrá descargar el software.
\end{enumerate}

\emph{(aclaración. estamos usando la última versión de Firefox para este
manual, las pantallas pueden lucir algo diferentes si usted usa otro
navegador)}

\begin{figure}[htbp]
\centering
\includegraphics{gpg_mac_inst_1.jpg}
\caption{Instalando GPG}
\end{figure}

\begin{enumerate}
\def\labelenumi{\arabic{enumi}.}
\setcounter{enumi}{1}
\tightlist
\item
  Descargue el software seleccionando `Save File' y haciendo click en
  `OK' en el diálogo.
\end{enumerate}

\begin{figure}[htbp]
\centering
\includegraphics{gpg_mac_inst_2.jpg}
\caption{Descarga de GPG}
\end{figure}

\begin{enumerate}
\def\labelenumi{\arabic{enumi}.}
\setcounter{enumi}{2}
\tightlist
\item
  Navegue a la carpeta donde guarda habitualmente sus descargas
  (generalmente el escritorio o la carpeta de descargas) y haga doble
  click en el archivo `.DMG' para abrir el disco virtual que contiene el
  instalador.
\end{enumerate}

\begin{figure}[htbp]
\centering
\includegraphics{gpg_mac_inst_3.jpg}
\caption{Lanzando el instalador}
\end{figure}

\begin{enumerate}
\def\labelenumi{\arabic{enumi}.}
\setcounter{enumi}{3}
\tightlist
\item
  Abra el instalador con un doble click en el ícono.
\end{enumerate}

\begin{figure}[htbp]
\centering
\includegraphics{gpg_mac_inst_4.jpg}
\caption{Inicio de instalación}
\end{figure}

\begin{enumerate}
\def\labelenumi{\arabic{enumi}.}
\setcounter{enumi}{4}
\tightlist
\item
  El programa analizará su computadora para determinar si ésta puede
  ejecutarlo.
\end{enumerate}

(Observe que si su Mac fue construida antes del 2006 no tendrá el
procesador de Intel requerido para ejecutar este software y la
instalación fallará. Por desgracia, está más allá del alcance de este
manual tener también en cuenta a estos equipos de más de cinco años de
edad)

\begin{figure}[htbp]
\centering
\includegraphics{gpg_mac_inst_5.jpg}
\caption{Observación}
\end{figure}

Usted será guiado por el programa a través de los pasos siguientes para
aceptar el acuerdo de licencia. Presione todo los OK y deténgase al
llegar a la pantalla de `Tipo de instalación':

\begin{figure}[htbp]
\centering
\includegraphics{gpg_mac_inst_6.jpg}
\caption{Tipo de instalación}
\end{figure}

\begin{enumerate}
\def\labelenumi{\arabic{enumi}.}
\setcounter{enumi}{5}
\tightlist
\item
  Haga click en `Customize', se abrirá una pantalla donde habrá
  distintas opciones de programas y software para instalar. Haciendo
  click en cada uno de ellos tendrá una breve información acerca de qué
  es, qué hace y por qué puede necesitarlo.
\end{enumerate}

\begin{figure}[htbp]
\centering
\includegraphics{gpg_mac_inst_7.jpg}
\caption{Opciones de instalación}
\end{figure}

Como se dijo en la introducción; advertimos en contrario al uso de Apple
Mail en combinación con PGP. Por lo tanto, no va a necesitar `GPGMail',
ya que éste habilita PGP en el correo de Apple, y usted puede
desactivarlo.

`\textbf{Enigmail}' por otra parte es muy importante, ya que es el
componente que permitirá a Thunderbird utilizar PGP. En la captura de
pantalla aquí es gris vemos como el programa de instalación no pudo
identificar mi instalación de Thunderbird. Dicho así parece ser un
error. También puede instalar Enigmail desde dentro de Thunderbird como
se explica en otro capítulo.

Si la opción no aparece en gris en la instalación, debe funcionar.

Una vez comprobados todos los componentes que desea instalar, haga clic
en ``Install'' para continuar. El instalador le preguntará por su
contraseña y después de escribirla la instalación se ejecutará y
completará; ¡Hurra!

\begin{figure}[htbp]
\centering
\includegraphics{gpg_mac_inst_8.jpg}
\caption{Instalando}
\end{figure}

\section{Instalación de Enigmail}\label{instalaciuxf3n-de-enigmail}

\begin{enumerate}
\def\labelenumi{\arabic{enumi}.}
\item
  Abra \textbf{Thunderbird}, luego
  \texttt{Select\ Tools\ \textgreater{}\ Add-ons} para activar la
  ventana de los \emph{complementos}; aparecerá con el panel \emph{Get
  Add-ons} habilitado por defecto.
\item
  Después de abierta la ventana de complementos, busque `Enigmail' e
  instale la extensión haciendo click en `Add to Thunderbird \ldots{}'
\end{enumerate}

\begin{figure}[htbp]
\centering
\includegraphics{enigmail_mac_inst_1.jpg}
\caption{Buscando el complemento}
\end{figure}

\begin{enumerate}
\def\labelenumi{\arabic{enumi}.}
\setcounter{enumi}{2}
\tightlist
\item
  Haga click en `Install Now' para descargar e instalar la extensión.
\end{enumerate}

\begin{figure}[htbp]
\centering
\includegraphics{enigmail_mac_inst_2.jpg}
\caption{Instalando}
\end{figure}

\textbf{¡Tenga en cuenta que deberá reiniciar Thunderbird para usar la
funcionalidad de esta extensión!}

Ahora que ha descargado e instalado exitosamente Enigmail y PGP pase al
capítulo que trata acerca de cómo configurar el software para su uso.

\chapter{Instalación de PGP en
Ubuntu}\label{instalaciuxf3n-de-pgp-en-ubuntu}

Usaremos el Ubuntu Software Center para instalar PGP (Enigmail y
accesorios). Primero ábralo desde el menú Unity menu tipeando `software'
en el área de búsqueda

\begin{figure}[htbp]
\centering
\includegraphics{pgp_ubuntu_inst_1.png}
\caption{Buscando PGPl}
\end{figure}

Haga click en el `Ubuntu Software Center'.

Tipee `Enigmail' dentro del área de búsqueda, los resultados lo
devolverán automáticamente:

Resalte el ítem Enigmail item (debería estarlo por defecto) y haga click
en `Install', se le pedirá autenticar el proceso de instalación.

\begin{figure}[htbp]
\centering
\includegraphics{pgp_ubuntu_inst_2.png}
\caption{Instalando PGP}
\end{figure}

Ingrese su contraseña y pulse `Authenticate'. El proceso de instalación
comenzará.

Cuando el proceso se complete tendrá escasa respuesta desde Ubuntu. La
barra de progreso arriba a la izquierda simplemente desaparecerá. El
texto `In Progress' a la derecha también desaparecerá. Enigmail debería
estar instalado.

\chapter{Instalación de GPG en
Android}\label{instalaciuxf3n-de-gpg-en-android}

Con el uso creciente de teléfonos móviles para acceder al correo
electrónico, es interesante aprender a usar GPG también en su teléfono.
De esta manera, podrá leer sus mensajes en GPG no sólo en su
computadora.

Instale las aplicaciones \emph{Android Privacy Guard (APG)} y \emph{K-9
Mail} en su dispositivo Android desde Google Play Store u otra fuente
verificada.

\begin{enumerate}
\def\labelenumi{\arabic{enumi}.}
\tightlist
\item
  Genere una nueva clave privada que use DSA-Elgamal con la GPG
  instalada en su computadora (Sólo se pueden crear claves con una
  longitud máxima de 1024 bits en Android).
\item
  Copie la clave privada a su dispositivo Android.
\item
  Importe la clave privada a APG. Es posible que desee que APG elimine
  automáticamente la copia en texto plano de la clave privada del
  sistema de archivos de su dispositivo Android
\item
  Configure sus cuentas de correo electrónico en \emph{K-9 Mail}.
\item
  En la configuración de cada cuenta, en \emph{Cryptography}, asegúrese
  que K-9 Mail sabe cómo usar APG. También puede hacer que K-9 Mail
  firme automáticamente sus mensajes y/o los descifre si APG puede
  encontrar una clave pública para sus destinatarios.
\item
  Pruébelo.
\end{enumerate}

\section{APG}\label{apg}

Es una pequeña herramienta que hace posible el cifrado GPG en su
teléfono. Puede usar APG para administrar sus claves públicas y
privadas. Las opciones de la aplicación son bastante sencillas si tiene
algún conocimiento en GPG.

La administración de claves no está bien muy implementada aún. La mejor
manera es copiar manualmente sus claves públicas en una tarjeta SD en la
carpeta de APG. Entonces será muy sencillo importar sus claves. Después
que las haya importado, cifrado con GPG, firmado y descifrado estarán
disponibles para otras aplicaciones siempre y cuando estén integradas
con el cifrado/GPG.

\section{Cómo habilitar GPG en correos electrónicos en Android: K-9
Mail}\label{cuxf3mo-habilitar-gpg-en-correos-electruxf3nicos-en-android-k-9-mail}

La aplicación de correo no soporta GPG por defecto. Afortunadamente
existe una alternativa excelente: K-9 Mail. Esta aplicación está basada
en la aplicación original de Android pero con algunas mejoras. La
aplicación puede utilizar APG, ya que es proveedor de GPG. La
configuración de K-9 Mail es sencilla y similar a la configuración del
correo electrónico en la aplicación de Android por defecto. En el menú
de configuración hay una opción para habilitar ``Cryptography'' para la
firma del correo GPG.

Si desea tener acceso a su correo electrónico en su teléfono GPG esta
aplicación es una necesidad.

Por favor, note que debido a algunos errores pequeños en K-9 Mail y/o
APG, es muy recomendable deshabilitar el correo HTML y utilizar sólo
texto sin formato. Los mensajes del tipo HTML no están bien cifrados y a
menudo son ilegibles.

\chapter{Creación de sus claves
PGP}\label{creaciuxf3n-de-sus-claves-pgp}

Enigmail presenta un agradable asistente que le ayudará a crear su par
de claves pública/privada (vea el capítulo Introducción a PGP para una
explicación). Puede iniciar el asistente en cualquier momento dentro de
Thunderbird seleccionando
\texttt{OpenPGP\ \textgreater{}\ Setup\ Wizard} desde el menú superior.

\begin{enumerate}
\def\labelenumi{\arabic{enumi}.}
\tightlist
\item
  Así luce el asistente. Por favor, lea el texto en todas las ventanas
  con cuidado. Proporciona información útil y lo ayudará a configurar
  PGP de acuerdo con sus preferencias personales. En la primera
  pantalla, haga click en Next para iniciar la configuración.
\end{enumerate}

\begin{figure}[htbp]
\centering
\includegraphics{gpg_keys_1.png}
\caption{Inicio}
\end{figure}

\begin{enumerate}
\def\labelenumi{\arabic{enumi}.}
\setcounter{enumi}{1}
\tightlist
\item
  El asistente le preguntará si desea firmar todos los mensajes de
  correo salientes. La firma de todos los mensajes es una buena opción.
  Si usted no la elige, todavía puede decidir hacerlo de forma manual
  para firmar un mensaje cuando lo está redactando. Haga clic en el
  botón `Next' una vez que haya tomado una decisión.
\end{enumerate}

\begin{figure}[htbp]
\centering
\includegraphics{gpg_keys_2.png}
\caption{Opciones de firma}
\end{figure}

\begin{enumerate}
\def\labelenumi{\arabic{enumi}.}
\setcounter{enumi}{2}
\tightlist
\item
  En la siguiente pantalla, el asistente le preguntará si desea cifrar
  \emph{todos} los mensajes de correo salientes. A diferencia de la
  firma de correo electrónico, el cifrado requiere que el destinatario
  disponga de software de PGP instalado. Probablemente debería responder
  `no' a esta pregunta, por lo que se va a enviar normal (sin cifrar) de
  correo por defecto. Una vez que haya tomado su decisión, haga clic en
  el botón `Next'.
\end{enumerate}

\begin{figure}[htbp]
\centering
\includegraphics{gpg_keys_3.png}
\caption{Opciones de cifrado}
\end{figure}

\begin{enumerate}
\def\labelenumi{\arabic{enumi}.}
\setcounter{enumi}{3}
\tightlist
\item
  En la siguiente pantalla el asistente le preguntará si quiere cambiar
  algo en su configuración del formato del correo para trabajar mejor
  con PGP. Es una buena opción responder `Sí' aquí. Esto significa que,
  por defecto, el correo estará integrado en texto sin formato en lugar
  de HTML. Haga clic en el botón `Next' después de que usted haya tomado
  su decisión.
\end{enumerate}

\begin{figure}[htbp]
\centering
\includegraphics{gpg_keys_4.png}
\caption{Formato de correo}
\end{figure}

\begin{enumerate}
\def\labelenumi{\arabic{enumi}.}
\setcounter{enumi}{4}
\tightlist
\item
  En la siguiente pantalla, seleccione una de las cuentas de correo. En
  el cuadro de texto `Contraseña' debe introducir una. Se trata de un
  nuevo archivo \emph{contraseña} que se utiliza para proteger su clave
  privada. Es \textbf{muy importante} recordar esta contraseña, ya que
  no puede leer sus mensajes de correo electrónico cifrados propios en
  caso de olvido. Debe ser una contraseña \textbf{fuerte}, lo ideal es
  20 caracteres o más. Por favor, vea el capítulo sobre las contraseñas
  para obtener ayuda sobre la creación de contraseñas únicas, largas y
  fácil de recordar. Una vez seleccionada una su cuenta creada una
  contraseña, haga click en el botón ``Siguiente''.
\end{enumerate}

\begin{figure}[htbp]
\centering
\includegraphics{gpg_keys_5.png}
\caption{Selección de cuenta}
\end{figure}

\begin{enumerate}
\def\labelenumi{\arabic{enumi}.}
\setcounter{enumi}{5}
\tightlist
\item
  En la siguiente pantalla del asistente resume las acciones que tomará
  para habilitar el cifrado PGP para su cuenta. Si está satisfecho, haga
  clic en el botón ``Siguiente''.
\end{enumerate}

\begin{figure}[htbp]
\centering
\includegraphics{gpg_keys_6.png}
\caption{Resumen de acciones a tomar}
\end{figure}

\begin{enumerate}
\def\labelenumi{\arabic{enumi}.}
\setcounter{enumi}{6}
\tightlist
\item
  Sus claves serán creados por el asistente, que tomará algún tiempo.
  Cuando se haya completado, haga clic en el botón ``Siguiente''.
\end{enumerate}

\begin{figure}[htbp]
\centering
\includegraphics{gpg_keys_7.png}
\caption{Creación de claves}
\end{figure}

\begin{enumerate}
\def\labelenumi{\arabic{enumi}.}
\setcounter{enumi}{7}
\tightlist
\item
  Ahora tiene su propio par de claves PGP. El asistente le preguntará si
  también desea crear un ``certificado de revocación. Este es un archivo
  que se puede utilizar para informar a todo el mundo si la clave
  privada está en peligro, por ejemplo, si su portátil es robado. Piense
  en ello como un''kill switch" para su identidad PGP. Usted también
  puede desear revocar la clave, simplemente porque usted ha generado
  una nueva, y el viejo es obsoleto.
\end{enumerate}

\begin{figure}[htbp]
\centering
\includegraphics{gpg_keys_8.png}
\caption{Certificado de revocación}
\end{figure}

\begin{enumerate}
\def\labelenumi{\arabic{enumi}.}
\setcounter{enumi}{8}
\tightlist
\item
  Si ha decidido generar un certificado de revocación, el asistente le
  pedirá la ubicación del archivo debe ser guardado. El diálogo tendrá
  un aspecto diferente dependiendo del sistema operativo que utilice. Es
  una buena idea cambiar el nombre del archivo a algo sensato como
  \emph{my\_revocation\_certificate}. Haga clic en ``Guardar'' cuando
  usted haya decidido sobre un lugar.
\end{enumerate}

\begin{figure}[htbp]
\centering
\includegraphics{gpg_keys_9.png}
\caption{Guardando el certificado}
\end{figure}

\begin{enumerate}
\def\labelenumi{\arabic{enumi}.}
\setcounter{enumi}{9}
\tightlist
\item
  Si ha decidido generar un certificado de revocación, el asistente le
  informa de que se ha almacenado correctamente. Si lo desea, imprimirlo
  o grabarlo en un CD y guárdelo en un lugar seguro.
\end{enumerate}

\begin{figure}[htbp]
\centering
\includegraphics{gpg_keys_10.png}
\caption{Confirmación}
\end{figure}

\begin{enumerate}
\def\labelenumi{\arabic{enumi}.}
\setcounter{enumi}{10}
\tightlist
\item
  El asistente le informará de que ha completado.
\end{enumerate}

\begin{figure}[htbp]
\centering
\includegraphics{gpg_keys_11.png}
\caption{Finalizando}
\end{figure}

Felicitaciones, ahora tiene un cliente de correo electrónico totalmente
configurado con PGP. En el próximo capítulo vamos a explicar cómo
manejar sus llaves, mensajes de muestra y hacer la encriptación.
Thunderbird puede ayudarle a hacer un montón de estas cosas
automáticamente. Uso cotidiano de GPG ====================

En los capítulos previos hemos explicado como configurar un ambiente
seguro para el correo electrónico usando Thunderbird, GPG y Enigmail.
Asumiremos que ya tiene instalado el software mencionado y que ha
seguido exitosamente y paso a paso las instrucciones del asistente para
generar un par de claves de cifrado como se describió anteriormente.
Este capítulo explicará como usar Thunderbird en forma segura
cotidianamente para proteger sus comunicaciones por correo electrónico.
En particular, nos enfocaremos en:

\begin{enumerate}
\def\labelenumi{\arabic{enumi}.}
\tightlist
\item
  Cifrado de archivos adjuntos
\item
  Ingreso de una frase de paso
\item
  Recepción de mensajes cifrados
\item
  Envío y recepción de claves públicas
\item
  Recepción de claves públicas y agregado de las mismas a su anillo de
  claves
\item
  Uso de servidores de claves públicas
\item
  Firma de un mensaje en particular
\item
  Envío de mensajes cifrados a una destinatario en particular
\item
  Cifrado automático para destinatarios específicos
\item
  Verificación de mensajes entrantes
\item
  Revocación de su par de claves GPG
\item
  Qué hacer si pierde su clave secreta, u olvida se frase de paso
\item
  Qué hacer si robaron su clave secreta, o si la misma está comprometida
\item
  Copias de resguardo de sus claves
\end{enumerate}

Primero vamos a explicar dos ventanas de diálogo que inevitablemente
aparecen después de empezar a usar Thunderbird para cifrar sus correos
electrónicos.

\section{Cifrado de archivos
adjuntos}\label{cifrado-de-archivos-adjuntos}

La ventana de diálogo siguiente aparece cada vez que se envía un correo
electrónico con archivos adjuntos cifrados por primera vez. Thunderbird
hace una pregunta técnica sobre cómo cifrar los archivos adjuntos en el
correo. La opción predeterminada (la segunda) es la mejor opción, ya que
combina la seguridad con la máxima compatibilidad. También debe
seleccionar la opción `Use the selected method for all future
attachments'. A continuación, haga click en `Aceptar' y su correo será
enviado de inmediato.

\begin{figure}[htbp]
\centering
\includegraphics{daily_gpg_1.png}
\caption{Forma de cifrado}
\end{figure}

\section{Ingreso de una frase de
paso}\label{ingreso-de-una-frase-de-paso}

Por razones de seguridad, la frase de paso para su clave secreta se
almacena temporalmente en la memoria. De vez en cuando la ventana de
diálogo siguiente aparecerá. Thunderbird le pide la frase de paso para
su clave secreta. Esto debe ser diferente de su contraseña de correo
electrónico normal. Es la frase de paso que ha introducido al crear el
par de claves en el capítulo anterior. Introduzca la frase de paso en el
cuadro de texto y haga click en `OK'

\begin{figure}[htbp]
\centering
\includegraphics{daily_gpg_2.png}
\caption{Frase de paso}
\end{figure}

\section{Recepción de mensajes
cifrados}\label{recepciuxf3n-de-mensajes-cifrados}

El descifrado de mensajes de correo electrónico es manejado
automáticamente por Enigmail, la única acción que tendrá que hacer
eventualmente es introducir la frase de paso para su clave secreta. Sin
embargo, para mantener cualquier tipo de correspondencia cifrada con
alguien, primero tienen que intercambiar sus claves públicas.

\section{Envío y recepción de claves
públicas}\label{envuxedo-y-recepciuxf3n-de-claves-puxfablicas}

Existen varias formas de distribuir su clave pública a los amigos o
compañeros de trabajo. Con mucho, la forma más simple consiste en
adjuntar su clave a un correo electrónico. Para que su lista de amigos
pueda confiar en que el mensaje procede realmente de usted, debe
informarles en persona (si es posible) y también obligarles a que
respondan a su correo. Esto debería al menos evitar falsificaciones
fáciles. Usted tiene que decidir por sí mismo cuál es el nivel de la
validación necesario. Esto también es válido cuando se reciben mensajes
de correo electrónico de terceros que contienen las claves públicas.
Póngase en contacto con su interlocutor a través de algún medio de
comunicación alternativo. Puede utilizar un teléfono, mensajes de texto,
voz sobre protocolo de Internet (VoIP) o cualquier otro método, pero
debe estar absolutamente seguro de que usted está realmente hablando con
la persona correcta. Como resultado de ello, las conversaciones
telefónicas y reuniones cara a cara funcionan mejor, si ellas son
convenientes y si se pueden organizar de manera segura.

El envío de la clave pública es simple

\begin{enumerate}
\def\labelenumi{\arabic{enumi}.}
\item
  En Thunderbird, pulse el ícono \includegraphics{gpg_write.png}.
\item
  Envíe un mensaje a su amigo o colega y dígale que le ha enviado su
  clave PGP pública. Si sus amigos no saben qué significa esto, debe
  explicárselo y referirles alguna documentación.
\item
  Antes de enviar el correo, haga clic en la opción
  \texttt{OpenPGP\textgreater{}\ Adjuntar\ mi\ clave\ pública} en la
  barra de menús de la ventana de redacción de correo. Junto a esta
  opción aparecerá un signo marcado. Vea el siguiente ejemplo.
\end{enumerate}

\begin{figure}[htbp]
\centering
\includegraphics{daily_gpg_3.png}
\caption{Opción de envío}
\end{figure}

\begin{enumerate}
\def\labelenumi{\arabic{enumi}.}
\setcounter{enumi}{3}
\tightlist
\item
  Envíe su correo haciendo click en el botón
  \includegraphics{gpg_send.png}.
\end{enumerate}

\section{Recepción de claves públicas y agregado de las mismas a su
anillo de
claves}\label{recepciuxf3n-de-claves-puxfablicas-y-agregado-de-las-mismas-a-su-anillo-de-claves}

Supongamos que recibe una clave pública de un amigo por correo. La clave
se mostrará en Thunderbird como un \emph{archivo adjunto}. Desplácese
por el mensaje y por debajo verá las pestañas con uno o dos nombres de
archivo. La extensión de este archivo de clave pública será .asc, a
diferencia de la extensión de un archivo adjunto de firma GPG, que
termina en .asc.sig

Observe el ejemplo de correo electrónico en la imagen siguiente, que es
un mensaje GPG recibido firmado que contiene una clave pública adjunta.
Verá una barra amarilla con un mensaje de advertencia: `OpenPGP:
Unverified signature, click on 'Details' button for more information'.
Thunderbird nos advierte de que el remitente no se conoce todavía, lo
que es correcto. Esto va a cambiar una vez que aceptemos la clave
pública.

¿Qué están haciendo todos esos caracteres extraños en el mensaje de
correo? Debido a que Thunderbird aún no reconoce la firma como válida,
se imprime la firma cruda entera, al igual que lo que ha recibido. Así
es como aparecerán los mensajes GPG firmados digitalmente a todos
aquellos destinatarios que no tengan su clave pública.

Lo más importante en este caso es encontrar la clave pública GPG
adjunta. Hemos mencionado que es un archivo que termina en .asc. En este
ejemplo, es el primer archivo adjunto a la izquierda, en el círculo
rojo. Si hace doble clic sobre este archivo adjunto, Thunderbird
reconocerá la clave.

\begin{figure}[htbp]
\centering
\includegraphics{daily_gpg_4.png}
\caption{Clave GPG adjunta}
\end{figure}

Después de hacer click en el archivo adjunto, la siguiente ventana
aparecerá.

\begin{figure}[htbp]
\centering
\includegraphics{daily_gpg_5.png}
\caption{Ventana de confirmación}
\end{figure}

Thunderbird ha reconocido el archivo de clave pública GPG. Seleccione
`Import' para añadir esta clave a su anillo. La siguiente ventana
aparecerá. Thunderbird le indica que la operación ha sido exitosa. Pulse
`OK'.

\begin{figure}[htbp]
\centering
\includegraphics{daily_gpg_6.png}
\caption{Importación de clave pública}
\end{figure}

De vuelta en la pantalla principal de Thunderbird, actualizamos la vista
de este ejemplo de mensaje en concreto, haciendo clic en algún otro
mensaje. Ahora, el cuerpo del mensaje se ve diferente (véase más
adelante). Esta vez Thunderbird \emph{podrá} reconocer la firma, ya que
hemos añadido la clave pública del remitente.

\begin{figure}[htbp]
\centering
\includegraphics{daily_gpg_7.png}
\caption{Reconocimiento de la firma}
\end{figure}

Aún falta algo. Aunque Thunderbird reconoce ahora la firma, debemos
verificar explícitamente que la clave pública realmente pertenece al
remitente en la vida real. Nos damos cuenta de esto cuando echamos un
vistazo más de cerca a la barra verde (ver más abajo). Si bien la firma
es buena, todavía no es confiable.

\begin{figure}[htbp]
\centering
\includegraphics{daily_gpg_8.png}
\caption{Verificando la confianza}
\end{figure}

Si decide confiar en esta clave pública particular y las firmas hechas
por ella, haga click en `Details'. Un pequeño menú aparecerá (ver más
abajo). Desde este menú se debe hacer click en la opción `Sign Sender's
Key \ldots{}'.

\begin{figure}[htbp]
\centering
\includegraphics{daily_gpg_9.png}
\caption{Detalles}
\end{figure}

Después de elegir `Sign Sender's Key \ldots{}' aparecerá otra ventana de
selección (ver más abajo). Nos pedirá que indiquemos qué tan
cuidadosamente hemos seleccionado esta clave para su validez. La
explicación de los niveles de confianza y redes de confianza en GPG
queda fuera del alcance de este documento. No utilizaremos esta
información, por lo tanto, nos limitaremos a seleccionar la opción `I
will not answer' (``No voy a responder''). También seleccione la opción
`Local signature (cannot be exported)'. Haga click en el botón
``Aceptar'' para terminar de firmar esta clave. Esto completa la
aceptación de la clave pública. Ahora puede enviar correo cifrado a este
individuo.

\begin{figure}[htbp]
\centering
\includegraphics{daily_gpg_10.png}
\caption{Aceptación de la clave}
\end{figure}

\section{Uso de servidores de claves
públicas}\label{uso-de-servidores-de-claves-puxfablicas}

Otro método para distribuir claves públicas es colocarlas en un
servidor. Esto permite que cualquier persona pueda comprobar si su
dirección de correo electrónico soporta GPG, y luego descargar su clave
pública.

Para guardar su propia clave en un servidor, haga lo siguiente: i 1.
Diríjase hacia el administrador de claves utilizando el menú de
Thunderbird y haga click en
\texttt{OpenPGP\ \textgreater{}\ Key\ Management}

\begin{figure}[htbp]
\centering
\includegraphics{daily_gpg_11.png}
\caption{Administración de claves}
\end{figure}

\begin{enumerate}
\def\labelenumi{\arabic{enumi}.}
\setcounter{enumi}{1}
\tightlist
\item
  Aparecerá la siguiente ventana:
\end{enumerate}

\begin{figure}[htbp]
\centering
\includegraphics{daily_gpg_12.png}
\caption{Ventana de administración}
\end{figure}

\begin{enumerate}
\def\labelenumi{\arabic{enumi}.}
\setcounter{enumi}{2}
\tightlist
\item
  Seleccione ahora la opción `Display All Keys by Default' para acceder
  a la lista de todas sus claves. Busque su dirección de correo
  electrónico en la lista y haga click derecho. Aparecerá una ventana de
  selección con algunas opciones. Elija `Upload Public Keys to
  Keyserver'.
\end{enumerate}

\begin{figure}[htbp]
\centering
\includegraphics{daily_gpg_13.png}
\caption{Selección de opciones}
\end{figure}

\begin{enumerate}
\def\labelenumi{\arabic{enumi}.}
\setcounter{enumi}{3}
\tightlist
\item
  Ahora verá una pequeña ventana de diálogo. El servidor por defecto
  para distribuir sus claves es correcto. Presione `OK' y distribuya su
  clave pública por el mundo.
\end{enumerate}

\begin{figure}[htbp]
\centering
\includegraphics{daily_gpg_14.png}
\caption{Distribución de claves}
\end{figure}

Para saber si alguna dirección de correo electrónico posee una clave
pública disponible, siga los siguientes pasos:

\begin{enumerate}
\def\labelenumi{\arabic{enumi}.}
\item
  Diríjase al administrador de claves mediante el menú de Thunderbird y
  seleccione \texttt{OpenPGP\ \textgreater{}\ Key\ Management}
\item
  En la barra de menú de la ventana del administrador de claves,
  seleccione \texttt{Keyserver\ \textgreater{}\ Search\ for\ Keys}
\end{enumerate}

\begin{figure}[htbp]
\centering
\includegraphics{daily_gpg_15.png}
\caption{Búsqueda de claves}
\end{figure}

\begin{enumerate}
\def\labelenumi{\arabic{enumi}.}
\setcounter{enumi}{2}
\tightlist
\item
  En este ejemplo buscaremos l clave del creador del software PGP,
  Philip Zimmermann. Después de ingresar la dirección de correo
  electrónico, pulse `OK'.
\end{enumerate}

\begin{figure}[htbp]
\centering
\includegraphics{daily_gpg_16.png}
\caption{Buscando\ldots{}}
\end{figure}

\begin{enumerate}
\def\labelenumi{\arabic{enumi}.}
\setcounter{enumi}{3}
\tightlist
\item
  La ventana próxima mostrará el resultado de nuestra búsqueda. Nosotros
  hemos encontrado la clave pública. Se ha seleccionado automáticamente.
  Solo presione `OK' para importar la clave.
\end{enumerate}

\begin{figure}[htbp]
\centering
\includegraphics{daily_gpg_17.png}
\caption{Claves halladas}
\end{figure}

\begin{enumerate}
\def\labelenumi{\arabic{enumi}.}
\setcounter{enumi}{4}
\tightlist
\item
  Importar la clave tomará algo de tiempo. Al completarse, se debería
  mostrar una ventana como la siguiente:
\end{enumerate}

\begin{figure}[htbp]
\centering
\includegraphics{daily_gpg_18.png}
\caption{Clave importada}
\end{figure}

\begin{enumerate}
\def\labelenumi{\arabic{enumi}.}
\setcounter{enumi}{5}
\tightlist
\item
  El paso final es firmar localmente la clave, para indicar que
  confiamos en ella.Cuando esté de vuelta en el gestor de claves,
  asegúrese de que ha seleccionado la opción `Display All Keys by
  Default'. Ahora debería ver la clave recién importada en la lista.
  Haga click en la dirección y seleccione `Key Sign'.
\end{enumerate}

\begin{figure}[htbp]
\centering
\includegraphics{daily_gpg_19.png}
\caption{Firmando la clave localmente}
\end{figure}

\begin{enumerate}
\def\labelenumi{\arabic{enumi}.}
\setcounter{enumi}{6}
\tightlist
\item
  Seleccione `I will not answer' y `Local signature (cannot be
  exported)', luego pulse `OK'. Ya puede enviarle correo cifrado a
  Philip Zimmermann.
\end{enumerate}

\begin{figure}[htbp]
\centering
\includegraphics{daily_gpg_20.png}
\caption{Aceptando}
\end{figure}

\section{Firma de un mensaje en
particular}\label{firma-de-un-mensaje-en-particular}

Firmar digitalmente sus mensajes es la manera de probar al destinatario
que usted los ha enviado. Quienes reciban su clave pública serán capaces
de \emph{verificar} que su mensaje es auténtico. Sin embargo, tome nota
que firmar un mensaje hará que sea muy difícil (si no imposible) negar
que usted ha sido el autor del mensaje.

\begin{enumerate}
\def\labelenumi{\arabic{enumi}.}
\item
  Ofrezca a sus amigos su clave pública, usando los métodos descriptos
  anteriormente en este capítulo.
\item
  En Thunderbird, pulse en el ícono \emph{Write}.
\item
  Antes de enviar el mensaje, habilite la opción
  \texttt{OpenPGP\ \textgreater{}\ Sign\ Message} desde la barra de menú
  de la ventana de redacción del mensaje, si aún no está habilitado.
  Luego, pulse sobre la opción y aparecerá una firma marcada. Al hacer
  otro click, debería deshabilitarse el cifrado. Vea el ejemplo más
  abajo:
\end{enumerate}

\begin{figure}[htbp]
\centering
\includegraphics{daily_gpg_21.png}
\caption{Ejemplo}
\end{figure}

\begin{enumerate}
\def\labelenumi{\arabic{enumi}.}
\setcounter{enumi}{3}
\tightlist
\item
  Pulse el botón \emph{Send} y su mensaje firmado será enviado.
\end{enumerate}

\section{Envío de mensajes cifrados a una destinatario en
particular}\label{envuxedo-de-mensajes-cifrados-a-una-destinatario-en-particular}

\begin{enumerate}
\def\labelenumi{\arabic{enumi}.}
\item
  Intercambie previamente claves públicas con sus amigos y colegas como
  explicamos anteriormente en este capítulo.
\item
  En Thunderbird, presione el ícono \emph{Write}.
\item
  Redacte un mensaje a su amigo o colega, del cual haya recibido
  previamente su clave pública. \textbf{Recuerde que la línea del asunto
  del mensaje no será cifrada}, sólo se cifrará el cuerpo del mensaje y
  sus archivos adjuntos.
\item
  Antes de enviar el mensaje, habilite la opción
  \texttt{OpenPGP\ \textgreater{}\ Encrypt\ Message} en la barra de menú
  de la ventana de redacción del mensaje, si aún no está habilitada.
  Hecho esto, al pulsar sobre ella, aparecerá una firma marcada.
  Haciendo otro click debería deshabilitarse el cifrado. Observe el
  ejemplo más abajo.
\end{enumerate}

\begin{figure}[htbp]
\centering
\includegraphics{daily_gpg_22.png}
\caption{Cifrado del mensaje}
\end{figure}

\begin{enumerate}
\def\labelenumi{\arabic{enumi}.}
\setcounter{enumi}{4}
\tightlist
\item
  Presione el botón \emph{Send} para enviar su mensaje cifrado.
\end{enumerate}

\section{Cifrado automático para destinatarios
específicos}\label{cifrado-automuxe1tico-para-destinatarios-especuxedficos}

A menudo querrá asegurarse de que todos sus mensajes a un colega o amigo
estén firmados y cifrados. Esta es una buena práctica, porque es posible
que se olvide de habilitar el cifrado manualmente. Usted puede hacer
esto mediante la modificación de las normas por receptores. Para ello
accedamos al editor de reglas OpenPGP por destinatario.

Seleccione \texttt{OpenPGP\ \textgreater{}\ Preferences} desde la barra
de menú de Thunderbird.

\begin{figure}[htbp]
\centering
\includegraphics{daily_gpg_23.png}
\caption{Editando reglas}
\end{figure}

la ventana de preferencias aparecerá. Pulse `Display Expert Settings'.

\begin{figure}[htbp]
\centering
\includegraphics{daily_gpg_24.png}
\caption{Preferencias}
\end{figure}

Aparecerán nuevas pestañas en la ventana. Vaya a la pestaña `Key
Selection' y haga click en el botón etiquetado como `Edit Rules
\ldots{}'

\begin{figure}[htbp]
\centering
\includegraphics{daily_gpg_25.png}
\caption{Selección y edición}
\end{figure}

Ahora veremos el editor de reglas por (ver más abajo). Este editor puede
ser usado para especificar la forma en cómo los mensajes a ciertos
destinatarios son enviados. Ahora vamos a agregar una regla que diga que
queremos cifrar y firmar todos los mensajes de correo para
\texttt{maildemo@greenhost.nl}

Primero haga click en el botón `Add'.

\begin{figure}[htbp]
\centering
\includegraphics{daily_gpg_26.png}
\caption{agregando\ldots{}}
\end{figure}

Aparecerá la ventana para añadir una nueva regla.

Lo primero que deberíamos ingresar es la dirección de correo electrónico
del destinatario. En el ejemplo de más abajo, hemos ingresado
\texttt{maildemo@greenhost.nl}

\begin{figure}[htbp]
\centering
\includegraphics{daily_gpg_27.png}
\caption{Ingresando la dirección de correo}
\end{figure}

Ahora vamos a configurar los valores predeterminados de cifrado mediante
el uso de los menús desplegables. Para firmar seleccione `Always'. Para
cifrar seleccione `Always'.

\begin{figure}[htbp]
\centering
\includegraphics{daily_gpg_28.png}
\caption{Configurndo\ldots{}}
\end{figure}

Finalmente seleccionemos la \emph{clave pública} del destinatario, con
la cual cifraremos nuestro mensaje. No olvide este paso, es muy
importante, de otra forma su mensaje no será cifrado. Pulse el botón
etiquetado como `Select Key(s)\ldots{}'. La ventana de selección de
claves aparecerá. La clave más obvia se seleccionará por defecto. En el
ejemplo debajo, solo hay disponible una única clave pública. Podemos
seleccionar claves haciendo click sobre la pequeña casilla cercana a la
dirección. Luego, presionando `OK', cerramos todas las ventanas
relevantes y habremos terminado.

\begin{figure}[htbp]
\centering
\includegraphics{daily_gpg_29.png}
\caption{Finalizando\ldots{}.}
\end{figure}

\section{Verificación de mensajes
entrantes}\label{verificaciuxf3n-de-mensajes-entrantes}

El descifrado de sus mensajes entrantes será automático y transparente.
Pero es obvio que es muy importante que usted verifique que el mensaje
estaba cifrado y/o firmado. Esta información está disponible en la barra
especial sobre el cuerpo del mensaje.

Una firma válida será reconocida con una barra verde sobre el mensaje
tal como se muestra la imagen debajo:

\begin{figure}[htbp]
\centering
\includegraphics{daily_gpg_30.png}
\caption{Firma válida}
\end{figure}

El último ejemplo estaba firmado pero no cifrado. Si el mensaje ha sido
cifrado, lucirá algo así:

\begin{figure}[htbp]
\centering
\includegraphics{daily_gpg_31.png}
\caption{Mensaje cifrado}
\end{figure}

Cuando aparezca un mensaje que ha sido cifrado, pero sin firmar, puede
resultar una falsificación hecha por alguien. La barra de estado se
volverá gris, como en la imagen de abajo y le dirá que aunque el mensaje
ha sido enviado de manera segura (cifrada), el remitente puede ser otro
y no la persona detrás de la dirección de correo electrónico que se verá
en el campo `From'. La firma es necesaria para verificar el remitente
real del mensaje. Por supuesto, es perfectamente posible que usted haya
publicado su clave pública en Internet y permiten que las personas le
envíen mensajes de correo electrónico anónimos. Pero también es posible
que alguien está tratando de hacerse pasar por uno de sus amigos.

\begin{figure}[htbp]
\centering
\includegraphics{daily_gpg_32.png}
\caption{Mensaje cifrado sin firma}
\end{figure}

De forma similar, si usted recibe un mensaje firmado de alguien que
conozca y posee su clave pública, pero la barra de estado se ha vuelto
amarilla y muestra un mensaje de advertencia, es posible que alguien
esté intentando enviarle correos electrónicos falsos.

\begin{figure}[htbp]
\centering
\includegraphics{daily_gpg_33.png}
\caption{Advertencia}
\end{figure}

A veces las claves secretas son robadas o perdidas. El propietario de la
clave deberá informar a sus amigos y enviarles un certificado de
revocación (más explicación de esto en el siguiente párrafo). La
revocación significa que ya no confía en la clave antigua. El ladrón
podría probar suerte más tarde enviándole un mensaje de correo
electrónico firmado falsamente. La barra de estado ahora se verá así:

\begin{figure}[htbp]
\centering
\includegraphics{daily_gpg_34.png}
\caption{Barra de estado}
\end{figure}

Curiosamente Thunderbird en esta situación ¡seguirá mostrando una barra
de estado verde! Es importante tener en cuenta el contenido de la barra
de estado con el fin de entender los aspectos de cifrado de un mensaje.
GPG permite una gran seguridad y privacidad, pero sólo si está
familiarizado con su uso y conceptos. Preste atención a las advertencias
en la barra de estado.

\section{Revocación de su par de claves
GPG}\label{revocaciuxf3n-de-su-par-de-claves-gpg}

Su clave secreta ha sido robada por alguien. Su disco duro se rompió y
ha perdido todos sus datos. Si la clave se pierde, ya no se pueden
descifrar los mensajes. Si la clave ha sido robada, alguien puede
descifrar su comunicación. Es necesario hacer un nuevo juego de claves.
El proceso de creación de claves, utilizando el asistente OpenPGP en
Thunderbird, ha sido descrita en este manual. Pero primero debe decirle
al mundo que su clave pública vieja ya no tiene valor, e incluso es
peligroso su uso.

\section{Qué hacer si pierde su clave secreta, u olvida se frase de
paso}\label{quuxe9-hacer-si-pierde-su-clave-secreta-u-olvida-se-frase-de-paso}

Durante la creación de su par de claves, el asistente OpenPGP le ofreció
la posibilidad de crear un certificado de revocación. Este es un archivo
especial que usted envíe a los demás para advertirles que hay que
desactivar la clave. Si usted tiene una copia de este archivo, el envío
de la clave de revocación es simplemente enviar el archivo como un
archivo adjunto a todos sus amigos. Ya no puede enviar correos firmados
(obviamente, porque ha perdido su clave secreta). Eso no tiene
importancia. Envíelo como un correo normal. El certificado de revocación
sólo pudo haber sido creado por el propietario de la clave secreta y
prueba que él desea revocarla. Es por eso que normalmente debe
mantenerse oculto a los demás.

Si usted no tiene el certificado de revocación, no existe otra opción
que ponerse en contacto con sus amigos e informarles personalmente que
su llave se ha perdido y que ya no deberían confiar en ella.

\section{Qué hacer si robaron su clave secreta, o si la misma está
comprometida}\label{quuxe9-hacer-si-robaron-su-clave-secreta-o-si-la-misma-estuxe1-comprometida}

Si tiene razones para sospechar que su clave secreta ha sido
comprometida, o peor, su clave y contraseña, es muy importante ponerse
en contacto con los demás para decirles que dejen de enviarle mensajes
cifrados. Con su clave privada, otras personas serán capaces de romper
el cifrado de los mensajes de correo electrónico si también tienen su
frase de contraseña. Esto también es cierto para aquellos mensajes que
ha enviado en el pasado. Descifrar la frase de paso no es sencillo, pero
puede ser posible si la persona tiene muchos recursos, como un estado o
una gran organización, por ejemplo, o si su contraseña es demasiado
débil. En cualquier caso, debe asumir lo peor y asumir que la frase de
contraseña puede haber sido comprometida. Envíe un archivo de revocación
de certificados a todos tus amigos o póngase en contacto con ellos
personalmente para informarles de la situación.

Incluso después de haber revocado su par de claves viejas, la clave
robado todavía se puede utilizar para descifrar su correspondencia
anterior. Usted debe considerar otras maneras de proteger la
correspondencia antigua, por ejemplo, volver a cifrarla con una clave
nueva. La última operación no se discutirá en este manual. Si no está
seguro de cómo hacerlo, debe buscar la ayuda de expertos o más
información en la web.

\section{Recepción de un mensaje de
revocación}\label{recepciuxf3n-de-un-mensaje-de-revocaciuxf3n}

Si uno de sus amigos le envía a usted un certificado de revocación, le
está pidiendo que desconfíe de su clave pública a partir de ahora. Usted
siempre debe aceptar la solicitud y debe `importar' el certificado para
desactivar su clave. El proceso de aceptación de un certificado de
revocación es exactamente el mismo que aceptar una clave pública, como
ya se ha descrito en el capítulo. Thunderbird le preguntará si desea
importar el archivo `OpenPGP key'. Una vez que lo ha hecho, una ventana
emergente de confirmación similar a la siguiente deberá aparecer.

\begin{figure}[htbp]
\centering
\includegraphics{daily_gpg_35.png}
\caption{Aceptación de una revocación}
\end{figure}

\section{Preparándose para lo peor: copias de resguardo de sus
claves}\label{preparuxe1ndose-para-lo-peor-copias-de-resguardo-de-sus-claves}

Sus claves son almacenados en el disco rígido como archivos normales.
Pueden perderse si el equipo se daña. Se recomienda encarecidamente
mantener una copia de seguridad de sus claves en un lugar seguro, como
una caja fuerte. Hacer una copia de seguridad de su clave secreta tiene
otra ventaja de seguridad también. Cada vez que usted tema que su
computadora se encuentra en peligro inmediato de ser confiscada, puede
eliminar el par de claves. Su correo electrónico será ilegible
inmediatamente. En una etapa posterior, puede recuperar las claves de la
bóveda y volver a importarlas en Thunderbird.

Para realizar una copia de seguridad de su par de claves, diríjase al
administrador de claves utilizando el menú de Thunderbird y haga clic en
\texttt{OpenPGP\ \textgreater{}\ Key\ Management}.

Es necesario haber seleccionado la opción `Display All Keys by Default'
para obtener una lista de todas sus claves. Busque su propia dirección
de correo en la lista y haga click con el botón derecho sobre ella. Una
ventana de selección aparecerá con algunas opciones. Seleccione la
opción `Export Keys to File'.

\begin{figure}[htbp]
\centering
\includegraphics{daily_gpg_36.png}
\caption{Copia de seguridad}
\end{figure}

Ahora deberá grabar el par de claves en un archivo. Thunderbird le
preguntará si desea incluir la clave secreta. Si así lo desea,
seleccione `Export Secret Keys'.

\begin{figure}[htbp]
\centering
\includegraphics{daily_gpg_37.png}
\caption{Exportando las claves}
\end{figure}

Finalmente Thunderbird le preguntará dónde almacenar el archivo de
claves. Puede hacerlo donde más lo desee, disco de red, memoria USB,
etc. Solo recuerde ocultarla de otras personas.

\section{Lecturas adicionales}\label{lecturas-adicionales}

Más documentación referida al uso de GPG con Thunderbird puede ser
encontrada en el sitio web del plugin Enigmail. Consulte el manual de
Enigmail

\url{http://enigmail.mozdev.org/documentation/handbook.php.html} Webmail
y PGP ===============

La única forma segura de cifrar el correo electrónico dentro de la
ventana del navegador es cifrar el texto fuera de la ventana y luego
copiarlo y pegarlo adentro.

Por ejemplo, escriba el texto en un editor de texto como gedit, kate o
vim y guárdelo con extensión .txt (en este ejemplo ``mensaje.txt''). A
continuación, escriba

\begin{verbatim}
gpg -ase -r <dirección de email/gpg id> -r <gpg id> mensaje.txt
\end{verbatim}

Un nuevo archivo llamado ``mensaje.asc'' se creará. Contiene el mensaje
cifrado y por lo tanto ya puede adjuntarse en un correo electrónico o se
puede copiar y pegar el contenido de forma segura en la ventana del
navegador.

Para descifrar un mensaje desde la ventana del navegador, basta con
escribir \texttt{gpg} en la línea de comandos y pulsar Enter. A
continuación, copie y pegue el mensaje a descifrar en la ventana de
línea de comandos y, después que le pregunten su contraseña, pulse
Ctrl+D (esto añade un carácter de fin de archivo y le solicita a gpg el
mensaje de texto descifrado).

Si utilizar la línea de comandos le parece demasiado complicado, podría
considerar la posibilidad de instalar una aplicación de ayuda como
gpgApplet, kgpg o cualquier otra aplicación que posea su sistema
operativo. ¿Por qué usar Firefox? ======================

Firefox es software de código abierto desarrollado por una organización
sin fines de lucro, la Fundación Mozilla. Por eso, es independiente de
los intereses de cualquier empresa aunque un
\href{https://en.wikipedia.org/wiki/Mozilla_Foundation\#Financing}{gran
porcentaje de su financiación proviene de Google} para poder colocar a
su motor de búsqueda como la opción por defecto dentro del navegador web
Firefox. Además, es altamente extensible a través de sus complementos y
plugins, que le permiten al usuario mantener un mayor control acerca de
cómo actúa el navegador comparado a Internet Explorer o Chrome (y a su
versión de código abierto, Chromium). Sin embargo, debe señalarse que
esta extensibilidad a través de sus complementos es un arma de doble
filo ya que dichos complementos pueden subvertir el normal
funcionamiento del navegador además de mejorarlo.

Si no está cómodo con Google como su motor de búsqueda por defecto,
puede ser cambiado por medio de la opción `Manage Search
Engines\ldots{}' del menú desplegable de la caja de búsqueda. Algunos de
los motores de búsqueda pro-privacidad más recomendables son
\href{https://www.startpage.com/}{Startpage} y
\href{https://duckduckgo.com/}{DuckDuckGo}.

\chapter{Accediendo a Firefox en
Ubuntu}\label{accediendo-a-firefox-en-ubuntu}

Firefox viene instalado en Ubuntu por defecto. Para abrirlo, haga click
en el ícono de Firefox en la barra lateral de Unity:

\begin{figure}[htbp]
\centering
\includegraphics{ff_ubuntu_1.png}
\caption{Abriendo Firefox}
\end{figure}

Firefox comienza abriendo una ventana de bienvenida:

\begin{figure}[htbp]
\centering
\includegraphics{ff_ubuntu_2.png}
\caption{Bienvenida}
\end{figure}

\chapter{Instalación en Mac OS X}\label{instalaciuxf3n-en-mac-os-x}

\begin{enumerate}
\def\labelenumi{\arabic{enumi}.}
\tightlist
\item
  Para descargar Firefox, visite su
  \url{https://www.mozilla.org/firefox} y haga click en el botón verde
  etiquetado como ``Firefox Free Download''. La descarga debería
  comenzar automáticamente, si no es así, haga click en el enlace para
  descargarlo manualmente.
\end{enumerate}

\begin{figure}[htbp]
\centering
\includegraphics{ff_mac_inst_1.png}
\caption{Descargando Firefox}
\end{figure}

\begin{enumerate}
\def\labelenumi{\arabic{enumi}.}
\setcounter{enumi}{1}
\tightlist
\item
  Cuando se lo pidan, haga click en \textbf{OK}.
\end{enumerate}

\begin{figure}[htbp]
\centering
\includegraphics{ff_mac_inst_2.png}
\caption{Inicio de descarga}
\end{figure}

Cuando se complete la descarga aparecerá una ventana similar a la
siguiente:

\begin{figure}[htbp]
\centering
\includegraphics{ff_mac_inst_3.png}
\caption{Finalizando la descarga}
\end{figure}

\begin{enumerate}
\def\labelenumi{\arabic{enumi}.}
\setcounter{enumi}{2}
\item
  Haga click y arrastre el ícono de \textbf{Firefox} sobre la parte
  superior del ícono de \textbf{Applications}.
\item
  Cuando la instalación haya terminado, cierre las dos ventanas pequeñas
  de Firefox.
\item
  Elimine la imagen de disco de Firefox. Si esto no funciona de forma
  normal, seleccione el ícono de imagen de disco y luego, en el menú
  Finder, seleccione \texttt{File\ \textgreater{}\ Eject\ Firefox}.
\item
  Ahora, abra el directorio \textbf{Applications} y arrastre el ícono de
  \textbf{Firefox} al dock:
\end{enumerate}

\begin{figure}[htbp]
\centering
\includegraphics{ff_mac_inst_4.png}
\caption{Poniendo el ícono en el dock}
\end{figure}

\begin{enumerate}
\def\labelenumi{\arabic{enumi}.}
\setcounter{enumi}{6}
\tightlist
\item
  Haga click en el ícono de \textbf{Firefox} en el Dock para ejecutarlo.
  Aparecerá la casilla de diálogos del asistente de importación:
\end{enumerate}

\begin{figure}[htbp]
\centering
\includegraphics{ff_mac_inst_5.png}
\caption{Asistente de importación}
\end{figure}

\begin{enumerate}
\def\labelenumi{\arabic{enumi}.}
\setcounter{enumi}{7}
\tightlist
\item
  Para importar sus marcadores, contraseñas y otros datos de Safari,
  haga click en \textbf{Continue}. Si no desea importar nada, solo
  seleccione \textbf{Cancel}.
\end{enumerate}

Felicitaciones, ¡ya está preparado para usar Firefox!

\begin{figure}[htbp]
\centering
\includegraphics{ff_mac_inst_8.png}
\caption{Instalación finalizada}
\end{figure}

\chapter{Instalación de Firefox en
Windows}\label{instalaciuxf3n-de-firefox-en-windows}

\begin{enumerate}
\def\labelenumi{\arabic{enumi}.}
\tightlist
\item
  Para descargar Firefox, visite \url{https://www.mozilla.com/firefox/}.
\end{enumerate}

\begin{figure}[htbp]
\centering
\includegraphics{ff_win_inst_1.png}
\caption{Descarga de Firefox}
\end{figure}

\begin{enumerate}
\def\labelenumi{\arabic{enumi}.}
\setcounter{enumi}{1}
\item
  Haga click en el botón de descarga y el archivo de instalación se
  descargará en su computadora.
\item
  Una vez completada la descarga, haga doble click en el archivo de
  instalación para iniciar el asistente.

  \begin{itemize}
  \tightlist
  \item
    Si está ejecutando Windows Vista, debería tener acceso al control de
    cuentas de usuario. En este caso, permita que se ejecute la
    configuración pulsando \textbf{Continue}.
  \item
    Si está ejecutando Windows 7, se le preguntará si le permite a
    Firefox realizar cambios en su computadora. Haga click en
    \textbf{Yes}.
  \end{itemize}

  Aparecerá una pantalla de bienvenida.
\item
  Pulse \textbf{Next} para continuar. Se le preguntará si desea la
  instalación estándar, o si quieres personalizarla. Elija la
  instalación estándar y haga click en \textbf{Next}.
\end{enumerate}

\begin{figure}[htbp]
\centering
\includegraphics{ff_win_inst_2.png}
\caption{Instalación de Firefox}
\end{figure}

\begin{enumerate}
\def\labelenumi{\arabic{enumi}.}
\setcounter{enumi}{4}
\tightlist
\item
  Se le preguntará si desea que Firefox sea su navegador por defecto. Se
  recomienda que sí lo sea.
\end{enumerate}

\begin{figure}[htbp]
\centering
\includegraphics{ff_win_inst_3.png}
\caption{Windows Firefox Install}
\end{figure}

\begin{enumerate}
\def\labelenumi{\arabic{enumi}.}
\setcounter{enumi}{5}
\item
  Haga click en \textbf{Install}.
\item
  Para importar sus marcadores y otros datos de otros navegadores (por
  ejemplo Internet Explorer),haga click en \textbf{Continue}. Si no
  desea importar nada, solo seleccione \textbf{Cancel}.
\end{enumerate}

\begin{figure}[htbp]
\centering
\includegraphics{ff_win_inst_4.png}
\caption{Importando marcadores}
\end{figure}

\begin{enumerate}
\def\labelenumi{\arabic{enumi}.}
\setcounter{enumi}{7}
\tightlist
\item
  Una vez instalado Firefox, haga click en \textbf{Finish} para cerrar
  el asistente de configuración.
\end{enumerate}

Si tilda la casilla de verificación \textbf{Launch Firefox now}, Firefox
se ejecutará después que usted pulse \textbf{Finish}. Otra forma de
ejecutarlo es a través del menú de inicio.

\subsection{Usuarios de Windows Vista}\label{usuarios-de-windows-vista}

Si en ningún momento a través del proceso de instalación le piden
ingresar con la ventana de control de cuenta de usuario, presione
Continue, Allow, o Accept.

\section{Problemas}\label{problemas}

Si surgen problemas con el uso de Firefox, consulte
\url{https://support.mozilla.com/kb/Firefox+will+not+start}

\chapter{Extensiones de Firefox}\label{extensiones-de-firefox}

La primera vez que descargue e instale Firefox, puede manejar las tareas
básicas del navegador inmediatamente. También puede agregar capacidades
adicionales o cambiar la forma en que se comporta con la instalación de
complementos, pequeños añadidos que extienden el poder de Firefox.

Las extensiones de Firefox optimizan su navegador, pero también pueden
recoger y transmitir información sobre usted. Antes de instalar
cualquier complemento, tenga en cuenta elegir los complementos a partir
de fuentes confiables. De lo contrario, un complemento puede enviar
información acerca de usted sin que usted lo sepa, mantener un registro
de los sitios que ha visitado, o incluso dañar el equipo.

Hay varios tipos de complementos:

\begin{itemize}
\tightlist
\item
  \emph{Extensiones} que agregan funcionalidad a Firefox
\item
  \emph{Temas} que permiten cambiar la apariencia de Firefox.
\item
  \emph{Plugins} que ayudan a Firefox manejar las cosas que normalmente
  no puede procesar (por ejemplo, películas Flash, aplicaciones Java).
\end{itemize}

Para los temas que se tratan en este libro sólo vamos a necesitar
extensiones. Vamos a ver algunos complementos que son particularmente
importantes para hacer frente a la seguridad en Internet. La variedad de
extensiones disponibles es enorme. Puede añadir diccionarios de
diferentes idiomas, realizar el seguimiento del clima en otros países,
obtener sugerencias de los sitios web que son similares a la que usted
está viendo en ese momento, y mucho más. Firefox mantiene una lista de
las extensiones actuales disponibles en
(\url{https://addons.mozilla.org/firefox}); también puede buscar por
categoría en \url{https://addons.mozilla.org/firefox/browse}.

\textbf{Atención}: Nosotros le recomendamos que nunca instale un
complemento si no está disponible en la página de complementos de
Firefox. Usted nunca debe instalar Firefox a menos que obtenga los
archivos de instalación de una fuente de confianza. Es importante tener
en cuenta que el uso de Firefox en el ordenador de alguien o en un café
con Internet aumenta su vulnerabilidad potencial. Sepa que usted puede
tener Firefox en un CD o en una memoria USB (consulte el capítulo sobre
este tema).

Si bien ningún instrumento puede protegerlo completamente contra todas
las amenazas a su privacidad y seguridad en línea, las extensiones de
Firefox que se describen en este capítulo pueden reducir
significativamente su exposición a las más comunes, y aumentar sus
posibilidades de permanecer en el anonimato.

\section{HTTPS Everywhere}\label{https-everywhere}

HTTP es considerada insegura, porque la comunicación se transmite en
texto plano. Muchos sitios en la Web ofrecen algún soporte para el
cifrado HTTPS, pero es difícil de usar. Por ejemplo, pueden conectarse a
HTTP de forma predeterminada, incluso cuando HTTPS está disponible, o
pueden llenar las páginas cifradas con vínculos que se remontan al sitio
sin cifrar. La extensión HTTPS Everywhere soluciona estos problemas al
volver a escribir todas las solicitudes a estos sitios a HTTPS. Aunque
la extensión se llama ``HTTPS Everywhere'', sólo se activa HTTPS en una
lista particular de sitios y sólo pueden utilizar HTTPS en los sitios
que han decidido apoyarlo. No se puede hacer la conexión a un sitio
seguro si ese sitio no ofrece HTTPS como opción.

\begin{figure}[htbp]
\centering
\includegraphics{https_schema.jpg}
\caption{Esquema de HTTPS}
\end{figure}

Por favor, tenga en cuenta que algunos de estos sitios todavía incluyen
una gran cantidad de contenido, como imágenes o íconos, de dominios de
terceros que no están disponibles a través de HTTPS. Como siempre, si el
ícono de la cerradura del navegador está roto o tiene un signo de
exclamación, es posible que sigan siendo vulnerables a algunos
adversarios que usan ataques activos o análisis de tráfico. Sin embargo,
el esfuerzo que se requiere para controlar su navegación será mucho
mayor.

Algunos sitios web (como Gmail) proporcionan soporte HTTPS
automáticamente, pero utilizar HTTPS Everywhere también lo protegerá de
ataques de eliminación de TLS/SSL, en los que un atacante oculta la
versión HTTPS del sitio desde su computadora si en un inicio se intenta
acceder a la versión HTTP.

Información adicional se puede encontrar en su
\url{https://www.eff.org/https-everywhere}.

\section{Instalación}\label{instalaciuxf3n}

Primero, descargue la extensión HTTPS Everywhere desde el sitio web
oficial \url{https://www.eff.org/https-everywhere}

\begin{figure}[htbp]
\centering
\includegraphics{https_everywhere.png}
\caption{HTTPS Everywhere}
\end{figure}

Seleccione la versión más nueva. En el ejemplo debajo, usamos la versión
2.2 de HTTPS Everywhere. (Podría estar disponible una versión más nueva
en este momento.)

\begin{figure}[htbp]
\centering
\includegraphics{https_everywhere_2.png}
\caption{Seleccionando la versión más reciente}
\end{figure}

Haga click en ``Allow''. Tendrá que reiniciar Firefox pulsando el botón
``Restart Now''. HTTPS Everywhere está instalado.

\section{Configuración}\label{configuraciuxf3n}

Para acceder al panel de configuración de HTTPS Everywhere en Firefox 4
(GNU/Linux), haga click en el menú Tools en la parte superior de su
pantalla y luego seleccione complementos. (Observe que en diferentes
versiones de Firefox y en diferentes sistemas operativos, el
administrador de complementos puede estar en diferentes lugares en la
interfaz.)

\begin{figure}[htbp]
\centering
\includegraphics{https_everywhere_3.png}
\caption{Configurando HTTPS Everywhere}
\end{figure}

Haga click en el botón Preferences.

\begin{figure}[htbp]
\centering
\includegraphics{https_everywhere_4.png}
\caption{Preferencias}
\end{figure}

Se mostrará una lista de todos los sitios web soportados donde las
reglas de redirección de HTTPS pueden aplicarse. Si tiene problemas con
una regla específica de redirección, puede desmarcarla aquí. En este
caso, HTTPS Everywhere no modificará su conexión con el sitio
específico.

\section{Uso}\label{uso}

Una vez habilitado y configurado, HTTPS Everywhere es muy fácil y
transparente para usar. Tipee una URL como HTTP insegura (por ejemplo,
\url{http://www.google.com}).

\begin{figure}[htbp]
\centering
\includegraphics{https_everywhere_5.png}
\caption{Usando HTTPS Everywhere}
\end{figure}

Presione Enter. Será redirigido automáticamente al sitio web seguro
HTTPS cifrado (en este ejemplo: \url{https://encrypted.google.com}). No
se necesita ninguna otra acción.

\begin{figure}[htbp]
\centering
\includegraphics{https_everywhere_6.png}
\caption{Redireccionamiento de HTTPS Everywhere}
\end{figure}

\section{Si las redes bloquean HTTPS}\label{si-las-redes-bloquean-https}

Su operador de red puede decidir bloquear las versiones seguras de los
sitios web para aumentar su capacidad de espiar qué es lo que usted
hace. En tales casos, HTTPS Everywhere puede advertirlo de usar estos
sitios porque usted puede forzarlo para que nunca use las versiones
inseguras. (Por ejemplo, sabemos acerca de una red wifi de un aeropuerto
donde todas las conexiones HTTP estaban permitidas, pero no las HTTPS.
Quizás los operadores WiFi estaban interesados en ver que hacían los
usuarios. En el aeropuerto, los usuarios con HTTPS Everywhere no podrán
navegar por determinados sitios web a menos que deshabiliten
temporariamente HTTPS Everywhere.)

En este escenario, usted debería elegir usar HTTPS Everywhere junto con
una tecnología de evasión tal como Tor o una VPN para eludir a la red
que está bloqueando el acceso seguro a los sitios web.

\section{Añadir soporte para sitios adicionales en HTTPS
Everywhere}\label{auxf1adir-soporte-para-sitios-adicionales-en-https-everywhere}

Usted puede agregar sus propias reglas a HTTPS Everywhere para sus
sitios web favoritos. Puede encontrar cómo hacer esto en el siguiente
\url{https://www.eff.org/https-everywhere/rulesets}. El beneficio de
añadir reglas es que ellas le enseñan a HTTPS Everywhere cómo asegurarse
que su acceso a estos sitios sea seguro. Pero recuerde: HTTPS Everywhere
no le permitirá acceder a sitios seguros a menos que los operadores de
los sitios hayan elegido ponerlos disponibles a través de HTTPS. Si un
sitio no soporta HTTPS, no tendrá ningún beneficio añadir una regla para
él.

Si usted administra un sitio web y dispone de una versión HTTPS del
sitio disponible, una buena práctica sería la de presentar su sitio web
al lanzamiento oficial de HTTPS Everywhere.

\section{Forzando conexiones seguras sobre servidor
HTTPS}\label{forzando-conexiones-seguras-sobre-servidor-https}

Aún cuando usted le de instrucciones a su navegador para que use el
protocolo HTTPS cuando se comunique con un servidor web, es posible que
el servidor (debido a una configuración insegura de su lado) fuerce a un
protocolo cifrado SSL inseguro para la conexión. La única forma de
prevenir es diciéndole al navegador que no acepte dichos protocolos
inseguros SSL (como aquellos basados en cifrado RC4).

Para deshabilitar el cifrado RC4 para las conexiones HTTPS haga lo
siguiente. En la barra de direcciones vacía tipee ``about:config'',
presione enter y cierre la ventana de diálogo de precaución que aparece
(puede deshabilitar este diálogo si lo desea la próxima vez que
configure Firefox). En el campo de búsqueda ingrese ``rc4'' y observe la
lista desplegada como resultado de su búsqueda:

\begin{figure}[htbp]
\centering
\includegraphics{disable_rc4.png}
\caption{Deshabilitando RC4}
\end{figure}

Una entrada con un ``true'' en la última columna (campo ``Value'')
estará activa, debe desactivarla. Simplemente, con un click derecho en
la entrada y cambie el valor a false. Proceda de la misma forma para
todas las entradas.

\section{Adblock Plus}\label{adblock-plus}

Adblock Plus (\url{http://www.adblockplus.org}) es conocido
principalmente por bloquear publicidad en los sitios web. Pero también
se puede usar para bloquear otro contenido que intente rastrearlo. Para
mantenerse actualizado con las últimas amenazas, Adblock Plus depende de
las listas negras mantenidas por voluntarios.

Información extra para Geeks: ¿Cómo bloquea direcciones Adblock Plus?

El trabajo duro aquí está hecho realmente por Gecko, el motor sobre al
cual se construyen aplicaciones tales como Firefox, Thunderbird y otros.
Permiten lo se conoce como ``políticas de contenido''. Una política de
contenido no es más que un objeto JavaScript (o C++) que se llama cada
vez que el navegador tiene que cargar algo. A continuación, puede ver la
dirección que debe cargarse y algunos otros datos y decidir si se debe
permitir o no. Existe una serie de directivas integradas de contenido
(cuando usted define a qué sitios no se les debe permitir cargar las
imágenes en Firefox o SeaMonkey, en realidad se está configurando una de
estas políticas de contenido integrado) y ninguna extensión puede
registrar alguna. Así que todo lo que Adblock Plus tiene que hacer es
registrar su política de contenidos, aplicar una lógica para decidir qué
direcciones bloquear e implementar la interfaz de usuario para permitir
la configuración de los filtros.

\section{Comenzando con Adblock Plus}\label{comenzando-con-adblock-plus}

Una vez que está instalado Firefox:

\begin{enumerate}
\def\labelenumi{\arabic{enumi}.}
\tightlist
\item
  Descargue la última versión de Adblock Plus desde la base de datos de
  los complementos de Firefox.
\item
  Confirme que quiere instalar Adblock haciendo click en ``Install
  Now''.
\item
  Después que Adblock Plus se ha instalado, Firefox se reiniciará.
\end{enumerate}

\section{Elección de una suscripción a un
filtro}\label{elecciuxf3n-de-una-suscripciuxf3n-a-un-filtro}

Adblock Plus por sí mismo no hace nada. Puede ver cada elemento que un
sitio web intenta cargar, pero no sabe a cuál bloquear. Para eso están
los filtros de Adblock's. Después de reiniciar Firefox, se le pedirá que
elija una suscripción a un filtro (gratuita).

\begin{figure}[htbp]
\centering
\includegraphics{abp_1.png}
\caption{Ad Block Plus}
\end{figure}

¿Cuál elegir? Adblock Plus ofrece algunos en un menú desplegable y
posiblemente usted quiera saber algo acerca de las fortalezas y
debilidades de cada uno. Un buen filtro para comenzar a proteger su
privacidad es EasyList (también disponible en
\url{http://easylist.adblockplus.org/en}).

Por muy tentador que pueda parecer, no se suscriba a demasiados filtros,
ya que algunos pueden superponerse, lo que resulta en resultados
inesperados. EasyList (principalmente dirigido a sitios en idioma
inglés) funciona bien con otras extensiones EasyList (tales como
extensiones específicas de la región, como las listas de RuAdList o
listas temáticas como EasyPrivacy). Pero choca con la lista de Fanboy
(otra lista con foco principal en sitios en idioma inglés).

Usted puede cambiar sus suscripciones de filtro en cualquier momento.
Una vez hechos sus cambios, haga click en OK.

\section{Creación de filtros
personalizados}\label{creaciuxf3n-de-filtros-personalizados}

AdBlock Plus también le permite crear sus propios filtros, si así lo
desea. Para agregar un filtro, vaya a las preferencias de Adblock Plus
preferencias y haga clic en ``Add Filter'' en la esquina inferior
izquierda de la ventana. Los filtros personalizados no pueden reemplazar
los beneficios de las listas negras bien mantenidas como EasyList, pero
son muy útiles para bloquear el contenido específico que no está
cubierto en las listas públicas. Por ejemplo, si desea evitar la
interacción con Facebook en otros sitios web, puede agregar el siguiente
filtro:

\begin{verbatim}
||facebook.*$domain=~facebook.com|~127.0.0.1
\end{verbatim}

La primera parte (\texttt{\textbar{}\textbar{}facebook.*}) bloqueará
inicialmente todo lo que venga desde el dominio de Facebook. La segunda
parte
(\texttt{\$domain=\textasciitilde{}facebook.com\textbar{}\textasciitilde{}127.0.0.1})
es una excepción que le dice al filtro que permita solicitudes de
Facebook solamente cuando usted está en Facebook o si sus solicitudes de
Facebook proceden desde 127.0.0.1 (su propia computadora) para mantener
ciertas características de Facebook trabajando.

Puede encontrar una guía acerca de cómo crear sus propios filtros en
\url{http://adblockplus.org/en/filters}.

\section{Habilitación y deshabilitación de AdBlock Plus para elementos o
sitios web
específicos}\label{habilitaciuxf3n-y-deshabilitaciuxf3n-de-adblock-plus-para-elementos-o-sitios-web-especuxedficos}

Usted puede ver los elementos identificados por AdBlock Plus pulsando en
el ícono ABP de AdBlock Plus en su navegador (habitualmente cerca de la
barra de búsqueda) y seleccionar ``Open blockable items''. Una ventana
abajo en su navegador habilitará o deshabilitará cada elemento de la
base caso por caso. Alternativamente, puede deshabilitar AdBlock Plus
para un dominio o una página específica haciendo click en el ícono ABP y
marcando la opción ``Disable on {[}nombre del dominio{]}'' o ``Disable
on this page only''.

\section{Otras extensiones que pueden mejorar su
seguridad}\label{otras-extensiones-que-pueden-mejorar-su-seguridad}

Debajo hay una breve lista de extensiones que no están cubiertas en este
libro y son de gran ayuda para su protección.

\begin{itemize}
\item
  \textbf{Flagfox} - pone una bandera en la barra de localización que le
  informa el lugar más probable en donde se encuentre el servidor web
  que hospeda la página web que está visitando
  \href{https://addons.mozilla.org/en-US/firefox/addon/flagfox/}{https://addons.mozilla,org/en-US/firefox/addon/flagfox/}
\item
  \textbf{BetterPrivacy} - administra las ``cookies'' usadas para
  rastreaslo mientras visita sitios web. Las cookies son pequeñas
  cantidades de información almacenada en su navegador. Algunas de ellas
  son usadas por los publicistas para rastrear los sitios que usted
  visita.
  \url{https://addons.mozilla.org/en-US/firefox/addon/betterprivacy/}
\item
  \textbf{GoogleSharing} - Si le preocupa que Google conozca su
  historial de búsqueda, esta extensión lo ayudará a evitarlo
  \href{https://addons.mozilla.org/en-us/firefox/addon/googlesharing/}{https://addons.mozilla.org/en-US/firefox/addon/googlesharing/}
\item
  \textbf{NoScript} - Aunque no es demasiado amigable para los
  principiantes, este complemento bloqueará los scripts y el contenido
  de los plugins de terceras partes (por ejemplo, Adobe Flash) a menos
  de que usted se lo permita específicamente, también brinda una
  protección general contra simples vectores de cross site scripting
  \url{http://noscript.net}
\item
  \textbf{User Agent Switcher} - Su navegador envía gran cantidad de
  información a cualquier servidor remoto a través del encabezado
  `User-Agent', incluyendo su sistema operativo e información específica
  sobre su versión. Este complemento le permite a usted falsificar dicha
  información o enviar un User-Agent genérico al servidor.
  \url{http://chrispederick.com/work/user-agent-switcher/}
\end{itemize}

\chapter{Configuración de proxy}\label{configuraciuxf3n-de-proxy}

Un servidor proxy le permite a usted alcanzar un sitio web u otro lugar
que esté bloqueado por su país o por su ISP. Existen muchas clases
diferentes de proxies, que incluyen:

\begin{itemize}
\tightlist
\item
  Proxies web, que sólo requieren que usted conozca la dirección del
  sitio web del proxy. Una URL puede lucir así
  \texttt{http://www.example.com/cgi-bin/nph-proxy.cgi}
\item
  HTTP proxies, que requieren que modifique la configuración de su
  navegador. Los proxies HTTP sólo trabajan en contenido web. Puede
  obtener más información acerca de un proxy HTTP en el formato
  \texttt{proxy.example.com:3128} o \texttt{192.168.0.1:8080}.
\item
  Proxies SOCKS, que requieren modificar la configuración de su
  navegador. Los proxies SOCKS trabajan para muchas aplicaciones
  diferentes de Internet, incluso correo electrónico y herramientas de
  mensajería instantánea. La información sobre proxy SOCKS se parece a
  la información sobre proxy HTTP.
\end{itemize}

Usted puede usar un proxy web directamente sin ninguna configuración
tipeando en la URL. Los proxies HTTP y SOCKS, sin embargo, tienen que
configurarse en su navegador web.

\section{Configuración del proxy por
defecto}\label{configuraciuxf3n-del-proxy-por-defecto}

En Firefox usted puede cambiar la configuración para usar un proxy.
Necesitará abrir las opciones en la ventana de preferencias de Firefox.
Puede encontrar esto en el menú, haciendo click en la parte superior y
seleccionando \texttt{Edit\ \textgreater{}\ Preferences} en GNU/Linux o
\texttt{Tools\ \textgreater{}\ Options} en Windows.

Vaya a la sección Network y abra la pestaña Advanced.

\begin{figure}[htbp]
\centering
\includegraphics{ff_proxy_1.png}
\caption{Configurando un proxy}
\end{figure}

Seleccione Settings, haga click en ``Manual proxy configuration'' e
ingrese la información del servidor proxy que desea usar. Por favor
recuerde que los proxies HTTP y los proxies SOCKS trabajan de distinta
forma y tienen que ingresarse en sus correspondientes campos. Si hay dos
puntos (:) en la información de su proxy, esto es una separación entre
la dirección y el número de puerto. Su pantalla lucirá como esta:

\begin{figure}[htbp]
\centering
\includegraphics{ff_proxy_2.png}
\caption{Proxy para Firefox}
\end{figure}

Después de hacer click en OK, su configuración será grabada y su
navegador web se conectará automáticamente a través del proxy en todas
sus conexiones futuras. Si obtiene un mensaje de error tal como, ``The
proxy server is refusing connections'' o ``Unable to find the proxy
server'', existe algún problema con su configuración del proxy. En este
caso, repita los pasos anteriores y seleccione ``No proxy'' en la última
pantalla para desactivar el proxy.

\chapter{Uso de Tor}\label{uso-de-tor}

Tor es un sistema pensado para facilitar el anonimato online, y está
compuesto por un software cliente y una red de servidores los cuales
pueden ocultar información acerca de la localización de los usuarios y
otros factores que podrían identificarlos. Imagine un mensaje envuelto
en muchas capas de protección: cada servidor tiene que quitar una capa,
con lo que inmediatamente elimina la información del remitente del
servidor anterior.

Si Alice desea visitar el sitio web de Bob en forma directa, lo
representamos de la siguiente forma:

\begin{verbatim}
Alicia -> Bob
\end{verbatim}

Esto está bien, y Alicia y Bob podrán usar cifrado punto a punto para
asegurarse la privacidad, la integridad y la autenticidad de sus
comunicaciones. Sin embargo, si Alice no quiere que Bob sepa que ella
está visitando su sitio web o no quiere que Eva (una hipotética espía,
del lado de Alicia o del de Bob en la conexión) sepa que ella y Bob
están comunicándose, se deben establecer algunos pasos extra.

Alicia debe entablar una conexión cifrada con un nodo de entrada de la
red Tor, aquí se establecerá una conexión TLS y el nodo de entrada le
permitirá a Alicia establecer una comunicación a través de él. Una vez
establecida dicha conexión TLS, se repite este proceso con un nodo de
repetición, y entre éste y un nodo de salida. En este punto, Alicia
cifrará sus datos 3 veces, primero a través del nodo de salida, luego a
través del nodo repetidor y finalmente a través del nodo de entrada. La
ruta establecida en la red luce de la siguiente manera:

\begin{verbatim}
Alicia -> Nodo de entrada -> Nodo repetidor -> Nodo de salida -> Bob
\end{verbatim}

Cuando el nodo de entrada recibe los datos de Alicia estos están aún
cifrados por el nodo repetidor y el nodo de salida. El nodo de entrada
conoce su procedencia (Alicia) pero no su destino final (Bob) ni su
contenido. El nodo repetidor recibe los datos del nodo de entrada y los
trasmite al nodo de salida. Los datos aún están cifrados por el nodo de
salida, y no conoce el origen (Alicia) ni el destino (Bob). Cuando el
nodo de salida recibe los datos del nodo repetidor, se remueve la última
capa de cifrado: el nodo de salida puede ver los datos y el destino
(Bob) pero no conoce su origen (Alicia).

Esta aproximación por capas da su nombre a Tor (The Onion Router, el
enrutador cebolla), cada capa conoce la capa en contacto con ella, y
significa que nadie en la cadena excepto Alicia conoce la ruta completa
que los datos están siguiendo; sin embargo, Alicia, Bob y el nodo de
salida son capaces de leer el contenido del mensaje, por eso el cifrado
punto a punto es requerido para asegurar la privacidad, la integridad y
la autenticidad de las comunicaciones a través de la red Tor.

El uso de este sistema hace que sea más difícil de rastrear el tráfico
de Internet del usuario, que incluye visitas a sitios web, publicaciones
en línea, mensajes instantáneos y otras formas de comunicación. Su
objetivo es proteger la libertad personal de los usuarios, la privacidad
y capacidad de hacer negocios confidenciales, al mantener sus
actividades en Internet a salvo del monitoreo. Tor es software libre y
la red es de uso gratuito.

Como todas las redes actuales de anonimato de baja latencia, Tor no
puede y no trata de protegerlo contra la vigilancia del tráfico en los
extremos de la red, es decir, el tráfico que entra y sale de la red.
Mientras que Tor proporciona protección contra el análisis de tráfico,
no puede evitar la confirmación del tráfico (también llamada correlación
de extremo a extremo)

Precaución: Como Tor no lo hace, y por diseño, no puede, cifrar el
tráfico entre un nodo de salida y el servidor de destino, cualquier nodo
de salida está en disposición de capturar cualquier tráfico que pasa a
través de él, que no utilice cifrado de extremo a extremo, tal como TLS.
(Si el cartero es corrupto, podría abrir el sobre y leer el contenido).
Si bien esto puede o no violar el anonimato de la fuente, si los
usuarios de Tor no cifran la comunicación de extremo a extremo ellos
puede estar sujeto a un riesgo adicional de la interceptación de datos
por parte de terceros. Resumiendo: la ubicación del usuario permanece
oculta, sin embargo, en algunos casos el contenido es vulnerable al
análisis a través del cual también se puede obtener información sobre el
usuario.

\section{Uso del paquete Tor para
navegadores}\label{uso-del-paquete-tor-para-navegadores}

El paquete Tor para navegadores le permite usar Tor en Windows, OSX o
GNU/Linux sin necesidad de configurar al navegador web. Aún mejor,
también es una aplicación portable que se puede ejecutar desde una
unidad flash USB, lo que le permite llevarlo a cualquier PC sin
necesidad de instalarlo en el disco rígido de cada computadora.

\section{Descarga del paquete Tor para
navegadores}\label{descarga-del-paquete-tor-para-navegadores}

Puede descargarlo desde el sitio web de torproject.org
(\url{https://www.torproject.org}).

Si su país restringe el acceso al sitio web de tor, tipee ``tor
mirrors'' en su motor de búsqueda web favorito: el resultado
probablemente incluya algunas direcciones alternativas para descargarlo.

Por favor, siga las instrucciones del sito web del proyecto Tor acerca
de cómo instalarlo.

Precaución: cuando usted descarga el paquete Tor (en sus versiones
completa o dividida), debería verificar las firmas de los archivos,
especialmente si lo está descargando desde un sitio espejo. Este paso le
asegura que los archivos no han sido falsificados. Para aprender más
acerca de archivos de firma y cómo revisarlos, consulte
\url{https://www.torproject.org/docs/verifying-signatures}

\section{Ejecutando un repetidor o un
puente}\label{ejecutando-un-repetidor-o-un-puente}

Tor es una red de voluntarios que ejecutan repetidores y puentes. Si
desea ayudar al crecimiento de la red Tor contribuyendo con ancho de
banda y ciclos extra de CPU, considere ejecutar un repetidor. Además, al
correr un repetidor puede mejorar su anonimato ya que un atacante no
puede distinguir entre el tráfico originado por usted o por el
repetidor. Consulte
\url{https://www.torproject.org/docs/faq.html.en\#BetterAnonymity} para
obtener más detalles.

Sin embargo, si usted corre un repetidor, su dirección IP será listada
en Internet como un repetidor Tor. Los clientes Tor dependen de esta
lista, provista por los servidores del directorio de Tor, para poder
establecer los circuitos. Si desea contribuir con Tor, pero no desea
correr un repetidor público, considere ejecutar un puente. Ya que los
repetidores Tor son públicos, algunos ISP bloquean el acceso a la red
Tor bloqueando \emph{todos los repetidores.} Los puentes Tor, sin
embargo, no están listados y además, son más difíciles de hallar.

La meta de Tor es proteger el anonimato en Internet, pero algunas veces
se usa con fines ilegales. Como operador de un repetidor, consulte
\url{https://www.torproject.org/eff/tor-legal-faq.html}, escrito por la
Electronic Frontier Foundation (EFF). La EFF es una organización sin
fines de lucro de EE.UU. cuya misión es ``proteger sus derechos
digitales.'' En otros países, deberían buscar asesoramiento de
organizaciones similares. Sin embargo, los riesgos legales pueden ser
minimizados corriendo un repetidor que no sea de salida o un puente.

Si desea configurar su computadora para correr un repetidor o un puente,
visite el \url{https://www.torproject.org/docs/tor-doc-relay.html.en}
para obtener instrucciones. Extendiendo Google Chrome
=========================

Chrome es el navegador de Google. Aquí daremos algunos consejos y
extensiones:

\section{Deshabilitación de búsqueda
instantánea}\label{deshabilitaciuxf3n-de-buxfasqueda-instantuxe1nea}

Chrome puede buscar por tipo. La ventaja de esto es que usted puede
recibir sugerencias de búsqueda y usar las predicciones de Google - pero
las desventajas son que cada carácter que usted tipee puede ser enviado
a los servidores de Google, donde queda registrado.

Para deshabilitarlo, abra la configuración de Chrome haciendo click en
el botón del menú a la derecha la barra de direcciones y pulsando
Settings. O, simplemente tipee \texttt{chrome://settings/} en su barra
de direcciones.

Asegúrese de la casilla de verificación \textbf{Enable Instant for
faster searching (omnibox input may be logged)} esté desmarcada.

\section{AdBlock para Chrome}\label{adblock-para-chrome}

Como en Firefox, AdBlock remueve la publicidad. Puede instalarlo a
partir de la página de
\href{https://chrome.google.com/webstore/detail/adblock/gighmmpiobklfepjocnamgkkbiglidom}{Chrome
Webstore}.

\section{HTTPS Everywhere}\label{https-everywhere-1}

Fuerza conexiones cifradas https donde sean posibles. El enlace de
instalación puede encontrarse en su
\href{https://www.eff.org/https-everywhere}{página web}.

\section{PrivacyFix}\label{privacyfix}

PrivacyFix (beta) le proporciona la vista de tablero de mandos de la
configuración de su privacidad en Facebook y Google, así como los
encabezados de Do-Not-Track y las cookies de rastreo. Esto proporciona
enlaces para cambiar rápidamente estos ajustes de privacidad sin excavar
a través de muchas páginas de desglose. Se puede instalar desde la
página de
\href{https://chrome.google.com/webstore/detail/privacyfix-by-privacychoi/pmejhjjecaldkllonlokhkglbdbkdcni}{Chrome
web store}

\chapter{Manteniendo contraseñas
seguras}\label{manteniendo-contraseuxf1as-seguras}

Las contraseñas son como las llaves del mundo físico. Si pierde una
contraseña no será capaz de entrar, y si los demás la copian o roban la
podrán utilizar para entrar. Una buena contraseña no debe ser fácil de
adivinar para los demás ni fácil de romper con las computadoras, sin
dejar de ser fácil de recordar.

\section{Extensión y complejidad de la
contraseña}\label{extensiuxf3n-y-complejidad-de-la-contraseuxf1a}

Para evitar que sus contraseñas sean adivinadas, la longitud y
complejidad son importantes. Las contraseñas como el nombre de su
mascota o la fecha de nacimiento son muy inseguras, ya que utiliza una
sola palabra que se puede encontrar en un diccionario. No utilice una
contraseña que contenga solamente números. Lo más importante en una
contraseña segura es que sea larga. El uso de combinaciones de letras
minúsculas, mayúsculas, números y caracteres especiales pueden mejorar
la seguridad, pero la longitud sigue siendo el factor más importante.

Utilice contraseñas de 20 caracteres por lo menos (cuanto más caracteres
tenga, mejor) para asegurar cuentas importantes, como la frase de paso
que protege su PGP/GPG o sus datos cifrados TrueCrypt, o la contraseña
de su cuenta de correo electrónico principal . Ver {[}esta caricatura
XKCD{]} (https://xkcd.com/936/)
\texttt{"correct\ horse\ battery\ staple"} vis-à-vis
\texttt{"Tr0ub4dor\&3"} para obtener una explicación.

\section{Contraseñas seguras y fáciles de
recordar}\label{contraseuxf1as-seguras-y-fuxe1ciles-de-recordar}

Una forma de crear una contraseña fuerte y fácil de recordar es usar
frases.

Unos pocos ejemplos:

\begin{itemize}
\tightlist
\item
  `AmoAdouglasAdamsPorqueEsRealmenteGenial. '
\item
  `LaGenteAmaAlasMaquinasEnEl2029. '
\item
  `BarneyDe¡ComoConociAtuMadreEsImpresionante! '
\end{itemize}

Las frases son fáciles de recordar, incluso si son 50 caracteres y
contiene caracteres en mayúsculas, en minúsculas, símbolos y números.

\section{Minimizar los daños}\label{minimizar-los-dauxf1os}

Es importante minimizar el daño si uno de sus contraseñas está siempre
en peligro. Utilice diferentes contraseñas para diferentes sitios web o
cuentas, de esa manera que si una se ve comprometida, los demás no lo
estarán. Cambie sus contraseñas de vez en cuando, especialmente si hay
cuentas que consideran sensibles. De esta manera usted puede bloquear el
acceso a un atacante que puede haber aprendido su antigua contraseña.

\section{El uso de un gestor de
contraseñas}\label{el-uso-de-un-gestor-de-contraseuxf1as}

Recordar un montón de contraseñas diferentes puede ser difícil. Una
solución es utilizar una aplicación dedicada a gestionar la mayor parte
de sus contraseñas. La siguiente sección de este capítulo discutiremos
\emph{Keepass}, un gestor de contraseñas libre sin vulnerabilidades
conocidas, siempre y cuando elijas una ``contraseña maestra''
suficientemente largo y compleja para asegurarlo.

Para guardar contraseñas de sitios web sólo, otra opción es el
administrador de contraseñas integrado del navegador Firefox. ¡Asegúrese
de establecer una contraseña maestra, de lo contrario esto es muy
inseguro!

\section{La protección física}\label{la-protecciuxf3n-fuxedsica}

Cuando se utiliza un equipo público, como en una biblioteca, un
cibercafé o cualquier equipo que no es de su propiedad, existen varios
peligros. Usando el método de la vigilancia ``sobre el hombro'' alguien,
posiblemente con una cámara, puede ver sus acciones y puede ver la
cuenta con la que inicia sesión y la contraseña que escribe. Una amenaza
menos evidente son los programas de software o dispositivos de hardware
llamados ``keyloggers'', que registran lo que escribe. Ellos pueden
estar ocultos dentro de una computadora o un teclado y no verse
fácilmente. No utilice computadoras públicas para iniciar sesión en sus
cuentas privadas, tales como el correo electrónico. Si lo hace, cambie
sus contraseñas tan pronto como vuelva a una computadora que posee y en
la cual confíe.

\section{Otras advertencias}\label{otras-advertencias}

Algunas aplicaciones, como programas de chat o de correo electrónico
puede pedirle guardar o ``recordar'' su nombre de usuario y contraseña,
por lo que no tendrá que introducirlas cada vez que se abre el programa.
Si lo hace, puede significar que su contraseña puede ser recuperada por
otros programas que se ejecutan en la máquina, o directamente desde el
disco duro por alguien con acceso físico a la misma.

Si la información de inicio de sesión se envía a través de una conexión
o canal inseguro, podría caer en las manos equivocadas. Consulte los
capítulos sobre la navegación segura para más información. Instalación
de KeePass ======================

Explicaremos la instalación de KeePass en Ubuntu, en Windows y en Mac
OSX.

Mac OS X viene con un excelente archivo administrador integrado de
contraseña llamada Keychain que es bastante seguro Los inconvenientes
son que no es de código abierto y no funciona en otros sistemas
operativos. Si lo necesitas para llevar tus contraseñas de un sistema
operativo a otro, es mejor quedarse con Keepass después de todo. Cómo
utilizar Keychain se explicará en el capítulo siguiente.

\section{Instalación de KeePassX en
Ubuntu}\label{instalaciuxf3n-de-keepassx-en-ubuntu}

Para instalarlo en Ubuntu vamos a utilizar el Ubuntu Software Center.
Escriba KeePass en el campo de búsqueda en la parte superior derecha y
la aplicación KeePassX deberá aparecer automáticamente en el listado.

Resalte el elemento (que puede estar ya resaltado por defecto) y pulse
en ``Instalar''. Se le pedirá autorización para el proceso de
instalación:

\begin{figure}[htbp]
\centering
\includegraphics{keepass_1.png}
\caption{Instalación de Keepass}
\end{figure}

Introduzca la contraseña y pulse «verificar», el proceso de instalación
comenzará entonces.

Ubuntu no ofrece una respuesta muy buena para mostrar que el software
está instalado. Si el indicador de progreso verde en la izquierda ha
desaparecido y la barra de progreso de la derecha se ha ido entonces se
puede suponer que el software está instalado.

\section{Instalación de KeePass en
Windows}\label{instalaciuxf3n-de-keepass-en-windows}

Primero visite la \href{http://keepass.info/download.html}{página web de
descarga de KeePass} y seleccione el instalador apropiado. En este
capítulo se utiliza el
\href{http://downloads.sourceforge.net/keepass/KeePass-2.15-Setup.exe}{Instalador
actual}.

Descárguelo a su computadora y haga doble click en el instalador.
Primero se le pedirá que seleccione un idioma, vamos a elegir el idioma
Inglés:

\begin{figure}[htbp]
\centering
\includegraphics{keepass_2.png}
\caption{Seleccionando el idioma de Keepass}
\end{figure}

Presione `OK' y se mostrará la siguiente pantalla:

\begin{figure}[htbp]
\centering
\includegraphics{keepass_3.png}
\caption{Ventana de instalación}
\end{figure}

Sólo pulse en `Next\textgreater{}' y vaya a la siguiente pantalla:

\begin{figure}[htbp]
\centering
\includegraphics{keepass_4.png}
\caption{Pantalla de acuerdo}
\end{figure}

En la pantalla que se muestra arriba hay que seleccionar ``Acepto el
acuerdo'' de lo contrario no podrá instalar el software. Elija esta
opción y luego pulse `Next\textgreater{}'. En la siguiente pantalla se
le pedirá determinar la ubicación de la instalación. Puede dejar los
valores por defecto a menos que tenga una buena razón para cambiarlos.

\begin{figure}[htbp]
\centering
\includegraphics{keepass_5.png}
\caption{Configurando la ruta de instalación}
\end{figure}

Haga clic en `Next\textgreater{}' y continúe.

\begin{figure}[htbp]
\centering
\includegraphics{keepass_6.png}
\caption{Keepass Install}
\end{figure}

La imagen de arriba muestra los componentes de KeePass que usted puede
elegir. Deje los valores por defecto como están y pulse
`Next\textgreater{}'. Llegará a una nueva pantalla:

\begin{figure}[htbp]
\centering
\includegraphics{keepass_7.png}
\caption{Instalando Keepass}
\end{figure}

Esto no hace otra cosa que mostrarle un resumen de sus opciones.
Presione ``Instalar'' y el proceso de instalación comenzará.

\begin{figure}[htbp]
\centering
\includegraphics{keepass_8.png}
\caption{Confirmación de las opciones}
\end{figure}

\section{Instalación de KeePass en Mac OS
X}\label{instalaciuxf3n-de-keepass-en-mac-os-x}

Aunque KeyChain de Mac OS X hace un trabajo excelente al almacenar sus
contraseñas, es posible que desee ejecutar su propia base datos y
administrador de contraseñas. KeePass permite esta flexibilidad
adicional. Primero visite la página web de descarga KeePass
\url{http://keepass.info/download.html} y seleccione el instalador
apropiado. Aunque los instaladores oficiales se enumeran en la parte
superior de la página, hay instaladores no oficiales/contribuidos más
abajo. Desplácese hacia abajo para encontrar {[}KeePass 2.x para Mac OS
X{]}{[}http://keepass2.openix.be/{]}(http://keepass2.openix.be/):

\begin{figure}[htbp]
\centering
\includegraphics{keepass_9.png}
\caption{Keepass para Mac OS X}
\end{figure}

Como se trata de un enlace externo, su navegador será redirigido a
\url{http://keepass2.openix.be/}:

\begin{figure}[htbp]
\centering
\includegraphics{keepass_10.png}
\caption{Redirección del navegador}
\end{figure}

Nótese aquí que debe instalar el framework Mono primero, para que
KeePass puede ejecutarlo en Mac OS X. Haga un click sobre cada uno de
los enlaces
\href{http://download.mono-project.com/archive/2.10.5/macos-10-x86/0/MonoFramework-MRE-2.10.5_0.macos10.xamarin.x86.dmg}{Mono
2.10.5} y \href{http://keepass2.openix.be/KeePass2.18.dmg}{KeePass2.18}
para descargar los archivos DMG a su computadora. Haga doble click en
cada uno de los DMGS en tus carpeta de descargas para descomprimir los
volúmenes en el escritorio.

El programa de instalación del paquete Mono se llama
`MonoFramework-MRE-2.10.5\_0.macos10.xamarin.x86.pkg', por lo que debe
hacer doble click en este documento en el volumen MonoFramework en el
escritorio:

\begin{figure}[htbp]
\centering
\includegraphics{keepass_11.png}
\caption{MonoFramework}
\end{figure}

El instalador se abre y ejecuta:

\begin{figure}[htbp]
\centering
\includegraphics{keepass_12.png}
\caption{Instalando MonoFramework}
\end{figure}

Siga cada uno de los pasos, haga clic en ``Continuar'', el siguiente
paso es ver la sección `Read me'. Esta es información importante, ya que
posee todos los archivos que el paquete instalará, incluyendo
información sobre cómo desinstalar Mono:

\begin{figure}[htbp]
\centering
\includegraphics{keepass_13.png}
\caption{Readme de MonoFramework}
\end{figure}

Haga click en `Continue' en la pantalla siguiente, la licencia.
Aparecerá el cuadro de diálogo de acuerdo/desacuerdo. Si está de acuerdo
con las condiciones de la licencia, la instalación continuará:

\begin{figure}[htbp]
\centering
\includegraphics{keepass_14.png}
\caption{Acuerdo}
\end{figure}

Los siguientes dos pasos de la instalación le pedirán que elija un
destino y comprobar que haya espacio suficiente en el disco. Cuando la
instalación se haya completado, verá la siguiente pantalla:

\begin{figure}[htbp]
\centering
\includegraphics{keepass_15.png}
\caption{Instalación terminada}
\end{figure}

Ahora puede salir del instalador. A continuación, eche un vistazo a la
imagen del disco KeePass, haga doble click para abrirlo y arrastre la
aplicación KeePass a su carpeta de Aplicaciones:

\begin{figure}[htbp]
\centering
\includegraphics{keepass_16.png}
\caption{Arrastrando a la carpeta de Aplicaciones}
\end{figure}

Ahora KeePass está listo para usar en Mac OS X.

\chapter{Cifrado de contraseñas con un
administrador}\label{cifrado-de-contraseuxf1as-con-un-administrador}

Para cifrar contraseñas utilizamos KeePass en Windows, KeePassX en
Ubuntu y KeyChain en OSX. El principio básico es el mismo: usted tiene
un archivo en su computadora, que se cifra con una \emph{contraseña
única muy segura}. Esto se refiere a veces como una `contraseña
maestra', `contraseña de administrador', `contraseña raíz', etc, pero
todos ellos son \emph{la clave definitiva} para todas sus claves y otros
datos seguros. Por esta razón no se puede ni se debe pensar a la luz
sobre la creación de esta contraseña. Si un administrador de contraseñas
es parte de su sistema operativo (como sucede con OSX) se desbloquea
automáticamente para usted después de que usted ingrese a su cuenta y le
permite acceder a información segura, como contraseñas. Por esto y otras
razones, se debe desactivar `Iniciar sesión automáticamente'. Debería
iniciar una sesión siempre que arranque el equipo y, mejor aún, debería
configurarlo para que automáticamente la cierre o bloquee la pantalla
después de un período de tiempo determinado

\section{Cifrado de contraseñas con KeePassX en
Ubuntu}\label{cifrado-de-contraseuxf1as-con-keepassx-en-ubuntu}

Abra primero KeePassX desde el menú Applications -\textgreater{}
KeePassX.

La primera vez que utilice KeePassX es necesario establecer una nueva
base de datos para almacenar sus contraseñas. Haga clic en File
-\textgreater{} New Database.

Se le pedirá que establezca una clave principal (clave).

\begin{figure}[htbp]
\centering
\includegraphics{mng_1.png}
\caption{Estableciendo una clave principal}
\end{figure}

Elija una contraseña fuerte para este campo - consulte el capítulo
acerca de contraseñas si desea algunos consejos sobre cómo hacer esto.
Ingrese la contraseña y presione `OK'. Se le pedirá que lo ingrese
nuevamente. Hecho esto, presione `OK'. Si las contraseñas son iguales
verá una nueva `base de datos' KeePassX lista para usar.

\begin{figure}[htbp]
\centering
\includegraphics{mng_2.png}
\caption{Ingresando la contraseña}
\end{figure}

Ahora tiene un lugar para almacenar todas sus contraseñas y protegerlas
con la contraseña `maestra' que acaba de establecer. Verá dos categorías
por defecto `Internet' y `Correo electrónico' - se pueden almacenar las
contraseñas sólo en estas dos categorías, puede eliminar categorías,
añadir subgrupos, o crear nuevas categorías. Por ahora sólo nos
quedaremos con estas dos y añadiremos la contraseña de nuestro correo
electrónico al grupo de correo electrónico. Haga clic en esta categoría
y seleccione ``Agregar nueva entrada \ldots{} ':

\begin{figure}[htbp]
\centering
\includegraphics{mng_3.png}
\caption{Agregando una nueva entrada}
\end{figure}

\begin{figure}[htbp]
\centering
\includegraphics{mng_4.png}
\caption{Nueva entrada}
\end{figure}

Así que ahora llene este formulario con los detalles para que usted
pueda identificar correctamente la cuenta de correo electrónico y las
contraseñas asociadas. Usted necesita llenar los campos `Título' y los
campos de la contraseña. Todo lo demás es opcional.

\begin{figure}[htbp]
\centering
\includegraphics{mng_5.png}
\caption{Formulario}
\end{figure}

KeePassX da alguna indicación de si las contraseñas que se utilizan son
`fuertes' o `débiles' \ldots{} usted debe tratar de hacer que las
contraseñas sean lo más fuertes posible; consejo sobre esto, lea el
capítulo acerca de cómo crear una buena contraseña. Pulse `OK' cuando
haya terminado y usted verá algo como esto:

\begin{figure}[htbp]
\centering
\includegraphics{mng_6.png}
\caption{Contraseña en KeePassX}
\end{figure}

Para recuperar las contraseñas deberá hacer doble click en la entrada y
verá la ventana que utilizó para registrar la información. Si hace click
en el icono del `ojo' a la derecha de las contraseñas, pasarán de
asteriscos (***) a texto plano para que pueda leerlo.

Ahora usted puede utilizar KeePassX para almacenar sus contraseñas. Sin
embargo, antes de emocionarse demasiado usted debe hacer una última
cosa. Al cerrar KeePassX (elija File-\textgreater{}Quit) se le pregunta
si desea guardar los cambios que haya realizado.

\begin{figure}[htbp]
\centering
\includegraphics{mng_7.png}
\caption{Grabando los cambios}
\end{figure}

Pulse ``Sí''. Si es la primera vez que se utiliza KeePassX (o acaba de
crear una nueva base de datos), debe elegir un lugar para almacenar sus
contraseñas. De lo contrario, se guardará la información actualizada en
el archivo que ha creado anteriormente.

Si desea acceder a las contraseñas a continuación, debe abrir KeePassX y
se le pedirá la clave maestra. Después de escribir esto usted puede
agregar todas tus contraseñas de la base de datos y ver todas las
entradas. No es una buena idea abrir KeePassX y tenerlo abierto
permanentemente ya que alguien podría ver sus contraseñas si pueden
accede a su computadora. En lugar de entrar, en la práctica limítese a
abrirlo cuando lo necesite y luego ciérrelo de nuevo.

\section{Cifrado de contraseñas con KeePass en
Windows}\label{cifrado-de-contraseuxf1as-con-keepass-en-windows}

Después de instalar KeePass en Windows se puede encontrar en el menú de
aplicaciones. Inicie la aplicación y la siguiente ventana debe aparecer.

\begin{figure}[htbp]
\centering
\includegraphics{mng_8.png}
\caption{Lanzando KeePass}
\end{figure}

Se empieza haciendo una base de datos, el archivo que contendrá su
clave. En el menú seleccione \texttt{File\ \textgreater{}\ new}. Usted
tiene que elegir el nombre y la ubicación del archivo en la ventana de
diálogo siguiente. En este ejemplo llamamos a nuestra base de datos
\texttt{my\_password\_database}.

\begin{figure}[htbp]
\centering
\includegraphics{mng_9.png}
\caption{Nuestra base de datos}
\end{figure}

La siguiente pantalla le pedirá la contraseña maestra. Introdúzcala y
haga click en `OK'. Usted no tendrá que elegir otra cosa.

\begin{figure}[htbp]
\centering
\includegraphics{mng_10.png}
\caption{Contraseña maestra}
\end{figure}

La siguiente ventana le permite añadir opciones especiales de
configuración para su nueva base de datos. No es necesario modificar
nada. Haga clic en `Aceptar'.

\begin{figure}[htbp]
\centering
\includegraphics{mng_11.png}
\caption{Opciones de configuración}
\end{figure}

Ahora aparece la ventana principal de nuevo y vemos algunas categorías
de contraseñas por defecto en el lado izquierdo. Permite añadir una
nueva contraseña en la categoría `Internet'. Primero haga clic en la
palabra `Internet', luego pulse en el ícono de agregar entrada debajo de
la barra de menús.

\begin{figure}[htbp]
\centering
\includegraphics{mng_12.png}
\caption{Categorías de Internet}
\end{figure}

Aparecerá una ventana como la de abajo. Utilice los campos para dar una
descripción de esta contraseña especial, y por supuesto, ingresar la
propia contraseña. Cuando termine, haga click en `OK'.

\begin{figure}[htbp]
\centering
\includegraphics{mng_13.png}
\caption{Descripción de contraseñas}
\end{figure}

\section{Contraseñas cifradas con Keychain en Mac
OSX}\label{contraseuxf1as-cifradas-con-keychain-en-mac-osx}

Mac OSX viene preinstalado con el gestor de contraseñas `Keychain'.
Debido a su estrecha integración con la mayoría de los OS la mayoría de
las veces es casi imposible saber que existe. Pero de vez en cuando
usted tendrá una ventana pop-up en casi cualquier aplicación preguntando
`¿quiere guardar esta contraseña en Keychain?'. Esto sucede cuando se
agregan nuevas cuentas de correo electrónico a su cliente de correo,
cuando se accede a una red inalámbrica protegida, cuando introduce sus
datos en el cliente de chat, etc, etc, etc.

Básicamente lo que ocurre es que Mac OS X le ofrece a usted almacenar
todos los datos de usuario y contraseñas diferentes en un archivo
cifrado que se abre tan pronto como se inicie sesión en su cuenta. A
continuación, puede revisar su correo, iniciar sesión con su WiFi y
utilizar el cliente de chat sin tener que introducir sus datos de acceso
en todo momento una y otra vez. Este es un proceso totalmente
automatizado, pero si usted quiere ver lo que está almacenado, dónde lo
está, alterar contraseñas, o buscar una contraseña entonces tendrás que
abrir el programa Keychain.

Usted puede encontrar el programa de llavero en la carpeta Utilities,
que está dentro de la carpeta Applications.

\begin{figure}[htbp]
\centering
\includegraphics{mng_14.png}
\caption{Contraseñas cifradas}
\end{figure}

Cuando lo abra, verá que su `Login' de Keychain está desbloqueado y verá
todos los elementos contenidos en el mismo en la parte inferior derecha
de la ventana.

(Nota: la ventana aquí está vacía porque sería engañoso para el
propósito de este manual hacer una captura de pantalla de mis claves
personales y compartirlas con ustedes)

\begin{figure}[htbp]
\centering
\includegraphics{mng_15.png}
\caption{Claves personales}
\end{figure}

Puede hacer doble click en cualquiera de los elementos en Keychain para
poder ver los detalles y marque la casilla `Show password:' para ver la
contraseña asociada con el elemento.

\begin{figure}[htbp]
\centering
\includegraphics{mng_16.png}
\caption{Mostrando las contraseñas}
\end{figure}

Usted notará que se le pedirá su contraseña maestra o contraseña de
inicio de sesión para ver el elemento.

\begin{figure}[htbp]
\centering
\includegraphics{mng_17.png}
\caption{Contraseña maestra}
\end{figure}

Se puede acceder a modificar cualquiera de los elementos y también
utilizar el Keychain para guardar con seguridad las partes y piezas de
texto con las notas. Para ello haga click en las notas y elija `New
secure Note item' desde el menú archivo.

\chapter{Obtención, configuración y prueba de una cuenta
VPN}\label{obtenciuxf3n-configuraciuxf3n-y-prueba-de-una-cuenta-vpn}

En todos los sistemas VPN, existe una computadora configurada como un
servidor (en alguna país sin demasiadas restricciones), a la cual se
conectan uno o más clientes. La configuración del servidor está fuera
del alcance de este manual y la configuración de su sistema está
cubierto, en general, por su proveedor de VPN. Este servidor es uno de
los dos extremos del túnel cifrado. Es muy importante que la
organización que provee el servidor sea confiable y esté ubicada en un
país o región que también sea confiable. Para correr una VPN, se
necesita una cuenta en dicho servidor.

Por favor recuerde que cada cuenta puede usarse, a menudo, en un sólo
dispositivo a la vez. Si quiere usar una VPN con un teléfono móvil y una
computadora personal simultáneamente, es muy posible que necesite dos
cuentas.

\section{Una cuenta de un proveedor comercial de
VPN}\label{una-cuenta-de-un-proveedor-comercial-de-vpn}

Hay múltiples proveedores de VPN ahí afuera. Algunos le ofrecerán
probarlo gratis por un tiempo, otros comenzarán a cobrarle una tarifa
fija por mes. Busque un proveedor de VPN que ofrezca cuentas con OpenVPN
- una solución libre disponible para GNU/Linux, OS X y Windows, además
de Android e iOS.

Cuando elija un proveedor VPN debe considerar los siguientes puntos:

\begin{itemize}
\tightlist
\item
  Cuando menos información le sea pedida para registrar una cuenta
  mejor. Un proveedor de VPN verdaderamente preocupado por su privacidad
  sólo le pedirá una dirección de correo electrónico (¡haga una
  temporal!), nombre de usuario y contraseña. No se necesita más, a
  menos que el proveedor cree una base de datos de usuarios, que es muy
  probable que no quieran ser parte de ello.
\item
  Formas de pago que se utilizarán para pagar su suscripción: la
  transferencia de efectivo es probablemente el método más propenso a la
  privacidad, ya que no vincula su cuenta bancaria con su identificación
  en la red VPN. PayPal también puede ser una opción aceptable
  suponiendo que usted puede registrar y utilizar una cuenta temporal
  para cada pago. El pago a través de una transferencia bancaria o con
  tarjeta de crédito puede socavar gravemente su anonimato incluso más
  allá de la VPN.
\item
  Evite los proveedores de VPN que le obliguen a instalar su propio
  software cliente propietario. Existen soluciones libres disponibles
  para todas las plataformas, y tener que ejecutar un cliente
  ``especial'' es una clara señal de un servicio falso.
\item
  Evite el uso de VPN basada en PPTP, dicho protocolo presenta
  vulnerabilidades de seguridad. De hecho, si dos proveedores son
  iguales en todo, elija el que \emph{no le ofrezca} PPTP.
\item
  Busque un proveedor de VPN que está utilizando OpenVPN - una solución
  VPN libre y multiplataforma.
\item
  Puertas de salida en los países de su interés: poder elegir entre
  varios países le permite cambiar su contexto geopolítico y aparentar
  provenir de una parte diferente del mundo. ¡Tiene que ser consciente
  de los detalles de la legislación y las leyes de privacidad de ese
  país en particular!
\item
  Considere la política de anonimato con respecto a su tráfico: un
  proveedor de VPN seguro debe tener una política de no divulgación. La
  información personal, como nombre de usuario y los tiempos de
  conexión, tampoco se deben registrar.
\item
  Se deben admitir dentro de VPN a la gran mayoría de los protocolos de
  Internet.
\item
  Compare el precio con la calidad del servicio y su fiabilidad.
\item
  Investigue todos los problemas conocidos en cuanto al anonimato de los
  usuarios que el proveedor de VPN podría haber tenido en el pasado.
  Mire en línea, lea los foros y pregunte por ahí. No se deje tentar por
  nuevas ofertas o proveedores desconocidos, baratos o poco fiables.
\end{itemize}

Hay disponibles en varios sitios web comparaciones entre distintos
servicios VPN que lo pueden ayudar a seleccionar la mejor opción:

\begin{itemize}
\tightlist
\item
  \href{http://www.bestvpnservice.com/vpn-providers.php}{Best VPN
  Service}
\item
  \href{http://vpncreative.com/complete-list-of-vpn-providers}{VPN
  Creative}
\item
  \href{http://en.cship.org/wiki/VPN}{Cship}
\end{itemize}

\section{Configuración de su cliente
VPN}\label{configuraciuxf3n-de-su-cliente-vpn}

\begin{quote}
``OpenVPN {[}..{]} es una solución completa de software VPN SSL que
integra capacidades de servidor OpenVPN, capacidades de administración
simplificada, interfaz de usuario OpenVPN Connect, y paquetes de
software cliente de OpenVPN que se adaptan a GNU/Linux, OSX, Windows y
entornos. El servidor OpenVPN Access es compatible con una amplia gama
de configuraciones, incluyendo acceso remoto seguro y granular a la red
interna y a los recursos privados en la red y a las aplicaciones en la
nube con un control de acceso riguroso.''
(\url{http://openvpn.net/index.php/access-server/overview.html})
\end{quote}

Hay muchos estándares diferentes para configurar VPN, incluyendo PPTP,
LL2P/IPSec y \textbf{OpenVPN}. Varían en complejidad, nivel de seguridad
provisto y disponibilidad de sistemas operativos. No use PPTP porque
presenta importantes fallas de seguridad. En este manual nos
concentraremos en OpenVPN. Funciona en la mayoría de versiones de
GNU/Linux, OSX y Windows. OpenVPN se basa en TLS/SSL - usa el mismo tipo
de \textbf{cifrado} que usa HTTPS (HTTP segura) y una gran cantidad de
otros protocolos de cifrado. El cifrado de OpenVPN se basa en el
algoritmo de intercambio de claves \textbf{RSA}. Para poder establecer
una comunicación, tanto el servidor como el cliente necesitan claves RSA
públicas y privadas.

Una vez que obtiene el acceso a una cuenta VPN el servidor genera las
claves y usted simplemente necesita descargarlas del sitio web de su
proveedor o recibirlas por medio de un correo electrónico. Junto con sus
claves recibirá un certificado de raíz (*.ca) y un archivo de
configuración principal (*.conf o *.ovpn). En la mayoría de los casos
solamente se necesitan los siguientes archivos para configurar y correr
un cliente OpenVPN:

\begin{itemize}
\tightlist
\item
  \textbf{client.conf} (o client.ovpn) - archivo de configuración que
  incluye todos los parámetros necesarios. NOTA: en algunos casos las
  claves y los certificados pueden estar embebidos dentro del archivo de
  configuración principal. En tal caso los archivos mencionados más
  abajo no son necesarios.
\item
  \textbf{ca.crt} (excepto en el archivo de configuración) - certificado
  de autoridad raíz de su servidor VPN, usado para firmar y y comprobar
  otras claves emitidas por el proveedor.
\item
  \textbf{client.crt} (excepto en el archivo de configuración) - su
  certificado de cliente, le permite comunicarse con su servidor VPN.
\end{itemize}

En base a su configuración particular, su proveedor VPN puede requerirle
nombre de usuario y contraseña para autenticar su conexión. A menudo,
por conveniencia, el nombre de usuario y la contraseña pueden grabarse
en un archivo separado o agregado al archivo de configuración principal.
En otros casos, se usa autenticación basada en claves, que se almacenan
en un archivo separado:

\begin{itemize}
\tightlist
\item
  \textbf{client.key} (excepto en el archivo de configuración) - clave
  de autenticación de cliente, usada para autenticar el servidor VPN y
  establecer un canal de datos cifrado.
\end{itemize}

En la mayoría de los casos, a menos que sea necesario, no necesitará
cambiar nada en el archivo de configuración, y (¡téngalo por seguro!)
\textbf{¡nunca necesitará editar los archivos de certificación o las
claves!} Todos los proveedores VPN tienen instrucciones detalladas
acerca de la instalación. Lea y siga estas directrices para asegurarse
de que su cliente VPN está configurado correctamente.

NOTA: Por lo general, sólo está permitido el uso de una clave por
conexión, por lo que probablemente no debería estar usando las mismas en
dispositivos diferentes al mismo tiempo. Obtenga un nuevo conjunto de
claves para cada dispositivo que va a utilizar con una VPN, o intente
establecer un gateway VPN local (de un nivel más avanzado, no cubierto
aquí).

Descargue sus archivos de configuración y de claves de OpenVPN y
cópielos en un lugar seguro. Luego pase al capítulo siguiente.

\section{Configuración del cliente
OpenVPN}\label{configuraciuxf3n-del-cliente-openvpn}

En los capítulos siguientes se dan algunos ejemplos de configuración de
software cliente OpenVPN. En cualquier distribución de GNU/Linux utilice
su gestor de paquetes preferido e instale \textbf{openvpn} ** u
\textbf{openvpn-client}.

Si desea utilizar OpenVPN en Windows u OSX, consulte:

\begin{itemize}
\tightlist
\item
  \href{http://openvpn.se}{OpenVPN} (interfaz Windows)
\item
  \href{http://code.google.com/p/tunnelblick}{Tunnel Blick} (interfaz
  OSX)
\end{itemize}

\section{Advertencias\ldots{}¡Cuidado!}\label{advertenciascuidado}

Aunque un VPN ocultará su dirección IP, debido a la naturaleza de la
mayoría de los VPN los metadatos de su pila TCP/IP y otra información de
identificación puede ser enviada.

Esto puede parecer trivial, pero fíjese, un encabezado IP estándar tiene
20 bytes de tamaño, algunos de los cuales se llenan con información
obvia (4 bytes para la IP origen, 4 bytes para la IP destino, etc.) pero
algunos bytes del encabezado pueden tener otras opciones arbitrarias; el
encabezado TCP tiene al menos 20 bytes también, con el potencial para
otros 20. La configuración específica de estas opciones varía según el
sistema operativo, incluso según su versión, así como un simple paquete
SYN de TCP es a menudo suficiente para identificar el sistema operativo
en uso, la versión y otra información reveladora, como el tiempo de
actividad del sistema. Existen herramientas fácilmente disponibles
\url{http://lcamtuf.coredump.cx/p0f3/} que puede ser usado para obtener
la huella de esta información; como prueba, intente conectarse a un
servidor que ejecute esta herramienta con su conexión normal a Internet,
luego conéctese nuevamente a través de su VPN. Muy probablemente verá
que las huellas son idénticas, y que si su amigo se conecta su huella
será diferente.

Por eso, es importante que recuerde lo siguiente: * nadie irá a la
cárcel por usted, si su proveedor VPN es alcanzado por una requisitoria
legal para obtener información acerca de usted, ellos la brindarán.
Porque declaren que no mantienen registros de actividad no significa que
no los tengan. * las VPN proveen privacidad, no anonimato

\chapter{VPN en Ubuntu}\label{vpn-en-ubuntu}

Si usa Ubuntu, puede conectarse a una VPN usando el
\emph{NetworkManager} incorporado. Esta aplicación es capaz de
configurar redes con OpenVPN. No se debe usar PPTP por razones de
seguridad. Desafortunadamente al momento de escribir este texto, no hay
disponible en Ubuntu una interfaz L2TP. (Puede hacerse manualmente, pero
está más allá del alcance de este documento).

El siguiente ejemplo explicara como conectarse con un servidor OpenVPN.
En todos los casos supondremos que tiene una cuenta VPN.

\section{Preparación del Network Manager para redes
VPN}\label{preparaciuxf3n-del-network-manager-para-redes-vpn}

Existe una excelente utilidad de red para Ubuntu: Network Manager. Es la
misma utilidad que usted usa para configurar su red inalámbrica (o
cableada)que está ubicada habitualmente en la esquina superior derecha
de su pantalla (al lado del reloj). Esta herramienta también es capaz de
administrar VPN, pero antes de hacer esto, es necesario instalar algunas
extensiones.

\subsection{Instalación de la extensión OpenVPN para Network
Manager}\label{instalaciuxf3n-de-la-extensiuxf3n-openvpn-para-network-manager}

Para instalar los plugins para Network Manager usaremos el Ubuntu
Software Center.

\begin{enumerate}
\def\labelenumi{\arabic{enumi}.}
\tightlist
\item
  Abra el Ubuntu Software Center escribiendo ``software'' en la barra de
  búsqueda Unity
\end{enumerate}

\begin{figure}[htbp]
\centering
\includegraphics{vpn_ubuntu_001.png}
\caption{Abriendo el Ubuntu Software Center}
\end{figure}

\begin{enumerate}
\def\labelenumi{\arabic{enumi}.}
\setcounter{enumi}{1}
\tightlist
\item
  El Ubuntu Software Center habilita la búsqueda, instala y remueve
  software de su computadora. Haga click en la casilla de búsqueda en la
  esquina superior derecha de la ventana.
\end{enumerate}

\begin{figure}[htbp]
\centering
\includegraphics{vpn_ubuntu_002.png}
\caption{Casilla de búsqueda}
\end{figure}

\begin{enumerate}
\def\labelenumi{\arabic{enumi}.}
\setcounter{enumi}{2}
\tightlist
\item
  En la casilla de búsqueda, escriba ``network-manager-openvpn-gnome''
  (es una extensión que habilitará OpenVPN). Es necesario tipear los
  nombres completos. Estos paquetes incluyen todos los archivos
  necesarios para establecer una conexión VPN exitosa.
\end{enumerate}

\begin{figure}[htbp]
\centering
\includegraphics{vpn_ubuntu_003.png}
\caption{Buscando el software}
\end{figure}

\begin{enumerate}
\def\labelenumi{\arabic{enumi}.}
\setcounter{enumi}{3}
\tightlist
\item
  Ubuntu puede pedirle permisos adicionales para instalar el programa.
  Si este es su caso, tipee su contraseña y haga click en Authenticate.
  Una vez instalados los paquetes, puede cerrar la ventana del Software
  Center.
\end{enumerate}

\begin{figure}[htbp]
\centering
\includegraphics{vpn_ubuntu_004.png}
\caption{Instalando los paquetes necesarios}
\end{figure}

\begin{enumerate}
\def\labelenumi{\arabic{enumi}.}
\setcounter{enumi}{4}
\tightlist
\item
  Para comprobar si las extensiones se instalaron correctamente, haga
  click en NetworkManager (el ícono a la izquierda de su reloj del
  sistema) y seleccione VPN Connections \textgreater{} Configure VPN.
\end{enumerate}

\begin{figure}[htbp]
\centering
\includegraphics{vpn_ubuntu_005.png}
\caption{Abriendo VPN}
\end{figure}

\begin{enumerate}
\def\labelenumi{\arabic{enumi}.}
\setcounter{enumi}{5}
\tightlist
\item
  Haga click en Add bajo la pestaña de VPN.
\end{enumerate}

\begin{figure}[htbp]
\centering
\includegraphics{vpn_ubuntu_006.png}
\caption{Agregando una conexión VPN}
\end{figure}

\begin{enumerate}
\def\labelenumi{\arabic{enumi}.}
\setcounter{enumi}{6}
\tightlist
\item
  Si aparece un pop-up preguntando por el tipo de VPN y la opción de
  tecnología del túnel (OpenVPN) está disponible, esto significa que
  usted tiene instalada la extensión VPN correctamente. Si tiene lista
  la información de logueo a su VPN, puede continuar, en caso contrario
  debe adquirir una cuenta VPN de un proveedor. Si este es el caso,
  cancele y cierre el Network Manager.
\end{enumerate}

\begin{figure}[htbp]
\centering
\includegraphics{vpn_ubuntu_007.png}
\caption{Creación de una cuenta VPN}
\end{figure}

\section{Configuración de una red
OpenVPN}\label{configuraciuxf3n-de-una-red-openvpn}

Supondremos que su proveedor de VPN ya le ha entregado a usted sus
archivos de configuración. Esta información debería consistir en lo
siguiente:

\begin{itemize}
\tightlist
\item
  un archivo *.ovpn, ex. air.ovpn
\item
  El archivo ca.crt (específico de cada proveedor OpenVPN)
\item
  El archivo user.crt (este archivo es su certificado personal, usado
  para cifrado de datos)
\item
  El archivo: user.key (este archivo contiene su clave privada. Debería
  ser protegido cuidadosamente. Perder este archivo volverá insegura a
  su conexión)
\end{itemize}

En la mayoría de los casos su proveedor le enviará estos archivos a
usted en un archivo comprimido. Algunos proveedores de openvpn utilizan
nombres de usuario y contraseña de autenticación que no están cubiertos.

\begin{enumerate}
\def\labelenumi{\arabic{enumi}.}
\tightlist
\item
  Descomprima el archivo que ha descargado en una carpeta de su disco
  rígido rígido (por ejemplo: ``/home/{[}sunombredeusuario{]}/.vpn'').
  Debería tener ahora cuatro archivos. El archivo ``air.ovpn'' es el
  archivo de configuración que usted necesita importar en
  NetworkManager.
\end{enumerate}

\begin{figure}[htbp]
\centering
\includegraphics{vpn_ubuntu_008.png}
\caption{Archivo de configuración}
\end{figure}

\begin{enumerate}
\def\labelenumi{\arabic{enumi}.}
\setcounter{enumi}{1}
\tightlist
\item
  Para importar el archivo de configuración, abra NetworkManager y vaya
  a VPN Connections \textgreater{} Configure VPN.
\end{enumerate}

\begin{figure}[htbp]
\centering
\includegraphics{vpn_ubuntu_009.png}
\caption{Configurando VPN}
\end{figure}

\begin{enumerate}
\def\labelenumi{\arabic{enumi}.}
\setcounter{enumi}{2}
\tightlist
\item
  Bajo la pestaña VPN, pulse Import.
\end{enumerate}

\begin{figure}[htbp]
\centering
\includegraphics{vpn_ubuntu_010.png}
\caption{Importando el archivo}
\end{figure}

\begin{enumerate}
\def\labelenumi{\arabic{enumi}.}
\setcounter{enumi}{3}
\tightlist
\item
  Localice el archivo air.ovpn que ha descomprimido. Pulse Open.
\end{enumerate}

\begin{figure}[htbp]
\centering
\includegraphics{vpn_ubuntu_011.png}
\caption{Abriendo air.ovpn}
\end{figure}

\begin{enumerate}
\def\labelenumi{\arabic{enumi}.}
\setcounter{enumi}{4}
\tightlist
\item
  Se abrirá una nueva ventana. Deje todo como está y seleccione Apply.
\end{enumerate}

\begin{figure}[htbp]
\centering
\includegraphics{vpn_ubuntu_012.png}
\caption{Terminando la configuración}
\end{figure}

\begin{enumerate}
\def\labelenumi{\arabic{enumi}.}
\setcounter{enumi}{5}
\tightlist
\item
  ¡Felicitaciones! Su conexión VPN está lista para ser usada y debería
  aparecer en la lista de conexiones bajo la pestaña VPN. Ahora puede
  cerrar NetworkManager.
\end{enumerate}

\begin{figure}[htbp]
\centering
\includegraphics{vpn_ubuntu_013.png}
\caption{Cerrando Network Manager}
\end{figure}

\section{Uso de su nueva conexión
VPN}\label{uso-de-su-nueva-conexiuxf3n-vpn}

Ahora que ha configurado NetworkManager para conectarse a un servicio
VPN usando el cliente OpenVPN, puede usar su nueva conexión VPN para
eludir la censura en Internet. Para comenzar siga estos pasos:

\begin{enumerate}
\def\labelenumi{\arabic{enumi}.}
\tightlist
\item
  En el menú NetworkManager, seleccione su nueva conexión de VPN
  Connections.
\end{enumerate}

\begin{figure}[htbp]
\centering
\includegraphics{vpn_ubuntu_014.png}
\caption{Buscando la conexión VPN}
\end{figure}

\begin{enumerate}
\def\labelenumi{\arabic{enumi}.}
\setcounter{enumi}{1}
\tightlist
\item
  Espere que se establezca la conexión VPN. Cuando está conectado, un
  pequeño candado aparecerá justo arriba del ícono de NetworkManager,
  indicando que usted ahora está usando una conexión segura. Mueva el
  cursos sobre el ícono para confirmar que su conexión VPN está activa.
\end{enumerate}

\begin{figure}[htbp]
\centering
\includegraphics{vpn_ubuntu_015.png}
\caption{Conexión activa}
\end{figure}

\begin{enumerate}
\def\labelenumi{\arabic{enumi}.}
\setcounter{enumi}{2}
\item
  Compruebe su conexión, usando el método descripto en la sección
  ``Asegúrese que funciona'' en este capítulo.
\item
  Para desconectarse de su VPN, seleccione VPN Connections
  \textgreater{} Disconnect VPN en el menú NetworkManager. Ahora está
  usando su conexión normal nuevamente.
\end{enumerate}

\begin{figure}[htbp]
\centering
\includegraphics{vpn_ubuntu_016.png}
\caption{Desconectando VPN}
\end{figure}

\chapter{VPN en MacOSX}\label{vpn-en-macosx}

Configurar un servicio VPN en MacOSX es muy sencillo una vez que tiene
habilitada su cuenta. Supondremos que ya tiene en su poder las
credenciales suministradas por su proveedor de VPN, para establecer una
conexión L2TP/IPSec. Esta información debería contener lo siguiente:

\begin{itemize}
\tightlist
\item
  Nombre de usuario, por ejemplo \texttt{bill2}
\item
  Contraseña, por ejemplo \texttt{verysecretpassword}
\item
  servidor VPN, por ejemplo \texttt{tunnel.greenhost.nl}
\item
  Una clave pre-compartida o un certificado de máquina
\end{itemize}

\section{Configuración}\label{configuraciuxf3n-1}

\begin{enumerate}
\def\labelenumi{\arabic{enumi}.}
\item
  Antes de comenzar, por favor asegúrese de leer el párrafo ``prueba
  antes y después de configurar una cuenta'', para comprobar si su
  conexión trabaja correctamente.
\item
  Un servicio VPN se configura en network settings, que son accesibles
  por medio de ``System Preferences..'' en el menú de Apple.
\end{enumerate}

\begin{figure}[htbp]
\centering
\includegraphics{vpn_osx_02.jpg}
\caption{Configurando VPN}
\end{figure}

\begin{enumerate}
\def\labelenumi{\arabic{enumi}.}
\setcounter{enumi}{2}
\tightlist
\item
  Abra Network preferences.
\end{enumerate}

\begin{figure}[htbp]
\centering
\includegraphics{vpn_osx_03.jpg}
\caption{Preferencias de red}
\end{figure}

\begin{enumerate}
\def\labelenumi{\arabic{enumi}.}
\setcounter{enumi}{3}
\tightlist
\item
  OSX usa este sistema para bloquear la pantalla. Para agregar una VPN
  es necesario desbloquearla, haciendo doble click en el candado de la
  parte inferior izquierda de su pantalla.
\end{enumerate}

\begin{figure}[htbp]
\centering
\includegraphics{vpn_osx_04.jpg}
\caption{Desbloqueo de VPN}
\end{figure}

\begin{enumerate}
\def\labelenumi{\arabic{enumi}.}
\setcounter{enumi}{4}
\tightlist
\item
  Ingrese sus credenciales.
\end{enumerate}

\begin{figure}[htbp]
\centering
\includegraphics{vpn_osx_05.jpg}
\caption{Ingresando credenciales}
\end{figure}

\begin{enumerate}
\def\labelenumi{\arabic{enumi}.}
\setcounter{enumi}{5}
\tightlist
\item
  Ahora puede agregar una nueva red, haciendo click en el signo ``+''.
\end{enumerate}

\begin{figure}[htbp]
\centering
\includegraphics{vpn_osx_06.jpg}
\caption{Agregando una nueva red}
\end{figure}

\begin{enumerate}
\def\labelenumi{\arabic{enumi}.}
\setcounter{enumi}{6}
\tightlist
\item
  En la ventana emergente deberá especificar el tipo de conexión. En
  este caso elija una interfaz VPN con L2TP sobre IPSec. Es el sistema
  más común. No olvide darle a la conexión un nombre bonito.
\end{enumerate}

\begin{figure}[htbp]
\centering
\includegraphics{vpn_osx_07.jpg}
\caption{Estableciendo un nombre}
\end{figure}

\begin{enumerate}
\def\labelenumi{\arabic{enumi}.}
\setcounter{enumi}{7}
\tightlist
\item
  Sigamos con los datos de conexión. Complete el nombre del servidor del
  proveedor y su nombre de usuario (denominado `Account Name'). Hecho
  esto, pulse el botón ``Authentication Settings\ldots{}''.
\end{enumerate}

\begin{figure}[htbp]
\centering
\includegraphics{vpn_osx_08.jpg}
\caption{Configurando la autenticación}
\end{figure}

\begin{enumerate}
\def\labelenumi{\arabic{enumi}.}
\setcounter{enumi}{8}
\tightlist
\item
  En la nueva ventana emergente puede especificar la información de la
  conexión. Esta es la manera en que el usuario y la máquina se
  autentifican. El usuario generalmente se autentifica con una
  contraseña, aunque se podrían usar otros métodos. La autentificación
  es realizada a menudo por un secreto compartido (Pre-Shared-Key/PSK),
  pero también muy a menudo mediante el uso de un certificado. En este
  caso se utiliza el método del secreto compartido. Hecho esto, haga
  click en OK.
\end{enumerate}

\begin{figure}[htbp]
\centering
\includegraphics{vpn_osx_09.jpg}
\caption{Configurando la autenticación}
\end{figure}

\begin{enumerate}
\def\labelenumi{\arabic{enumi}.}
\setcounter{enumi}{9}
\tightlist
\item
  Volvamos a la pantalla de red. El próximo paso es muy importante, por
  eso haga click en ``Advanced\ldots{}''
\end{enumerate}

\begin{figure}[htbp]
\centering
\includegraphics{vpn_osx_09b.jpg}
\caption{Pantalla de red}
\end{figure}

\begin{enumerate}
\def\labelenumi{\arabic{enumi}.}
\setcounter{enumi}{10}
\tightlist
\item
  En la nueva ventana emergente verá la opción para enrutar todo el
  tráfico a través de una conexión VPN. Habilítela para cifrar todo su
  tráfico.
\end{enumerate}

\begin{figure}[htbp]
\centering
\includegraphics{vpn_osx_10.jpg}
\caption{Cifrando el tráfico}
\end{figure}

\begin{enumerate}
\def\labelenumi{\arabic{enumi}.}
\setcounter{enumi}{11}
\tightlist
\item
  Bueno, hemos terminado. ¡Ahora conéctese!
\end{enumerate}

\begin{figure}[htbp]
\centering
\includegraphics{vpn_osx_11.jpg}
\caption{Conectándose\ldots{}}
\end{figure}

\begin{enumerate}
\def\labelenumi{\arabic{enumi}.}
\setcounter{enumi}{12}
\tightlist
\item
  Aparecerá una ventana emergente. Para confirmar los cambios, sólo
  pulse ``Apply''
\end{enumerate}

\begin{figure}[htbp]
\centering
\includegraphics{vpn_osx_12.jpg}
\caption{Confirmando cambios}
\end{figure}

\begin{enumerate}
\def\labelenumi{\arabic{enumi}.}
\setcounter{enumi}{13}
\tightlist
\item
  Después de unos pocos segundos, en el lado izquierdo la conexión debe
  tornarse verde. Si esto sucede, entonces ¡ya está conectado!
\end{enumerate}

\begin{figure}[htbp]
\centering
\includegraphics{vpn_osx_13.jpg}
\caption{¡Conexión establecida!}
\end{figure}

\begin{enumerate}
\def\labelenumi{\arabic{enumi}.}
\setcounter{enumi}{14}
\tightlist
\item
  Ok, ahora pruebe su conexión VPN en Windows ==============
\end{enumerate}

Configurar un servicio VPN en MacOSX es muy sencillo una vez que tiene
habilitada su cuenta. Supondremos que ya tiene en su poder las
credenciales suministradas por su proveedor de VPN, para establecer una
conexión L2TP/IPSec. Esta información debería contener lo siguiente:

\begin{itemize}
\tightlist
\item
  Nombre de usuario, por ejemplo \texttt{bill2}
\item
  Contraseña, por ejemplo \texttt{verysecretpassword}
\item
  servidor VPN, por ejemplo \texttt{tunnel.greenhost.nl}
\item
  Una clave pre-compartida o un certificado de máquina
\end{itemize}

\section{Setup}\label{setup}

\begin{enumerate}
\def\labelenumi{\arabic{enumi}.}
\item
  Antes de comenzar, por favor asegúrese de leer el párrafo ``prueba
  antes y después de configurar una cuenta'', para comprobar si su
  conexión trabaja correctamente.
\item
  Necesitamos ir al ``Network and Sharing Center'' de Windows para crear
  una nueva conexión VPN. Nosotros podemos acceder fácilmente haciendo
  click en el ícono de red cercano al reloj de sistema.
\end{enumerate}

\begin{figure}[htbp]
\centering
\includegraphics{vpn_windows_01.jpg}
\caption{Creación de una cuenta VPN}
\end{figure}

\begin{enumerate}
\def\labelenumi{\arabic{enumi}.}
\setcounter{enumi}{2}
\tightlist
\item
  Aparecerá el ``Network and Sharing Center''. Revise la información
  referida a su red actual. Seleccione ``Connect to a network'' para
  añadir una conexión VPN.
\end{enumerate}

\begin{figure}[htbp]
\centering
\includegraphics{vpn_windows_02.jpg}
\caption{Añadiendo una cuenta VPN}
\end{figure}

\begin{enumerate}
\def\labelenumi{\arabic{enumi}.}
\setcounter{enumi}{3}
\tightlist
\item
  El asistente de configuración aparecerá. Seleccione la opción
  ``connect to a workplace'', que es el nombre dado por Microsoft a una
  conexión VPN.
\end{enumerate}

\begin{figure}[htbp]
\centering
\includegraphics{vpn_windows_03.jpg}
\caption{Asistente de configuración de VPN}
\end{figure}

\begin{enumerate}
\def\labelenumi{\arabic{enumi}.}
\setcounter{enumi}{4}
\tightlist
\item
  La próxima pantalla le preguntará si desea usar su conexión a Internet
  o una antigua conexión por línea telefónica para conectarse a una VPN.
  Elija la primera opción.
\end{enumerate}

\begin{figure}[htbp]
\centering
\includegraphics{vpn_windows_04.jpg}
\caption{Seleccionando la conexión}
\end{figure}

\begin{enumerate}
\def\labelenumi{\arabic{enumi}.}
\setcounter{enumi}{5}
\tightlist
\item
  La próxima pantalla le pedirá los detalles de su conexión. Ingrese
  aquí el servidor de su proveedor de VPN (denominado ``Internet
  address'' en este diálogo). En la parte inferior, por favor marque la
  casilla ``No conectarse ahora; sólo configurar''. Con esta opción, la
  conexión se guarda automáticamente y es más fácil controlar la
  configuración extra. Hecho esto, pulse el botón ``next''
\end{enumerate}

\begin{figure}[htbp]
\centering
\includegraphics{vpn_windows_05.jpg}
\caption{Detalles de conexión}
\end{figure}

\begin{enumerate}
\def\labelenumi{\arabic{enumi}.}
\setcounter{enumi}{6}
\tightlist
\item
  Lo siguiente es su nombre de usuario y contraseña. Sólo tiene que
  ingresar los datos recibidos de su proveedor de VPN. Si falla la
  conexión, Windows los olvida. Así que recuérdelos, tal vez los
  necesite más tarde. Después de esto, pulse ``create''.
\end{enumerate}

\begin{figure}[htbp]
\centering
\includegraphics{vpn_windows_06.jpg}
\caption{Datos del usuaario}
\end{figure}

\begin{enumerate}
\def\labelenumi{\arabic{enumi}.}
\setcounter{enumi}{7}
\tightlist
\item
  Ya está disponible su conexión, si hace click en el ícono de red
  nuevamente, verá una nueva opción en el menú de red, el nombre de su
  conexión VPN. Conéctese haciendo click sobre ella.
\end{enumerate}

\begin{figure}[htbp]
\centering
\includegraphics{vpn_windows_07.jpg}
\caption{Opciones de conexión}
\end{figure}

\begin{enumerate}
\def\labelenumi{\arabic{enumi}.}
\setcounter{enumi}{8}
\tightlist
\item
  Pulse ``connect''
\end{enumerate}

\begin{figure}[htbp]
\centering
\includegraphics{vpn_windows_08.jpg}
\caption{Conectándose}
\end{figure}

\begin{enumerate}
\def\labelenumi{\arabic{enumi}.}
\setcounter{enumi}{9}
\tightlist
\item
  Aparecerá un diálogo de conexión VPN. Se le ofrece la oportunidad de
  revisar su configuración y conexión.Puede intentas conectarse, Windows
  tratará de descubrir el resto de la configuración automáticamente.
  Desafortunadamente, no siempre funciona, por eso, si no le es muy
  trabajoso, pulse el botón ``properties''.
\end{enumerate}

\begin{figure}[htbp]
\centering
\includegraphics{vpn_windows_09.jpg}
\caption{Revisando la configuración}
\end{figure}

\begin{enumerate}
\def\labelenumi{\arabic{enumi}.}
\setcounter{enumi}{10}
\tightlist
\item
  La ventana de propiedades aparecerá. La página más importante es
  ``Security'', haga click en la pestaña de seguridad para abrirla.
\end{enumerate}

\begin{figure}[htbp]
\centering
\includegraphics{vpn_windows_10.jpg}
\caption{Seguridad}
\end{figure}

\begin{enumerate}
\def\labelenumi{\arabic{enumi}.}
\setcounter{enumi}{11}
\tightlist
\item
  En la pestaña de seguridad puede especificar el tipo de VPN,
  normalmente L2TP/IPSec. No use PPTP que posee varias vulnerabilidades
  de seguridad. Para L2TP/IPSec eche un vistazo a Advanced settings.
\end{enumerate}

\begin{figure}[htbp]
\centering
\includegraphics{vpn_windows_11.jpg}
\caption{Tipo de VPN}
\end{figure}

\begin{enumerate}
\def\labelenumi{\arabic{enumi}.}
\setcounter{enumi}{12}
\tightlist
\item
  En la ventana de Advanced Settings, puede especificar si está usando
  una clave pre-compartida o un certificado. Esto depende de su
  proveedor de VPN. Si ha recibido una clave pre compartida, seleccione
  esta opción y complétela con este valor. Presione OK, y volverá a la
  pantalla anterior. Pulse OK nuevamente.
\end{enumerate}

\begin{figure}[htbp]
\centering
\includegraphics{vpn_windows_12.jpg}
\caption{Configuración avanzada}
\end{figure}

\begin{enumerate}
\def\labelenumi{\arabic{enumi}.}
\setcounter{enumi}{13}
\tightlist
\item
  De regreso a la ventana de conexión intente conectarse ahora. Complete
  con su nombre de usuario y contraseña.
\end{enumerate}

\begin{figure}[htbp]
\centering
\includegraphics{vpn_windows_13.jpg}
\caption{Datos del usuario}
\end{figure}

\begin{enumerate}
\def\labelenumi{\arabic{enumi}.}
\setcounter{enumi}{14}
\tightlist
\item
  Aparecerá una ventana emergente de conexión
\end{enumerate}

\begin{figure}[htbp]
\centering
\includegraphics{vpn_windows_14.jpg}
\caption{Ventana de conexión}
\end{figure}

\begin{enumerate}
\def\labelenumi{\arabic{enumi}.}
\setcounter{enumi}{15}
\tightlist
\item
  ¡Online! No se olvide de comprobar que su VPN esté trabajando
  correctamente.
\end{enumerate}

\chapter{Asegúrese que funcione}\label{aseguxfarese-que-funcione}

Probablemente una de las primeras cosas que debería asegurarse después
que la conexión VPN se ha establecido correctamente es si sus datos
realmente pasan a través de la red VPN. La prueba más simple (y fiable)
es comprobar cuál es su dirección IP ``externa'' que está exponiendo en
Internet.

Existen numerosos sitios web en línea que puede reportar su dirección IP
y su ubicación geográfica (también llamada geolocalización). Muchos
motores de búsqueda reportarán su dirección IP si busca ``Mi IP'', pero
también puede usar servidores dedicados como \url{http://www.myip.se} y
\href{http://ipchicken.com}{http://www.ipchicken.com}.

Verifique su dirección IP antes de conectarse a su VPN. Una vez
conectado a su VPN, la dirección IP pública de su computadora debería
cambiar por el brindado por su servidor VPN, y su geolocalización
debería cambiar al lugar en donde su servidor VPN está localizado.

Una vez que sepa que su IP externa se ha cambiado a la IP de su servidor
VPN, usted puede estar seguro de que su comunicación está cifrada.

\chapter{Instalando TrueCrypt}\label{instalando-truecrypt}

TrueCrypt puede ser instalado en Windows, GNU/Linux, o Mac OSX. Los
archivos de instalación están disponibles en su página web
\url{http://www.truecrypt.org/downloads}

La sección siguiente explica detalladamente como instalar TrueCrypt en
su computadora para diversos sistemas operativos, comenzando con Ubuntu.

\section{Instalación en Ubuntu}\label{instalaciuxf3n-en-ubuntu}

TrueCrypt no está disponible en los repositorios estándares de Ubuntu.
Esto significa que no puede usar el centro de software Ubuntu o
\emph{apt-get} (el método de la línea de comandos para instalar software
en Ubuntu). Además, debería visitar primeramente la página de descargas
(\url{http://www.truecrypt.org/downloads}.

Usted verá primero un menú desplegable bajo el encabezado Linux.

\begin{figure}[htbp]
\centering
\includegraphics{tc_001.png}
\caption{Descarga de TrueCrypt}
\end{figure}

Del menú desplegable `(Select a package)' puede elegir una entre cuatro
opciones:

\begin{figure}[htbp]
\centering
\includegraphics{tc_002.png}
\caption{Seleccionando paquetes}
\end{figure}

Esta es una versión para instalar desde la consola - la que debe elegir
si tiene habilidades técnicas y no quiere usar interfaces gráficas de
usuario o si desea correr TrueCrypt en una máquina que sólo posea una
terminal (línea de comandos o `shell') de acceso (como un servidor
remoto por ejemplo).

Suponiendo que usted ejecutará TrueCrypt en su computadora personal la
mejor opción es la `standard', la más sencilla, que le proporcionará una
agradable interfaz de usuario. Debe elegir entre las dos opciones
disponibles la más adecuada para la \emph{arquitectura} de su máquina.
¿No sabe qué significa? Bueno, esto se relaciona básicamente con el tipo
de hardware del procesador que posee su computadora, las opciones son
32-bits o 64-bits. Desafortunadamente Ubuntu no facilita información si
no dispone de algunos conocimientos. Necesita abrir una desde el menú
Aplicaciones-\textgreater{}Accesorios y tipear

\begin{verbatim}
uname -a
\end{verbatim}

La salida será algo como
\texttt{Linux\ bigsy\ 2.6.32-30-generic\ \#59-Ubuntu\ SMP\ Tue\ Mar\ 1\ 21:30:46\ UTC\ 2011\ x86\_64\ GNU/Linux}.
En esta instancia usted puede ver que la arquitectura es de 64-bit
(\texttt{x86\_64}). En este ejemplo debería elegir la opción `Standard -
64-bit (x64)'. Si lee \texttt{i686} en algún en la salida del comando
uname entonces debería elegir la opción restante para descargar.

Una vez elegida presione el botón `download' y grabe el archivo en su
computadora.

El proceso de instalación aún no ha terminado. El archivo que ha
descargado está comprimido (para acelerar la descarga) y debe
descomprimirlo antes de instalarlo. Afortunadamente Ubuntu lo hace muy
fácilmente - simplemente vaya al archivo en su computadora y haga click
con el botón derecho sobre él y elija `Extraer aquí'.

\begin{figure}[htbp]
\centering
\includegraphics{tc_003.png}
\caption{Extracción del archivo}
\end{figure}

Verá que aparece un nuevo archivo cerca del original comprimido:

\begin{figure}[htbp]
\centering
\includegraphics{tc_004.png}
\caption{Archivo extraído}
\end{figure}

¡Casi hemos terminado! Ahora haga click derecho sobre el nuevo archivo y
elija `open':

\begin{figure}[htbp]
\centering
\includegraphics{tc_005.png}
\caption{Abriendo el archivo}
\end{figure}

Si todo va bien verá una ventana abierta como esta:

\begin{figure}[htbp]
\centering
\includegraphics{tc_006.png}
\caption{Ventana de instalación}
\end{figure}

Elija `Run' y verá lo siguiente:

\begin{figure}[htbp]
\centering
\includegraphics{tc_007.png}
\caption{Instalando TrueCrypt}
\end{figure}

Ahora estamos llegando a alguna parte \ldots{} oprima el botón `Instalar
TrueCrypt. Se le mostrará un acuerdo de usuario. En la parte inferior
elija ``I accept and agree to be bound by the license terms'' (suena
serio). A continuación, se muestra otra pantalla de información que le
dice que usted puede instalar TrueCrypt. Pulse 'OK' y luego se le pedirá
la contraseña para instalar software en su ordenador. Introdúzcala y
entonces por fin verá una pantalla como ésta:

\begin{figure}[htbp]
\centering
\includegraphics{tc_008.png}
\caption{Instalación finalizada}
\end{figure}

Créalo o no ya está hecho\ldots{}TrueCrypt está instalado y usted puede
acceder desde el menú
Aplicaciones-\textgreater{}accesorios\ldots{}cierre la ventana de
configuración. Ahora vaya al capítulo Uso de TrueCrypt.

\section{Instalación en OSX}\label{instalaciuxf3n-en-osx}

\begin{enumerate}
\def\labelenumi{\arabic{enumi}.}
\tightlist
\item
  Para instalar TrueCrypt en OSX primeramente visite la
  \href{http://www.truecrypt.org/downloads}{página de descarga} y
  presione el botón de descarga en la sección OSX.
\end{enumerate}

\begin{figure}[htbp]
\centering
\includegraphics{tc_009.jpg}
\caption{Página de descargas}
\end{figure}

\begin{enumerate}
\def\labelenumi{\arabic{enumi}.}
\setcounter{enumi}{1}
\tightlist
\item
  Finalizada la descarga busque el archivo .dmg y ábralo para acceder al
  paquete de instalación
\end{enumerate}

\begin{figure}[htbp]
\centering
\includegraphics{tc_010.jpg}
\caption{Abriendo el archivo}
\end{figure}

\begin{enumerate}
\def\labelenumi{\arabic{enumi}.}
\setcounter{enumi}{2}
\tightlist
\item
  Abra el paquete de instalación, y presione en algún lugar del diálogo.
\end{enumerate}

\begin{figure}[htbp]
\centering
\includegraphics{tc_011.jpg}
\caption{Abriendo el paquete de instalación}
\end{figure}

\begin{enumerate}
\def\labelenumi{\arabic{enumi}.}
\setcounter{enumi}{3}
\tightlist
\item
  Elija la instalación estándar. (usted puede elegir una instalación
  personalizada y no seleccionar FUSE, pero no tiene sentido hacerlo. Lo
  necesita)
\end{enumerate}

\begin{figure}[htbp]
\centering
\includegraphics{tc_012.jpg}
\caption{Instalando TrueCrypt}
\end{figure}

\begin{enumerate}
\def\labelenumi{\arabic{enumi}.}
\setcounter{enumi}{4}
\tightlist
\item
  Terminada la instalación podrá hallar el programa en su carpeta de
  aplicaciones
\end{enumerate}

\begin{figure}[htbp]
\centering
\includegraphics{tc_013.jpg}
\caption{Instalación finalizada}
\end{figure}

\section{Instalación en Windows}\label{instalaciuxf3n-en-windows}

Para instalar TrueCrypt en Windows primeramente visite la página de
descarga (\url{http://www.truecrypt.org/downloads}) y presione el botón
de descarga de la sección de Windows.

Descárguelo a su computadora y haga doble click en el archivo. Verá un
acuerdo de usuario.

Haga click en `I accept and agree to be bound by the license terms' y
presione `Accept'.

Salga de la pantalla anterior con los valores por defecto y presione
`Next \textgreater{}', aparecerá la ventana de opciones de
configuración:

Puede dejar los valores predeterminados. Si desea instalar TrueCrypt
sólo para usted, entonces no se recomienda seleccionar la opción
``Instalar para todos los usuarios''. Sin embargo, si va a instalar esto
en su propia máquina y nadie más usa la computadora, entonces esto no es
necesario. Puede que también desee considerar la instalación de
TrueCrypt en una carpeta distinta de la predeterminada. En este caso,
haga clic en ``Examinar'' y elija otra ubicación. Cuando haya terminado,
haga clic en ``Instalar'' y el proceso continuará:

Cuando se complete la instalación aparecerá una confirmación de se ha
realizado correctamente. Cierre esta ventana y haga clic en
``Finalizar''. Ahora continúe con el capítulo sobre el uso de TrueCrypt.

\chapter{Uso de TrueCrypt}\label{uso-de-truecrypt}

Las siguientes instrucciones le indican paso a paso cómo crear, montar y
usar un volumen TrueCrypt.

\section{Crear un contenedor
TrueCrypt}\label{crear-un-contenedor-truecrypt}

\begin{enumerate}
\def\labelenumi{\arabic{enumi}.}
\item
  Instale TrueCrypt. Luego ejecútelo mediante

  \begin{itemize}
  \tightlist
  \item
    haciendo doble click en el archivo TrueCrypt.exe en Windows
  \item
    abriendo
    Aplicaciones-\textgreater{}Accesorios-\textgreater{}TrueCrypt en
    Ubuntu
  \item
    abriéndolo en MacOSX haciendo click en Go \textgreater{}
    Applications. Busque TrueCrypt en la carpeta de aplicaciones y haga
    doble click en él.
  \end{itemize}
\item
  Cuando aparezca la ventana principal de TrueCrypt seleccione Create
  Volume.
\end{enumerate}

\begin{figure}[htbp]
\centering
\includegraphics{using_tc_001.png}
\caption{Creando un volumen}
\end{figure}

\begin{enumerate}
\def\labelenumi{\arabic{enumi}.}
\setcounter{enumi}{2}
\tightlist
\item
  Aparecerá en pantalla el asistente de creación de volumen TrueCrypt.
\end{enumerate}

\begin{figure}[htbp]
\centering
\includegraphics{using_tc_002.png}
\caption{Asistente}
\end{figure}

Elija dónde crear el volumen TrueCrypt. Puede ser dentro de un archivo,
que será llamado contenedor, en una partición o disco. Para lo siguiente
supondremos que ha elegido la primera opción, crearlo dentro de un
archivo.

Haga click en next

\begin{enumerate}
\def\labelenumi{\arabic{enumi}.}
\setcounter{enumi}{3}
\tightlist
\item
  Ahora debe elegir crear un volumen estándar o uno oculto. Nosotros
  crearemos un volumen estándar.
\end{enumerate}

\begin{figure}[htbp]
\centering
\includegraphics{using_tc_003.png}
\caption{Tipo de volumen}
\end{figure}

Haga click en next

\begin{enumerate}
\def\labelenumi{\arabic{enumi}.}
\setcounter{enumi}{4}
\tightlist
\item
  Ahora debe especificar el archivo contenedor del volumen TrueCrypt.
  Nótese que puede ser cualquiera, y podrá ser movido o borrado como
  cualquier archivo normal.
\end{enumerate}

\begin{figure}[htbp]
\centering
\includegraphics{using_tc_004.png}
\caption{Archivo contenedor}
\end{figure}

Haga click en Select File.

El selector de archivo estándar aparecerá en pantalla (el asistente
permanece abierto por detrás). Ahora necesita navegar hasta la carpeta
que el archivo debería haber creado dentro entonces escriba en el campo
`nombre' el nombre del archivo que usted desea crear.

\begin{figure}[htbp]
\centering
\includegraphics{using_tc_005.png}
\caption{Escribiendo el nombre del archivo}
\end{figure}

Nosotros crearemos nuestro volumen TrueCrypt dentro de la carpeta
`adam/true' y el nombre del archivo del volumen (contenedor) será
`myencryptedfile'. Usted puede, por supuesto, elegir cualquier otro
archivo y cualquier otro soporte (por ejemplo, un pendrive). Observe que
el archivo `myencryptedfile' aún no existe - TrueCrypt lo creará.
Presione `Save' cuando esté listo. La ventana de selección de archivo se
cerrará.

\textbf{IMPORTANTE:} Note que TrueCrypt no cifrará ningún archivo
existente. Si selecciona un archivo existente en este paso, será
sobreescrito y reemplazado por el volumen creado (los datos se
perderán). Podrá cifrar archivos existentes más tarde o moviéndolos al
volumen TrueCrypt.

\begin{enumerate}
\def\labelenumi{\arabic{enumi}.}
\setcounter{enumi}{5}
\item
  En la ventana del asistente de configuración (que estaba corriendo en
  el fondo) haga click en next.
\item
  Elija un algoritmo de cifrado y un algoritmo hash para el volumen.
\end{enumerate}

\begin{figure}[htbp]
\centering
\includegraphics{using_tc_006.png}
\caption{Selección del algoritmo de cifrado}
\end{figure}

El manual TrueCrypt sugiere que si usted no está seguro de su elección,
utilice la configuración predeterminada y haga clic en Siguiente (para
más información sobre cada ajuste echar un vistazo a la página web de
documentación de TrueCrypt).

\begin{enumerate}
\def\labelenumi{\arabic{enumi}.}
\setcounter{enumi}{7}
\tightlist
\item
  Ahora elija el tamaño de su contenedor. Debería estar bien con 1
  megabyte pero en el ejemplo nosotros ingresamos `20'.
\end{enumerate}

\begin{figure}[htbp]
\centering
\includegraphics{using_tc_007.png}
\caption{Fijando el tamaño del contenedor}
\end{figure}

Usted puede, por supuesto, especificar un tamaño diferente. Luego, haga
click en next.

\begin{enumerate}
\def\labelenumi{\arabic{enumi}.}
\setcounter{enumi}{8}
\tightlist
\item
  Este paso es realmente muy importante, elija una contraseña.
\end{enumerate}

La información mostrada en la ventana del asistente le dirá si su
contraseña es buena, debería leerla cuidadosamente.

Elija una contraseña fuerte, escríbala en el primer campo de entrada.
Luego, repítala en el campo que está por debajo.

\begin{figure}[htbp]
\centering
\includegraphics{using_tc_008.png}
\caption{Estableciendo contraseñas}
\end{figure}

Presione entonces next.

\begin{enumerate}
\def\labelenumi{\arabic{enumi}.}
\setcounter{enumi}{9}
\tightlist
\item
  Elija el formato de su partición (este paso no está disponible en
  Windows o OSX). Si usa Ubuntu puede elegir un tipo de archivo
  GNU/linux o FAT (Windows) por simplicidad use la opción por default.
\end{enumerate}

\begin{figure}[htbp]
\centering
\includegraphics{using_tc_009.png}
\caption{Formato de la partición}
\end{figure}

Luego presione next.

\begin{enumerate}
\def\labelenumi{\arabic{enumi}.}
\setcounter{enumi}{10}
\tightlist
\item
  En este momento TrueCrypt intentará generar información aleatoria para
  poder cifrar su contenedor. Por un minuto mueva su ratón tan
  aleatoriamente como le sea posible. Esto aumenta considerablemente su
  seguridad incrementando la fortaleza criptográfica de su clave de
  cifrado.
\end{enumerate}

\begin{figure}[htbp]
\centering
\includegraphics{using_tc_010.png}
\caption{Cifrando el contenedor\ldots{}}
\end{figure}

Luego haga click en Format.

TrueCrypt ahora creará un archivo en la carpeta que usted eligió. Este
archivo será un contenedor TrueCrypt, y contendrá un volumen TrueCrypt
cifrado. Tomará algún tiempo dependiendo del tamaño del volumen. Cuando
finalice debería aparecer:

\begin{figure}[htbp]
\centering
\includegraphics{using_tc_011.png}
\caption{Cifrado terminado}
\end{figure}

Cierre la casilla de diálogos.

\begin{enumerate}
\def\labelenumi{\arabic{enumi}.}
\setcounter{enumi}{11}
\tightlist
\item
  ¡Bien hecho! Ha creado satisfactoriamente un volumen TrueCrypt
  (archivo contenedor).
\end{enumerate}

Cierre la ventana del asistente de creación de volúmenes TrueCrypt.

\section{Montando el volumen cifrado}\label{montando-el-volumen-cifrado}

\begin{enumerate}
\def\labelenumi{\arabic{enumi}.}
\tightlist
\item
  Abra nuevamente TrueCrypt.
\end{enumerate}

2.Asegúrese de elegir uno de los `Slots' (no interesa cuál - puede optar
por la opción por defecto, el primero de la lista). Haga click en Select
File.

\begin{figure}[htbp]
\centering
\includegraphics{using_tc_012.png}
\caption{Eligiendo un slot}
\end{figure}

Aparecerá la ventana del selector de archivo estándar.

\begin{enumerate}
\def\labelenumi{\arabic{enumi}.}
\setcounter{enumi}{2}
\tightlist
\item
  En el selector de archivo, navegue al archivo contenedor creado
  anteriormente y elíjalo.
\end{enumerate}

\begin{figure}[htbp]
\centering
\includegraphics{using_tc_013.png}
\caption{Selector de archivo}
\end{figure}

Haga click en Open (en la ventana del selector de archivo).

La ventana desaparecerá.

\begin{enumerate}
\def\labelenumi{\arabic{enumi}.}
\setcounter{enumi}{3}
\tightlist
\item
  En la ventana principal de TrueCrypt, haga click en Mount.
\end{enumerate}

\begin{figure}[htbp]
\centering
\includegraphics{using_tc_014.png}
\caption{Montando el volumen}
\end{figure}

Aparecerá una ventana de diálogo de la contraseña.

\begin{enumerate}
\def\labelenumi{\arabic{enumi}.}
\setcounter{enumi}{4}
\tightlist
\item
  Ingrese su contraseña.
\end{enumerate}

\begin{figure}[htbp]
\centering
\includegraphics{using_tc_015.png}
\caption{Ingresando la contraseña}
\end{figure}

\begin{enumerate}
\def\labelenumi{\arabic{enumi}.}
\setcounter{enumi}{5}
\tightlist
\item
  Presione OK para aceptar.
\end{enumerate}

TrueCrypt procederá a montar el volumen si la contraseña es correcta.

\begin{figure}[htbp]
\centering
\includegraphics{using_tc_016.png}
\caption{Volumen montado}
\end{figure}

Si la contraseña es incorrecta TrueCrypt le avisará y deberá repetir el
paso anterior (tipear la contraseña y presionar OK).

\begin{enumerate}
\def\labelenumi{\arabic{enumi}.}
\setcounter{enumi}{6}
\tightlist
\item
  Nosotros hemos montado exitosamente el contenedor como virtual disk 1.
  El contenedor aparecerá en su Escritorio o deberá buscarlo con su
  navegador de archivos.
\end{enumerate}

\section{¿Qué significa esto?}\label{quuxe9-significa-esto}

El disco que ha creado está completamente cifrado y se comporta como un
disco real. Grabar (o mover, copiar, etc.) archivos en el disco le
permitirá a usted cifrar archivos sobre la marcha.

Podrá abrir un archivo almacenado en un volumen TrueCrypt, el cual será
descifrado automáticamente a la RAM mientras es leído, y no necesitará
ingresar su contraseña en cada ocasión. Solamente deberá ingresarla al
montar el volumen.

\section{¡Recuerde desmontarlo!}\label{recuerde-desmontarlo}

Haga click en el disco con el botón derecho y seleccione unmount. Esto
sucederá automáticamente cuando apague su computadora pero no cuando se
encuentre en modo suspendido. Configuración de un volumen oculto
==================================

Un volumen TrueCrypt oculto existe en el espacio libre de un volumen
TrueCrypt típico. Suponiendo que accedamos al `volumen externo', es
(casi) imposible determinar si hay un volumen oculto dentro de él. Esto
es así porque TrueCrypt \emph{siempre} llena el espacio vacío de un
volumen cifrado con datos aleatorios. Por eso un volumen oculto se ve
igual que un volumen vacío.

Para crear y utilizar un volumen oculto se necesitan dos contraseñas -
una para cada uno de los volúmenes, el exterior y el interior (oculto).
Cuando monte (abra) el volumen puede utilizar cualquiera de ellos y esto
determinará cuál de los dos estará abierto. Si desea abrir sólo el
volumen oculto utilice una contraseña, y si usted desea tener acceso
sólo al volumen cifrado no oculta deberá utilizar la otra contraseña.

Para crear un volumen abra TrueCrypt oculto y pulse el botón `Crear
volumen':

\begin{figure}[htbp]
\centering
\includegraphics{hidden_vol_001.png}
\caption{Creando un volumen oculto}
\end{figure}

Las opciones para la mitad de este proceso son casi todas iguales a las
usadas para configurar un volumen TrueCrypt estándar, no obstante
indicaremos el proceso completo paso por paso. En la pantalla mostrada
debajo deberá optar por la configuración por defecto `Create an
encrypted file container':

\begin{figure}[htbp]
\centering
\includegraphics{hidden_vol_002.png}
\caption{Opciones de cifrado}
\end{figure}

Presione `Next \textgreater{}' y continúe a la próxima pantalla.

Elija la segunda opción `Hidden TrueCrypt Volume'. Haga click en `Next
\textgreater{}' entonces le pedirán que seleccione un lugar y un nombre
para el volumen TrueCrypt \emph{externo}.

\begin{figure}[htbp]
\centering
\includegraphics{hidden_vol_004.png}
\caption{Seleccionando la ruta}
\end{figure}

Haga click en `Select File\ldots{}' y navegue hasta el lugar donde
ubicará el nuevo volumen. Nosotros usaremos el nombre `myencryptedfile'
en este ejemplo. Es el mismo nombre usado en el último ejemplo por eso
tenga cuidado si ha seguido nuestras instrucciones anteriores porque
ahora debe crear un volumen nuevo con un nombre diferente.

\begin{figure}[htbp]
\centering
\includegraphics{hidden_vol_005.png}
\caption{Eligiendo el nombre}
\end{figure}

Navegue al directorio donde desea colocar el volumen externo e ingrese
el nombre en el campo `Name' como en el ejemplo anterior. Grabe los
cambios. El navegador de archivos se cerrará y usted volverá al
asistente. Presione `Next \textgreater{}'. Aquí se encontrará con
algunas decisiones técnicas. No se preocupe. Acepte todas por defecto.
La próxima pantalla le pedirá que determine el tamaño de su volumen
externo. Note que cuando haga esto el tamaño máximo del volumen interior
`oculto' lo determina TrueCrypt. Este tamaño máximo será de hecho menor
que el tamaño que usted está configurando en su pantalla. Si no está
seguro de cuál es la relación entre el tamaño del volumen exterior y el
tamaño del volumen interior (oculto) a partir de ahora deberá simular el
proceso - siempre podrá desechar el volumen cifrado y empezar de nuevo
(sin provocar ningún daño).

Elegimos entonces el tamaño del volumen exterior, que será de 20MB:

\begin{figure}[htbp]
\centering
\includegraphics{hidden_vol_006.png}
\caption{Eligiendo el tamaño}
\end{figure}

No se puede configurar el tamaño del volumen exterior más grande que el
espacio libre disponible en su disco. TrueCrypt le informa el tamaño
máximo posible en negrita. Luego haga clic en `Siguiente \textgreater{}'
y pasará a una pantalla que le solicitará que establezca una contraseña
para el volumen exterior (no el oculto, esto viene después).

\begin{figure}[htbp]
\centering
\includegraphics{hidden_vol_007.png}
\caption{Estableciendo una contraseña}
\end{figure}

Ingrese una contraseña fuerte (consulte el capítulo acerca de crear
buenas contraseñas) y presione `Next \textgreater{}'. Ahora TrueCrypt lo
ayudará a crear datos aleatorios para completar el volumen con ellos.
Mueva zigzagueando su ratón por toda la pantalla, navegue por internet,
haga cualquier cosa que se le ocurra. Cuando crea que TrueCrypt ya está
satisfecho, pulse `Format'. Usted verá una barra de progreso rápida y
luego aparecerá la siguiente pantalla:

Usted puede abrir el volumen exterior si quiere, pero en este capítulo
lo vamos a saltear y seguir adelante para crear el volumen oculto. Pulse
`Siguiente \textgreater{}' y TrueCrypt le informará el tamaño máximo
posible del volumen oculto.

Cuando vea la pantalla mostrada más abajo, presione `Next
\textgreater{}'. Ahora podrá elegir el tipo de cifrado para el volumen
oculto. Mantenga el valor por defecto y presione `Next \textgreater{}'.

Ahora le pedirá que le indique el tamaño del volumen oculto.

Nosotros hemos configurado (como puede ver más abajo) el tamaño máximo
en 10MB. Cuando usted configure el suyo presione `Next \textgreater{}' y
podrá crear una contraseña para el volumen oculto.

Cuando cree la contraseña par el volumen oculto asegúrese que sea
sustancialmente diferente de la contraseña del volumen exterior. Si
alguien puede acceder a su disco y encuentra la contraseña del volumen
oculto podría intentar ligeras variaciones de esta contraseña para ver
si puede obtener también la contraseña del volumen oculto. Asegúrese que
las dos contraseñas no son similares.

Ingrese su contraseña por duplicado y presione `Next \textgreater{}'.

Mantenga los valores por defecto y presione `Next \textgreater{}'
entonces verá la misma pantalla que se le presentaba cuando generaba
datos aleatorios para TrueCrypt. Cuando le dé la gana, presione `Format'
y verá lo siguiente :

\begin{figure}[htbp]
\centering
\includegraphics{hidden_vol_014.png}
\caption{Dando formato}
\end{figure}

El manual de TrueCrypt al cual se refiere no es este, es el que se
encuentra en http://www.truecrypt.org/docs/

Presione `OK' y cierre TrueCrypt. Ahora puede montar el volumen como se
describió en el capítulo anterior. Destrucción segura de datos
===========================

No crea que con presionar el botón delete todo estará hecho. No es tan
sencillo. Para comprender cómo borrar datos en forma segura, debemos
comprender cómo se almacenan. Haciendo una analogía con el mundo real,
explicaremos el almacenamiento de datos como sigue:

Supongamos que usted posee una pequeña agenda con 10 páginas y quiere
escribir algunos datos en ella. Comenzará a escribir a partir de la
primera página. Puede ser que decida destruir la información de la
página 5. Probablemente arranque la hoja y le prenda fuego.

Desafortunadamente los datos en un disco rígido no trabajan de la misma
manera. Un disco rígido no contiene diez, sino miles o tal vez millones
de páginas. También es imposible sacar una ``página'' y destruirla. Para
explicar cómo trabaja un disco rígido, vamos a seguir con nuestro
ejemplo de la agenda de 10 páginas. Pero ahora vamos a trabajar un poco
diferente con ella. Vamos a trabajar de una manera similar a cómo
funciona un disco rígido.

Esta vez usaremos la primera página como índice. Supongamos que
escribimos un texto sobre ``WikiLeaks'', entonces en la primera página
escribiremos ``texto sobre WikiLeaks: vea la página 2''. Entonces el
texto se escribe luego en la página 2.

Para los próximos documentos, añadimos una línea en la página 1 referida
a ``Goldman Sachs'', ``Goldman Sachs: vea la página 3''. Continuamos de
esta forma hasta completar la agenda. Supongamos que la primera página
resulta de la siguiente manera:

\begin{itemize}
\tightlist
\item
  WikiLeaks -\textgreater{} vea la página 2
\item
  Goldman Sachs -\textgreater{} vea la página 3
\item
  Monsanto scandal -\textgreater{} vea la página 4
\item
  Holiday pictures -\textgreater{} vea la página 5
\item
  KGB Investigation -\textgreater{} vea la página 6
\item
  Al Jazeeraa contacts -\textgreater{} vea la página 7
\item
  Iran nuclear program -\textgreater{} vea la página 8
\item
  Sudan investigation -\textgreater{} vea la página 9
\item
  Infiltration in EU-politics -\textgreater{} vea la página 10
\end{itemize}

Ahora, supongamos que decidimos borrar el documento ``Goldman Sachs'',
lo que el disco rígido va a hacer es eliminar sólo la entrada en la
primera página, pero no los datos reales, el índice quedará así:

\begin{itemize}
\tightlist
\item
  WikiLeaks -\textgreater{} vea la página 2
\item
  Monsanto scandal -\textgreater{} vea la página 4
\item
  Holiday pictures -\textgreater{} vea la página 5
\item
  KGB Investigation -\textgreater{} vea la página 6
\item
  Al Jazeeraa contacts -\textgreater{} vea la página 7
\item
  Iran nuclear program -\textgreater{} vea la página 8
\item
  Sudan investigation -\textgreater{} vea la página 9
\item
  Infiltration in EU-politics -\textgreater{} vea la página 10
\end{itemize}

Como eliminamos sólo la referencia al artículo, si abrimos la página 3,
todavía podremos leer el documento acerca de Goldman Sachs. Esta es
exactamente la manera en un disco rígido ``borra'' un archivo. Con algun
software especializado todavía puede ``recuperarse'' de la página 3.

Para eliminar de forma segura los datos, se debe hacer lo siguiente:

\begin{enumerate}
\def\labelenumi{\arabic{enumi}.}
\tightlist
\item
  Abrir la página del documento ``Goldman Sachs'' (página 3)
\item
  Usar un borrador para eliminar el artículo de la página
\item
  Eliminar la referencia en el índice de la página 1
\end{enumerate}

Bueno, usted se sorprenderá por la similitud entre este ejemplo y el
mundo real. Usted sabe que cuando usted borra el artículo con una goma
de borrar, todavía es posible leer algo de él. El lápiz deja una huella
en el papel debido a la presión sobre el papel y también quedará algo de
grafito sin borrar. Pequeñas huellas así quedan en el papel. Si
realmente necesita este artículo, se pueden reconstruir (partes) de él,
aunque haya sido borrado.

Con un disco rígido la situación es muy similar. Incluso si se borra
cada pieza de datos, a veces es posible recuperar parte de estos datos
usando un hardware (muy) especializado. Si los datos son muy
confidenciales y deben ser borrados con el mayor cuidado, se puede
utilizar software para ``sobreescribir'' todas las piezas de datos con
datos aleatorios. Si se hace esto varias veces, será virtualmente
imposible recuperar los datos.

\section{Nota acerca de los discos rígidos de estado
sólido}\label{nota-acerca-de-los-discos-ruxedgidos-de-estado-suxf3lido}

Las siguientes instrucciones explican cómo utilizar las herramientas
para eliminar archivos de forma segura de sus discos rígidos. Estas
herramientas dependen de que el sistema operativo que esté utilizando
sea capaz de direccionar directamente cada byte del disco rígido en
forma ordenada para decirle ``configure número de byte de X a 0''.
Desafortunadamente, debido a una serie de tecnologías avanzadas
utilizadas por unidades de estado sólido (SSD) como TRIM, no siempre es
posible asegurar con 100\% de certeza de que cada parte de un archivo en
un disco SSD ha sido borrada usando las herramientas mostradas a
continuación.

\section{Borrado seguro de datos en
Windows}\label{borrado-seguro-de-datos-en-windows}

Para Windows existe una buena herramienta de código abierto llamado
``File Shredder''. Esta herramienta se puede descargar desde
http://www.fileshredder.org

La instalación es muy sencilla, sólo tiene que descargar la aplicación e
instalarla pulsando el botón next una y otra vez. Después de la
instalación esta aplicación se iniciará automáticamente. A continuación,
puede empezar a usarlo para borrar archivos. Sin embargo, la mejor parte
del programa es que puedes usarlo desde dentro del propio Windows,
haciendo click con el botón derecho sobre un archivo.

\begin{enumerate}
\def\labelenumi{\arabic{enumi}.}
\tightlist
\item
  Haga click derecho en el archivo que desea borrar, y elija File
  Shredder -\textgreater{} Secure delete files
\end{enumerate}

\begin{figure}[htbp]
\centering
\includegraphics{destroy_data_001.png}
\caption{Borrado seguro}
\end{figure}

\begin{enumerate}
\def\labelenumi{\arabic{enumi}.}
\setcounter{enumi}{1}
\tightlist
\item
  Un pop-up le preguntará si realmente desea borrar este archivo
\end{enumerate}

\begin{figure}[htbp]
\centering
\includegraphics{destroy_data_002.png}
\caption{Confirmación del borrado}
\end{figure}

\begin{enumerate}
\def\labelenumi{\arabic{enumi}.}
\setcounter{enumi}{2}
\tightlist
\item
  Tras su confirmación, dependiendo del tamaño del archivo, el borrado
  tardará unos minutos
\end{enumerate}

\begin{figure}[htbp]
\centering
\includegraphics{destroy_data_003.png}
\caption{Borrando\ldots{}}
\end{figure}

\section{Borrado seguro de datos en
MacOSX}\label{borrado-seguro-de-datos-en-macosx}

Siga los pasos siguientes para borrar datos en forma segura en su Mac
OSX.

\begin{enumerate}
\def\labelenumi{\arabic{enumi}.}
\item
  Borre el espacio libre en su disco rígido que contiene todos los datos
  de los ítem que fueron borrados en forma insegura.
\item
  Asegúrese de que todos los archivos a partir de ahora se eliminen
  siempre de forma segura.
\end{enumerate}

Empecemos con el primer paso:

\subsection{Borrando el espacio libre}\label{borrando-el-espacio-libre}

\begin{enumerate}
\def\labelenumi{\arabic{enumi}.}
\tightlist
\item
  Abra la utilidad de disco la cual reside en la carpeta Utilities
  dentro de la carpeta Applications.
\end{enumerate}

\begin{figure}[htbp]
\centering
\includegraphics{destroy_data_004.jpg}
\caption{Borrado del espacio libre}
\end{figure}

\begin{enumerate}
\def\labelenumi{\arabic{enumi}.}
\setcounter{enumi}{1}
\tightlist
\item
  Seleccione su disco rígido y haga click en `Erase Free Space'.
\end{enumerate}

\begin{figure}[htbp]
\centering
\includegraphics{destroy_data_005.jpg}
\caption{Confirmación del borrado}
\end{figure}

\begin{enumerate}
\def\labelenumi{\arabic{enumi}.}
\setcounter{enumi}{2}
\tightlist
\item
  Aparecerán tres opciones, aumentando la seguridad de arriba hacia
  abajo, pero también tardará mucho más tiempo en completarse. Lea las
  descripciones de cada uno de ellos para poder seleccionar lo que se
  adecúe mejor a sus necesidades y haga click en `Erase free Space'.
\end{enumerate}

Si el tiempo no es un problema, use el método más seguro y disfrute de
su tiempo libre para obtener un buen café mientras su Mac cruje con esta
tarea. Si los ladrones ya están llamando a su puerta principal es
posible que desee utilizar la forma más rápida.

\begin{figure}[htbp]
\centering
\includegraphics{destroy_data_006.jpg}
\caption{Selección del nmétodo de borrado}
\end{figure}

\subsection{Borrado seguro de
archivos}\label{borrado-seguro-de-archivos}

Ahora que sus datos han sido eliminados para siempre debe asegurarse de
que usted no creará nuevos datos que podrían ser recuperados en una
fecha posterior.

\begin{enumerate}
\def\labelenumi{\arabic{enumi}.}
\tightlist
\item
  Para hacer esto,abra el buscador de preferencias bajo el menú Finder.
\end{enumerate}

\begin{figure}[htbp]
\centering
\includegraphics{destroy_data_007.jpg}
\caption{Buscador de preferencias}
\end{figure}

\begin{enumerate}
\def\labelenumi{\arabic{enumi}.}
\setcounter{enumi}{1}
\tightlist
\item
  Vaya a la pestaña advanced y marque `Empty trash securely'. Esto le
  asegurará que cuando vacíe su papelera todos los items serán borrados
  en forma segura.
\end{enumerate}

\begin{figure}[htbp]
\centering
\includegraphics{destroy_data_008.jpg}
\caption{Borrando en forma segura}
\end{figure}

\textbf{Nota:} borrar sus archivos en forma segura tardará mucho tiempo
más tiempo que el borrado simple. Si tiene que borrar grandes porciones
de datos sin importancia (por ejemplo su colección de películas y mp3)
debería desmarcar esta opción.

\section{Borrado seguro de datos en
Ubuntu}\label{borrado-seguro-de-datos-en-ubuntu}

Desafortunadamente no existe en la actualidad una interfaz gráfica
disponible en Ubuntu para borrar archivos en forma segura. Existen dos
comandos disponibles:

\begin{itemize}
\tightlist
\item
  shred
\item
  wipe
\end{itemize}

Shred está instalado en Ubuntu por defecto y puede borrar archivos
simples. Wipe no está instalado por defecto pero puede instalarse
fácilmente con el Ubuntu Software Center o mediante línea de comandos
con \texttt{apt-get\ install\ wipe}. Wipe es un poco más seguro y una
mejor opción.

Es posible acceder a estos programas fácilmente agregándolos como una
opción de menú adicional.

\begin{enumerate}
\def\labelenumi{\arabic{enumi}.}
\tightlist
\item
  Suponemos que está familiarizado con el Ubuntu Software Center. Para
  agregar la opción \emph{securely wipe}, deberá instalar los programas
  \emph{wipe} y \emph{nautilus-actions}
\end{enumerate}

Si estos dos programas están instalados siga con el paso siguiente. Si
no es su caso, instálelos usando el Ubuntu Software Center o la línea de
comandos tipeando apt-get install nautilus-actions wipe

\begin{enumerate}
\def\labelenumi{\arabic{enumi}.}
\setcounter{enumi}{1}
\tightlist
\item
  Abra ``Nautilus Actions Configuration'' desde System -\textgreater{}
  Preferences menu
\end{enumerate}

\begin{figure}[htbp]
\centering
\includegraphics{destroy_data_009.png}
\caption{AConfigurando Nautilus}
\end{figure}

\begin{enumerate}
\def\labelenumi{\arabic{enumi}.}
\setcounter{enumi}{2}
\tightlist
\item
  Agregaremos una nueva acción haciendo click en ``create new action
  button'', la primera opción en la barra de herramientas
\end{enumerate}

\begin{figure}[htbp]
\centering
\includegraphics{destroy_data_010.png}
\caption{Creando una nueva acción}
\end{figure}

\begin{enumerate}
\def\labelenumi{\arabic{enumi}.}
\setcounter{enumi}{3}
\tightlist
\item
  Lo que sigue es describir la nueva acción. Puede darle a la acción el
  nombre que quiera. Colóquelo en el campo ``Context label''. En este
  ejemplo usamos ``Delete file securely''
\end{enumerate}

\begin{figure}[htbp]
\centering
\includegraphics{destroy_data_011.png}
\caption{Describiendo la acción}
\end{figure}

\begin{enumerate}
\def\labelenumi{\arabic{enumi}.}
\setcounter{enumi}{4}
\tightlist
\item
  Haga click en la segunda pestaña (``Command''), aquí especificaremos
  la acción que queremos realizar. En el campo ``Path'', tipee ``wipe'',
  en tipos de parámetro escriba ``-rf \%M'', asegúrese de hacerlo
  correctamente, es muy importante.
\end{enumerate}

\begin{figure}[htbp]
\centering
\includegraphics{destroy_data_012.png}
\caption{Configurando la acción}
\end{figure}

\begin{enumerate}
\def\labelenumi{\arabic{enumi}.}
\setcounter{enumi}{5}
\tightlist
\item
  Especifiquemos las condiciones, haga click en la pestaña de
  condiciones y elija ``Both'' en la caja de diálogos ``Appears if
  selection contains\ldots{}''. Con esta opción podrá borrar archivos y
  carpetaas en forma segura. Grabe los cambios
\end{enumerate}

\begin{figure}[htbp]
\centering
\includegraphics{destroy_data_013.png}
\caption{Terminando la configuración}
\end{figure}

\begin{enumerate}
\def\labelenumi{\arabic{enumi}.}
\setcounter{enumi}{6}
\item
  Cierre la herramienta de configuración de acciones Nautilus. Tendrá
  que reiniciar su sesión para que los cambios surtan efecto.
\item
  Ahora navegue hasta el archivo que desea borrar en forma segura y haga
  un click derecho:
\end{enumerate}

\begin{figure}[htbp]
\centering
\includegraphics{destroy_data_014.png}
\caption{Borrando un archivo}
\end{figure}

Elija `Delete File Securely'. El archivo será borrado `tranquilamente' -
no se dará cuenta que el proceso ha comenzado o ya concluyó. Sin
embargo, el proceso está en marcha. Se necesita algún tiempo para
eliminar de forma segura los datos y el más grande es el archivo que más
tardará. Cuando se complete, el ícono del archivo a ser borrado
desaparecerá. Si usted quisiera añadir algunos comentarios puede cambiar
el campo de los parámetros en la herramienta de configuración de
acciones Nautilius, por ejemplo así:

\texttt{-rf\ \%M\ \textbar{}\ zenity\ -\/-info\ -\/-text\ "your\ wipe\ is\ underway\ please\ be\ patient.\ The\ icon\ of\ the\ file\ to\ be\ wiped\ will\ disappear\ shortly."}

La línea de arriba le indicará que el proceso está en marcha pero el
archivo no será eliminado hasta que desaparezca el ícono. About LUKS
==========

\textbf{LUKS}, short for \emph{Linux Unified Key Setup}, is the default
method for disk encryption on Linux. It can be used to enable \emph{Full
Disk Encryption} during installation with a single click, or to encrypt
individual partitions on external hard disks or usb sticks later on.
Please note that \emph{Full Disk Encryption} is hard to enable
\textbf{after} the installation as it requires moving all existing files
temporarily as encrypting a device requires formatting it.

\begin{itemize}
\item
  Advantages: LUKS is available through dm-crypt which is part of the
  Linux kernel, so it doesn't need any further software to be installed.
\item
  Disadvantages: Unlike with Truecrypt, it is not possible to use it
  with other Operating Systems (yet), so if you use LUKS to encrypt a
  USB drive, you can only use it on Linux machines, but not on Windows
  or Mac OS.
\end{itemize}

If you want to encrypt a device after the Linux installation completed,
you can use the \emph{Disks} utility which can be found in most Linux
distribution's \emph{System Settings}.

\section{\texorpdfstring{Starting
\emph{Disks}}{Starting Disks}}\label{starting-disks}

On Ubuntu, start \emph{Disks} by pressing the Windows key and A, typing
``disks'' and selecting the corresponding program as shown below:

\begin{figure}[htbp]
\centering
\includegraphics{disks_000_launch.png}
\caption{Launching Disks}
\end{figure}

\section{Encrypting a device}\label{encrypting-a-device}

\begin{figure}[htbp]
\centering
\includegraphics{disks_001_with_steps.png}
\caption{Disks main window}
\end{figure}

On the left hand side you will find a list of all storage devices
plugged into your computer.

Select the one you want to encrypt (step 1) (in this case a usb stick),
and then on the right hand side, click on the cog wheels and
``Format\ldots{}''. A dialog will appear where you can select if the
existing data on the device shall be completely overwritten (that can
take up to several hours depending on the size and performance of the
device) or just formatted. Please note that even if you choose to
encrypt the device, data, that was present before will be recoverable if
you don't choose to overwrite it completely.

No matter what you choose for the field \emph{Erase}, select
``Encrypted, compatible with Linux systems (LUKS+Ext4)'' for
\emph{Type}, give it a name and a strong passphrase (see chapter 8 on
that matter), and click \emph{Format\ldots{}}

\begin{figure}[htbp]
\centering
\includegraphics{disks_003_formatencryptedfilledout.png}
\caption{``Format\ldots{}'' dialog}
\end{figure}

On the confirmation screen make sure you selected the correct device as
data recovery is a cumbersome tasks -- if possible at all.

\begin{figure}[htbp]
\centering
\includegraphics{disks_004_formatconfirmation.png}
\caption{Confirmation step}
\end{figure}

Back on the main window the device now consists of two layers. One is
the physical storage (here called ``Partition 1'') and the other a
virtual device which is created by the LUKS system to give you access to
the encrypted device (here called ``cryptostick''). The pad lock on
``Partition 1'' is open as the \emph{Disks} utility needed to open it in
order to create a file system (how would you store files on a device
without a file system?). You can click on the (other) pad lock as shown
below to close the decryption channel and the \emph{eject} button in the
upper right corner to safely remove the device.

\includegraphics{disks_005_with_steps.png}
\includegraphics{disks_006_with_steps.png}

\section{Using an encrypted device}\label{using-an-encrypted-device}

This is quite straight-forward. Plug it in, enter the passphrase and
click \emph{Connect}. If the file manager does not open automatically,
the device will be available when you do.

\includegraphics{disks_007_passphrase_prompt.png} Instalación de
CSipSimple =========================

CSipSimple es un programa para dispositivos Android que permite hacer
llamadas cifradas. Naturalmente, el software no es suficiente por sí
solo y necesitamos una red de comunicación que nos permita hacer
llamadas.

\section{Introducción a la red
OSTN}\label{introducciuxf3n-a-la-red-ostn}

Si conoce acerca de OSTN y tiene una cuenta, puede saltear esta sección.

La red de telefonía abierta (segura, de código abierto, estándar) OSTN
del proyecto Guardian (\url{https://guardianproject.info/wiki/OSTN}) es
un intento de definir una configuración estándar de voz sobre IP (VoIP)
usando el protocolo de inicio de sesión SIP que permite llamadas
cifradas de extremo a extremo. De manera similar al correo electrónico,
SIP le permite a las personas elegir su proveedor de servicios sin
perder su capacidad de llamar a los demás, incluso si no está utilizando
el mismo proveedor. Sin embargo, no todos los proveedores de SIP ofrecen
OSTN y los proveedores tienen que apoyar OSTN para que las llamadas sean
seguras. Una vez que una relación entre dos personas se ha establecido,
los datos de audio se intercambian directamente entre las dos partes.
Los datos se cifran de acuerdo con el protocolo de transporte seguro en
tiempo real (SRTP).

La mayoría de las aplicaciones de cifrado de VoIP utilizan actualmente
el protocolo de descripción de sesión denominado Descripciones de
Seguridad para flujos de medios (SDES) con la seguridad de la capa de
transporte (TLS) salto por salto para intercambiar claves maestras
secretas para SRTP. Este método no es de extremo a extremo seguro como
las claves de SRTP ya que son visibles en texto claro a cualquier proxy
SIP o proveedor involucrado en la llamada.

ZRTP es un protocolo de acuerdo de clave cifrada para negociar las
claves de cifrado entre dos partes. Los puntos extremos ZRTP utilizan el
flujo de medios de comunicación en lugar del flujo de señalización para
establecer las claves de cifrado SRTP. Puesto que la corriente de medios
de comunicación es una conexión directa entre las partes que llaman, no
hay manera para que los proveedores de SIP o proxies para interceptar
las claves SRTP. ZRTP proporciona una tranquilidad razonable para los
usuarios finales que tienen una línea segura. Al leer y comparar un par
de palabras, los usuarios pueden estar seguros de que el intercambio de
claves se ha completado.

\section{CSipSimple}\label{csipsimple}

CSipSimple es un cliente libre y open source para Android que trabaja
bien con OSTN. Puede encontrarloi en
\url{https://market.android.com/details?id=com.csipsimple}

Para usar CSipSimple con ostel.me, elija OSTN en el asistente genérico
cuando cree una cuenta e ingrese un nombre de usuario, contraseña y
servidor según lo previsto después de inscribirse en
\href{https://ostel.me/users/sign_up}{Ostel}

Una vez que llame a otra persona con CSipSimple aparecerá una barra
amarilla con ZRTP y el par de verificación de palabra. Ahora se ha
establecido una conexión de voz segura que no puede ser interceptada.
Sin embargo, usted debe ser consciente de que el teléfono o el teléfono
de la otra parte pueden estar configurados para grabar la conversación.

Pasos básicos:

\begin{enumerate}
\def\labelenumi{\arabic{enumi}.}
\tightlist
\item
  Instalar CSipSimple desde Google Play store u otra fuente verificada
\item
  Ponerlo en marcha y elegir si desea realizar llamadas SIP a través de
  conexión de datos o sólo Wi-Fi
\item
  Configurar su cuenta
\end{enumerate}

Para usar CSipSimple con ostel.me, elija OSTN en la sección Generic
Wizards cuando cree una cuenta. Puede alternar entre los proveedores de
los ``Estados Unidos'' haciendo clic en ``Estados Unidos''. Ahora
seleccione \emph{OSTN}:

\begin{figure}[htbp]
\centering
\includegraphics{ostn_1.png}
\caption{OSTN}
\end{figure}

Ahora puede ingresar su usuario (número), contraseña y servidor
(ostel.me) según lo previsto después de inscribirse en
\href{https://ostel.me/users/sign_up}{Ostel}.

\begin{figure}[htbp]
\centering
\includegraphics{ostn_2.png}
\caption{OSTN}
\end{figure}

Ahora usted puede hacer una llamada. La primera vez que se conecte a una
persona con ZRTP usted tiene que comprobar que el intercambio de claves
se ha realizado correctamente. En el siguiente ejemplo la palabra de
confirmación es ``cieh'', usted puede hablar con la otra parte, y
asegurarse de que ambos ven la misma palabra. Una vez terminado, pulse
Aceptar.

\begin{figure}[htbp]
\centering
\includegraphics{ostn_3.png}
\caption{OSTN}
\end{figure}

Usted ha establecido una conexión de voz segura que no puede ser
interceptada. Tenga en cuenta que usted o el teléfono de la otra parte
podría estar grabando la conversación.

\chapter{Configuración de mensajería instantánea
cifrada}\label{configuraciuxf3n-de-mensajeruxeda-instantuxe1nea-cifrada}

\section{Android - Instalación de
Gibberbot}\label{android---instalaciuxf3n-de-gibberbot}

\url{https://guardianproject.info/apps/gibber/}

Gibberbot es un cliente de chat seguro capaz de cifrado de extremo a
extremo. Funciona con Google, Facebook, y Jabber o cualquier servidor
XMPP. Gibberbot usa cifrado Off-The-Record estándar (OTR) para habilitar
comunicaciones cifradas extremo a extremo verificables verdaderas.

Usted puede instalar Gibberbot a través del Google Play o a partir de
otras fuentes autenticadas.

También pueder chatear en forma segura con otros programas con soporte
OTR tales como Adium, Pidgin en la computadora, Gibberbot en Android o
ChatSecure en iOS.

\section{iOS - Instalación de
ChatSecure}\label{ios---instalaciuxf3n-de-chatsecure}

\url{http://chrisballinger.info/apps/chatsecure/}

ChatSecure es un cliente de chat seguro capaz de cifrar de extremo a
extremo. Funciona con Google, Facebook y Jabber o cualquier servidor
XMPP. ChatSecure usa cifrado estándar Off-the-Record (OTR) para
habilitar las comunicaciones cifradas de extremo a extremo verificables
verdaderas.

Puede instalar ChatSecure desde iTunes store.

Puede chatear en forma segura con otros programas con soporte OTR tales
como Adium, Pidgin en la computadora, Gibberbot en Android o ChatSecure
en iOS.

\section{Ubuntu - Instalación de
Pidgin}\label{ubuntu---instalaciuxf3n-de-pidgin}

\url{http://pidgin.im/}

Pidgin es un cliente de chat seguro capaz de cifrar de extremo a
extremo. Funciona con Google, Facebook, Jabber o cualquier servidor
XMPP. Pidgin utiliza el cifrado estándar Off-the-Record (OTR) para
habilitar las comunicaciones cifradas de extremo a extremo verificables
verdaderas.

Puede instalarlo desde el Ubuntu Software Center, busque pidgin-otr para
instalar pidgin y el plugin otr.

Una vez instalado puede habilitar otr para cualquier cuenta que
configure en pidgin.

Puede chatear en forma segura con otros programas con soporte OTR tales
como Adium, Pidgin en la computadora, Gibberbot en Android o ChatSecure
en iOS.

\section{OS X - Instalación de
Adium}\label{os-x---instalaciuxf3n-de-adium}

\url{http://www.adium.im/}

Adium es un cliente de chat seguro capaz de cifrado de extremo a
extremo. Funciona con Google, Facebook, Jabber o cualquier servidor
XMPP. Adium utiliza el cifrado estándar Off-the-Record (OTR) para
habilitar las comunicaciones cifradas de extremo a extremo verificables
verdaderas.

La instalación de Adium es similar a la instalación de la mayoría de las
aplicaciones de Mac OS X.

\begin{enumerate}
\def\labelenumi{\arabic{enumi}.}
\tightlist
\item
  Descargar la imagen de disco de Adium \url{http://www.adium.im/}.
\item
  Si una ventana Adium no se abre automáticamente, haga doble click en
  el archivo descargado
\item
  Arrastre la aplicación Adium a la carpeta Aplicaciones.
\item
  ``Expulse'' la imagen de disco Adium, que tiene un ícono de un disco
\item
  La imagen de disco Adium todavía estará presente en la carpeta de
  descarga (probablemente en el escritorio). Puede arrastrar el archivo
  a la papelera, ya que ya no es necesaria.
\item
  Para cargar Adium, colóquela en la carpeta Aplicaciones y haga doble
  click.
\end{enumerate}

Puede chatear en forma segura con otros programas con soporte OTR tales
como Adium, Pidgin en la computadora, Gibberbot en Android o ChatSecure
en iOS.

\section{Windows - Instalación de
Pidgin}\label{windows---instalaciuxf3n-de-pidgin}

\url{http://pidgin.im/}

Pidgin es un cliente de chat seguro capaz de cifrado de extremo a
extremo. Funciona con Google, Facebook, Jabber o cualquier servidor
XMPP. Pidgin utiliza el cifrado estándar Off-the-Record (OTR) para
habilitar las comunicaciones cifradas de extremo a extremo verificables
verdaderas.

Para utilizar Pidgin con OTR en Windows, es necesario instalar el plugin
de Pidgin y OTR para Pidgin.

\begin{enumerate}
\def\labelenumi{\arabic{enumi}.}
\tightlist
\item
  Descargue la última versión de Pidgin para Windows desde
  \url{http://www.pidgin.im/download/windows/}
\item
  Ejecute el instalador de Pidgin
\item
  Descargue la versión más reciente del ``plugin OTR para Pidgin'' de
  \url{http://www.cypherpunks.ca/otr/\#downloads}
\item
  Ejecute el Instalador de Complementos OTR
\end{enumerate}

Una vez instalado puede habilitar otr para cualquier cuenta que
configure en pidgin.

Puede chatear en forma segura con otros programas con soporte OTR tales
como Adium, Pidgin en la computadora, Gibberbot en Android o ChatSecure
en iOS.

\section{Todos los OS - crypto.cat}\label{todos-los-os---crypto.cat}

\url{https://crypto.cat}

Cryptocat es una aplicación web de código abierto destinado a permitir
chat online de forma segura y cifrada. Cryptocat cifra los chats en el
lado del cliente, sólo confía en el servidor con los datos que ya están
cifrados. Cryptocat se entrega como una extensión del navegador y ofrece
plugins para Google Chrome, Mozilla Firefox y Apple Safari.

Cryptocat tiene la intención de proporcionar medios para las
comunicaciones cifradas improvisadas, que ofrecen más privacidad que
servicios como Google Talk, manteniendo al mismo tiempo un mayor nivel
de accesibilidad que otras plataformas de cifrado de alto nivel, y
además permite múltiples usuarios en una sala de chat.

\section{Archivos de registros de
chat}\label{archivos-de-registros-de-chat}

Algunos de los clientes de chat antes mencionados, por ejemplo, Adium,
almacenan texto plano, sin cifrar los registros de chat, a menudo de
forma predeterminada, incluso aún cuando está instalado el plugin de
``seguridad/privacidad'' OTR.

Si usted está tomando precauciones OTR para proteger sus sesiones de
chat de fisgones a través en el cable o inalámbricos, debería comprobar
que se ha apagado manualmente el Inicio de sesión de Chat, o asegurarse
de que el registros de chat destinados deliberadamente para ser
guardados son creados en un disco o volumen cifrado, en caso de que su
equipo se pierda, sea robado o decomisado. También vale la pena
preguntarle a la persona con la que está conversando si está registrando
inadvertidamente el contenido de la charla con su propio software
cliente de chat. Instalación de I2P on Ubuntu Lucid Lynx (y posteriores)
y sus derivados como Linux Mint \& Trisquel ============================

\begin{enumerate}
\def\labelenumi{\arabic{enumi}.}
\tightlist
\item
  Abra una terminal y escriba:
\end{enumerate}

\texttt{sudo\ apt-add-repository\ ppa:i2p-maintainers/i2p}

Este comando agregará el PPA a /etc/apt/sources.list.d y descargará la
clave gpg con la cual se ha firmado el repositorio. La clave GPG se
asegura de que los paquetes no se han alterado desde que fue construido.

\begin{enumerate}
\def\labelenumi{\arabic{enumi}.}
\setcounter{enumi}{1}
\tightlist
\item
  Notifique a su administrador de paquetes del nuevo PPA ingresando
\end{enumerate}

\texttt{sudo\ apt-get\ update}

Este comando recuperará la lista más reciente de software de cada
repositorio que está habilitado en el sistema, incluyendo el PPA I2P que
se ha añadido con el comando anterior.

\begin{enumerate}
\def\labelenumi{\arabic{enumi}.}
\setcounter{enumi}{2}
\tightlist
\item
  Ahora está listo para instalar I2P
\end{enumerate}

\texttt{sudo\ apt-get\ install\ i2p}

\begin{enumerate}
\def\labelenumi{\arabic{enumi}.}
\setcounter{enumi}{3}
\tightlist
\item
  Su explorador debería abrir con la consola del router I2P local, para
  navegar por dominios I2P tiene que configurar su navegador para usar
  el proxy I2P. También puede ver el estado de conexión en el lado
  izquierdo de la consola del router. Si su estado es \textbf{Network:
  Firewalled} su conexión va a ser bastante lenta. La primera vez que
  inicie I2P puede tardar algunos minutos para integrarse en la red y
  encontrar pares adicionales para optimizar su integración, por lo que
  por favor sea paciente.
\end{enumerate}

En el menú Tools, seleccione Options para abrir el panel de
configuración de Firefox. Haga click en el icono con la etiqueta
Advanced, haga clic en la ficha Network. En la sección Connections, haga
clic en el botón Settings. Verá una ventana como la siguiente:

\begin{figure}[htbp]
\centering
\includegraphics{i2p_1.jpg}
\caption{I2P}
\end{figure}

En la ventana de Connection Settings, haga click en el círculo cercano a
Manual proxy configuration, luego ingrese 127.0.0.1, port 4444 en el
campo HTTP Proxy. Ingrese 127.0.0.1, port 4445 en el campo SSL Proxy.
Asegúrese de ingresar localhost y 127.0.0.1 en la casilla de ``No Proxy
for''.

\begin{figure}[htbp]
\centering
\includegraphics{i2p_1.jpg}
\caption{I2P}
\end{figure}

Para más información y cómo configurar proxies para otros navegadores
consulte \url{https://www.i2p2.de/htproxyports.htm}

\chapter{Instrucciones para Debian Lenny y
posteriores}\label{instrucciones-para-debian-lenny-y-posteriores}

Para más información visita esta página
\href{https://www.i2p2.de/debian-html}{https://www.i2p2.de/debian.html}

\chapter{Empezando con I2P}\label{empezando-con-i2p}

Usando estos paquetes I2P el router I2P puede iniciar de alguna de las
siguientes formas:

\begin{itemize}
\tightlist
\item
  ``on demand'' usando el script i2prouter. Simplemente ejecute
  ``i2prouter start'' en un terminal. (Nota: ¡no use sudo ni lo ejecute
  como root!)
\item
  como un servicio que se ejecuta automáticamente cuando inicia su
  sistema, aún antes de loguearse. El servicio puede ser habilitado con
  ``dpkg-reconfigure i2p'' como root o mediante sudo. Esta es la manera
  recomendada.
\end{itemize}

\chapter{Bittorrent anónimos con
I2PSnark}\label{bittorrent-anuxf3nimos-con-i2psnark}

Podemos utilizar la red I2P para compartir y descargar archivos sin
necesidad de que todo el mundo sepa que los está compartiendo, o que se
está ejecutando un cliente de torrent, ya que la red I2P está cifrada de
extremo a extremo y lo único que se ve desde afuera es que está
corriendo I2P.

I2p viene con un cliente de torrent incorporado que se ejecuta dentro
del navegador llamado I2PSnark. Puede acceder a través del siguiente
enlace:

\url{http://localhost:7657/i2psnark/}

o a través de la consola del router: \url{http://localhost:7657/}
haciendo click en el ícono del torrent. Una vez lanzado aparecerá una
pantalla similar a la siguiente:

\begin{figure}[htbp]
\centering
\includegraphics{i2p_3.jpg}
\caption{I2P}
\end{figure}

Puede buscar un torrent usando uno de los siguientes trackers de
bittorrent:

\begin{itemize}
\item
  \url{http://tracker.postman.i2p/}
\item
  \url{http://diftracker.i2p/}
\end{itemize}

Copie el torrent o el enlace magnet y péguelo en la ventana de I2PSnark,
luego haga click en \textbf{Add torrent}. El archivo se descargará en la
carpeta \textbf{/home/user/.i2p/i2psnark}.

\textbf{NOTA:}

\begin{itemize}
\item
  Como I2P es una red cerrada, no se pueden descargar los torrents
  normales que se encuentran en Internet, ¡y no puede ser utilizada para
  hacer la descarga anónimamente!
\item
  La velocidad parece ser un poco más baja de lo habitual debido a la
  anonimización. Las velocidades de descarga son aceptables si se tiene
  en cuenta que lo está haciendo de forma anónima. \# OnionShare
\end{itemize}

\section{Introducción}\label{introducciuxf3n-2}

¿Qué es\href{https://onionshare.org/}{OnionShare}? Según las palabras de
los propios dueños del proyecto (cita extraída de
\url{https://github.com/micahflee/onionshare/blob/master/README.md}):

\begin{quote}
\href{https://onionshare.org/}{OnionShare} le permite a usted compartir
archivos de cualquier tamaño segura y anónimamente. Funciona mediante un
servidor web, que es accesible mediante un servicio Tor oculto, y genera
una URL imposible de adivinar para acceder y descargar los archivos.
Esto no requiere que se configure un servidor en algún lugar de Internet
o usar algún servicio de terceros para compartir servicios. Usted
hospeda el archivo en su propia computadora y usa un servicio oculto de
Tor para que esté temporalmente accesible en Internet. El resto de los
usuarios solamente necesitan usar el
\href{https://www.torproject.org/download/download-easy.html.en}{navegador
web de Tor} para descargar su archivo.
\end{quote}

\section{Instalación}\label{instalaciuxf3n-1}

Las instrucciones de instalación se encuentran en el sitio web de
\href{https://onionshare.org/}{OnionShare}.

\section{Como usar OnionShare}\label{como-usar-onionshare}

La siguiente es la pantalla de inicio de
\href{https://onionshare.org}{OnionShare}.

\begin{figure}[htbp]
\centering
\includegraphics{onionshare_1.png}
\caption{started OnionShare}
\end{figure}

Usted puede compartir tantos archivos y carpetas como quiera. Para
añadirlas puede usar el botón correspondiente o arrastrar y soltar las
carpetas dentro de la ventana. Por favor seleccione la opción
\texttt{Stop\ sharing\ automatically}. Esto le asegura que los archivos
que usted comparte puedan ser descargados solamente una vez.

\begin{figure}[htbp]
\centering
\includegraphics{onionshare_2.png}
\caption{added files and folders}
\end{figure}

Cliqueando el botón \texttt{Start\ Sharing} se lanza un pequeño servidor
web en segundo plano. Esto le permite a su amigo descargar el archivo
pero solamente a través de la red \href{https://torproject.org}{Tor}
porque dicho pequeño servidor es un
\href{https://tor.eff.org/docs/hidden-services.html.en}{servicio Tor
oculto}. El inicio del servicio oculto puede demorar un poco, por favor,
sea paciente.

\begin{figure}[htbp]
\centering
\includegraphics{onionshare_3.png}
\caption{preparing to share files}
\end{figure}

Una vez que el servicio oculto inicie, copie su url mediante el botón
\texttt{Copy\ URL}. Luego, envíe dicha dirección a su amigo (si es
necesario, a través de un canal cifrado).

\begin{figure}[htbp]
\centering
\includegraphics{onionshare_4.png}
\caption{sharing files}
\end{figure}

Después de recibir la dirección su amigo debe abrirla en su
\href{https://www.torproject.org/download/download-easy.html.en}{navegador
web Tor}. Esta no será accesible desde otros navegadores web. Su amigo
verá un enlace a un archivo comprimido en formato zip y una lista de
archivos contenidos en su interior. La descarga se inicia cliqueando el
gran botón azul.

\begin{figure}[htbp]
\centering
\includegraphics{onionshare_5.png}
\caption{downloading through TorBrowser}
\end{figure}

Usted puede ver cuando su amigo descarga los archivos mediante la barra
de progreso azul. Una vez que todo ha terminado,
\href{https://onionshare.org}{OnionShare} detendrá el intercambio de
archivos automáticamente (excepto que haya deseleccionado
\texttt{Stop\ sharing\ automatically}).

\begin{figure}[htbp]
\centering
\includegraphics{onionshare_6.png}
\caption{completed download as seen in OnionShare}
\end{figure}

Para verificar que \href{https://onionshare.org}{OnionShare} ha detenido
efectivamente el intercambio de archivos puede abrir la dirección que le
había enviado a su amigo en su propio
\href{https://www.torproject.org/download/download-easy.html.en}{navegador
Tor}. La descarga ya no está disponible.

\begin{figure}[htbp]
\centering
\includegraphics{onionshare_7.png}
\caption{trying download through TorBrowser a second time}
\end{figure}

0xcaca0 Adam Hyde Ahmed Mansour Alice Miller A Ravi Ariel Viera Asher
Wolf AT Austin Martin Ben Weissmann Bernd Fix Brendan Howell Brian
Newbold Carola Hesse Chris Pinchen Dan Hassan Daniel Kinsman Danja
Vasiliev Dévai Nándor djmattyg007 Douwe Schmidt Edward Cherlin Elemar
Emile Denichaud Emile den Tex Erik Stein Erinn Clark Freddy Martinez
Freerk Ohling Greg Broiles Haneef Mubarak helen varley jamieson Janet
Swisher Jan Gerber Jannette Mensch Jens Kubieziel jmorahan Josh Datko
Joshua Datko Julian Oliver Kai Engert Karen Reilly l3lackEyedAngels
leoj3n LiamO Lonneke van der Velden Malte Malte Dik Marta Peirano Mart
van Santen mdimitrova Michael Henriksen Nart Villeneuve Nathan Andrew
Fain Nathan Houle Niels Elgaard Larsen Petter Ericson Piers Plato
Punkbob Roberto Rastapopoulos Ronald Deibert Ross Anderson Sacha van
Geffen Sam Tennyson Samuel Carlisle Samuel L. Tennyson Seth Schoen
Steven Murdoch StooJ Story89 Ted W Ted Wood Teresa Dillon therealplato
Tomas Krag Tom Boyle Travis Tueffel Uwe Lippmann WillMorrison Ximin Luo
Yuval Adam zandi Zorrino Zorrinno Criptografía y cifrado
======================

Criptografía y cifrado son términos similares, el primero es la ciencia
y el segundo la implementación. La historia del tema se remonta a las
civilizaciones antiguas, cuando los primeros seres humanos comenzaron a
organizarse en grupos. Esto se debió, en parte, al darse cuenta de que
estábamos en competencia por los recursos y la organización tribal,
conflictos y demás necesarios para mantenerse en la cima. En este
sentido, la criptografía y el cifrado se basan en la guerra, la
progresión y la gestión de recursos, en los que es necesario enviar
mensajes secretos el uno al otro sin que el enemigo pueda descifrarlos.

La escritura es en realidad una de las primeras formas de cifrado que no
todo el mundo puede leer. La palabra criptografía proviene de las
palabras griegas kryptos (oculto) y graphein (escritura). En este
sentido la criptografía y el cifrado en su forma más simple se refieren
a la escritura de mensajes ocultos, que requieren un sistema o regla
para descifrar y leer. Básicamente, esto le permite proteger su
privacidad mediante la codificación de información de una manera que
sólo es recuperable con cierto conocimiento (contraseñas o frases de
paso) o posesión (una clave).

Dicho de otro modo, el cifrado es la traducción de la información
escrita en texto plano en una forma no legible (texto cifrado) mediante
esquemas algorítmicos (cifrados). El objetivo es utilizar la clave
correcta para abrir el mensaje y regresarlo de nuevo a su forma original
de texto plano para que sea legible.

Aunque la mayoría de los métodos de cifrado se refieren a la palabra
escrita, durante la Segunda Guerra Mundial, el ejército de EE.UU.
utilizó indios Navajos, que viajaban entre los campamentos enviando
mensajes en su lengua nativa. La razón por la que el ejército utilizó la
tribu Navajo era proteger la información que se enviaba de las tropas
japonesas, que no podían descifrar el idioma navajo hablado. Este es un
ejemplo muy simple de usar un lenguaje para enviar mensajes que no
queremos que la gente escuche o que sepa lo que estamos discutiendo.
¿Por qué es tan importante el cifrado?
--------------------------------------

Las redes informáticas y de telecomunicaciones almacenan los ecos
digitales o huellas de nuestros pensamientos y los registros de nuestra
vida personal.

Desde la banca, hasta las reservaciones, pasando por la socialización:
nosotros enviamos una gran variedad de información detallada y
personalizada, que está impulsando nuevos modelos de negocios, de
interacción social y de conducta. Ahora nos hemos acostumbrado a dar lo
que era (y sigue siendo) información considerada privada a cambio de lo
que se presenta como un servicio más personalizado y a nuestra medida,
que podría satisfacer nuestras necesidades, pero en realidad alimenta
nuestra codicia.

Pero, ¿cómo protegemos de quienes nos observan, controlan y utilizan
esta información?

Vamos a considerar un escenario en el que las cosas funcionan muy bien y
podemos enviar toda nuestra comunicación en tarjetas postales abiertas
escritas a mano. Desde las conversaciones con su médico, pasando por los
momentos íntimos con sus amantes, hasta las discusiones legales que
usted pueda tener con abogados o contadores. Es poco probable que nos
guste que todas las personas sean capaces de leer dichas comunicaciones.
Por esto, escribimos cartas en sobres cerrados, monitoreamos los envíos
del correo, disponemos de oficinas cerradas y acuerdos confidenciales,
que ayudan a mantener la comunicación privada. Sin embargo, dado el
cambio en la forma en que nos comunicamos, mucho más que este tipo de
interacción se está llevando a cabo online. Más importante aún, se lleva
a cabo a través de los espacios online, que no son privados de forma
predeterminada y están abierta a personas con pocos conocimientos
técnicos para husmear en los asuntos más importantes de nuestra vida.

La privacidad online y el cifrado es algo de lo que tiene que ser
consciente y practicarlo todos los días. De la misma manera que pondría
una carta importante en un sobre o tendría una conversación privada
detrás de una puerta cerrada. Teniendo en cuenta que gran parte de
nuestra comunicación privada está pasando ahora en los espacios en red y
online, debemos considerar la interfaz, como los sobres o sellos que
protegen este material como una necesidad básica y un derecho humano.

\section{Ejemplos de cifrado}\label{ejemplos-de-cifrado}

A lo largo de la historia encontramos ejemplos de métodos de cifrado,
que han sido usados para mantener a los mensajes privados y secretos.

\section{¡Una advertencia!}\label{una-advertencia}

\begin{quote}
``Existen dos clases de criptografía en el mundo: la que evitará que tu
hermanita acceda a tus archivos, y la que detendrá a la mayoría de los
gobiernos cuando quieran acceder a tus archivos'' - Bruce Schneier,
Applied Cryptography, 1996
\end{quote}

Este capítulo primero explica un número de sistemas de cifrado
históricos y luego proporciona un resumen de las técnicas más modernas.
Los ejemplos históricos ilustran como surgió la criptografía, pero se
considera obsoleta desde la aparición de las computadoras modernas.
Pueden ser divertidos para aprender, pero por favor no los use para nada
realmente importante.

\section{Cifrado histórico}\label{cifrado-histuxf3rico}

El cifrado clásico se refiere al cifrado histórico, que está fuera de
uso o no es muy aplicable. Existen dos categorías generales de cifrado
clásico: trasposición y sustitución.

En el cifrado por trasposición, las cartas mismas se mantienen sin
cambios, pero el orden dentro del mensaje se codifica de acuerdo con un
esquema bien definido. Un ejemplo de un cifrado de transposición es la
escítala, que fue utilizada en la antigua Roma y Grecia. Se envolvía una
cinta de papel alrededor de una vara y se escribía el mensaje a lo
largo. De esta forma el mensaje no se podía leer a menos que la cinta se
envolviese nuevamente alrededor de una vara del mismo diámetro.

\begin{figure}[htbp]
\centering
\includegraphics{crypto_1.png}
\caption{Escítala}
\end{figure}

\emph{Imagen: escítala, extraída de Wikimedia Commons (3.10.12)}

El cifrado por sustitución es una forma clásica de cifrado mediante el
cual las letras o un grupo de ellas se reemplazan sistemáticamente a
través del mensaje por otras letras (o un grupo de ellas). El cifrado de
sustitución se divide en monoalfabético y polialfabético. El cifrado por
desplazamiento del César es un ejemplo común de cifrado por sustitución
monoalfabética, donde las letras del abecedario son desplazadas en un u
otra dirección.

\begin{figure}[htbp]
\centering
\includegraphics{crypto_2.png}
\caption{Cifrado del César}
\end{figure}

\emph{Imagen: Cifrado por desplazamiento del César, extraída de
Wikimedia Commons (3.10.12)}

Las sustituciones polialfabéticas son más complejas que el cifrado por
sustitución porque usan mas de un alfabeto y girado. Por ejemplo, el
cifrado de Alberti, el primer cifrado polialfabético, fue creado en el
siglo XV por León Battista Alberti, un erudito y humanista renacentista
italiano al que también se lo conoce como el fundador de la criptografía
occidental. Su cifrado es similar al cifrado de Vigenère, donde cada
letra del alfabeto tiene un número único (por ejemplo, 1-26). El mensaje
se cifra escribiendo la contraseña repetidamente por debajo de él.

En el cifrado Vigenère los números correspondientes a las letras del
mensaje y la clave se suman (los números que exceden al alfabeto se
redondean hacia abajo) haciendo que el mensaje fuera tan difícil de leer
que no podía ser descifrado por siglos (hoy en día, con la ayuda de las
computadoras, esto obviamente no es cierto ya).

\begin{figure}[htbp]
\centering
\includegraphics{crypto_3.png}
\caption{Cifrado de Vigenère}
\end{figure}

\emph{Imagen: Cifrado de Vigenère, extraída de Wikimedia Commons
(3.10.12)}

Durante la Segunda Guerra Mundial se produjo la explosión de la
criptografía, que condujo al desarrollo de nuevos algoritmos, como la
libreta de un solo uso (OTP). El algoritmo de OTP combina texto plano
con una clave aleatoria que es tan larga como el texto plano de forma
que cada carácter sólo se utiliza una vez. Para utilizarlo se necesitan
dos copias de la libreta, una para cada usuario y el intercambio a
través de un canal seguro. Una vez que el mensaje se codifica con la
libreta, ésta se destruye y el mensaje codificado es enviado. Del lado
del receptor, se utiliza la libreta para descifrar el mensaje. Una
manera de entender el algoritmo de OTP es pensar en él como una fuente
de ruido del 100\%, que se utiliza para enmascarar el mensaje. Dado que
ambas partes de la comunicación tienen copias de la fuente de ruido son
las únicas personas que pueden filtrarlo.

El algoritmo OTP se encuentra presente en varios sistemas de cifrado de
flujo, que se explican a continuación. Claude Shannon, (un participante
clave en la criptografía moderna y en la teoría de la información), en
su artículo fundamental de 1949, ``Teoría de la Comunicación de Sistemas
Secretos'' demuestra teóricamente que todo sistema de cifrado
irrompibles debe incluir el cifrado OTP, el cual de ser usado
correctamente es imposible de descifrar.

\section{Cifrado moderno}\label{cifrado-moderno}

Tras las guerras mundiales del campo de la criptografía se alejó del
servicio público y se redujo más al ámbito de los gobiernos. Los
principales avances en el campo aparecieron en la década del 70 con el
advenimiento de las computadoras personalizadas y la introducción del
estándar de cifrado de datos (DES, desarrollado por IBM en 1977 y
adoptada más tarde por el gobierno de los EE.UU.). Ahora, a partir del
2001, utilizamos la AES, (Advanced Encryption Standard), que se basa en
formas de cifrado simétrico.

La criptografía moderna se puede dividir generalmente en tres partes:
criptografía simétrica, asimétrica y cuántica.

La criptografía simétrica o de clave secreta, se refiere a sistemas de
cifrado que utilizan la misma clave para cifrar y descifrar el texto o
la información involucrada. En esta clase de sistemas de cifrado la
clave se comparte y se mantiene en secreto dentro de un grupo
restringido y por lo tanto no es posible ver la información cifrada sin
tener la clave. Una simple analogía con la criptografía de clave secreta
es el acceso restringido a un parque comunitario, donde existe llave
para abrir la puerta, que es compartida por la comunidad. No puede abrir
la puerta, a menos que tenga la llave. Obviamente, el problema aquí con
la llave del jardín y con la clave del cifrado simétrico es si caen en
las manos equivocadas, entonces un intruso o un atacante puede ingresar
y la seguridad del jardín, o los datos y la información se verá
comprometida. Por lo tanto uno de los principales problemas con este
tipo de cifrado es el tema de la gestión de claves. Por lo dicho
anteriormente, este método es más utilizado cuando existe un único
usuario o en contextos o entornos de grupos pequeños.

A pesar de esta limitación los métodos de cifrado simétricos son
considerablemente más rápido que los métodos asimétricos y también son
el mecanismo preferido para cifrar grandes trozos de texto.

Los sistemas de cifrado simétricos suelen ser implementado usando
\textbf{cifrados de bloque} ** o \textbf{cifrados de flujo}.

Los cifrados de bloque trabajan tomando los datos de entrada en bloques
de 8, 16 o 32 bytes a la vez y mezclando los datos con la clave dentro
de dichos bloques. Se realizan diferentes operaciones sobre los datos
con el fin de transformarlos y mezclarlos dentro de los bloques. Tales
sistemas de cifrado utilizan una clave secreta para convertir un bloque
fijo de texto plano en texto cifrado. La misma clave se utiliza entonces
para descifrar el texto cifrado.

En comparación con el cifrado de flujo (también conocido como cifrado de
estado), trabaja con cada dígito de texto plano mediante la creación de
un flujo de clave correspondiente que forma el texto cifrado. El flujo
de clave se refiere a una secuencia de caracteres aleatorios (bits,
bytes, números o letras) en el que se llevan a cabo diversas sumas o
restas combinadas sobre un carácter en el mensaje de texto plano, que
produce entonces el texto cifrado. Aunque este método es muy seguro, no
siempre es práctico, ya que la clave de la misma longitud que el mensaje
tiene que ser transmitido de alguna manera segura de modo que el
receptor puede descifrar el mensaje. Otra limitación es que la clave
sólo puede ser utilizado una vez y después debe ser desechada. Aunque
esto puede significar mayor seguridad, limita el uso del cifrado.

Los sistemas de cifrado asimétricos trabajan con problemas matemáticos
más complejos con puertas traseras, lo que permite soluciones más
rápidas en las piezas de datos pequeñas muy importantes. También
trabajan con tamaños de datos fijos, por lo general 1024-2048 bits y 384
bits. Lo que los hace especiales es que ayudan a resolver algunos de los
problemas con la distribución de claves mediante la asignación de una
par, una clave pública y otra privada por persona, así que todo el mundo
sólo necesita saber todas los demás claves públicas. Los sistemas de
cifrado asimétricos se usan también para firmas digitales. Los cifrados
simétricos se utilizan generalmente para la autentificación del mensaje.
Los sistemas de cifrado simétrico no pueden no repudiar firmas (es
decir, las firmas que después usted no puede negar que firmó). Las
firmas digitales son muy importantes en la criptografía moderna. Son
similares a los sellos de cera con los cuales se verificaba de quién
provenía el mensaje y al igual que ellos, son exclusivos de cada
persona. Las firmas digitales son uno de los métodos utilizados en
sistemas de clave pública, que han transformado el campo de la
criptografía y son esenciales para la seguridad en Internet y en las
transacciones online.

\section{Criptografía cuántica}\label{criptografuxeda-cuuxe1ntica}

Criptografía cuántica es el término usado para describir el tipo de
criptografía necesaria para tratar con la velocidad con la cual nosotros
procesamos información y las medidas de seguridad relacionadas que son
necesarias. Esencialmente consiste en usar comunicación cuántica para
asegurar el intercambio de una clave y sus distribuciones asociadas.
Como las máquinas que usamos se han vuelto más rápidas las posibles
combinaciones de cifrado de clave pública y firmas digitales se han
vuelto más vulnerables y la criptografía cuántica trata con los tipos de
algoritmos que son necesarios para mantenerse al tanto con las redes más
avanzadas.

\section{Desafíos e implicaciones}\label{desafuxedos-e-implicaciones}

En el corazón de la criptografía está el reto de cómo usar y comunicar
información. Los métodos anteriores describen cómo cifrar la
comunicación escrita, pero, obviamente, como se muestra en el ejemplo
Navajo, otros medios de comunicación (voz, sonido, imagen, etc) también
se pueden cifrar utilizando diferentes métodos.

El objetivo principal y la habilidad del cifrado consiste en aplicar los
métodos adecuados para brindar una comunicación confiable. Esto se logra
mediante la comprensión de las ventajas y desventajas, fortalezas y
debilidades de los diferentes métodos de cifrado y su relación con el
nivel de seguridad y privacidad necesarias. Obtener este derecho depende
de la tarea y el contexto.

Es importante destacar que cuando hablamos de comunicación, estamos
hablando acerca de la confianza. Tradicionalmente, la criptografía se
ocupaba de los escenarios hipotéticos, donde el desafío era cómo hacer
que `Bob' pudiera hablar con `Alice' de una manera privada y segura.

Nuestras vidas están ahora fuertemente entrelazadas con las computadoras
e Internet. De modo que los límites entre Bob, Alice y el ``otro'' (Eva,
Oscar, el Gran Hermano, su jefe, el ex-novio o el gobierno) son mucho
más borrosos. Teniendo en cuenta el gran avance en el procesamiento de
datos con computadoras, para `nosotros', para que Bob y Alice puedan
confiar en el sistema, tenemos que saber con quién están hablando
demasiado, necesitamos saber quién está escuchando y lo más importante
quién tiene el potencial para espiar . Lo que resulta importante es cómo
navegar por esta complejidad y sentir que mantenemos el control y la
seguridad, para poder participar y comunicarnos de una manera confiable,
respetando nuestras libertades individuales y nuestra privacidad.
Glosario ========

Gran parte de este contenido está basado en
\url{http://en.cship.org/wiki/Special:Allpages}

\section{Administrador de
contraseñas}\label{administrador-de-contraseuxf1as}

Un administrador de contraseñas es software que ayuda al usuario para
organizar sus contraseñas y códigos PIN. El software generalmente tiene
una base de datos local o un archivo que mantiene los datos de las
contraseñas cifrados para conectarse en forma segura con otras
computadoras, redes, sitios web y archivos de aplicaciones. KeePass
http://keepass.info/ es un ejemplo.

\section{Agregador}\label{agregador}

Un agregador es un servicio que ofrece información recolectada de un
sitio y lo pone disponible en diferentes direcciones. Se lo conoce
también como agregador RSS, agregador de feeds, lector de feeds, o
lector de noticias (No se debe confundir con el lector de noticias de
Usenet)

\section{Análisis de amenazas}\label{anuxe1lisis-de-amenazas}

Un análisis de amenazas a la seguridad es un estudio formal, adecuado al
detalle, de todas las maneras conocidas de ataques a la seguridad de los
servidores o los protocolos, o de los métodos usados para un propósito
particular tales como evasión. Las amenazas pueden ser de carácter
técnico, como romper el código o explotar los errores de software, o
sociales, como el robo de contraseñas o sobornar a alguien que tiene un
conocimiento especial. Pocas compañías o individuos tienen el
conocimiento y la habilidad para hacer un análisis global, pero todos
los implicados tienen que hacer alguna estimación de los temas.

\section{Análisis de tráfico}\label{anuxe1lisis-de-truxe1fico}

El análisis de tráfico consiste en el análisis estadístico de las
comunicaciones cifradas. En algunas circunstancias puede revelar
información acerca de la gente comunicada y la información que están
compartiendo.

\section{Ancho de banda}\label{ancho-de-banda}

El ancho de banda de una conexión es la máxima velocidad de
transferencia de datos, limitada por su capacidad y las características
de las computadoras en ambos extremos de la conexión.

\section{Anonimato}\label{anonimato-1}

(No se debe confundir con privacidad, seudoanonimato, seguridad o
confidencialidad)

El anonimato en Internet es la capacidad de utilizar los servicios sin
dejar pistas sobre su identidad o sin ser espiado. El nivel de
protección depende de las técnicas de anonimato utilizados y el grado de
seguimiento. Los más fuertes técnicas en uso para proteger el anonimato
implican la creación de una cadena de comunicación a través de un
proceso aleatorio para seleccionar algunos de los enlaces, en la que
cada eslabón tiene acceso a la información parcial sobre el proceso. El
primero conoce la dirección del usuario de Internet (IP), pero no el
contenido, el destino o finalidad de la comunicación, ya que el
contenido del mensaje e información de destino están cifrados. El último
conoce la identidad del sitio que se está en contacto, pero no la fuente
de la sesión. Algunos pasos intermedios entre los enlaces impiden que el
primero y el último compartan su conocimiento parcial con el fin de
conectar el usuario y el sitio de destino.

\section{Archivo de registro}\label{archivo-de-registro}

Un archivo de registro es un archivo que guarda una secuencia de
mensajes enviados por algún proceso de software, el cual puede ser una
aplicación o un componente del sistema operativo. Por ejemplo, los
servidores web o los proxies pueden mantener registros que contienen
información acerca de cuáles direcciones IP usan estos servicios y
cuándo acceden y a que páginas.

\section{ASP (proveedor de servicios de
aplicaciones)}\label{asp-proveedor-de-servicios-de-aplicaciones}

Un ASP es una organización que ofrece software sobre Internet,
permitiendo actualizarlo y mantenerlo en forma centralizada.

\section{Ataque por fuerza bruta}\label{ataque-por-fuerza-bruta}

Un ataque por fuerza bruta consiste en tratar de averiguar una
contraseña probando todos las variantes posibles. Es uno de los ataques
de hacking más básicos.

\section{Backbone}\label{backbone}

Un backbone (a veces llamado red troncal) es uno de los enlaces de
comunicaciones de gran ancho de banda que une redes en diferentes países
y organizaciones alrededor del mundo en Internet.

\section{Badware}\label{badware}

Consulte \emph{malware}.

\section{Bash (Bourne-again shell)}\label{bash-bourne-again-shell}

El shell bash es una interfaz de línea de comandos para sistemas
operativos GNU/Linux o Unix, basado en el shell Bourn.

\section{BitTorrent}\label{bittorrent-1}

BitTorrent es un protocolo para compartir archivos entre pares,
inventado por Bram Cohen en 2001. Permite a los individuos distribuir de
forma barata y eficaz archivos de gran tamaño, como imágenes de CD,
video o archivos de música.

\section{Bluebar}\label{bluebar}

La barra azul de URL (llamada en la jerga Bluebar Psiphon) es la forma
en la parte superior de la ventana del navegador del nodo Psiphon que le
permite acceder al sitio bloqueado escribiendo su URL en el interior.

Vea también \emph{nodo Psiphon}.

\section{Bloqueo}\label{bloqueo}

El bloqueo impide el acceso a un recurso de Internet, basado en un gran
número de métodos.

\section{Caché}\label{cachuxe9}

La caché es una parte de un sistema de procesamiento de información
usada para almacenar datos usados en forma reciente o muy frecuente con
el fin de acelerar el acceso repetido a ellos. Una caché web mantiene
copias de los archivos de la página web.

\section{Censorware}\label{censorware}

Censorware es software usado para filtrar o bloquear el acceso a
Internet. Este término se usa a menudo para referirse al software
instalado en la máquina cliente (la computadora usada para acceder a
Internet). La mayoría del censorware se utiliza con propósitos de
control parental. Algunas veces el término censorware se utiliza también
para referirse al software usado con los mismos propósitos pero
instalado en un servidor de red o un router.

\section{Censura}\label{censura}

Censurar es evitar la publicación o recuperación de información, o tomar
medidas, legales o de otro tipo, contra los editores y lectores.

\section{CGI (Interfaz de gateway
común)}\label{cgi-interfaz-de-gateway-comuxfan}

CGI es un estándar de uso común que permite a los programas de un
servidor web ejecutarse como aplicaciones. Algunas páginas web basadas
en proxy usan CGI, por lo que se denominan ``proxies CGI''. (Una de las
más populares aplicaciones escrita por James Marshall usa el lenguaje de
programación Perl y se denomina CGIProxy.)

\section{Cifrado}\label{cifrado}

Se denomina así a todo método usado para recodificar y mezclar datos o
transformarlos matemáticamente para que sea ilegible a las terceras
partes y por lo tanto, no puedan descifrar el secreto que oculta. Es
posible cifrar datos en su disco rígido local usando software como
TrueCrypt (http://www.truecrypt.org) o cifrar el tráfico de Internet con
TLS/SSL o SSH.

vea también \emph{descifrado}.

\section{Cifrado completo de disco}\label{cifrado-completo-de-disco}

vea \emph{cifrado de disco}.

\section{Cifrado de disco}\label{cifrado-de-disco}

El cifrado de disco es una tecnología que protege la información al
convertirla en ilegible para que no pueda ser descifrada fácilmente por
personas no autorizadas. Usa software o hardware para cifrar cada bit de
datos que está en un disco o en un volumen de disco. El cifrado de disco
previene el acceso no autorizado a los datos almacenados.

\section{Clave pública}\label{clave-puxfablica}

vea \emph{criptografía de clave pública/cifrado de clave pública}.

\section{Código (de cifrado)}\label{cuxf3digo-de-cifrado}

En criptografía, un código es un algoritmo para realizar cifrado o
descifrado de mensajes.

\section{Confidencialidad directa perfecta
(PFS)}\label{confidencialidad-directa-perfecta-pfs}

En un protocolo de acuerdo de claves autenticadas que utiliza la
criptografía de clave pública, la confidencialidad directa perfecta (PFS
o) es la propiedad que asegura que una clave de sesión derivada de un
conjunto de claves públicas y privadas de largo plazo no se verá
comprometida si una de las claves privadas (a largo plazo) se ve
comprometida en el futuro.

\section{Cookie}\label{cookie}

Un cookie es una cadena de texto enviado por un servidor web al
navegador del usuario para almacenarla en su computadora, conteniendo
información necesaria para mantener la continuidad en las sesiones a
través de múltiples páginas web, o a través de múltiple sesiones.
Algunos sitios web no se pueden usar sin aceptar ni almacenar una
cookie. Algunas personas consideran esto como una invasión a la
privacidad y/o un riesgo de seguridad.

\section{Criptografía}\label{criptografuxeda}

La criptografía es la práctica y el estudio de las técnicas para
establecer comunicaciones seguras en presencia de terceras partes (los
llamados adversarios). En forma más general, consiste en la construcción
y análisis de protocolos para vencer a nuestros adversarios en varios
aspectos relacionados con la seguridad de la información tales como
confidencialidad, integridad, autentificación y no repudio de datos. La
criptografía moderna abarca un amplio rango de disciplinas tales como
matemáticas, ciencias de la computación e ingeniería eléctrica. Sus
aplicaciones incluyen tarjetas ATM, contraseñas de computadoras y
comercio electrónico.

\section{Criptografía de clave pública/cifrado de clave
pública}\label{criptografuxeda-de-clave-puxfablicacifrado-de-clave-puxfablica}

La criptografía de clave pública se refiere al sistema de cifrado que
requiere dos claves separadas, una de las cuales es secreta y la otra
pública. Aunque son diferentes, ambas claves están relacionadas
matemáticamente. Una clave cifra el texto plano y la otra lo descifra.
Ninguna clave puede realizar ambas funciones. Una de ellas puede ser
publicad, mientras que la otra debe mantenerse en privado.

La criptografía de clave pública usa algoritmos de clave asimétricos
(tales como RSA), y se suele llamar por el término mas general
\emph{criptografía de clave asimétrica}.

\section{Clave privada}\label{clave-privada}

vea \emph{criptografía de clave pública/cifrado de clave pública}.

\section{Chat}\label{chat}

El chat, también llamado mensajería instantánea, es un método habitual
de comunicación entre dos o más personas en la cual cada línea tipeada
por un participante en una sesión es vista por los otros. Existen
numerosos protocolos, incluyendo aquellos creados por empresas
específicas (AOL, Yahoo!, Microsoft, Google, y otros) y los definidos
públicamente. Algunos clientes soportan un único protocolo, pero la
mayoría utiliza una variedad de los protocolos más populares.

\section{DARPA}\label{darpa}

DARPA (Defense Advanced Projects Research Agency, Agencia de
investigación de proyectos avanzados de defensa) es el sucesor de ARPA,
que fundó Internet y su predecesor, ARPAnet.

\section{Descifrado}\label{descifrado}

Descifrar es recuperar el texto plano u otros mensajes a partir de un
mensaje cifrado mediante el uso de una clave.

Consulte también \emph{cifrado}.

\section{Dirección IP (dirección del protocolo de
Internet)}\label{direcciuxf3n-ip-direcciuxf3n-del-protocolo-de-internet}

Una dirección IP es un número que identifica una computadora en
particular en Internet. En la versión 4 (IPv4) consiste en cuatro bytes
(32 bits), a menudo representada por cuatro números enteros del rango de
0 a 255 separados por puntos, tal como 74.54.30.85. En IPv6, versión a
la cual actualmente está cambiando la red, una dirección IP es cuatro
veces más larga, y consiste de 128 bits. Puede ser escrita en 8 grupos
de de 4 dígitos hexadecimales separados por dos puntos, por ejemplo
2001:0db8:85a3:0000:0000:8a2e:0370:7334.

\section{Dirección IP públicamente
ruteable}\label{direcciuxf3n-ip-puxfablicamente-ruteable}

Las direcciones IP públicamente ruteables (a veces llamadas direcciones
IP públicas) son aquellas que pueden alcanzarse de forma normal en
Internet, a través de una cadena de enrutadores. Algunas direcciones IP
son privadas, como el bloque 192.168.x.x, y muchas no están asignadas.

\section{DNS (Sistema de nombres de
dominio)}\label{dns-sistema-de-nombres-de-dominio}

El sistema de nombres de dominio (DNS) convierte los nombres de dominio,
compuestos por combinaciones de letras fáciles de recordar, a las
direcciones IP, que son cadenas de números difíciles de recordar. Cada
computadora en Internet tiene una dirección única (algo parecido a un
código de área + número telefónico)

\section{Dominio}\label{dominio}

Un dominio puede ser un dominio de nivel superior (TLD) o un dominio
secundario de Internet.

Vea también \emph{dominio de nivel superior}, \emph{dominio de nivel
superior con código país} y \emph{dominio secundario}.

\section{Dominio de alto nivel con código de país
(ccTLD)}\label{dominio-de-alto-nivel-con-cuxf3digo-de-pauxeds-cctld}

Cada país tiene un código de dos letras, y un TLD (dominio de alto
nivel) basado en él, tal como .ca para Canada; este dominio se llama
dominio de alto nivel con código de país. Cada ccTLD tiene un servidor
DNS que lista todos los dominios de segundo nivel dentro del TLD. Los
servidores raíz apuntan a todos los TLD, y almacenan la información
usada frecuentemente en los dominios de nivel inferior.

\section{Dominio de nivel superior
(TLD)}\label{dominio-de-nivel-superior-tld}

En el ámbito de los nombres de Internet, el TLD es el último componente
del nombre de dominio. Existen diversos TLD genéricos, los más
importantes son .com, .org, .edu, .net, .gov, .mil, .int, y un código de
país de dos letras (ccTLD) diferente para cada uno de ellos, por
ejemplo, .ca for Canada. La Unión Europea también tiene código propio,
.eu.

\section{E-mail (correo
electrónico)}\label{e-mail-correo-electruxf3nico}

E-mail, abreviatura en inglés de correo electrónico, es un método para
enviar y recibir mensajes por Internet. Se puede usar un servicio de web
mail o enviarlos con el protocolo SMTP y recibirlos con el protocolo
POP3 mediante un cliente de correo electrónico tal como Outlook Express
o Thunderbird. Es raro que un gobierno bloquee el correo electrónico,
sin embargo, la vigilancia es muy común. Si el correo electrónico no
está cifrado, será muy fácil de leer por algún operador de red o un
gobierno.

\section{Escuchas ilegales}\label{escuchas-ilegales}

Las escuchas ilegales consisten en interceptar al tráfico de voz o la
lectura o filtrar el tráfico de datos en una línea telefónica o una
conexión de datos digitales, por lo general para detectar o prevenir
actividades ilegales o no deseadas o para controlar o monitorear lo que
la gente está hablando.

\section{Esquema}\label{esquema}

En la Web, un esquema es una asignación de un nombre a un protocolo.
Así, el esquema HTTP asigna URLs que comienzan con HTTP: el Protocolo de
Transferencia de Hipertexto. El protocolo determina la interpretación
del resto de la URL, por lo que http://www.example.com/dir/content.html
identifica un sitio web y un archivo específico en un directorio
específico, y mailto: user@somewhere.com es una dirección de correo
electrónico de una persona o grupo específico en un dominio específico.

\section{Esteganografía}\label{esteganografuxeda}

Esta palabra, que en griego significa escritura ocultar, se refiere a la
variedad de métodos para enviar mensajes ocultos donde no sólo lo está
el contenido, sino que también es muy probable que algo que lo encubra
también lo esté. Generalmente se oculta una cosa dentro de otra, como
una fotografía o un texto dentro de algo sin relación alguna.
Contrariamente a la criptografía, donde está claro que se está
transmitiendo un mensaje secreto, la estenografía intenta ocultar
también a la comunicación en sí misma.

\section{Evasión}\label{evasiuxf3n}

La evasión es publicar o dar acceso a los contenidos, a pesar de los
intentos de censura.

\section{Expresión regular}\label{expresiuxf3n-regular}

Una expresión regular (también conocida como regexp o RE) es un patrón
de texto que especifica un conjunto de cadena de textos en una
implementación particular de una expresión regular tal como la utilidad
grep de UNIX. Una cadena de texto ``encuentra'' una expresión regular si
la cadena concuerda con el patrón, como está definido en la sintaxis de
la expresión regular. En cada sintaxis de RE, algunos caracteres tienen
significados especiales, para permitir que un patrón encuentre múltiples
cadenas. Por ejemplo, la expresión regular lo+se encuentra lose, loose,
and looose.

\section{Filtro}\label{filtro}

Es alguna forma de búsqueda de patrones de datos para bloquear o
permitir las comunicaciones.

\section{Filtro de bajo ancho de
banda}\label{filtro-de-bajo-ancho-de-banda}

Un filtro de bajo ancho de banda es un servicio web que remueve
elementos extraños tales como publicidades e imágenes de una página web
y además la comprime, haciendo mucho más rápida la descarga.

\section{Filtro de palabra clave}\label{filtro-de-palabra-clave}

Un filtro de palabra clave escanea todo el tráfico de Internet que pasa
a través de un servidor para hallar palabras o términos prohibidos para
poder bloquearlas.

\section{Firefox}\label{firefox}

Firefox es el navegador web open source más popular, desarrollado por la
Fundación Mozilla.

\section{Foro}\label{foro}

En un sitio web, un foro es un lugar para la discusión, donde los
usuarios pueden publicar mensajes y comentar otros previamente
publicados. Se diferencia de una lista de correo o un grupo de Usenet
por la persistencia de las páginas que contienen los hilos de los
mensajes. Los archivos de grupo de noticias y listas de correo, sin
embargo, muestran habitualmente un mensaje por página, con páginas de
navegación que listan solamente las cabeceras de los mensajes en un
hilo.

\section{Frame (marco)}\label{frame-marco}

Un frame es una porción de una página web que posee su propia URL. Por
ejemplo, los frames se usan habitualmente para colocar un menú estático
cercano a una ventana con texto deslizante.

\section{FTP (Protocolo de transferencia de
archivo)}\label{ftp-protocolo-de-transferencia-de-archivo}

El protocolo FTP se usa para transferir archivos. Mucha gente lo usa
generalmente para descargas; aunque se puede usar también para cargar
páginas web y scripts para algunos servidores web. Usa habitualmente los
puertos 20 y 21, que a veces están bloqueados. Algunos servidores FTP
escuchan en otros puertos, pudiendo evadir el bloqueo basado en puertos.

Un cliente FTP popular libre para Windows y Mac OS es FileZilla. Existen
también algunos clientes FTP basados en la web que pueden usarse con un
navegador normal tal como Firefox.

\section{Fuga de DNS}\label{fuga-de-dns}

Una fuga de DNS ocurre cuando un equipo que está configurado para usar
un proxy para su conexión a Internet, sin embargo hace consultas DNS sin
usarlo, lo que expone a los intentos de los usuarios para conectarse con
sitios bloqueados. Algunos navegadores web tienen opciones de
configuración para forzar el uso del proxy.

\section{Gateway}\label{gateway}

Un gateway es un nodo que conecta dos redes en Internet. Un ejemplo
importante son los gateways nacionales a través de los cuales pasa todo
el tráfico, tanto entrante como saliente.

\section{GNU Privacy Guard}\label{gnu-privacy-guard}

GNU Privacy Guard (GnuPG o GPG) es una aplicación de software de
criptografía, alternativa de PGP, con licencia libre GPL. Cumple con la
especificación RFC 4880, la especificación estándar actual IETF de
OpenPGP.

vea también \emph{PGP}.

\section{GPG}\label{gpg}

vea \emph{GNU Privacy Guard}.

\section{Honeypot}\label{honeypot}

Un honeypot es un sitio web que simula ofrecer un servicio para tentar a
usuarios potenciales para que lo usen, y poder capturar información
sobre ellos o sus actividades.

\section{HTTP (Protocolo de transferencia de
hipertexto)}\label{http-protocolo-de-transferencia-de-hipertexto}

HTTP es el protocolo fundamental de la World Wide Web, que proporciona
métodos para solicitar y mostrar páginas Web, consultar y generar
respuestas a las consultas, y el acceso a una amplia gama de servicios.

\section{HTTPS (HTTP seguro)}\label{https-http-seguro}

Es un protocolo de comunicación segura mediante cifrado de mensajes
HTTP. Los mensajes entre el cliente y el servidor se cifran en ambas
direcciones, utilizando claves generadas cuando la conexión se solicitó
y se intercambiaron con seguridad. Las direcciones IP de origen y
destino están en las cabeceras de cada paquete, así que HTTPS no puede
ocultar el hecho de la comunicación, sólo el contenido de los datos
transmitidos y recibidos.

\section{IANA}\label{iana}

IANA (Internet Assigned Numbers Authority, autoridad de asignación de
números de internet) es la organización responsable de los trabajos
técnicos en la gestión de la infraestructura de Internet, incluyendo la
asignación de bloques de direcciones IP para dominios de nivel superior
y los registradores de licencias de dominio para los ccTLD y de los
dominios de nivel superior genéricos, la ejecución de los servidores
raíz de Internet, y otros funciones.

\section{ICANN}\label{icann}

ICANN (Internet Corporation for Assigned Names and Numbers, corporación
de internet para nombres y números asignados) es una corporación creada
por el Departamento de Comercio de los EEUU para administrar los niveles
más altos de Internet. El trabajo técnico lo lleva a cabo IANA.

\section{Mensajería instantánea
(IM)}\label{mensajeruxeda-instantuxe1nea-im}

La mensajería instantánea se refiere a chatear usando protocolos
propietarios, o a chatear en general. Los clientes de mensajería
instantánea más comunes son MSN Messenger, ICQ, AIM o Yahoo! Messenger.

\section{Intermediario}\label{intermediario}

vea \emph{man in the middle}.

\section{Intercambio de archivos}\label{intercambio-de-archivos}

El intercambio de archivos se refiere a cualquier sistema de computadora
donde mucha gente puede usar la misma información, pero a menudo se
refiere a música, películas u otros materiales disponibles libres de
cargo en Internet.

\section{Interfaz común de gateway}\label{interfaz-comuxfan-de-gateway}

vea \emph{CGI}.

\section{Interfaz de línea de
comandos}\label{interfaz-de-luxednea-de-comandos}

Es un método para controlar la ejecución de software usando comandos
ingresados con un teclado, tal como un shell de Unix o una línea de
comandos de Windows.

\section{Internet}\label{internet}

Internet es una red de redes interconectadas que usan TCP/IP y otros
protocolos de comunicación.

\section{IRC (Internet relay chat)}\label{irc-internet-relay-chat}

IRC es un protocolo de Internet de más de 20 años de antiguedad usado
para conversaciones de texto en tiempo real (chat o mensajería
instantánea). Existen distintas redes IRC, la mayor posee más de 50.000
usuarios.

\section{ISP (Proveedor de servicio de
internet)}\label{isp-proveedor-de-servicio-de-internet}

Un ISP (proveedor de servicio de Internet) es una empresa u organización
que suministra acceso a Internet para sus clientes.

\section{JavaScript}\label{javascript}

JavaScript es un lenguaje de scripting, de uso habitual en las páginas
web que suministran funciones interactivas.

\section{KeePass, KeePassX}\label{keepass-keepassx}

KeePass y KeePassX son dos tipos de administradores de contraseñas.

\section{Latencia}\label{latencia}

La latencia es una medida del tiempo de demora experimentado en un
sistema, en este contexto, en una computadora en la red. Se mide como el
tiempo transcurrido entre el comienzo de la transmisión de un paquete y
el comienzo de su recepción, entre un extremo de la red (por ejemplo
usted) y el otro (por ejemplo el servidor web). Una manera muy poderosa
de filtrado web es mantener una muy alta latencia, la que provoca que el
uso de muchas herramientas de evasión sea muy dificultoso.

\section{Lista blanca}\label{lista-blanca}

Una lista blanca (whitelist) es una lista de sitios específicamente
autorizados para establecer alguna forma particular de comunicación. Se
puede filtrar el tráfico con una lista blanca (bloqueando todo excepto
los sitios de la lista), una lista negra (permitiendo todo excepto los
sitios de la lista), una combinación de ambas u otras políticas basadas
en reglas y condiciones específicas.

\section{Lista negra}\label{lista-negra}

Una lista negra (blacklist) es una lista de cosas prohibidas. Para la
censura en Internet, es una lista de los sitios web o las direcciones IP
de las computadoras prohibidas; se permite acceder a todos los sitios
y/o computadoras excepto a aquellos listados específicamente. Una
alternativa es una lista blanca, o lista de cosas permitidas. Una lista
blanca bloquea el acceso a todos los sitios excepto a aquellos
específicamente listados. No es muy común. Es posible combinar ambos
tipos de listas usando cadenas de búsqueda u otras técnicas
condicionales en URL que no coincidan con ninguna de las listas.

\section{Malware}\label{malware}

Malware es un término general para referirse a software malicioso,
incluyendo a los virus, que pueden estar instalados o pueden ser
ejecutados sin su conocimiento. El malware toma el control de su
computadora para fines específicos como, por ejemplo, enviar spam.
(También se conoce al malware como badware.)

\section{Man in the middle}\label{man-in-the-middle}

Un man in the middle (\emph{hombre en el medio}) es una persona o
computadora que captura tráfico en un canal de comunicación,
principalmente para realizar cambios selectivos o bloquear el contenido
de una manera que socave la seguridad criptográfica. En general, el
ataque man-in-thr-middle implica pasar por un sitio Web, servicio o
individuo con el fin de registrar o alterar las comunicaciones. Los
gobiernos pueden ejecutar man-in-thr-middle en los gateways de entrada a
un país por donde pasa todo el tráfico.

\section{Marcador}\label{marcador}

Un marcador es una referencia a una posición dentro del software que
apunta a un recurso externo. En un navegador, es una referencia a una
página web -- al elegir un marcador usted puede cargar rápidamente el
sitio web sin necesidad de tipear la URL completa.

\section{Monitoreo}\label{monitoreo}

El monitoreo consiste en el control continuo de un flujo de datos en
busca de actividad no deseada.

\section{Motor de difusión de
archivos}\label{motor-de-difusiuxf3n-de-archivos}

Un motor de difusión de archivos es un sitio web editor que puede ser
usado para eludir la censura. Un usuario sólo tiene que cargar el
archivo a publicar una vez y el motor lo propaga a un conjunto de
servicios de almacenamiento compartido (como Rapidshare o Megaupload).

\section{NAT (Traducción de dirección de
red)}\label{nat-traducciuxf3n-de-direcciuxf3n-de-red}

NAT es una función de un router para ocultar un espacio de direcciones
de reasignación. Todo el tráfico que sale del router, utiliza su
dirección IP, y el router sabe cómo enrutar el tráfico entrante a quien
se lo solicite. NAT es frecuentemente aplicado por los cortafuegos.
Puesto que las conexiones entrantes son normalmente prohibidas por NAT,
se hace difícil ofrecer un servicio al público en general, como un sitio
Web o un proxy público. En una red donde NAT está en uso, ofrecer este
servicio requiere algún tipo de configuración de cortafuegos o método
NAT transversal.

\section{Nodo}\label{nodo}

Un nodo es un dispositivo activo en una red. Un router es un ejemplo de
un nodo. En las redes Psiphon y Tor, un servidor se conoce también como
nodo.

\section{Nodo abierto}\label{nodo-abierto}

Un nodo abierto es un nodo específico Psiphon que puede ser usado sin
loguearse. Éste carga automáticamente una página de inicio propia, y se
presenta a sí mismo en un lenguaje propio, pero puede ser usado por
cualquier navegador web.

vea también \emph{nodo Psiphon}.

\section{Nodo de enlace o intermedio}\label{nodo-de-enlace-o-intermedio}

Un nodo intermedio es un nodo Tor que no es un nodo de salida. La
ejecución de un nodo intermedio puede ser más segura que la ejecución de
un nodo de salida, porque un nodo intermedio no se mostrará en los
archivos de registro de terceros. (Un nodo intermediario a veces se
llama un nodo sin salida.)

\section{Nodo de salida}\label{nodo-de-salida}

Un nodo de salida es un nodo Tor que reenvía datos fuera de la red Tor.

\section{Nodo privado}\label{nodo-privado}

Un nodo privado es un nodo Psiphon que trabaja con autentificación, lo
que significa que usted debe registrarse antes de poder usarlo. Hecho
esto, podrá enviar invitaciones a su amigos para que usen este nodo
específico. Vea también \emph{nodo Psiphon}.

\section{Nodo Psiphon}\label{nodo-psiphon}

Un nodo Psiphon es un proxy web seguro diseñado para evadir la censura
en Internet. Fue desarrollado por Psiphon inc. Psiphon puede ser de
código libre o privativo.

\section{Nodo sin salida}\label{nodo-sin-salida}

vea \emph{nodo de enlace o intermedio}.

\section{Ofuscación}\label{ofuscaciuxf3n}

La ofuscación consiste en ocultar texto utilizando técnicas de
transformación fáciles de entender y de revertir que resistan la
inspección casual, pero no al criptoanálisis, o hacer cambios menores en
las cadenas de texto para prevenir comparaciones simples. Los proxies
Web suelen utilizar la ofuscación para ocultar ciertos nombres y
direcciones de los filtros de texto simples que pueden ser engañados.
Por ejemplo, cualquier nombre de dominio puede contener opcionalmente un
punto final, como en ``somewhere.com.'', Pero algunos filtros pueden
buscar sólo ``somewhere.com'' (sin el punto final).

\section{Operador de red}\label{operador-de-red}

Un operador de red es una persona u organización que mantiene o controla
una red y se encuentra en posición de monitorear, bloquear o alterar la
comunicación que pasa a través de su red.

\section{OTR (mensajes sin registro)}\label{otr-mensajes-sin-registro}

Un mensaje sin registro, comúnmente denominado OTR, es un protocolo
criptográfico que suministra un cifrado fuerte para conversaciones de
mensajería instantánea.

\section{Paquete}\label{paquete}

Un paquete es una estructura de datos definida por un protocolo de
comunicación que contiene información específica en formas
predeterminadas junto con datos arbitrarios para ser comunicados de un
punto a otro. Los mensajes se dividen en partes que se almacenan en
paquetes para ser transmitidos y luego se ensamblan en el otro extremo
del enlace.

\section{Pastebin}\label{pastebin}

Es un servicio web donde cualquier tipo de texto puede ser cargado y
leído sin tener que registrarse. Todos los textos son visibles
públicamente.

\section{P2P}\label{p2p}

Una red de pares (o P2P, \emph{peer to peer}) es una red de computadoras
entre iguales. A diferencia de las redes cliente-servidor no hay un
servidor central por lo que el tráfico se distribuye sólo entre
clientes. Esta tecnología se aplica sobre todo en los programas de
intercambio como BitTorrent, eMule y Gnutella. Pero también la
tecnología del muy antiguo Usenet o del programa de VoIP Skype VoIP se
pueden clasificar como sistemas P2P.

vea también \emph{compartir archivos}.

\section{\texorpdfstring{PGP (Pretty Good Privacidad, \emph{privacidad
bastante
buena})}{PGP (Pretty Good Privacidad, privacidad bastante buena)}}\label{pgp-pretty-good-privacidad-privacidad-bastante-buena}

PGP es un programa de computadora para cifrado que suministra privacidad
criptográfica y autentificación para comunicación de datos. Se utiliza a
menudo para firmar, cifrar y descifrar textos, correos electrónicos,
archivos, directorios y particiones de discos para incrementar la
seguridad de las comunicaciones por correo electrónico.

PGP y otros productos similares siguen el estándar OpenPGP (RFC 4880)
para cifrado y descifrado de datos.

\section{PHP}\label{php}

PHP es un lenguaje de scripting diseñado para crear sitios web dinámicos
y aplicaciones web. Se instala en un servidor. Por ejemplo, el popular
proxy web PHProxy usa esta tecnología.

\section{POP3}\label{pop3}

El protocolo POP3 (Post Office Protocol version 3) es usado para recibir
correos electrónicos de un servidor, por defecto en el puerto 110 con un
programa de correo electrónico tal como Outlook Express o Thunderbird.

\section{Privacidad}\label{privacidad}

La protección de la intimidad consiste en impedir la divulgación de
información privada personal sin el consentimiento de la persona
interesada. En este contexto, significa impedir que algunos observadores
se enteren de que una persona haya solicitado o recibido información que
ha sido bloqueada o es ilegal en el país donde se encuentre la persona
en cuestión.

\section{Protocolo}\label{protocolo}

Una definición formal de un método de comunicación, y la forma en que
los datos deben ser transmitidos. Además, se refiere al propósito de tal
método de comunicación. Por ejemplo, el protocolo para la transmisión de
paquetes de datos en Internet (IP), o el protocolo de transferencia de
hipertexto para las interacciones en la World Wide Web (HTTP).

\section{Proxy Web}\label{proxy-web}

Un proxy web es un script que se ejecuta en un servidor que actúa como
un proxy/gateway. Los usuarios pueden acceder al proxy web con su
navegador habitual (por ejemplo Firefox) e ingresar cualquier URL en el
formulario localizado en el sitio web. luego el programa del servidor
recibirá el contenido y lo mostrará al usuario. De esta forma el ISP
sólo verá una conexión al servidor con el proxy web ya que no se ha
establecido una conexión directa.

\section{Puente}\label{puente}

Vea \emph{Puente Tor}

\section{Puente Tor}\label{puente-tor}

Un puente es un nodo intermedio que no está listado en el directorio
público principal de Tor, por lo que es especialmente útil en países
donde las comunicaciones están bloqueadas. A diferencia del caso de los
nodos de salida, las direcciones IP de los nodos puente nunca aparecen
en los archivos de registro del servidor y nunca pasan a través de los
nodos de control de manera que pueden ser conectados con la evasión.

\section{Puerto}\label{puerto}

Un puerto de hardware en una computadora es un conector físico para un
propósito específico que usa un protocolo de hardware propio. Algunos
ejemplos son el puerto de la pantalla VGA o un conector USB.

Los puertos de software también conectan computadoras y otros
dispositivos en las redes usando distintos protocolos, pero existen en
el software solamente como números. Los puertos son algo así como los
números puestos sobre las puertas que dan acceso a distintas
habitaciones, cada uno con un servicio especial en un servidor o en una
PC. Están identificados por números enteros entre 0 y 65535.

\section{Remailer}\label{remailer}

Un remailer anónimo es un servicio que le permite a los usuarios enviar
correos electrónicos anónimamente. El remailer recibe los mensajes y lo
reenvía a su destinatario después de remover la información que podría
identificar al remitente original. Algunos servicios también proveen una
dirección anónima que puede ser usada para recibir las respuestas sin
descubrir su identidad. Algunos servicios de remailer conocidos incluyen
a Cypherpunk, Mixmaster y Nym.

\section{Router}\label{router}

Un router es una computadora que determina la ruta para reenviar
paquetes. Utiliza la información de la dirección en la cabecera del
paquete y la información de la caché en el servidor para hallar los
números de dirección con conexiones de hardware.

\section{RSS (agregador de noticias)}\label{rss-agregador-de-noticias}

RSS es un método y un protocolo que le permite a los usuarios de
Internet suscribirse al contenido de una página web, y recibir
actualizaciones tan pronto como sean publicadas.

\section{Salto (Hope)}\label{salto-hope}

Es un enlace en una cadena de paquetes transferidos desde una
computadora a otra, o alguna computadora a lo largo de la ruta. El
número de saltos entre computadoras puede brindar una estimación de la
demora (latencia) en las comunicaciones entre ellas. Cada salto
individual es también una entidad que posee la capacidad de escuchar,
bloquear o alterar las comunicaciones.

\section{Screenlogger}\label{screenlogger}

Un screenlogger es un software capaz de registrar todo lo que su
computadora muestra en la pantalla. Su principal característica es
capturar la pantalla y el login en archivos para consultarlos en otro
momento. Los screenloggers pueden utilizarse como una poderosa
herramienta de monitoreo. Debe ser precavido con todas las pantallas de
login que se ejecuten en la computadora que esté usando, en todo
momento.

\section{Script}\label{script}

Un script es un programa, generalmente escrito en un lenguaje
interpretado, no compilado (tal como JavaScript o Java), o en un
lenguaje interpretado de comandos tal como bash. Muchas páginas web
incluyen scripts para administrar la interacción con ella, y entonces el
servidor no necesita reenviar cada página ante un nuevo cambio.

\section{Script embebido}\label{script-embebido}

Un script embebido es una pieza de código de software.

\section{Servidor de nombre raíz}\label{servidor-de-nombre-rauxedz}

Un servidor de nombre raíz es uno de los trece grupos de servidores
administrados por la IANA para dirigir el tráfico a todos los dominios
de primer nivel, como el núcleo del sistema DNS.

\section{Servidor DNS}\label{servidor-dns}

A servidor DNS, o servidor de nombres, es un servidor que proporciona la
función de consulta del sistema de nombres de dominio. Esto se hace ya
sea mediante el acceso a un registro existente en caché de la dirección
IP de un dominio específico, o mediante el envío de una solicitud de
información a otro servidor de nombres.

\section{Servidor proxy}\label{servidor-proxy}

Un servidor proxy es un servidor, sistema de computadora o un programa
de aplicación que funciona como pasarela entre un cliente y un servidor
web. Un cliente se conecta al servidor proxy y envía una petición a una
página web desde un servidor diferente. Luego el servidor proxy accede
al recurso conectándose al servidor especificado, y devuelve la
información al sitio solicitante. Los servidores proxy pueden servir
para diferentes propósitos, incluyendo el acceso a páginas web
prohibidas o para ayudar a los usuarios a enrutarse sin obstáculos.

\section{Shell (terminal, consola)}\label{shell-terminal-consola}

Un shell de UNIX es una interfaz de usuario de línea de comandos
tradicional para sistemas operativos UNIX y GNU/Linux. Los shells más
comunes son sh y bash.

\section{Smartphone (teléfono
inteligente)}\label{smartphone-teluxe9fono-inteligente}

Un smartphone es un teléfono móvil que ofrece capacidades de
conectividad y computación más avanzadas que cualquier teléfono móvil
común contemporáneo, tales como acceso web, capacidad de ejecución de
sistemas operativos elaborados y de aplicaciones integradas.

\section{SOCKS}\label{socks}

Un proxy socks es una clase especial de servidor proxy. En el modelo OSI
opera entre las capas de aplicación y de transporte. El puerto estándar
para un proxy SOCKS es 1080, pero puede correr en otros. Algunos
programas soportan una conexión a través de un proxy SOCKS. Otra opción
es instalar un cliente como FreeCap, ProxyCap o SocksCap los cuales
pueden forzar a los programas a correr a través de un proxy Socks usando
reenvío por puerto dinámico. También es posible utilizar herramientas
SSH tales como OpenSSH como un servidor proxy SOCKS.

\section{Software de cadena de
claves}\label{software-de-cadena-de-claves}

vea \emph{administración de contraseñas}

\section{Spam}\label{spam}

El spam son los mensajes que saturan a un canal de comunicación
utilizado por la gente, sobre todo con publicidad comercial, enviados a
un gran número de individuos o a grupos de discusión. La mayoría del
spam anuncia productos o servicios que son ilegales en una o más formas,
casi siempre incluyendo el fraude. El filtrado de contenidos de correos
electrónicos para bloquear el spam, con el permiso del destinatario, es
una práctica universalmente extendida.

\section{SSH (shell seguro)}\label{ssh-shell-seguro}

El SSH o shell seguro es un protocolo de red que permite las
comunicaciones cifradas entre computadoras. Se inventó para suceder al
protocolo sin cifrado Telnet, usado para acceder a un shell en un
servidor remoto.

El puerto estándar SSH es el puerto 22. Puede ser usado para eludir la
censura en Internet con el reenvío de puertos o como túnel de otros
programas tales como VNC.

\section{\texorpdfstring{SSL (Secure Sockets Layer, \emph{capa de
conexión
segura})}{SSL (Secure Sockets Layer, capa de conexión segura)}}\label{ssl-secure-sockets-layer-capa-de-conexiuxf3n-segura}

SSL (o Secure Sockets Layer), es un estándar de cifrado usado para
realizar transacciones seguras en Internet. Es la base sobre la cual se
creó el TLS (Transport Layer Security, \emph{capa de transporte
segura}). Puede averiguar fácilmente si está usando SSL observando la
URL en su navegador web (por ejemplo Firefox o Internet Explorer): si
comienza con https en lugar de http, su conexión está cifrada.

\section{Subdominio}\label{subdominio}

Un subdominio es una parte de un dominio mayor. Por ejemplo,
``wikipedia.org'' es el dominio de Wikipedia, ``es.wikipedia.org'' es el
subdominio de la versión en español de Wikipedia.

\section{Texto plano}\label{texto-plano}

El texto plano es un texto sin formato que consiste en una secuencia de
códigos de caracteres, como en ASCII o en Unicode.

\section{Texto sin formato}\label{texto-sin-formato}

El texto sin formato es texto sin cifrar, o texto descifrado.

vea también \emph{cifrado, TLS/SSL, SSH}.

\section{TLS (Seguridad en capa de
transporte)}\label{tls-seguridad-en-capa-de-transporte}

TLS es un estándar de cifrado basado en SSL, usado para realizar
transacciones seguras en Internet.

\section{TCP/IP (Protocolo de control de transmisión sobre protocolo de
Internet)}\label{tcpip-protocolo-de-control-de-transmisiuxf3n-sobre-protocolo-de-internet}

TCP e IP son los protocolos fundamentales de Internet, ya que manejan la
transmisión de paquetes y su ruteo. Existen algunas pocos protocolos
alternativos para ser usados en este nivel de estructura de Internet,
por ejemplo UDP.

\section{Túnel}\label{tuxfanel}

Un túnel es una ruta alternativa desde una computadora a otra,
generalmente incluye un protocolo que especifica el cifrado de los
mensajes.

\section{Túnel DNS}\label{tuxfanel-dns}

Un túnel DNS es una forma de túnel a través de servidores de nombres
DNS.

Debido a que ``abusa'' del sistema de DNS para un propósito deseado,
sólo se permite una conexión muy lenta de aproximadamente 3 kbs/s que es
incluso menor que la velocidad de un módem analógico. Eso no es
suficiente para YouTube o para compartir archivos, pero debería ser
suficiente para la mensajería instantánea como ICQ o MSN Messenger y
también para el texto sin formato del correo electrónico.

En la conexión que desea utilizar un túnel de DNS, sólo tiene el puerto
53 disponible, pero aún funciona en muchos proveedores comerciales de
Wi-fi sin necesidad de pagar.

El problema principal es que no hay servidores de nombres públicos
modificados que se pueden utilizar. Usted tiene que configurar su
cuenta. Usted necesita un servidor con una conexión permanente a
Internet con Linux. Allí puede instalar el software libre ozymandns y en
combinación con SSH y un proxy como Squid puede utilizar el túnel. Más
información sobre esto en http://www.dnstunnel.de.

\section{UDP (Paquete de datagramas de
usuario)}\label{udp-paquete-de-datagramas-de-usuario}

UDP es un protocolo alternativo usado con IP. Se puede acceder a la
mayoría de los servicios de Internet usando TCP o UDP, pero existen
algunos servicios que están definidos para usar exclusivamente alguno de
los dos. Se usa habitualmente UDP en aplicaciones multimedia en tiempo
real tales como llamadas telefónicas en Internet (VoIP).

\section{URL (localizador uniforme de
recursos)}\label{url-localizador-uniforme-de-recursos}

La URL es la dirección del sitio web. Por ejemplo, la URL para la
sección de noticias internacionales del periódico New York Times es
http://www.nytimes.com/pages/world/index.html. Muchos sistemas de
censura pueden bloquear una URL simple. Algunas formas sencillas de
eludir el bloqueo es oscureciendo a la URL. Una manera de hacerlo es
agregando un punto al final del nombre del sitio, entonces la URL
http://en.cship.org/wiki/URL se convierte en
http://en.cship.org./wiki/URL. ; si tiene suerte con este truco podrá
acceder a sitios bloqueados.

\section{Usenet}\label{usenet}

Usenet es un sistema de foros de discusión de más de 20 años de
antiguedad al que se accede mediante el protocolo NNTP. Los mensajes no
se almacenan en un servidor pero se encuentran en muchos servidores que
distribuyen su contenido constantemente. Debido a esto es imposible
censurar Usenet como un todo, no obstante el acceso a Usenet puede y se
bloquea a menudo, y cualquier servidor en particular es probable que
lleve sólo un subconjunto de grupos de noticias de Usenet localmente
aceptables. Existen archivos de Google con toda la historia disponible
de mensajes de Usenet para su búsqueda.

\section{VoIP (Protocolo de voz sobre
Internet)}\label{voip-protocolo-de-voz-sobre-internet}

VoIP se refiere a uno de varios protocolos para comunicación entre dos
voces en tiempo real en Internet, que es notoriamente más barata que la
llamada entre redes de voz de compañías telefónicas estándares. Una
ventaja es que no pueden ser objeto de escuchas telefónicas como las
practicadas en la telefonía tradicional, pero pueden ser monitoreadas
usando tecnología digital. Muchas compañías producen software y
equipamiento para escuchar llamadas VoIP; las tecnologías de VoIP
cifrado y seguro recién se están desarrollando.

\section{VPN (red privada virtual)}\label{vpn-red-privada-virtual}

Una VPN es una red de comunicación privada usada por muchas empresas y
organizaciones para conectarse en forma segura sobre una red pública.
Generalmente está cifrada y nadie, excepto los extremos de la
comunicación pueden ver el tráfico de datos. Existen varios estándares
tales como IPSec, SSL, TLS. El uso de un proveedor de VPN es un método
muy rápido, seguro y conveniente para eludir la censura en Internet con
bajo riesgo pero generalmente tiene un costo mensual. Sin embargo, tenga
en cuenta que el estándar PPTP no es considerado muy seguro, por lo cual
se desaconseja su uso.

\section{Webmail}\label{webmail}

Webmail es un servicio a través de un sitio web. El servicio envía y
recibe los mensajes de correo para los usuarios de la manera habitual,
pero suministra una interfaz web para leer y administrar mensajes, como
una alternativa al uso de un cliente de correo tal como Outlook Express
o Thunderbird en la computadora del usuario. Por ejemplo, un popular
servicio de webmail libre es https://mail.google.com/

\section{WHOIS}\label{whois}

WHOIS (who is, \emph{¿quién es?}) es la función de Internet que permite
realizar consultas a bases de datos WHOIS remotas para obtener
información de registro de dominios. Mediante la realización de una
simple búsqueda WHOIS puede descubrir cuándo y quién ha registrado un
dominio, información de contacto y más.

Una búsqueda WHOIS también puede revelar el nombre o la red mapeada a
una dirección IP numérica.

\section{World Wide Web (WWW)}\label{world-wide-web-www}

La World Wide Web es la red de dominios y páginas de contenido con
hipervínculos accesibles usando el protocolo de transferencia de
hipertexto y sus numerosas extensiones. La World Wide Web es la parte
más famosa de Internet. La necesidad del Open Source
============================

Los últimos 20 años han visto un inserción de las tecnologías de redes
cada vez más profunda en nuestras vidas, informando la forma de
comunicarnos y de actuar en el mundo. Esto acarrea algunos riesgos:
cuanto menos sepamos acerca del entorno de red del cual dependemos, más
vulnerables somos a la explotación.

Esta ignorancia es algo que tradicionalmente aprovechan los criminales.
En los últimos años, sin embargo, algunas empresas y gobiernos se han
aprovechado de la ignorancia civil en la búsqueda de un mayor control.
Este flagrante y a menudo encubierto ataque a la dignidad afecta a
muchos derechos básicos, y en particular, al derecho a la intimidad.

El software de código cerrado ha sido una gran bendición para esa
explotación - principalmente debido al hecho de que no hay código
abierto disponible para que la comunidad pueda auditar la seguridad en
forma descentralizada. Bajo la premisa de proteger secretos comerciales,
los desarrolladores de software de código cerrado han demostrado no
estar dispuestos a explicar a los usuarios cómo trabajan sus programas.
Esto puede que no sea siempre un problema si no fuera por los altos
riesgos: el robo de identidad, la distribución de opiniones y
sentimientos profundamente personales, los intereses de diferentes
personas e incluso su casa cada vez más entran en contacto directo con
el software en una contexto de redes mundiales. Muchas personas usan
software con propósitos personales con la plena confianza de que es
seguro. El sistema operativo Windows en sí es el más obvio ejemplo del
mundo real. OS X de Apple lo sigue de cerca, con un gran porcentaje del
funcionamiento interno del sistema operativo excluido de la inspección
pública.

En criptografía existe un principio poderoso, establecido en el siglo 19
por \emph{Auguste Kerckhoff} que estipula que

\begin{quote}
``{[}el método de cifrado{]} no debe exigirse que sea secreto, y debe
ser capaz de caer en manos del enemigo sin inconvenientes''.
\end{quote}

Mientras este principio ha sido llevado lejos por la mayoría de los
científicos y (por supuesto) las comunidades de código abierto -
mediante la publicación de sus métodos y su funcionamiento interno, por
lo que las posibles deficiencias pueden señalarse y corregirse
previamente a su distribución - la mayoría de los distribuidores de
software propietario dependen del secreto para ocultar las debilidades
de su software. De hecho, a menudo tratan las vulnerabilidades
recientemente descubiertas de una manera poco transparente - dejando a
muchos usuarios confiados expuestos a ser explotados.

Por supuesto, hay que decir que el software de código abierto es tan
seguro como usted lo pueda hacer (y existe gran cantidad de software de
código abierto escrito por principiantes). Sin embargo, hay muchos
buenos ejemplos de código bien escrito, programas bien administrados que
tienen una gran (y preocupada) base de usuarios que encuentran y
resuelven incluso el más pequeño de los errores rápidamente. Este es
especialmente el caso del software que depende de un contexto de red.

Utilizar software de código cerrado en un contexto de red no es sólo ser
parte de una minoría, significa excluirse de una amplia comunidad de
investigadores y especialistas interesados en su privacidad y en su
seguridad.

Nótese bien que existe una opinión más cínica del software de código
abierto, que señala que, dado que nadie es pagado a tiempo completo para
llevar a cabo una constante revisión y prueba de regresión, los últimos
retoques de los programadores inexpertos o maliciosos podrían provocar
importantes fallas de seguridad que no son detectadas por largos
períodos de tiempo, dejándolo vulnerable a los hackers, criminales,
agencias de inteligencia,etc. - por ejemplo, el problema (ahora
resuelto) del generador de números aleatorios predecibles de Debian
GNU/Linux que dio lugar a la creación de muchas claves criptográficas
débiles -.

\chapter{La necesidad del software libre (o por qué es preferible al
open
source)}\label{la-necesidad-del-software-libre-o-por-quuxe9-es-preferible-al-open-source}

¿Por qué usar software libre? El software libre se podría definir como
todos aquellos programas informáticos que respetan las cuatro libertades
fundamentales de los usuarios, a saber: \emph{La libertad de ejecutar el
programa para cualquier propósito} (libertad 0). \emph{La libertad de
estudiar cómo funciona el programa, y cambiarlo para que haga lo que
usted quiera} (libertad 1). El acceso al código fuente es una condición
necesaria para ello. \emph{La libertad de redistribuir copias para
ayudar a su prójimo} (libertad 2). \emph{La libertad de distribuir
copias de sus versiones modificadas a terceros} (libertad 3). Esto le
permite ofrecer a toda la comunidad la oportunidad de beneficiarse de
las modificaciones. El acceso al código fuente es una condición
necesaria para ello.

Como bien puede ver, el eje principal es la libertad de los usuarios. En
el momento en que los programadores liberan su software con licencias
libres, el mismo deja de pertenecerles, pasando a la comunidad. Lo que
importa es el propósito de los usuarios, no el de los programadores. La
importancia del software libre es que promueve la solidaridad social:
compartir y cooperar. La importancia de estas libertades aumenta a
medida que nuestra cultura y nuestras actividades cotidianas se vinculan
cada vez más con el mundo digital. El software libre se vuelve cada vez
más esencial para la libertad en general.

Aunque no era su propósito inicial, la expresión \emph{open source}, o
«código abierto» fue rápidamente asociada con ideas y argumentaciones
basadas únicamente en valores de orden práctico, tales como desarrollar
o usar software potente y confiable. La mayoría de los partidarios del
«código abierto» llegaron al movimiento después de entonces y hacen la
misma asociación de conceptos. Este criterio práctico es aceptado por la
inmensa mayoría de los partidarios del open source.

En lo que respecta al software en sí mismo, tanto el software libre como
el open source describen prácticamente lo mismo, entonces, ¿por qué es
preferible el software libre?. El open source es pragmatismo en acción,
aplicado a la programación. El software libre es un movimiento social,
de profundas raíces éticas. Bueno, en este punto usted podría decir ¿y a
mí qué me importa?.. lo que me interesa es el software, no la política
ni nada por el estilo. Esbozaremos una respuesta\ldots{}

Primero y principal, todo el software libre es software open source pero
lo inverso no es cierto. Existen algunos programas (afortunadamente muy
pocos) con licencia open source que no son software libre. Esto
significa que los usuarios no tienen la posibilidad de disponer de ellos
libremente.

En segundo lugar, muchos productos que funcionan como computadoras (por
ejemplo, muchos dispositivos Android) contienen programas ejecutables
cuyo código fuente es software libre, pero los dispositivos no permiten
que el usuario instale versiones modificadas de esos ejecutables, es una
empresa específica la que tiene el poder de modificarlos. El usuario no
es libre de elegir qué aplicaciones ejecutar. A esta práctica se la
denomina «tivoización», con referencia al producto donde por primera vez
se descubrió esta implementación. Aunque el código fuente sea libre,
estos ejecutables no lo son. Según los criterios del código abierto,
esto no es un problema; sólo les interesa la licencia del código fuente.
Al centrar su atención exclusivamente en el código, no pueden apreciar
estas nuevas amenazas a la libertad de los usuarios, ya que se olvida
del hardware.., con lo cual estamos en una situación potencialmente muy
peligrosa, es decir, si nos enfocamos únicamente en el valor
práctico\ldots{} ¿para qué nos sirve tener acceso al código si no
disponemos de dispositivos en dónde ejecutarlos? En este punto yo lo
interrogo a usted, estimado lector: ¿no le parece que \emph{mirar para
otro lado}, tratando de olvidarnos de toda cuestión \emph{filosófica}
para concentrarnos únicamente en la practicidad del uso no es un bumerán
que se le volverá en contra\ldots{}?

Tercero, y para finalizar, el omitir hablar de ética y de libertad,
mencionando únicamente los beneficios prácticos inmediatos, para
«vender» el software más fácilmente a ciertos usuarios, especialmente a
las empresas, puede ser contraproducente. Si sólo se valoran los
aspectos técnicos, como la confiabilidad y la potencia de un programa,
se habla de una manera superficial, puramente práctica. Y acá está el
germen de la autodestrucción. Si no se enseña a la gente a valorar y
respetar su libertad..¿cómo pretender que la defienda?\ldots{}El
software libre es una cuestión de LIBERTAD, no de conveniencia.

\end{document}
